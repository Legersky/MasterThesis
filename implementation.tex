\section{Implementation}

\begin{upravit}

The designed method requires to compute arithmetic operations in $\Zomega$. Therefore, we have chosen Python-based programming language SageMath for the implementation of the extending window method as it contains many ready-to-use mathematical structure.  Using SageMath is very convenient as it also offers easily usable data structures or tools for plotting. Thus the code is more readable and we may focus on the algorithmic part of problem. On the other hand, SageMath is considerably slower than for example C++ or other low-level languages. Nevertheless, it is sufficient for our purpose.

The implementation is object-oriented. It consists of four classes. Class \emph{AlgorithmForParallelAddition} contains structures for computations in $\Zomega$. Specifically, we use the provided class \emph{PolynomialQuotientRing} to represent elements of $\Zomega$ and  \emph{NumberField} for obtaining numerical complex value of them. The class also links necessary instances and functions to construct algorithm for parallel addition by the extending window method for an algebraic integer $\omega$ given by its minimal polynomial $p$ and approximate complex value, a base $\beta\in\Zomega$, an alphabet $\A\subset\Zomega$ and input alphabet $\B$. Phase 1 of the extending window method is implemented in class \emph{WeightCoefficientsSetSearch} and Phase 2 in \emph{WeightFunctionSearch}. Class \emph{WeightFunction} holds the weight function $q$ computed in Phase 2. All classes are described in the following sections including lists of the important methods  with a short description. Sometimes, the notation from Chapter \ref{chap:methodDescription} is used for better understanding. For all implemented methods, see commented source code.  

Basically, weight function can be found just by creating an instance of \emph{AlgorithmForParallelAddition} and calling \textbf{findWeightFunction()}. For more comfortable usage, our implementation includes two interfaces -- a shell version and graphic user interface using interact in SageMath Cloud. The whole implementation is on the attached DVD or it can be downloaded from  \url{https://github.com/Legersky/ParallelAddition}.

\subsection{Class AlgorithmForParallelAddition}
This class constructs necessary structures for computation in $\Zomega$. It is \emph{PolynomialQuotientRing} obtained as a \emph{PolynomialRing} over integers factored by polynomial $p$. This is used for representation of elements of $\Zomega$ and arithmetic. We remark that it is independent on the choice of root of the  minimal polynomial $p$. But as we need also comparisons of numbers in~$\Zomega$ in modulus, we specify $\omega$ by its approximate complex value and we form a factor ring of rational polynomials by using class \emph{NumberField}. This enables us to get absolute values of elements of $\Zomega$ which can be then compared.

Method \textbf{findWeightFunction()} links together both phases of the extending window method to find the weight function $q$. That is used in the methods for addition and digit set conversion to process them as local functions. There are also verification methods.

Moreover, many methods for printing, plotting and saving outputs are implemented.

The constructor of class \emph{AlgorithmForParallelAddition} is 

\begin{method}{\_\_init\_\_}{minPol\_str, embd, alphabet, base, name='NumerationSystem', inputAlphabet=' ',\\
 printLog=True, printLogLatex=False, verbose=0}
Take \var{minPol\_str} which is a string of symbolic expression in the variable $x$ of an irreducible polynomial~$p$. The closest root of  \var{minPol\_str} to \var{embd} is used as the ring generator $\omega$ (see more in documentation of \emph{NumberField} in SageMath \cite{sage}). The structures for $\Zomega$ are constructed as described above. Setters \fun{setAlphabet}{alphabet}, \fun{setInputAlphabet}{inputAlphabet} and \fun{setBase}{base} are called.  Messages saved to logfile during existence of an instance are printed (using \LaTeX) on standard output depending on \var{printLog} and \var{printLogLatex}. The level of messages for a development is set by \var{verbose}. 
\end{method}

Methods for the construction of an addition algorithm which is computable in parallel by the designed extending window method are the following:

\begin{method}{\_findWeightCoefSet}{ max\_iterations, method\_number}
Create an instance of \emph{WeightCoefficientsSetSearch(method\_number)} and call its method \fun{findWeightCoefficientsSet}{max\_iterations} to obtain a weight coefficients set $\Q$.
\end{method}

\begin{method}{\_findWeightFunction}{ max\_input\_length,method\_number}
Create an instance of \emph{WeightFunctionSearch(method\_number)} and call its methods \fun{check\_one\_letter\_inputs}{max\_input\_length} and \\ \fun{findWeightFunction}{max\_input\_length} to obtain a weight function $q$.
\end{method}


\begin{method}{findWeightFunction}{ max\_iterations, max\_input\_length, method\_weightCoefSet=None,\\ method\_weightFunSearch=None}
It is verified that there are all representatives mod $\beta$ in the alphabet and that all elements of the input alphabet have their representatives mod $\beta-1$ in the alphabet, see Section \ref{sec:alphabet}.
Return the weight function $q$ obtained by calling \fun{\_findWeightCoefSet\\}{max\_iterations, method\_weightCoefSet} and \fun{\_findWeightFunction}{max\_input\_length, method\_weightFunSearch}.
\end{method}

The important function for the searching for possible weight coefficients is

\begin{method}{divideByBase}{divided\_number}
Using Theorem \ref{thm:divisibility}, check if \var{divided\_number} is divisible by the base $\beta$. If it is so, return the result of division, else return \var{None}.
\end{method}


Methods for the addition and the digit set conversion computable in parallel are following:

\begin{method}{addParallel}{a,b}
Sum up numbers represented by the lists of digits \var{a} and \var{b} digitwise and convert the result by \fun{parallelConversion}{}. 
\end{method}


\begin{method}{parallelConversion}{\_w}
Return $(\beta,\A)$-representation of the number represented by the list \var{\_w} of digits in the input alphabet $\B$. According to the equation \eqref{eq:conversionFormula}, it is computed locally by using the weight function $q$.
\end{method}


\begin{method}{localConversion}{w}
Return converted digit according to the equation \eqref{eq:conversionFormula} for the list of input digits \var{w}.
\end{method}


The correctness of the implementation of the extending window for a given numeration system can be verified by
 
\begin{method}{sanityCheck\_conversion}{ num\_digits}
Check whether the values of all possible numbers of the length \var{num\_digits} with digits in the input alphabet $\B$ are the same as their $(\beta, \A)$-representation obtained by \fun{parallelConversion}{}.   
\end{method}




\subsection{Class WeightCoefficientsSetSearch}
Class \emph{WeightCoefficientsSetSearch} implements Phase 1 of the extending window method described in Section \ref{subsec:phase1} with different methods how an intermediate weight coefficients set $\Q_k$ is extended to $\Q_{k+1}$. Algorithm~\ref{alg:extendWeightCoefSet} explains some of these methods. The most important method of the class is  \fun{findWeightCoefficientsSet}{} which returns a weight coefficients set $\Q$.

The constructor of the class is 
\begin{upravit}
\begin{method}{\_\_init\_\_}{ algForParallelAdd, method}
Initialize a ring generator $\omega$, base $\beta$, an alphabet $\A$ and input alphabet $\B$ by values obtained from \var{algForParallelAdd}. The parameter \var{method} characterizes how an intermediate weight coefficients set $\Q_k$ is extended to $\Q_{k+1}$. If \var{None}, then method 13 from Algorithm~\ref{alg:extendWeightCoefSet} is used as default.
\end{method}

Methods implementing Phase 1 are the following:

\begin{method}{\_findCandidates}{C}
Following Algorithm \ref{alg:searchCand}, return the list of lists \var{candidates} such that each element in \var{C} is covered by any value of the appropriate list in \var{candidates}.  
\end{method}


\begin{method}{\_chooseQk\_FromCandidates}{candidates}
Take the previous intermediate weight coefficients set $\Q_{k}$ as the class attribute and choose from \var{candidates} the intermediate weight coefficients set $\Q_{k+1}$ by Algorithm \ref{alg:extendWeightCoefSet}.
\end{method}


\begin{method}{\_getQk}{C}
Links together methods \fun{\_findCandidates}{C} and \fun{\_chooseQk\_FromCandidates}{} to return itermediate weight coefficients set $\Q_k$.
\end{method}


\begin{method}{findWeightCoefficientsSet}{ maxIterations}
According to Algorithm \ref{alg:weightCoefSet}, return the weight coefficients set $\Q$ which is build iteratively by using \fun{\_getQk}{}. The computation is aborted if the number of iterations exceeds \var{maxIterations}. 
\end{method}




\subsection{Class WeightFunctionSearch}

This class implements modified Phase 2 of the extending window method from Section \ref{sec:modifiedPhase2} with different methods of choice of possible weight coefficients sets and control of non-convergence. Methods 2a, 2b, 2c, 2d and 2e from Algorithm~\ref{alg:pickElement} correspond to 9, 15, 22, 23 and 14 respectively. A weight function $q$ is returned by method \fun{findWeightFunction}{}. The constructor of the class is

\begin{method}{\_\_init\_\_}{algForParallelAdd, weightCoefSet, method, maxInputs}
The ring generator $\omega$, base $\beta$, alphabet $\A$ and input alphabet $\B$ are initialized by the values obtained from \var{algForParallelAdd}. The weight coefficients set $\Q$ is set to \var{weightCoefSet}. The parameter \var{method} (the number of an experimental method or \verb+'2a'+, \verb+'2b'+, etc.) determines the way of the choice of a possible weight coefficients set $\Qwo{k}$ from $\Qwo{(k-1)}$, see Algorithm~\ref{alg:pickElement}. If \var{method} is \var{None}, then is the default method is 2b. Possible weight coefficients sets  are saved in a dictionary \var{\_Qw\_w}, which is set to be empty.
\end{method}

The following methods are used for search for a weight function $q$:

\begin{method}{\_find\_weightCoef\_for\_comb\_B}{combinations}
Take all combinations of input digits $\tupleo{(k-1)} \in \B^{k}$ in \var{combinations} such that $\#\Qwo{(k-1)}>1$, extend them by all letters $w_{-k}\in\B$ and find a possible weight coefficients set $\Qwo{k}$ by the method \fun{\_findQw}{$\tupleo{(k-1)}$}. If there is only one element in $\Qwo{k}$, it is saved as a solved input of the weight function $q$. Otherwise, the set $\Qwo{k}$ is saved in \var{\_Qw\_w} as an unsolved combination which requires extending of the window. The unsolved combinations are returned.  
\end{method}

\begin{method}{\_findQw}{w\_tuple}
Return a set $\Qwo{k}=\Q_{[\text{w\_tuple}]}$ by wrapping \fun{\_findQw\_once}{} into a while loop according to Algorithm~\ref{alg:possibleWeightCoefSetStable}. If $\Qwo{k}=\Qwo{(k-1)}$, then add a vertex $\tupleo{k}$ to a Rauzy graph $G_{k+1}$ and call \fun{\_checkCycles}{w\_tuple} in $G_{k}$.
\end{method}

\begin{method}{\_findQw\_once}{w\_tuple,Qw\_prev} 
Return a set $\Qwo{k}'=\Q'_{[\text{w\_tuple}]}$ obtained by Algorithm~\ref{alg:minimalSet} as a subset of \var{Qw\_prev}$=\Qwo{(k-1)}$. The set of possible carries from the right $\Qw{1}{k}$ is taken from the class attribute \var{\_Qw\_w}. The methods of Algorithm~\ref{alg:pickElement} are implemented here along with the experimental ones.
\end{method}

\begin{method}{\_checkCycles}{ w\_tuple}
\end{method}

\begin{upravit}
\begin{method}{findWeightFunction}{max\_input\_length}
Return the weight function $q$ unless the length of window exceeds \var{max\_input\_length}. Then an exception is raised. It implements Algorithm \ref{alg:weightFunction} by repetetive calling of the method \fun{\_find\_weightCoef\_for\_comb\_B}{} which extends length of window. This is done until all possible combinations of input digits are solved for some length of window $m$, i.e. $\max\{\#\Q_{[w_j,\dots, w_{j-m+1}]}:(w_j,\dots, w_{j-m+1}) \in \B^m \} = 1$.
\end{method}


\begin{method}{check\_one\_letter\_inputs}{max\_input\_length}
The method checks by Algorithm \ref{alg:oneletterSets} if there is a unique weight coefficient for inputs $({b,b,\dots,b})\in\B^m$ for some length of window $m$. Using Theorem \ref{thm:suffCondPhase1}, an exception is raised in the case of an infinite loop. Otherwise the list of inputs $(b,b,\dots,b)$ which have the largest length of the window is returned.
\end{method}

%\begin{method}{\_checkCycles}{ w\_tuple}
%
%
%
%
%\begin{method}{isSublist}{    def isSublist(\_list, \_sublist}
%
%\end{method}
%
%
%\begin{method}{find\_next\_letter}{    def find\_next\_letter(\_w,witness\_seq}
%
%\end{method}
%
%
%
%
%\begin{method}{\_findQw\_once}{w\_tuple,Qw\_prev}
%
%\end{method}
%
%
%\begin{method}{findWeightFunction}{ max\_input\_length}
%
%\end{method}
%
%
%\begin{method}{check\_one\_letter\_inputs}{ max\_input\_length}
%
%\end{method}






\subsection{Class WeightFunction}
This class serves for saving the weight function $q$. The constructor is

\begin{method}{\_\_init\_\_}{B}
Set the input alphabet to \var{B} and the maximum length of window to 1. Initialize the attribute \var{\_mapping} to an empty dictionary for saving the weight function $q$. 
\end{method}

The methods for saving and calling are following:

\begin{method}{addWeightCoefToInput}{\_input, coef}
Save the weight coefficient \var{coef} for \var{\_input} to the class attribute \var{\_mapping}. The digits of \var{\_input} must be in the input alphabet.
\end{method}

\begin{method}{getWeightCoef}{w}
The digits of the list \var{w} are taken from the left until the weight coefficient in the dictionary \var{\_mapping} is found. 
\end{method}

The result of the method \fun{getWeightCoef}{} is used to make this class callable, i.e., if \var{\_q} is an instance of \emph{WeightFunction}, then \var{\_q}.\fun{getWeightCoef}{w} is the same as \var{\_q}(\var{w}).






\subsection{User interfaces}
We provide two interfaces for running of the implemented extending window method -- the~shell version and graphic user interface.

\subsubsection{Shell}
SageMath must be installed. The implementation of the extending window method is launched in a shell by typing \verb+sage extending_window_method.sage <input_sample.sage>+. The parameters of the numeration system and setting of outputs and computation is given by the SageMath file \verb+input_sample.sage+. See Appendix \ref{app:inputSample} for an example of such a file.

The name of the numeration system, minimal polynomial of generator $\omega$, an approximate value of $\omega$, the base $\beta$, alphabet $\A$ and input alphabet $\B$ are set in the part INPUTS. The maximum number of iterations in Phase 1, maximal length of the window in Phase 2 and the launching of the sanity check are set in SETTING. 

The boolean values in the part SAVING determines which formats of the outputs are saved. All outputs are saved in the folder \verb+./outputs/<name>/+. General information about the computation can be saved in the TeX format, the computed weight function and local digit set conversion in the CSV file format. An inputs setting saved as a dictionary can be loaded by the interact interface. The log of the whole computation can be saved as a text file. There is also an option to save unsolved combinations in Phase 2 in the CSV file format in the case of the interruption of the program.

According to the boolean values in the part IMAGES, figures of the alphabet, input alphabet, weight coefficients set or part of the set $\Zomega$ with marked alphabet shifted into points which are divisible by the base $\beta$ are saved in the PNG format to the folder \verb+./outputs/<name>/+ \verb+img/+. Optionally, the weight coefficients set is plotted with the  bound given by the proof of Theorem~\ref{thm:suffCondPhase1}. Images of individual steps of both phases of the extending window method can be saved, too. For Phase 2, the search for the weight coefficient  is plotted for the digits given by \verb+phase2_input+.  

The program prints out all inputs and then it computes the weight function $q$ by calling \fun{findWeightFunction}{ max\_iterations, max\_input\_length}. The increments of the weight coefficients set in each iteration of Phase 1 are printed and then also the obtained weight coefficients set $\Q$. The longest tested combinations given by repetition of one letter are printed after the computation of \fun{check\_one\_letter\_inputs}{max\_input\_length}. During computing of Phase 2, the current length of window and the number of saved combinations are printed. At the end, the final length of window, elapsed time and info about saved outputs are printed.  

It is possible to batch process all input files in one folder by executing the bash script \verb+ewm_batch <folder_name>+.  

\subsubsection{Interact in SageMath Cloud}
The graphic user interface is implemented using an interact in SageMath Cloud. The parts of the interact are on Figure \ref{fig:interact1}, \ref{fig:interact2} and \ref{fig:interact3} in Appendix \ref{app:interact}. The functionality is basically the same as the shell version. An account on the website \url{https://cloud.sagemath.com} is needed to use the interact. Create a new project and load files \verb+extending_window_method_GUI.sagews+ and \verb+interact_ewm.sage+. After executing of the cell by Shift+Enter in the first one, the parameters of the numeration system are filled in the corresponding spaces or one of the previous settings is loaded by typing its name.  By default, the last inputs are shown in the form. The inputs are submitted by pressing the button Update. Using check-boxes, the formats for saving of the output are chosen and the search for the weight function is launched by pressing second button Update.

The printed output is similar to the shell output. In addition, it contains figures and it is formatted using \LaTeX. Moreover, the sanity check can be run for a given length, the weight coefficient for a tuple of  input digits is returned or images of individual steps of both phases are shown and saved.


\end{upravit}