\section{Implementation}
\label{sec:implementation}
Our implementation of the design is based on the program attached to \cite{vu}. The chosen programming language is SageMath. It is Python-based language with numerous implemented mathematical structures. That is the main reason of our choice -- the extending window method requires to handle elements of $\Zomega$ and arithmetic operations in this set. Moreover, SageMath provides various data structures and plotting tools.  The code is simpler and more similar to pseudocode than if we implemented in pure \Cpp. Due to it, we may concern ourselves with the algorithmic part of the problem instead of difficult programming. Unfortunately, SageMath is much slower than \Cpp.


The implementation is object-oriented. It consists of five classes. Class \emph{AlgorithmForParallelAddition} contains structures for computations in $\Zomega$. The build-in classes \emph{PolynomialQuotientRing} and \emph{NumberField} are used to represent elements of $\Zomega$ as an algebraic and complex numbers. The class also links together all functions and instances of other classes which are necessary to construct an algorithm for digit set conversion from $\B$ to $\A$ by the extending window method. The input parameters are an algebraic integer $\omega$ given by its minimal polynomial $m_\omega$ and an approximate complex value, a base $\beta\in\Zomega$, an alphabet $\A\subset\Zomega$ and input alphabet $\B\subset\Zomega$. 

Phase 1 of the extending window method is implemented in class \emph{WeightCoefficientsSetSearch} and Phase 2 in \emph{WeightFunctionSearch}. Class \emph{WeightFunction} serves to save a function $q$ computed in Phase 2. The last class \emph{ExceptionParAdd} is inherited from build-in class \emph{Exception} to distinguish between errors which are raised by the algorithm  of the extending window method and other ones.

We use notation from previous chapters in descriptions of the classes. We list only the most important methods of each class, see commented source code for all of them.  

For basic use, load \verb+AlgorithmForParallelAddition.sage+, create an instance of \emph{AlgorithmForParallelAddition} and call \textbf{findWeightFunction()}.

We also provide an interface -- a shell script with given parameters is executed. If the network access is enabled and the modul \verb+gspread+ is installed (see \cite{gspread}), then results of computation are automatically saved to Google spreadsheet \href{https://docs.google.com/spreadsheets/d/1TnhrHdefHfHa0WSeVs4q6XVj3epjPlPlnoekE0E1xeM/edit?usp=sharing}{ParallelAddition\_results}. The spreadsheet can be also used for loading input parameters. Both interfaces are described Section~\ref{sec:interfaces}. The whole implementation is on the attached DVD or it can be downloaded from  \url{https://github.com/Legersky/ParallelAddition}.



\subsection*{Class AlgorithmForParallelAddition}
This class constructs necessary structures for computation in $\Zomega$. It is \emph{PolynomialQuotientRing} obtained as a \emph{PolynomialRing} over integers factored by polynomial $p$. This is used for representation of elements of $\Zomega$ and arithmetic. We remark that it is independent on the choice of root of the  minimal polynomial $p$. But as we need also comparisons of numbers in~$\Zomega$ in modulus, we specify $\omega$ by its approximate complex value and we form a factor ring of rational polynomials by using class \emph{NumberField}. This enables us to get absolute values of elements of $\Zomega$ which can be then compared.

Method \textbf{findWeightFunction()} links together both phases of the extending window method to find the weight function $q$. That is used in the methods for addition and digit set conversion to process them as local functions. There are also verification methods.

Moreover, many methods for printing, plotting and saving outputs are implemented.

The constructor of class \emph{AlgorithmForParallelAddition} is 

\begin{method}{\_\_init\_\_}{minPol\_str, embd, alphabet, base, name='NumerationSystem', inputAlphabet=' ',\\
 printLog=True, printLogLatex=False, verbose=0}
Take \var{minPol\_str} which is a string of symbolic expression in the variable $x$ of an irreducible polynomial~$p$. The closest root of  \var{minPol\_str} to \var{embd} is used as the ring generator $\omega$ (see more in documentation of \emph{NumberField} in SageMath \cite{sage}). The structures for $\Zomega$ are constructed as described above. Setters \fun{setAlphabet}{alphabet}, \fun{setInputAlphabet}{inputAlphabet} and \fun{setBase}{base} are called.  Messages saved to logfile during existence of an instance are printed (using \LaTeX) on standard output depending on \var{printLog} and \var{printLogLatex}. The level of messages for a development is set by \var{verbose}. 
\end{method}

Methods for the construction of an addition algorithm which is computable in parallel by the designed extending window method are the following:

\begin{method}{\_findWeightCoefSet}{ max\_iterations, method\_number}
Create an instance of \emph{WeightCoefficientsSetSearch(method\_number)} and call its method \fun{findWeightCoefficientsSet}{max\_iterations} to obtain a weight coefficients set $\Q$.
\end{method}

\begin{method}{\_findWeightFunction}{ max\_input\_length,method\_number}
Create an instance of \emph{WeightFunctionSearch(method\_number)} and call its methods \fun{check\_one\_letter\_inputs}{max\_input\_length} and \\ \fun{findWeightFunction}{max\_input\_length} to obtain a weight function $q$.
\end{method}


\begin{method}{findWeightFunction}{ max\_iterations, max\_input\_length, method\_weightCoefSet=None,\\ method\_weightFunSearch=None}
It is verified that there are all representatives mod $\beta$ in the alphabet and that all elements of the input alphabet have their representatives mod $\beta-1$ in the alphabet, see Section \ref{sec:alphabet}.
Return the weight function $q$ obtained by calling \fun{\_findWeightCoefSet\\}{max\_iterations, method\_weightCoefSet} and \fun{\_findWeightFunction}{max\_input\_length, method\_weightFunSearch}.
\end{method}

The important function for the searching for possible weight coefficients is

\begin{method}{divideByBase}{divided\_number}
Using Theorem \ref{thm:divisibility}, check if \var{divided\_number} is divisible by the base $\beta$. If it is so, return the result of division, else return \var{None}.
\end{method}


Methods for the addition and the digit set conversion computable in parallel are following:

\begin{method}{addParallel}{a,b}
Sum up numbers represented by the lists of digits \var{a} and \var{b} digitwise and convert the result by \fun{parallelConversion}{}. 
\end{method}


\begin{method}{parallelConversion}{\_w}
Return $(\beta,\A)$-representation of the number represented by the list \var{\_w} of digits in the input alphabet $\B$. According to the equation \eqref{eq:conversionFormula}, it is computed locally by using the weight function $q$.
\end{method}


\begin{method}{localConversion}{w}
Return converted digit according to the equation \eqref{eq:conversionFormula} for the list of input digits \var{w}.
\end{method}


The correctness of the implementation of the extending window for a given numeration system can be verified by
 
\begin{method}{sanityCheck\_conversion}{ num\_digits}
Check whether the values of all possible numbers of the length \var{num\_digits} with digits in the input alphabet $\B$ are the same as their $(\beta, \A)$-representation obtained by \fun{parallelConversion}{}.   
\end{method}




\subsection*{Class WeightCoefficientsSetSearch}
Class \emph{WeightCoefficientsSetSearch} implements Phase 1 of the extending window method described in Section \ref{subsec:phase1} with different methods how an intermediate weight coefficients set $\Q_k$ is extended to $\Q_{k+1}$. Algorithm~\ref{alg:extendWeightCoefSet} explains some of these methods. The most important method of the class is  \fun{findWeightCoefficientsSet}{} which returns a weight coefficients set $\Q$.

The constructor of the class is 
\begin{upravit}
\begin{method}{\_\_init\_\_}{ algForParallelAdd, method}
Initialize a ring generator $\omega$, base $\beta$, an alphabet $\A$ and input alphabet $\B$ by values obtained from \var{algForParallelAdd}. The parameter \var{method} characterizes how an intermediate weight coefficients set $\Q_k$ is extended to $\Q_{k+1}$. If \var{None}, then method 13 from Algorithm~\ref{alg:extendWeightCoefSet} is used as default.
\end{method}

Methods implementing Phase 1 are the following:

\begin{method}{\_findCandidates}{C}
Following Algorithm \ref{alg:searchCand}, return the list of lists \var{candidates} such that each element in \var{C} is covered by any value of the appropriate list in \var{candidates}.  
\end{method}


\begin{method}{\_chooseQk\_FromCandidates}{candidates}
Take the previous intermediate weight coefficients set $\Q_{k}$ as the class attribute and choose from \var{candidates} the intermediate weight coefficients set $\Q_{k+1}$ by Algorithm \ref{alg:extendWeightCoefSet}.
\end{method}


\begin{method}{\_getQk}{C}
Links together methods \fun{\_findCandidates}{C} and \fun{\_chooseQk\_FromCandidates}{} to return itermediate weight coefficients set $\Q_k$.
\end{method}


\begin{method}{findWeightCoefficientsSet}{ maxIterations}
According to Algorithm \ref{alg:weightCoefSet}, return the weight coefficients set $\Q$ which is build iteratively by using \fun{\_getQk}{}. The computation is aborted if the number of iterations exceeds \var{maxIterations}. 
\end{method}




\subsection*{Class WeightFunctionSearch}

This class implements modified Phase 2 of the extending window method from Section \ref{sec:modifiedPhase2} with different methods of choice of possible weight coefficients sets and control of non-convergence. Methods 2a, 2b, 2c, 2d and 2e from Algorithm~\ref{alg:pickElement} correspond to 9, 15, 22, 23 and 14 respectively. A weight function $q$ is returned by method \fun{findWeightFunction}{}. The constructor of the class is

\begin{method}{\_\_init\_\_}{algForParallelAdd, weightCoefSet, method, maxInputs}
The ring generator $\omega$, base $\beta$, alphabet $\A$ and input alphabet $\B$ are initialized by the values obtained from \var{algForParallelAdd}. The weight coefficients set $\Q$ is set to \var{weightCoefSet}. The parameter \var{method} (the number of an experimental method or \verb+'2a'+, \verb+'2b'+, etc.) determines the way of the choice of a possible weight coefficients set $\Qwo{k}$ from $\Qwo{(k-1)}$, see Algorithm~\ref{alg:pickElement}. If \var{method} is \var{None}, then is the default method is 2b. Possible weight coefficients sets  are saved in a dictionary \var{\_Qw\_w}, which is set to be empty.
\end{method}

The following methods are used for search for a weight function $q$:

\begin{method}{\_find\_weightCoef\_for\_comb\_B}{combinations}
Take all combinations of input digits $\tupleo{(k-1)} \in \B^{k}$ in \var{combinations} such that $\#\Qwo{(k-1)}>1$, extend them by all letters $w_{-k}\in\B$ and find a possible weight coefficients set $\Qwo{k}$ by the method \fun{\_findQw}{$\tupleo{(k-1)}$}. If there is only one element in $\Qwo{k}$, it is saved as a solved input of the weight function $q$. Otherwise, the set $\Qwo{k}$ is saved in \var{\_Qw\_w} as an unsolved combination which requires extending of the window. The unsolved combinations are returned.  
\end{method}

\begin{method}{\_findQw}{w\_tuple}
Return a set $\Qwo{k}=\Q_{[\text{w\_tuple}]}$ by wrapping \fun{\_findQw\_once}{} into a while loop according to Algorithm~\ref{alg:possibleWeightCoefSetStable}. If $\Qwo{k}=\Qwo{(k-1)}$, then add a vertex $\tupleo{k}$ to a Rauzy graph $G_{k+1}$ and call \fun{\_checkCycles}{w\_tuple} in $G_{k}$.
\end{method}

\begin{method}{\_findQw\_once}{w\_tuple,Qw\_prev} 
Return a set $\Qwo{k}'=\Q'_{[\text{w\_tuple}]}$ obtained by Algorithm~\ref{alg:minimalSet} as a subset of \var{Qw\_prev}$=\Qwo{(k-1)}$. The set of possible carries from the right $\Qw{1}{k}$ is taken from the class attribute \var{\_Qw\_w}. The methods of Algorithm~\ref{alg:pickElement} are implemented here along with the experimental ones.
\end{method}

\begin{method}{\_checkCycles}{ w\_tuple}
\end{method}

\begin{upravit}
\begin{method}{findWeightFunction}{max\_input\_length}
Return the weight function $q$ unless the length of window exceeds \var{max\_input\_length}. Then an exception is raised. It implements Algorithm \ref{alg:weightFunction} by repetetive calling of the method \fun{\_find\_weightCoef\_for\_comb\_B}{} which extends length of window. This is done until all possible combinations of input digits are solved for some length of window $m$, i.e. $\max\{\#\Q_{[w_j,\dots, w_{j-m+1}]}:(w_j,\dots, w_{j-m+1}) \in \B^m \} = 1$.
\end{method}


\begin{method}{check\_one\_letter\_inputs}{max\_input\_length}
The method checks by Algorithm \ref{alg:oneletterSets} if there is a unique weight coefficient for inputs $({b,b,\dots,b})\in\B^m$ for some length of window $m$. Using Theorem \ref{thm:suffCondPhase1}, an exception is raised in the case of an infinite loop. Otherwise the list of inputs $(b,b,\dots,b)$ which have the largest length of the window is returned.
\end{method}

%\begin{method}{\_checkCycles}{ w\_tuple}
%
%
%
%
%\begin{method}{isSublist}{    def isSublist(\_list, \_sublist}
%
%\end{method}
%
%
%\begin{method}{find\_next\_letter}{    def find\_next\_letter(\_w,witness\_seq}
%
%\end{method}
%
%
%
%
%\begin{method}{\_findQw\_once}{w\_tuple,Qw\_prev}
%
%\end{method}
%
%
%\begin{method}{findWeightFunction}{ max\_input\_length}
%
%\end{method}
%
%
%\begin{method}{check\_one\_letter\_inputs}{ max\_input\_length}
%
%\end{method}






\subsection*{Class WeightFunction}
This class serves for saving the weight function $q$. The constructor is

\begin{method}{\_\_init\_\_}{B}
Set the input alphabet to \var{B} and the maximum length of window to 1. Initialize the attribute \var{\_mapping} to an empty dictionary for saving the weight function $q$. 
\end{method}

The methods for saving and calling are following:

\begin{method}{addWeightCoefToInput}{\_input, coef}
Save the weight coefficient \var{coef} for \var{\_input} to the class attribute \var{\_mapping}. The digits of \var{\_input} must be in the input alphabet.
\end{method}

\begin{method}{getWeightCoef}{w}
The digits of the list \var{w} are taken from the left until the weight coefficient in the dictionary \var{\_mapping} is found. 
\end{method}

The result of the method \fun{getWeightCoef}{} is used to make this class callable, i.e., if \var{\_q} is an instance of \emph{WeightFunction}, then \var{\_q}.\fun{getWeightCoef}{w} is the same as \var{\_q}(\var{w}).







\section{User guide}
We provide two options of loading inputs for running the implemented extending window method. SageMath must be installed as they are executed as shell scripts.

The first one is launching in a shell by typing \verb+sage ewm_inputs.sage+. The parameters are given in the head of the file \verb+ewm_inputs.sage+, see 
Appendix~\ref{app:interfaces} for an example.

 We describe four parts of the file.
The name of the numeration system, minimal polynomial of generator $\omega$, an approximate value of $\omega$, the base $\beta$, alphabet $\A$ and input alphabet $\B$ are set in the part INPUTS. Different methods of choice for Phase 1 and 2 can be set. If there are more methods in the lists, then methods for Phase~1 are compared first. Next, each distinct result is processed with each method for Phase~2.  

For verification of output, \fun{sanityCheck\_conversion}{} is launched according to the boolean value in the part SANITY CHECK. 

The boolean values in the part SAVING determines which formats of the outputs are saved. All outputs are saved in the folder \verb+./outputs/<name>/+, where \verb+<name>+ is the name of the numeration system. General information about the computation can be saved in the \LaTeX{} format, a computed weight function and local digit set conversion in the CSV file format.  A log of the whole computation can be saved as a text file.

Figures of the alphabet, input alphabet or weight coefficients set are saved in the PNG format in the folder \verb+./outputs/<name>/img/+ according to the boolean values in the part IMAGES. Images of individual steps of both phases of the extending window method can be also saved. For Phase 2, searching for a weight coefficient  is plotted for given input digits.  

The program prints out all inputs and then it computes a weight function $q$ by calling \fun{findWeightFunction}{}. Intermediate weight coefficients sets in each iteration of Phase 1 and the obtained weight coefficients set $\Q$ are printed out. Non-convergence of Phase~2 for combinations given by repetition $b\in\B$ is verified by \fun{check\_one\_letter\_inputs}{}. The processed length of window is showed during computing of Phase 2. At the end, the final length of window, elapsed time and info about saved outputs are printed. Results are also saved in the Google spreadsheet \href{https://docs.google.com/spreadsheets/d/1TnhrHdefHfHa0WSeVs4q6XVj3epjPlPlnoekE0E1xeM/edit?usp=sharing}{ParallelAddition\_results} in the worksheet \verb+results+ and \verb+comparePhase1+ if there are more methods for Phase~1.

The second option of loading input parameters is to execute the script \verb+ewm_gspread+- \verb+sheet.sage+ (see Appendix~\ref{app:interfaces}). Parameters are loaded from the worksheet \verb+inputs+ in the Google spreadsheet \href{https://docs.google.com/spreadsheets/d/1TnhrHdefHfHa0WSeVs4q6XVj3epjPlPlnoekE0E1xeM/edit?usp=sharing}{ParallelAddition\_results}. The column A marks whether a row should be tested. The columns B--G, i.e., \emph{Name}, \emph{Alphabet}, \emph{Input alphabet}, \emph{Approximate value of ring generator omega}, \emph{Minimal polynomial omega} and \emph{Base} must be filled. If the column \emph{Input alphabet} is empty, then the input alphabet $\A+\A$ is used. Methods for Phase~1, resp. 2, are given in the header cell C1, resp. C2.

Program runs in the same way as before, but results are saved only to the Google spreadsheet. Notice that column A and cells with the methods may be changed  after executing the script, but other cells or order of rows should not be modified. The reason is that the program reads the methods at the beginning and it remembers position of rows to be tested, but the parameters are loaded on the fly.

