\newcommand{\Zomega}{\mathbb{Z}[\omega]}
\newcommand{\Zbeta}{\mathbb{Z}[\beta]}

\newcommand{\ZZ}{\mathbb{Z}}
\newcommand{\QQ}{\mathbb{Q}}
\newcommand{\CC}{\mathbb{C}}
\newcommand{\NN}{\mathbb{N}}
\newcommand{\RR}{\mathbb{R}}

% \newcommand{\OO}{\mathbb{O}}
\newcommand{\II}{\mathbb{I}}

\newcommand{\A}{\mathcal{A}}
\newcommand{\B}{\mathcal{B}}
\newcommand{\Q}{\mathcal{Q}}

\newcommand{\Qb}[1]{\mathcal{Q}_{[b^{#1}]}}

\newcommand{\fin}[1]{\text{Fin}_{#1}(\beta)}

\newcommand{\multMat}[1]{\sum_{i=0}^{d-1} {#1}_i S^i}

% \newcommand{\vect}[1]{\begin{pmatrix}
%             {#1}_0 \\
%             {#1}_1 \\
%             \vdots \\
%             {#1}_{d-1} 
%             \end{pmatrix}}

\newcommand{\norm}[2]{\left\lVert#1\right\rVert_{#2}}
\newcommand{\Mnorm}[2]{\left|\lVert#1\right|\rVert_{#2}}
\newcommand{\normBeta}[1]{\norm{#1}{\beta}}


% \def\changemargin#1#2{\list{}{\rightmargin#2\leftmargin#1}\item[]}
% \let\endchangemargin=\endlist 

% \newenvironment{method}[2]{
% \noindent \textbf{#1(}\textit{#2}\textbf{)}
% \vspace{-5pt}
% \begin{changemargin}{3em}{0em}}
% {\end{changemargin}}

% \newcommand{\var}[1]{\textit{#1}}
% \newcommand{\fun}[2]{\textbf{#1}(\var{#2})}


\usepackage{mathtools, mathdots}

% \usepackage{algorithm}
% \usepackage{algorithmic}
% \renewcommand{\algorithmicrequire}{\textbf{Input:}}
% \algsetup{indent=2em}

% \usepackage[nomessages]{fp}
% \usepackage{sidecap}
% \usepackage{enumerate}
% \usepackage{pgffor}


\renewcommand\Re{\operatorname{Re}}
\renewcommand\Im{\operatorname{Im}}

% \usepackage{pifont}
% \renewcommand\checkmark{\ding{51}}
% \newcommand\xmark{\ding{55}}


% \usepackage{pgfkeys}
% \pgfkeys{
%  /phaseOnecaptions array/.is family, /phaseOnecaptions array,
%  .unknown/.style = {\pgfkeyscurrentname/.initial = #1},
% }
% 
% \newcommand\figurehascaptionOne[1]{\pgfkeys{/phaseOnecaptions array, #1}}
% \newcommand\getcaptionOne[1]{\pgfkeysvalueof{/phaseOnecaptions array/#1}}
% 
% \pgfkeys{
%  /phase2captions array/.is family, /phase2captions array,
%  .unknown/.style = {\pgfkeyscurrentname/.initial = #1},
% }
% 
% \newcommand\figurehascaptionTwo[1]{\pgfkeys{/phase2captions array, #1}}
% \newcommand\getcaptionTwo[1]{\pgfkeysvalueof{/phase2captions array/#1}}


% \hyphenation{coef-fi-cient}
% \hyphenation{Algorithm-For-Parallel-Addition}
% \hyphenation{Polynomial-Quotient-Ring}



\newtheorem{theo}{Theorem}[chapter]
\newtheorem{lem}[theo]{Lemma}

\theoremstyle{definition}
\newtheorem{defn}{Definition}[chapter]
\newtheorem{exmp}{Example}[chapter]

