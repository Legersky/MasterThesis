\begin{upravit}
For construction of the set $\Q_{[w_j,\dots, w_{j-m+1}]}$ we first choose such elements of $\Q_{[w_j,\dots, w_{j-m+2}]}$ which are the only possible to cover some value $x \in w_0 + \Q_{[w_{j-1},\dots, w_{j-m+1}]}$. Other elements from $\Q_{[w_j,\dots, w_{j-m+2}]}$ which cover an uncovered value are added one by one to $\Q_{[w_j,\dots, w_{j-m+1}]}$ until each $x$ equals $a+\beta q_j$ for some $q_j$ in $\Q_{[w_j,\dots, w_{j-m+1}]}$ and $a\in\A$. The pseudocode is in Algorithm~\ref{alg:minimalSet}. 

\begin{algorithm}
  \caption{Search for set $\Qwo{k}$ }
    \label{alg:minimalSet}
  \begin{algorithmic}[1]
    \REQUIRE{Input digit $w_0$, set of possible carries $\Qw{1}{k}$, previous set of possible weight coefficients $\Qwo{k}$}
    \STATE $C:=\emptyset$
    \FORALL{$x \in w_0 + \Qw{1}{k}$}
    	\STATE $S_x:= \{q_0\in\Qwo{(k-1)}\colon\, \exists \, a\in\A: x=a+\beta q_0 \}$
        \STATE Add $C_x$ to $C$
    \ENDFOR
    \STATE $\Qwo{k}:=\emptyset$
    \FORALL{$C_x\in C$}
    	\IF {$\#C_x=1$}
    		\STATE Add the element $q\in C_x$ to $\Qwo{k}$
	        \STATE Remove sets $C_{x'}$ such that $q \in C_{x'}$  from the set $C$ 
	    \ENDIF
	\ENDFOR
    \WHILE{$C\neq\emptyset$}
        \STATE By Algorithm~\ref{alg:pickElement},  pick an element $q$ from $\bigcup C$
        \STATE Add the element $q$ to $\Qwo{k}$
        \STATE Remove sets $C_x$ such that $q \in C_x$  from the set $C$ 
    \ENDWHILE
    \RETURN $\Qwo{k}$
  \end{algorithmic}
\end{algorithm}


\begin{algorithm}
  \caption{Choose one element from the set of covering $C$ }
    \label{alg:pickElement}
  \begin{algorithmic}[1]
    \REQUIRE{set of covering $C$, method number $M$}
    \IF {$M=9$}
    	\STATE $g:=$ point of gravity $\bigcup C$ jernfkjnfk
    	\STATE 
    	\RETURN 
    \ENDIF

  \end{algorithmic}
\end{algorithm}


Notice that the result of Algorithm~\ref{alg:minimalSet} is influenced by the way how we pick an element on line~\ref{line:pickElement}. It can be done deterministically or non-deterministically. We use the following deterministic choice -- suppose that we want to choose from elements $x_1=\sum_{i=0}^{d-1}x_{1,i}\omega^i, x_2=\sum_{i=0}^{d-1}x_{2,i}\omega^i, \dots,x_n=\sum_{i=0}^{d-1}x_{n,i}\omega^i\in\Zomega$, where $d$ is the degree of $\omega$. Let $a_0,\dots,a_{d-1}\in\ZZ$ be such that 
$$
\sum_{i=0}^{d-1}a_i\omega^i=\sum_{j=1}^n x_j\,.
$$ 
Set $c:=\sum_{i=0}^{d-1}c_i\omega^i \in\Zomega$ with $c_i=[\frac{a_i}{n}]$ where $[\cdot]$ denotes rounding to the nearest integer. Let the index set $I_0\subset\{1,\dots,n\}$ be such that 
$$
|x_{j,0}-c_0|=\min\{|x_{k,0}-c_0|\colon k\in{1,\dots,n}\}
$$
for all $j\in I_0$. For all $i\in\{1,\dots,d-1\}$, let the index set $I_i\subset I_{i-1}$ be such that
$$
|x_{j,i}-c_i|=\min\{|x_{k,i}-c_i|\colon k\in I_{i-1}\}
$$
for all $j\in I_i$. If there is only one element in the index set $I_{d-1}=\{j_0\}$, choose the element~$x_{j_0}$. Otherwise choose $j_0\in I_{d-1}$ such that $x_{j_0,0}$ is the smallest from all $x_{j,0}$ such that $j\in I_{d-1}$. If there are more such elements, then choose from them according to the value $x_{j,1}$ etc. 

In other words, we take the elements which are the ``closest'' ones to the rounded center of gravity~$c$ of the values $x_1,\dots,x_n$ where ``closest'' is measured by absolute value of the first coordinate of $\pi(x_j)-\pi(c)$. In case of equality, according to the second coordinate etc. If there is more than one such element, we choose the element $x_{j_0}$ with the smallest first, resp. second, etc. coordinate of $\pi(x_{j_0})$.  
\end{upravit}