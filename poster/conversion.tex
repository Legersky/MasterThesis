\block{Digit set conversion from $\A+\A$ into $\A$}{
	A rewriting rule $R(x)= x-\beta$ is used, since $0=q_j \beta^j \cdot R(\beta) =q_j\cdot \beta^{j+1} -\beta  q_j \cdot \beta^{j}$:
	\begin{center}
	\begin{tabular}{rcccccccclcl}
	$w_{n'}$ & $w_{{n'}-1}$ & $\cdots$ & $w_{j+1}$ &\textcolor{red}{$w_{j}$} &$w_{j-1}$ & $\cdots$ & $w_1$ &$w_0$ &$\bullet$ & $=$&$(x+y)_{\beta,\A+\A}$\\
	  &   & &   &  &$q_{j-2}$ & $\iddots$ &  &  &\\
	   &   & &   & \textcolor{red}{$q_{j-1}$}  & $-\beta q_{j-1}$ &  &  &  &\\
	  &   &  $\iddots$&  $q_{j}$ &  \textcolor{red}{$-\beta q_{j}$} & & &  &  &\\ \hline
	$z_{n} \cdots z_{n'}$ & $z_{{n'}-1}$ & $\cdots$ & $z_{j+1}$ &\textcolor{red}{$z_{j}$} &$z_{j-1}$ & $\cdots$ & $z_1$ &$z_0$ &$\bullet$  & $=$&$(x+y)_{\beta,\A}$\\
	\end{tabular}	
	\end{center}
	\coloredbox{The crucial point of any conversion is the choice of the weight coefficient $q_j$ such that $$\textcolor{red}{z_j=w_j + q_{j-1} - q_j \beta} \in \A\,.$$}
	}