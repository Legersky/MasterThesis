\documentclass[24pt, a0paper, portrait, margin=0mm, innermargin=15mm, blockverticalspace=15mm, colspace=15mm, subcolspace=-5mm]{tikzposter}
\tikzposterlatexaffectionproofoff



\usepackage{tabulary}
\usepackage{caption}
\usepackage{amsmath, amsthm, amssymb, units, dsfont}
% \usepackage[nomessages]{fp}
\usepackage{sidecap}
\usepackage{enumerate}
\usepackage{xcolor}

\usepackage{mathtools, mathdots}
\usepackage{pgffor}
\usepackage{pdflscape}
\usepackage{afterpage}
\usepackage{chngcntr}
\usepackage{multirow}
\usepackage{tabulary}
\usepackage{listings}


\usepackage{breqn}
\usepackage{hyperref}

\newcommand{\komentar}[1]{\textcolor{red}{\MakeUppercase{#1}} \newline}
\newenvironment{upravit}{\color{blue}}{}

\newcommand{\Zomega}{\mathbb{Z}[\omega]}
\newcommand{\Zbeta}{\mathbb{Z}[\beta]}

\newcommand{\ZZ}{\mathbb{Z}}
\newcommand{\QQ}{\mathbb{Q}}
\newcommand{\CC}{\mathbb{C}}
\newcommand{\NN}{\mathbb{N}}
\newcommand{\RR}{\mathbb{R}}

% \newcommand{\OO}{\mathbb{O}}
\newcommand{\II}{\mathbb{I}}

\newcommand{\A}{\mathcal{A}}
\newcommand{\B}{\mathcal{B}}
\newcommand{\Q}{\mathcal{Q}}

\newcommand{\Qw}[3][w]{\Q_{[#1_{-#2}, \dots, #1_{-#3}]}}
\newcommand{\Qwo}[2][w]{\Q_{[#1_{0}, \dots, #1_{-#2}]}}

\newcommand{\tuple}[3][w]{(#1_{-#2}, \dots, #1_{-#3})}
\newcommand{\tupleo}[2][w]{(#1_{0}, \dots, #1_{-#2})}

%\newcommand{\Qb}[1]{\mathcal{Q}_{[b^{#1}]}}
\newcommand{\Qb}[1]{\mathcal{Q}_{[\scriptstyle b]}^{\scriptstyle #1}}

\newcommand{\fin}[1]{\text{Fin}_{#1}(\beta)}

\newcommand{\multMat}[1]{\sum_{i=0}^{d-1} {#1}_i S^i}



\newcommand{\vect}[1]{\begin{pmatrix}
             {#1}_0 \\
             {#1}_1 \\
             \vdots \\
             {#1}_{d-1} 
             \end{pmatrix}}
             
\newcommand{\enum}[1]{({#1}_0,\ldots,{#1}_{d-1})}             

\newcommand{\vertiii}[1]{{\left\vert\kern-0.25ex\left\vert\kern-0.25ex\left\vert #1\right\vert\kern-0.25ex\right\vert\kern-0.25ex\right\vert}}
    
\newcommand{\norm}[2]{\left\lVert#1\right\rVert_{#2}}
\newcommand{\Mnorm}[2]{\vertiii{#1}_{#2}}
\newcommand{\normBeta}[1]{\norm{#1}{\beta}}
\newcommand{\MnormBeta}[1]{\Mnorm{#1}{\beta}}

\renewcommand\Re{\operatorname{Re}}
\renewcommand\Im{\operatorname{Im}}

\usepackage{algorithm}
\usepackage{algorithmic}
\renewcommand{\algorithmicrequire}{\textbf{Input:}}
\renewcommand{\algorithmicensure}{\textbf{Ouput:}}
\algsetup{indent=2em}

 \usepackage{pifont}
 \renewcommand\checkmark{\ding{51}}
 \newcommand\xmark{\ding{55}}

 \newcommand{\var}[1]{\textit{#1}}
 \newcommand{\fun}[2]{\textbf{#1}(\var{#2})}

 \def\changemargin#1#2{\list{}{\rightmargin#2\leftmargin#1}\item[]}
 \let\endchangemargin=\endlist 

 \newenvironment{method}[2]{
 \noindent \textbf{#1(}\textit{#2}\textbf{)}
 \vspace{-5pt}
 \begin{changemargin}{3em}{0em}}
 {\end{changemargin}}

\def\Cpp{{C\nolinebreak[4]\hspace{-.05em}\raisebox{.4ex}{\tiny\bf ++}}}

 \usepackage{pgfkeys}
 \pgfkeys{
  /phaseOnecaptions array/.is family, /phaseOnecaptions array,
  .unknown/.style = {\pgfkeyscurrentname/.initial = #1},
 }
 
 \newcommand\figurehascaptionOne[1]{\pgfkeys{/phaseOnecaptions array, #1}}
 \newcommand\getcaptionOne[1]{\pgfkeysvalueof{/phaseOnecaptions array/#1}}
 
 \pgfkeys{
  /phase2captions array/.is family, /phase2captions array,
  .unknown/.style = {\pgfkeyscurrentname/.initial = #1},
 }
 
 \newcommand\figurehascaptionTwo[1]{\pgfkeys{/phase2captions array, #1}}
 \newcommand\getcaptionTwo[1]{\pgfkeysvalueof{/phase2captions array/#1}}


% \hyphenation{coef-fi-cient}
% \hyphenation{Algorithm-For-Parallel-Addition}
% \hyphenation{Polynomial-Quotient-Ring}







\title{Construction of algorithms for parallel addition} 
\author{Jan Legersk\'y, Czech Technical University in Prague} 
%\institute{Czech Technical University in Prague}


\definecolorstyle{Czech} {
\definecolor{colorOne}{HTML}{34888C}%116699
\definecolor{colorTwo}{HTML}{C1E1DC}
\definecolor{colorThree}{HTML}{FFF2BE}
%\definecolor{colorOne}{HTML}{34888C}%116699
%\definecolor{colorTwo}{HTML}{7CAA2D}
%\definecolor{colorThree}{HTML}{F5E356}
%\definecolor{colorOne}{HTML}{4F6457}%116699
%\definecolor{colorTwo}{HTML}{D9B44A}
%\definecolor{colorThree}{HTML}{ACD0C0}
%\definecolor{colorOne}{HTML}{0D5078}%116699
%\definecolor{colorTwo}{HTML}{A2C4D9}
%\definecolor{colorThree}{HTML}{FCF0AD}
}{
     % Background Colors
    \colorlet{backgroundcolor}{colorTwo}
    \colorlet{framecolor}{colorThree}
    % Title Colors
    \colorlet{titlebgcolor}{colorOne}
    \colorlet{titlefgcolor}{white}
    % Block Colors
    \colorlet{blocktitlebgcolor}{white}
    \colorlet{blocktitlefgcolor}{colorOne}
    \colorlet{blockbodybgcolor}{white}
    \colorlet{blockbodyfgcolor}{black}
    % Innerblock Colors
    \colorlet{innerblocktitlebgcolor}{white}
    \colorlet{innerblocktitlefgcolor}{black}
    \colorlet{innerblockbodybgcolor}{colorThree}
    \colorlet{innerblockbodyfgcolor}{black}
    % Note colors
    \colorlet{notefgcolor}{black}
    \colorlet{notebgcolor}{colorThree}
    \colorlet{notefrcolor}{colorThree}
}


\usebackgroundstyle{Default} %Rays
\usetitlestyle{Filled}
\usecolorstyle{Czech}

\useblockstyle[bodyoffsety=12mm]{Slide}
\usenotestyle{Default}



\settitle{ \centering \vbox{
%\@titlegraphic\\[\TP@titlegraphictotitledistance] 
\centering
\color{titlefgcolor} {\bfseries \Huge \sc \@title \par}
\vspace*{1em}
{\huge \@author \par}% \vspace*{1em} {\LARGE \@institute}
}}


\renewcommand{\Qw}[3][w]{\Q_{[#1_{j-#2}, \dots, #1_{j-#3}]}}
\renewcommand{\Qwo}[2][w]{\Q_{[#1_{j}, \dots, #1_{j-#2}]}}

\renewcommand{\tuple}[3][w]{(#1_{j-#2}, \dots, #1_{j-#3})}
\renewcommand{\tupleo}[2][w]{(#1_{j}, \dots, #1_{j-#2})}




\begin{document}
\maketitle[titletoblockverticalspace=15mm]
\begin{columns} 
\column{0.4}
\block{Abstract}{
Parallel addition is used for example in fast division algorithms with an integer base and alphabet.  An extending window method, which is due to M. Svobodov\'a, attempts to construct parallel addition algorithms in more general numeration systems. 

We discuss the sucessfulness of the method based on given parameters and we develop tools to reveal failure. 

Besides theoretical results, we implemented the method with several inner modifications in SageMath. The easy control of failure allowed to test more than 5000 different numeration systems to obtain a parallel addition algorithm in about 120 cases.
}

\block{Numeration system}{
	\begin{enumerate}[--]
		\item Let $\omega$ be an algebraic integer and let $\Zomega$ be the smallest ring containing $\ZZ$ and $\omega$
		\item Let $\beta\in\Zomega$ with the monic minimal polynomial $m_\beta$ be such that $|\beta|>1$ and let $\A\subset\Zomega$ be a finite set containing 0 and 1.
		\item A pair $(\beta, \A)$ is called a \emph{positional numeration system} with \emph{base} $\beta$ and \emph{digit set (alphabet)} $\A$.
		\item A complex number $x$ has a finite \emph{$(\beta, \A)$-representation} if~ there exist digits $x_n,x_{n-1}, \dots x_{-m}\in\A$ such that $x=\sum_{j=-m}^n x_j \beta^j$:
		$$
		(x)_{\beta,\A}= x_n x_{n-1}\cdots x_1 x_0 \bullet x_{-1} x_{-2} \cdots x_{-m}.
		$$
	\end{enumerate}
	} 
\block{Addition}{
	\begin{enumerate}
	\item $(\beta,\A)$-representations of $x$ and $y$ are summed up digitwise:
	\begin{center}
	\begin{tabular}{ccccclcl}
	$x_{n'}$ & $x_{{n'}-1}$ & $\cdots$ & $x_1$ &$x_0$ &$\bullet$ & $=$& $(x)_{\beta,\A}$ \\%&$x_{-1}$ & $x_{-2}$ & $\cdots$ & $x_{-m'}$ \\
	$y_{n'}$ & $y_{{n'}-1}$ & $\cdots$ & $y_1$ &$y_0$ &$\bullet$ & $=$& $(y)_{\beta,\A}$ \\ \hline % &$y_{-1}$ & $y_{-2}$ & $\cdots$ & $y_{-m'}$ \\ \hline
	$w_{n'}$ & $w_{{n'}-1}$ & $\cdots$ & $w_1$ &$w_0$ &$\bullet$ & $=$& $(x+y)_{\beta,\A+\A}$\,, % &$w_{-1}$ & $w_{-2}$ & $\cdots$ & $w_{-m'}$ \,,
	\end{tabular}	
	\end{center}
	  where
	  $$
	    w_j=x_j+y_j \in \A +\A\,.
	  $$
	\item The $(\beta,\A+\A)$-representation $w_{n'} w_{{n'}-1} \cdots w_1 w_0 \bullet$  is converted into $$z_{n} z_{n-1}\cdots z_1 z_0 \bullet=(x+y)_{\beta,\A}\,.$$
	\end{enumerate}
	}
\block{Digit set conversion from $\A+\A$ into $\A$}{
	A rewriting rule $R(x)= x-\beta$ is used, since $0=q_j \beta^j \cdot R(\beta) =q_j\cdot \beta^{j+1} -\beta  q_j \cdot \beta^{j}$:
		\vspace{-15pt}
	\begin{center}
	\begin{tabular}{rcccccccclcl}
	$w_{n'}$ & $w_{{n'}-1}$ & $\cdots$ & $w_{j+1}$ &\textcolor{red}{$w_{j}$} &$w_{j-1}$ & $\cdots$ & $w_1$ &$w_0$ &$\bullet$ & $=$&$(x+y)_{\beta,\A+\A}$\\
	  &   & &   &  &$q_{j-2}$ & $\iddots$ &  &  &\\
	   &   & &   & \textcolor{red}{$q_{j-1}$}  & $-\beta q_{j-1}$ &  &  &  &\\
	  &   &  $\iddots$&  $q_{j}$ &  \textcolor{red}{$-\beta q_{j}$} & & &  &  &\\ \hline
	$z_{n} \cdots z_{n'}$ & $z_{{n'}-1}$ & $\cdots$ & $z_{j+1}$ &\textcolor{red}{$z_{j}$} &$z_{j-1}$ & $\cdots$ & $z_1$ &$z_0$ &$\bullet$  & $=$&$(x+y)_{\beta,\A}$\\
	\end{tabular}	
	\end{center}
	\vspace{15pt}
	\coloredbox{The crucial point of any conversion is the choice of the weight coefficient $q_j$ such that $$\textcolor{red}{z_j=w_j + q_{j-1} - q_j \beta} \in \A\,.$$}
	}
\block{Standard vs. parallel conversion}{
	Standard conversion -- \textcolor{red}{$z_j=z_j(w_j, w_{j-1}, \dots, w_0)$}:
   	\begin{center}
	\begin{tabular}{rcccccccccl}
	&$w_{n'}$ & $w_{{n'}-1}$ & $\cdots$ &$w_{j+1}$ &\textcolor{red}{$w_{j}$} &\textcolor{red}{$w_{j-1}$}& \textcolor{red}{$\cdots$}  & \textcolor{red}{$w_1$} &\textcolor{red}{$w_0$} &$\bullet$ \\
	$\longrightarrow z_{n'+1}$ & $z_{{n'}}$ & $z_{n'+1}$&$\cdots$ & $z_{j+1}$ &\textcolor{red}{$z_{j}$} &$z_{j-1}$ & $\cdots$ & $z_1$ &$z_0$ &$\bullet$ \\
	\end{tabular}	
	\end{center}

%   Parallel conversion -- \textcolor{red}{$z_j=z_j(w_{j+t}, \dots, w_{j-r})$} for fixed $r,t\in\NN$ (Avizienis, 1961):
%      	\begin{center}
%	\begin{tabular}{rcccccccccl}
%	$\cdots$&$w_{j+t+1}$ & \textcolor{red}{$w_{j+t}$} & \textcolor{red}{$\cdots$} &\textcolor{red}{$w_{j+1}$} &\textcolor{red}{$w_{j}$} &\textcolor{red}{$w_{j-1}$}& \textcolor{red}{$\cdots$}  & \textcolor{red}{$w_{j-r}$} &$w_{j-r-1}$ &$\cdots$ \\
%	$\longrightarrow \cdots$ & $z_{j+t+1}$ & $z_{j+t}$&$\cdots$ & $z_{j+1}$ &\textcolor{red}{$z_{j}$} &$z_{j-1}$ & $\cdots$ & $z_{j-r}$ &$z_{j-r-1}$ &$\cdots$ \\
%	\end{tabular}	
%	\end{center}
	    
   Parallel conversion -- \textcolor{red}{$z_j=z_j(w_{j}, \dots, w_{j-r})$} for fixed $r\in\NN$:
      	\begin{center}
	\begin{tabular}{rcccccccccl}
	\cline{3-6}
	$\cdots$ &$w_{j+1}$ &\multicolumn{1}{|c}{\textcolor{red}{$w_{j}$}} &\textcolor{red}{$w_{j-1}$}& \textcolor{red}{$\cdots$}  & \multicolumn{1}{c|}{\textcolor{red}{$w_{j-r}$}} &$w_{j-r-1}$ &$\cdots$ \\ \cline{3-6}
	$\longrightarrow \cdots$  & $z_{j+1}$ &\textcolor{red}{$z_{j}$} &$z_{j-1}$ & $\cdots$ & $z_{j-r}$ &$z_{j-r-1}$ &$\cdots$ \\
	\end{tabular}	
	\end{center}
	\vspace{1em}
	}
	\note[targetoffsetx=0.08\colwidth,targetoffsety=2.5cm,innersep=0.5cm,angle=-90, width=.8\colwidth]{A carry may propagate throught the whole conversion in the standard case, consider for instance $999\dots 9+1$ in the decimal system.}
\block{Size of the digit set}{
	Let $\A\subset\Zbeta$. If there is a parallel addition algorithm with the rewriting rule $x-\beta$ in the numeration system $(\beta, \A)$, then 
	$$
	\#\A \geq \max \{|m_\beta(0)|, |m_\beta(1)|\}\,.
	$$
	Moreover, if $\beta$ is such that it has a real conjugate greater than 1, then 
	$$
	\#\A \geq \max \{|m_\beta(0)|, |m_\beta(1)|+2\}\,.
	$$
}

\note[targetoffsetx=0cm,targetoffsety=11pt,innersep=0.5cm,angle=-90, width=\colwidth]{\url{jan.legersky@gmail.com}, \url{github.com/Legersky/ParallelAddition}}
%\block{}{}

\column{0.6}
\block{Extending window method}{
	The length of window $r$ and a weight function $q:(\A+\A)^{r} \rightarrow \Q \subset \Zomega$ such that $q_j=q(w_j, \dots, w_{j-(r-1)})$ are determined in two phases:
    \begin{enumerate}
        \item The weight coefficients set $\Q$ is constructed as a finite subset of $\Zomega$.
        \item The length of window $r$ is extended until there is a unique weight coefficient from $\Q$ for all $(w_j,w_{j-1}, \dots , w_{j-(r-1)}) \in (\A+\A)^{r}$ to define the weight function $q$.
    \end{enumerate}
}



\begin{subcolumns}
\subcolumn{0.43}
\block{Phase 1}{
We construct the set $\Q$ such that
            $$
    		\underbrace{(\A+\A)}_{w_j \in}+ \underbrace{\Q}_{q_{j-1} \in} \subset \underbrace{\A}_{z_j \in} + \underbrace{\beta \Q}_{\beta q_j \in}
    		$$
    		iteratively:
  \begin{algorithmic}[0]
    \STATE $k:=0$; $Q_0:=\{0\}$
    \REPEAT
     \STATE  Extend $\Q_k$ to $\Q_{k+1}$ so that $$\A+\A+ \Q_k \subset \A + \beta \Q_{k+1}$$
     \vspace{-30pt}
      \STATE  $k:=k+1$
      \UNTIL{$\Q_k = \Q_{k+1}$}      
      \STATE $\Q:=\Q_k$
  \end{algorithmic}
}
\block{}{
  \begin{tikzfigure}[Two iterations of Phase 2 for input digits $\omega,1$.]
    %\centering
    \includegraphics[width=0.2\textwidth]{img/phase2_image_3a.png}
    \includegraphics[width=0.2\textwidth]{img/phase2_image_5.png}
\end{tikzfigure}
}
\subcolumn{0.57}
\block{}{
  \begin{tikzfigure}[First iteration of Phase 1]
    %\centering
    \includegraphics[width=0.2\textwidth]{img/phase1_image_3a.png}
    \includegraphics[width=0.2\textwidth]{img/phase1_image_4.png}
\end{tikzfigure}
}

\note[targetoffsetx=-.2\colwidth,targetoffsety=0\colwidth,innersep=0.5cm,angle=-135,connection, width=15cm]{\rule{15cm}{0cm} Eisenstein base $$\beta = \omega -1\,,\omega=-\frac{1}{2} + \frac{\imath \sqrt{3}}{2}\,.$$

The minimal polynomial $$m_\beta(x)=x^2+3x+3\,.$$

Alphabet $$\mathcal{A} =\{0, 1, -1, \omega, -\omega, -\omega - 1, \omega + 1\}$$}

\block{Phase 2}{
  \begin{algorithmic}[0]
 %   \FORALL{$w_j \in \A+\A$}
            \STATE Find set $\Q_{[w_{j}]} \subset \Q$ such that
              $$
              w_j + \Q \subset \A + \beta \Q_{[w_{j}]}\,.
              $$
			for all $w_j \in \A+\A$.              
              %\vspace{-20pt}
%    \ENDFOR
    \STATE $k:=0$
    \WHILE{$\max\#\Qwo{k} > 1$}	%$\max\{\#\Qwo{k}:\tupleo{k} \in(\A+\A)^{k+1} \} > 1$
        \STATE $k:= k +1$
        %\FORALL{$\tupleo{k} \in (\A+\A)^{k+1}$}
            \STATE Find set $\Qwo{k} \subset \Qwo{(k-1)}$ such that
              $$
              w_j + \Qw{1}{k} \subset \A + \beta \Qwo{k}\,.
              $$%\vspace{-20pt}
              for all $\tupleo{k} \in (\A+\A)^{k+1}$.
        %\ENDFOR  
    \ENDWHILE  
    %\STATE $r:= k+1$ 
    %\STATE Define $q\tupleo{(r-1)}$ by only element of $\Qwo{(r-1)}$ for all $\tupleo{(r-1)}\in(\A+\A)^{r}$.
    \STATE Define $q\tupleo{k}$ by the only element of $\Qwo{k}$ for all $\tupleo{k}\in(\A+\A)^{k+1}$.
%    \FORALL{$\tupleo{(r-1)}\in(\A+\A)^{r}$}
%	    \STATE $q\tupleo{(r-1)}:=$ only element of $\Qwo{(r-1)}$
%	\ENDFOR
  \end{algorithmic}
}
\end{subcolumns}

\block{Convergence of Phase 1}{
	If the extending window method with the rewriting rule $x-\beta$ converges for the numeration system $(\beta, \A)$, then the base $\beta$ is expanding.

\vspace{20pt}

	Let $\A$ contain at least one representative of each congruence class modulo $\beta$ in $\Zomega$. If $\beta$ is expanding, then Phase 1 of the extending window method converges.
}

%\begin{subcolumns}
%\subcolumn{0.5}
%
%
%\subcolumn{0.5}
%
%
%\end{subcolumns}


\block{Failure of Phase 2}{
(C1) If $\#\Q_{[b,b,\dots,b]}$ does not decrease for some $b\in\A+\A$ when the length of window is increased or% , then Phase 2 fails.

\vspace{10pt}
(C2) there is an infinite path in a specific graph,

\vspace{10pt}
then Phase 2 fails.
\vspace{10pt}
}
\note[targetoffsetx=0.25\colwidth,targetoffsety=0.13\colwidth,innersep=0.5cm,angle=-90, width=.45\colwidth]{In the implementation, the control of the condition (C1) runs before Phase 2, while the graph in (C2) is constructed and  checked during it.}
\block{Examples}{
\centering
\begin{tabular}{c|ccc|ccc|c|ccc}
$\omega$ & \rule{2em}{0cm}$\beta$\rule{2em}{0cm} & \rule{2em}{0cm}$m_\beta$\rule{2em}{0cm} & real conj. $>1$ &$\A\subset$& \rule{1em}{0cm}$\#\A$\rule{1em}{0cm} & min. & $\#\Q$ & $bb\dots b$ & \rule{1em}{0cm}Phase 2\rule{1em}{0cm} & $r$   \\ \midrule
%$ \frac{1}{2} i \, \sqrt{3} - \frac{1}{2} $ & $ \omega - 1 $ & $ x^{2} + 3 \, x + 3 $ & no &$\Zbeta$ & $ 7 $& yes & $ 19 $ & \checkmark & \checkmark & 3 \\ \midrule
%$ i - 1 $ & $ \omega $ & $ x^{2} + 2 \, x + 2 $ & no & $\Zbeta$ &$ 5 $& yes & $ 45 $ & \checkmark & \checkmark & 6 \\ \midrule
$ -\frac{1}{2} \, \sqrt{21} + \frac{3}{2} $ & $ 2 \, \omega - 3 $ & $ x^{2} - 21 $ & yes & $\ZZ$ &$ 22 $& yes & $ 9 $ & \checkmark & \checkmark & 4 \\ \midrule
$ -\frac{1}{2} \, \sqrt{17} + \frac{3}{2} $ & $ 2 \, \omega - 3 $ & $ x^{2} - 17 $ & yes & $\ZZ$ &$ 19 $& no & $ 9 $ & \checkmark & \checkmark & 2 \\
$ -\frac{1}{2} \, \sqrt{17} + \frac{3}{2} $ & $ 2 \, \omega - 3 $ & $ x^{2} - 17 $ & yes &$\ZZ$& $ 18 $ & yes & $ 9 $ & \checkmark & \checkmark & 4 \\ \midrule
$ \frac{1}{2} i \, \sqrt{11} - \frac{3}{2} $ & $ \omega $ & $ x^{2} + 3 \, x + 5 $ & no & $\Zbeta$ &$ 9 $& yes & $ 11 $ & \checkmark & \xmark & - \\
$ \frac{1}{2} i \, \sqrt{11} - \frac{3}{2} $ & $ \omega $ & $ x^{2} + 3 \, x + 5 $ & no & $\Zbeta$ &$ 9 $& yes & $ 17 $ & \xmark & \xmark & - \\ \midrule
$ i \, \sqrt{2} - 1 $ & $ -\omega - 2 $ & $ x^{2} + 2 \, x + 3 $ & no &  $\Zbeta$ & $ 6 $ & yes  &$ 27 $& \checkmark & \checkmark & 7 \\
$ i \, \sqrt{2} - 1 $ & $ -\omega - 2 $ & $ x^{2} + 2 \, x + 3 $ & no & $\Zbeta$& $ 6 $ & yes  &$ 26 $& \checkmark & \xmark & - \\
\end{tabular}

\vspace{7pt}
}


\block{}{
C.~Frougny, E.~Pelantov\'a, and M.~Svobodov\'a, \emph{Parallel addition in
  non-standard numeration systems}, Theoret. Comput. Sci. \textbf{412} (2011),
  5714--5727.

%C.~Frougny, E.~Pelantov{\'a}, and M.~Svobodov{\'a}, \emph{Minimal digit sets
%  for parallel addition in non-standard numeration systems}, J. Integer Seq.
%  \textbf{16} (2013), 36.
}


\end{columns}
%\center KM FJFI \v{C}VUT, Trojanova 13, 120 00 Praha 2, Czech Republic,  \url{jan.legersky@gmail.com}
\end{document}


