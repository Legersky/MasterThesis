
\section{NUTNOST REPREZENTANTU + DOLNI ODHAD VELIKOSTI ABECEDY}
\begin{thm}
Let $\beta$ be an algebraic integer such that $|\beta|>1$. Let $0\in \A\subset \Zbeta$ be an alphabet such that $1\in \A[\beta]$. If addition in the numeration system $(\beta, \A)$ which uses the rewriting rule $x-\beta$ is computable in parallel, then the alphabet $\A$ contains at least one representative of each congruence class modulo $\beta$ and $\beta-1$ in $\Zbeta$. 
\label{thm:representativesInAlphabet}
\end{thm}
\begin{proof}
The existence of an algorithm for addition with the rewriting rule $x-\beta$ implies that the set $\A[\beta]$ is closed under addition. By the assumption $1\in \A[\beta]$, the set $\NN$ is subset of  $\A[\beta]$. Since $0\in\A$, we have $\beta \cdot \A[\beta] \subset \A[\beta]$. Hence, $\NN[\beta] \subset \A[\beta]$.

For any element  $x=\sum_{i=0}^N x_i \beta^i\in \Zbeta$ there is an element $x'=\sum_{i=0}^N x'_i \beta^i\in \NN[\beta]$ such that $x\equiv_\beta x'$  since $m_\beta (0)\equiv_\beta 0$ and $\beta^i\equiv_\beta 0$. As $x'\in \NN[\beta] \subset \A[\beta]$, we have
$$
x\equiv_\beta x'=\sum_{i=0}^{n}a_i\beta^i \equiv_\beta a_0\,,
$$
where $a_i\in \A$. Hence, for any element $x\in\Zomega$, there is a letter $a_0\in\A$ such that $x\equiv_\beta a_0$.

In order to prove that there is at least one representative of each congruence class modulo $\beta-1$ in the alphabet $\A$, we consider again an element $x=\sum_{i=0}^N x_i \beta^i\in \Zbeta$. Similarly, there is an element $x'=\sum_{i=0}^N x'_i \beta^i\in \NN[\beta]$ such that $x\equiv_{\beta-1} x'$  since $m_{\beta-1} (0)\equiv_{\beta-1} 0$ and $(\beta-1)^i\equiv_{\beta-1} 0$.

Since $x'\in \NN[\beta]\subset \A[\beta]$,
$$
x'=\sum_{i=0}^{n}a_i\beta^i\,,
$$
where $a_i\in \A$. We prove by induction with respect to $n$ that $x'\equiv_{\beta-1} a$ for some $a\in\A$.
If $n=0$, $x'=a_0$. Now we use the fact that if there is a parallel addition algorithm, for each letter $b \in\A+\A$, there is $a\in\A$ such that $b \equiv_{\beta-1} a$ ODKAZ NA PRISLUSNOU VETU. For $n+1$, we have
\begin{align*}
x'&=\sum_{i=0}^{n+1}a_i\beta^i =a_0 + \sum_{i=1}^{n+1}a_{i}\beta^i\\
    &=a_0 + \beta \sum_{i=0}^{n}a_{i+1}\beta^i - \sum_{i=0}^{n}a_{i+1}\beta^i+ \sum_{i=0}^{n}a_{i+1}\beta^i \\
    &\equiv_{\beta-1} a_0 + (\beta-1)\sum_{i=0}^{n}a_{i+1}\beta^i + a \equiv_{\beta-1}a_0 +a \equiv_{\beta-1}a' \in\A\,,
\end{align*}
where we use the induction assumption
$$
\sum_{i=0}^{n}a_{i+1}\beta^i\equiv_{\beta-1} a\,.
$$
\end{proof}


For our purposes, we recall the proof of the following theorem for a parallel digit set conversion without anticipation. The full version can be found in \cite{minAlph}.
\begin{thm}
\label{thm:reprBetaMinusOne}
Let $\omega$ be an algebraic integer. Let the base $\beta\in\Zomega$ be such that $|\beta|>1$ and the alphabet $\A\subset\Zomega$ be such that $0\in\A$. If there exists a $p$-local digit set conversion defined by the function $\phi\colon (\A+\A)^p\rightarrow \A$ and $p=r+1$, then the number $\phi(b,\dots,b)-b$ belongs to the set $(\beta-1)\Zomega$ for any $b\in\A+\A$. 
\end{thm}
\begin{proof}
Let $b\in\A+\A$ and $a=\phi(b, \dots,b)$. For $n\in\NN, n\geq 1$, we denote $S_n$ the number represented by
$$
0^\omega \underbrace{b\dots b}_{n}\bullet \underbrace{b\dots b}_{r}0^\omega\,.
$$
The representation of $S_n$ after the digit set conversion is of the form
$$
0^\omega \underbrace{w_{r}\dots w_{1}}_{\beta^n W}\underbrace{a\dots a}_{n}\bullet \underbrace{\widetilde{w_1}\dots \widetilde{w_r}}_{\beta^{-r}\widetilde{W}}0^\omega\,,
$$
where 
$$W=\sum_{j=1}^r w_j \beta^{j-1} \qquad \text{and} \qquad \widetilde{W}=\sum_{j=1}^r\widetilde{w_j} \beta^{r-j}\,.$$
Since both representations have same value, we have
\begin{align}
\label{eq:reprBetaMinusOne}
b \sum_{j=-r}^{n-1} \beta^j &= W \beta^n + a \sum_{j=0}^{n-1} \beta^j + \beta^{-r}\widetilde{W} \notag \\
b \sum_{j=-r}^{-1} \beta^j +b\frac{\beta^n-1}{\beta-1} &= W \beta^n + a \frac{\beta^n-1}{\beta-1} + \beta^{-r}\widetilde{W}\,,
\end{align}
for all $n\geq 1$. We substract this equation for $n$ and $n-1$ to obtain
$$
b\frac{\beta^n-\beta^{n-1}}{\beta-1}=W(\beta^n-\beta^{n-1}) + a\frac{\beta^n-\beta^{n-1}}{\beta-1}\,.
$$
We simplify it to
\begin{equation}
\label{eq:reprBetaMinusOneFinal}
b=W(\beta-1) + a\,.
\end{equation}
Hence, $a=\phi(b, \dots,b)\equiv b$ modulo $\beta-1$.
\end{proof}

The following lemma is a modification of lemma in \cite{minAlph}.
\begin{lem}
Let $\omega$ be a real algebraic integer and the base $\beta\in\Zomega$ be such that $\beta>1$. Let the alphabet $\A\subset\Zomega$ be such that $0\in\A$ and denote $\lambda=\min \A$ and $\Lambda=\max \A$. If there exists a $p$-local digit set conversion defined by the function $\phi\colon (\A+\A)^p\rightarrow \A$ and $p=r+1$, then:
\begin{enumerate}[i)]
	\item $\phi(b,\dots,b)\neq \lambda$ for all $b\in\A+\A$ such that $b>\lambda$.
	\item $\phi(b,\dots,b)\neq \Lambda$ for all $b\in\A+\A$ such that $b<\Lambda$.
	\item If $\Lambda\neq 0$, then $\phi(\Lambda,\dots,\Lambda)\neq \Lambda$.
	\item If $\lambda\neq 0$, then $\phi(\lambda,\dots,\lambda)\neq \lambda$.
\end{enumerate}
\end{lem}
\begin{proof}
To prove \textit{i)}, assume in contradiction that $\phi(b,\dots,b)= \lambda$. We proceed in the same manner as in Theorem \ref{thm:reprBetaMinusOne}, the equation \eqref{eq:reprBetaMinusOne} implies
$$
b \sum_{j=-r}^{-1} \beta^j +b\frac{\beta^n-1}{\beta-1} =  \beta^n W + \lambda \frac{\beta^n-1}{\beta-1} + \beta^{-r}\widetilde{W}\,.
$$
We apply also the equation \eqref{eq:reprBetaMinusOneFinal} to obtain
\begin{align*}
b \sum_{j=-r}^{-1} \beta^j +b\frac{\beta^n-1}{\beta-1} &=  \beta^n \frac{b-\lambda}{\beta-1} + \lambda \frac{\beta^n-1}{\beta-1} + \beta^{-r}\widetilde{W}\,.
\end{align*}
Now we may simplify and estimate
\begin{align*}
b \sum_{j=-r}^{-1} \beta^j +\frac{-b}{\beta-1} &=  \frac{-\lambda}{\beta-1} + \beta^{-r}\sum_{j=1}^r\widetilde{w_j} \beta^{r-j} \\
b \underbrace{\left(\sum_{j=1}^{r} \frac{1}{\beta^j} -\frac{1}{\beta-1}\right)}_{-\sum_{j=r+1}^{\infty} \frac{1}{\beta^j}} &=  -\lambda\frac{1}{\beta-1} + \sum_{j=1}^r\widetilde{w_j} \beta^{-j}\geq  \lambda\underbrace{\left(-\frac{1}{\beta-1}+\sum_{j=1}^{r} \frac{1}{\beta^j} \right)}_{-\sum_{j=r+1}^{\infty}\frac{1}{\beta^j}}\,.
\end{align*}
Hence $b\leq\lambda$ which is a contradiction.
\end{proof}










\begin{thm}
Let $\beta$ be an algebraic integer such that $|\beta|>1$. Let $0\in \A\subset \Zbeta$ be an alphabet such that $1\in \A[\beta]$. If addition in the numeration system $(\beta, \A)$ which uses the rewriting rule $x-\beta$ is computable in parallel, then
$$
\#\A \geq \max \{|m_\beta(0)|, |m_\beta(1)|\}\,.
$$
\end{thm}
\begin{proof}
By Theorem~\ref{thm:representativesInAlphabet}, there are all representatives modulo $\beta$ and modulo $\beta-1$ in the alphabet $\A$. The numbers of congruence classes are $|m_\beta(0)|$ and $|m_{\beta-1}(0)|$ by Theorem~\ref{thm:numbCongruenceClasses}. Obviously, $m_{\beta-1}(x) = m_\beta (x+1)$. Thus $m_{\beta-1}(0) = m_\beta (1)$.
\end{proof}


\komentar{BYLO BY FAJN TO JESTE ZOBECNIT NA abecedu ze Z[OMEGA]}