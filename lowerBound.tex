\section{NUTNOST REPREZENTANTU + DOLNI ODHAD VELIKOSTI ABECEDY}

\begin{thm}
Let $\beta$ be an algebraic integer such that $|\beta|>1$. Let $0\in \A\subset \Zbeta$ be an alphabet such that $1\in \A[\beta]$. If addition in the numeration system $(\beta, \A)$ which uses the rewriting rule $x-\beta$ is computable in parallel, the alphabet $\A$ contains at least one representative of each congruence class modulo $\beta$ and $\beta-1$ in $\Zbeta$. 
\label{thm:representativesInAlphabet}
\end{thm}
\begin{proof}
The existence of an algorithm for addition with the rewriting rule $x-\beta$ implies that the set $\A[\beta]$ is closed under addition. By the assumption $1\in \A[\beta]$, the set $\NN$ is subset of  $\A[\beta]$. Since $0\in\A$, we have $\beta \cdot \A[\beta] \subset \A[\beta]$. Hence, $\NN[\beta] \subset \A[\beta]$.

For any element  $x=\sum_{i=0}^N x_i \beta^i\in \Zbeta$ there is an element $x'=\sum_{i=0}^N x'_i \beta^i\in \NN[\beta]$ such that $x\equiv_\beta x'$  since $m_\beta (0)\equiv_\beta 0$ and $\beta^i\equiv_\beta 0$. As $x'\in \NN[\beta] \subset \A[\beta]$, we have
$$
x\equiv_\beta x'=\sum_{i=0}^{n}a_i\beta^i \equiv_\beta a_0\,,
$$
where $a_i\in \A$. Hence, for any element $x\in\Zomega$, there is a letter $a_0\in\A$ such that $x\equiv_\beta a_0$.

In order to prove that there is at least one representative of each congruence class modulo $\beta-1$ in the alphabet $\A$, we consider again an element $x=\sum_{i=0}^N x_i \beta^i\in \Zbeta$. Similarly, there is an element $x'=\sum_{i=0}^N x'_i \beta^i\in \NN[\beta]$ such that $x\equiv_{\beta-1} x'$  since $m_{\beta-1} (0)\equiv_{\beta-1} 0$ and $(\beta-1)^i\equiv_{\beta-1} 0$.

Since $x'\in \NN[\beta]\subset \A[\beta]$,
$$
x'=\sum_{i=0}^{n}a_i\beta^i\,,
$$
where $a_i\in \A$. We prove by induction with respect to $n$ that $x'\equiv_{\beta-1} a$ for some $a\in\A$.
If $n=0$, $x'=a_0$. Now we use the fact, that if there is a parallel addition algorithm, for each letter $b \in\A+\A$, there is $a\in\A$ such that $b \equiv_{\beta-1} a$ ODKAZ NA PRISLUSNOU VETU. For $n+1$, we have
\begin{align*}
x'&=\sum_{i=0}^{n+1}a_i\beta^i =a_0 + \sum_{i=1}^{n+1}a_{i}\beta^i\\
    &=a_0 + \beta \sum_{i=0}^{n}a_{i+1}\beta^i - \sum_{i=0}^{n}a_{i+1}\beta^i+ \sum_{i=0}^{n}a_{i+1}\beta^i \\
    &\equiv_{\beta-1} a_0 + (\beta-1)\sum_{i=0}^{n}a_{i+1}\beta^i + a \equiv_{\beta-1}a_0 +a \equiv_{\beta-1}a' \in\A\,,
\end{align*}
where we use the induction assumption
$$
\sum_{i=0}^{n}a_{i+1}\beta^i\equiv_{\beta-1} a\,.
$$
\end{proof}

\begin{thm}
Let $\beta$ be an algebraic integer such that $|\beta|>1$. Let $0\in \A\subset \Zbeta$ be an alphabet such that $1\in \A[\beta]$. If addition in the numeration system $(\beta, \A)$ which uses the rewriting rule $x-\beta$ is computable in parallel, then
$$
\#\A \geq \max \{|m_\beta(0)|, |m_\beta(1)|\}\,.
$$
\end{thm}
\begin{proof}
By Theorem~\ref{thm:representativesInAlphabet}, there are all representatives modulo $\beta$ and modulo $\beta-1$ in the alphabet $\A$. The numbers of congruence classes are $|m_\beta(0)|$ and $|m_{\beta-1}(0)|$ by Theorem~\ref{thm:numbCongruenceClasses}. Obviously, $m_{\beta-1}(x) = m_\beta (x+1)$. Thus $m_{\beta-1}(0) = m_\beta (1)$.
\end{proof}


\komentar{BYLO BY FAJN TO JESTE ZOBECNIT NA abecedu ze Z[OMEGA]}