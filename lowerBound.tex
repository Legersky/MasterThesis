\section{Minimal alphabet $\A$}
\label{sec:minimalAlphabet}	

Frougny, Pelantov\'a and Svodov\'a \cite{minAlph} proved a lower bound on the size of an alphabet $0\in\A\subset\ZZ$ of consecutive integers which enables parallel addition. In this section, we prove the same bound for an arbitrary alphabet $\A\in\Zbeta$.  We recall their auxiliary results in Theorem~\ref{thm:reprBetaMinusOne} an Lemma~\ref{lem:alphabetRestrictions}, but only for a parallel digit set conversion without anticipation as our rewriting rule $x-\beta$ does not require memory. 

We remark that we indicate by the assumption $\A\in\Zbeta$  that we work in $\Zbeta$ instead of $\Zomega$. Notice that congruence classes modulo $\beta$ in $\Zomega$ and $\Zbeta$ are generally different. It implies that even an integer alphabet behaves differently if $\beta\in\Zomega$ and $\Zomega\neq\Zbeta$. On the other hand, if $\beta=\pm \omega +c$, where $c\in\ZZ$, then $\Zomega=\Zbeta$

The following theorem says that all classes modulo $\beta$ in $\Zomega$ which are contained in $\A+\A$ must have their representatives in $\A$.
\begin{thm}
\label{thm:reprBetaMinusOne}
Let $\omega$ be an algebraic integer. Let the base $\beta\in\Zomega$ be such that $|\beta|>1$ and the alphabet $\A\subset\Zomega$ be such that $0\in\A$. If there exists a $p$-local digit set conversion defined by the function $\phi\colon (\A+\A)^p\rightarrow \A$ and $p=r+1$, then the number $\phi(b,\dots,b)-b$ belongs to the set $(\beta-1)\Zomega$ for any $b\in\A+\A$. 
\end{thm}
\begin{proof}
Let $b\in\A+\A$ and $a=\phi(b, \dots,b)$. For $n\in\NN, n\geq 1$, we denote $S_n$ the number represented by
$$
^{\omega}\!0 \underbrace{b\dots b}_{n}\bullet \underbrace{b\dots b}_{r}0^\omega\,.
$$
The representation of $S_n$ after the digit set conversion is of the form
$$
^{\omega}\!0 \underbrace{w_{r}\dots w_{1}}_{\beta^n W}\underbrace{a\dots a}_{n}\bullet \underbrace{\widetilde{w_1}\dots \widetilde{w_r}}_{\beta^{-r}\widetilde{W}}0^\omega\,,
$$
where 
$$W=\sum_{j=1}^r w_j \beta^{j-1} \qquad \text{and} \qquad \widetilde{W}=\sum_{j=1}^r\widetilde{w_j} \beta^{r-j}\,.$$
Since both representations have same value, we have
\begin{align}
\label{eq:reprBetaMinusOne}
b \sum_{j=-r}^{n-1} \beta^j &= W \beta^n + a \sum_{j=0}^{n-1} \beta^j + \beta^{-r}\widetilde{W} \notag \\
b \sum_{j=-r}^{-1} \beta^j +b\frac{\beta^n-1}{\beta-1} &= W \beta^n + a \frac{\beta^n-1}{\beta-1} + \beta^{-r}\widetilde{W}\,,
\end{align}
for all $n\geq 1$. We subtract this equation for $n$ and $n-1$ to obtain
$$
b\frac{\beta^n-\beta^{n-1}}{\beta-1}=W(\beta^n-\beta^{n-1}) + a\frac{\beta^n-\beta^{n-1}}{\beta-1}\,.
$$
We simplify it to
\begin{equation}
\label{eq:reprBetaMinusOneFinal}
b=W(\beta-1) + a\,.
\end{equation}
Hence, $a=\phi(b, \dots,b)\equiv b$ modulo $\beta-1$.
\end{proof}

If a base $\beta$ has a real conjugate greater than one, then there are some extra requirements on the alphabet $\A$. For simplicity, we assume that the base $\beta$ itself is real and greater than one. We show later that this assumption is without loss of generality.
\begin{lem}
\label{lem:alphabetRestrictions}
Let $\omega$ be a real algebraic integer and the base $\beta\in\Zomega$ be such that $\beta>1$. Let the alphabet $\A\subset\Zomega$ be such that $0\in\A$ and denote $\lambda=\min \A$ and $\Lambda=\max \A$. If there exists a $p$-local digit set conversion defined by the function $\phi\colon (\A+\A)^p\rightarrow \A$ and $p=r+1$, then:
\begin{enumerate}[i)]
	\item $\phi(b,\dots,b)\neq \lambda$ for all $b\in\A+\A$ such that $b>\lambda$.
	\item $\phi(b,\dots,b)\neq \Lambda$ for all $b\in\A+\A$ such that $b<\Lambda$.
	\item If $\Lambda\neq 0$, then $\phi(\Lambda,\dots,\Lambda)\neq \Lambda$.
	\item If $\lambda\neq 0$, then $\phi(\lambda,\dots,\lambda)\neq \lambda$.
\end{enumerate}
\end{lem}
\begin{proof}
To prove \textit{i)}, assume in contradiction that $\phi(b,\dots,b)= \lambda$. We proceed in the same manner as in Theorem \ref{thm:reprBetaMinusOne}, the equation \eqref{eq:reprBetaMinusOne} implies
$$
b \sum_{j=-r}^{-1} \beta^j +b\frac{\beta^n-1}{\beta-1} =  \beta^n W + \lambda \frac{\beta^n-1}{\beta-1} + \beta^{-r}\widetilde{W}\,.
$$
We apply also the equation c to obtain
\begin{align*}
b \sum_{j=-r}^{-1} \beta^j +b\frac{\beta^n-1}{\beta-1} &=  \beta^n \frac{b-\lambda}{\beta-1} + \lambda \frac{\beta^n-1}{\beta-1} + \beta^{-r}\widetilde{W}\,.
\end{align*}
Now we may simplify and estimate
\begin{align*}
b \sum_{j=-r}^{-1} \beta^j +\frac{-b}{\beta-1} &=  \frac{-\lambda}{\beta-1} + \beta^{-r}\sum_{j=1}^r\widetilde{w_j} \beta^{r-j} \\
b \underbrace{\left(\sum_{j=1}^{r} \frac{1}{\beta^j} -\frac{1}{\beta-1}\right)}_{-\sum_{j=r+1}^{\infty} \frac{1}{\beta^j}} &=  -\lambda\frac{1}{\beta-1} + \sum_{j=1}^r\widetilde{w_j} \beta^{-j}\geq  \lambda\underbrace{\left(-\frac{1}{\beta-1}+\sum_{j=1}^{r} \frac{1}{\beta^j} \right)}_{-\sum_{j=r+1}^{\infty}\frac{1}{\beta^j}}\,.
\end{align*}
Hence $b\leq\lambda$ which is a contradiction. The proof of \textit{ii)} can be done in the same way.

For \textit{iii)}, assume that $\phi(\Lambda,\dots,\Lambda)= \Lambda$. Now consider a number $T_q$ represented by
$$
^{\omega}\!0 \bullet \underbrace{\Lambda\dots\Lambda}_{r} \underbrace{(2\Lambda)\dots(2\Lambda)}_{q} 0^\omega\,.
$$
After the digit set conversion, a representation is
$$
^{\omega}\!0  \underbrace{w_r\dots w_1}_{W} \bullet z_1\dots z_{r+q} 0^\omega\,.
$$
The value $T_q$  preserves, thus,
$$
\Lambda\sum_{j=1}^r \beta^{-j} +2\Lambda \sum_{j=r+1}^{r+q} \beta^{-j}=W+\sum_{j=1}^{r+q}z_j\beta^{-j}\,.
$$
But $W=0$ from the equation \eqref{eq:reprBetaMinusOneFinal}. We estimate
\begin{align*}
\Lambda\sum_{j=1}^{r+q} \beta^{-j} +\Lambda \sum_{j=r+1}^{r+q} \beta^{-j}&=\sum_{j=1}^{r+q}z_j\beta^{-j}\leq \Lambda\sum_{j=1}^{r+q}\beta^{-j}\\
\Lambda \sum_{j=r+1}^{r+q} \beta^{-j}&\leq 0\,.
\end{align*}
This contradicts that $\Lambda$ is positive as it is a nonzero, maximal element of the alphabet $\A$ which contains 0. The proof of \textit{iv)} is analogous.
\end{proof}

In order to prove the lower bound, we need to show that the alphabet $\A$ must contain all representatives modulo $\beta$ and $\beta-1$. 
\begin{thm}
Let $\beta$ be an algebraic integer such that $|\beta|>1$. Let $0\in \A\subset \Zbeta$ be an alphabet such that $1\in \A[\beta]$. If addition in the numeration system $(\beta, \A)$ which uses the rewriting rule $x-\beta$ is computable in parallel, then the alphabet $\A$ contains at least one representative of each congruence class modulo $\beta$ and $\beta-1$ in $\Zbeta$. 
\label{thm:representativesInAlphabet}
\end{thm}
\begin{proof}
The existence of an algorithm for addition with the rewriting rule $x-\beta$ implies that the set $\A[\beta]$ is closed under addition. By the assumption $1\in \A[\beta]$, the set $\NN$ is subset of  $\A[\beta]$. Since $0\in\A$, we have $\beta \cdot \A[\beta] \subset \A[\beta]$. Hence, $\NN[\beta] \subset \A[\beta]$.

For any element  $x=\sum_{i=0}^N x_i \beta^i\in \Zbeta$ there is an element $x'=\sum_{i=0}^N x'_i \beta^i\in \NN[\beta]$ such that $x\equiv_\beta x'$  since $m_\beta (0)\equiv_\beta 0$ and $\beta^i\equiv_\beta 0$. As $x'\in \NN[\beta] \subset \A[\beta]$, we have
$$
x\equiv_\beta x'=\sum_{i=0}^{n}a_i\beta^i \equiv_\beta a_0\,,
$$
where $a_i\in \A$. Hence, for any element $x\in\Zomega$, there is a letter $a_0\in\A$ such that $x\equiv_\beta a_0$.

In order to prove that there is at least one representative of each congruence class modulo $\beta-1$ in the alphabet $\A$, we consider again an element $x=\sum_{i=0}^N x_i \beta^i\in \Zbeta$. Similarly, there is an element $x'=\sum_{i=0}^N x'_i \beta^i\in \NN[\beta]$ such that $x\equiv_{\beta-1} x'$  since $m_{\beta-1} (0)\equiv_{\beta-1} 0$ and $(\beta-1)^i\equiv_{\beta-1} 0$.

Since $x'\in \NN[\beta]\subset \A[\beta]$,
$$
x'=\sum_{i=0}^{n}a_i\beta^i\,,
$$
where $a_i\in \A$. We prove by induction with respect to $n$ that $x'\equiv_{\beta-1} a$ for some $a\in\A$.
If $n=0$, $x'=a_0$. Now we use the fact that if there is a parallel addition algorithm, for each letter $b \in\A+\A$, there is $a\in\A$ such that $b \equiv_{\beta-1} a$ (Theorem~\ref{thm:reprBetaMinusOne}). For $n+1$, we have
\begin{align*}
x'&=\sum_{i=0}^{n+1}a_i\beta^i =a_0 + \sum_{i=1}^{n+1}a_{i}\beta^i\\
    &=a_0 + \beta \sum_{i=0}^{n}a_{i+1}\beta^i - \sum_{i=0}^{n}a_{i+1}\beta^i+ \sum_{i=0}^{n}a_{i+1}\beta^i \\
    &\equiv_{\beta-1} a_0 + (\beta-1)\sum_{i=0}^{n}a_{i+1}\beta^i + a \equiv_{\beta-1}a_0 +a \equiv_{\beta-1}a' \in\A\,,
\end{align*}
where we use the induction assumption
$$
\sum_{i=0}^{n}a_{i+1}\beta^i\equiv_{\beta-1} a\,.
$$
\end{proof}
Unfortunately, the claim cannot be generalized to modulo in $\Zomega$ -- there are numeration systems with integer alphabets which allow parallel addition, but these alphabets do not contain all representatives modulo $\beta-1$ in $\Zomega$, see Table~\ref{tab:resultsQuadrInt} and Examples \ref{ex:integerAB},  \ref{ex:integerAJ}, \ref{ex:integerAO} and \ref{ex:integerAR}. Nevertheless, each class modulo $\beta -1$ in $\Zomega$ which is contained in $\A+\A$ must still have its representative in $\A$ according to Theorem~\ref{thm:reprBetaMinusOne}.

The following lemma summarizes that if we have a parallel addition algorithm for a base $\beta$, then we easily obtain an algorithm also for conjugates of $\beta$.
\begin{lem}
\label{lem:parAddAlgForConjugate}
Let $\omega$ be an algebraic integer with a conjugate $\omega'$. Let $\beta\in\Zomega, |\beta|>1$ and let $\sigma:\QQ(\omega)\rightarrow \QQ(\omega')$ be an isomorphism such that $|\sigma(\beta)|>1$. Let $\varphi$ be a digit set conversion  in the base $\beta$ from $\A+\A$ to $\A$. There exists  is a digit set conversion $\varphi'$ in the base $\beta'$ from $\A'+\A'$ to $\A'$ where $\beta'=\sigma(\beta)$ and $\A'=\{\sigma(a) \colon a\in\A\}$.
\end{lem}
\begin{proof}
Let $\phi:\A^p\rightarrow\A$ be a mapping which defines $\varphi$ with $p=r+t+1$. We define a mapping $\phi':\A^p\rightarrow \A$ by 
$$
\phi'(w'_{j+t}, \dots, w'_{j-r})=\sigma\left(\phi\left(\sigma^{-1}(w'_{j+t}), \dots, \sigma^{-1}(w'_{j-r})\right)\right)\,.
$$
Next, we define a digit set conversion  $\varphi':(\A'+\A')\rightarrow\A'$ by $\varphi'(w')=(z'_j)_{j\in\ZZ}$ where $w'=(w'_j)_{j\in\ZZ}$ and $z'_j=\phi'(w'_{j+t}, \dots, w'_{j-r})$. Obviously, if $w'$ has only finitely many nonzero entries, then there is only finitely many nonzeros in $(z'_j)_{j\in\ZZ}$   since
$$
\phi'(0, \dots, 0)=\sigma\left(\phi\left(\sigma^{-1}(0), \dots, \sigma^{-1}(0)\right)\right)=\sigma\left(\phi\left(0, \dots, 0\right)\right)=\sigma\left(0\right)=0\,.
$$
The value of the number represented by $w'$ is also preserved:
\begin{align*}
\sum_{j\in\ZZ}w'_j {\beta'}^j&=\sum_{j\in\ZZ}\sigma(w_j) \sigma(\beta)^j=\sigma\left(\sum_{j\in\ZZ}w_j\beta^j\right) \\
&=\sigma\left(\sum_{j\in\ZZ}z_j\beta^j \right)=\sigma\left(\sum_{j\in\ZZ}\phi\left(w_{j+t}, \dots,w_{j-r}\right)\beta^j\right) \\
&=\sum_{j\in\ZZ}\sigma(\phi\left(w_{j+t}, \dots,w_{j-r}\right)){\beta'}^j=\sum_{j\in\ZZ}z'_j {\beta'}^j
\end{align*}
where $w_j=\sigma^{-1}(w'_j)$ for $j\in\ZZ$ and $\varphi((w_j)_{j\in\ZZ})=(z_j)_{j\in\ZZ}$.
\end{proof}

Finally, we put together that the alphabet $\A$ contains all representative modulo $\beta$ and $\beta-1$, number of congruence classes and restrictions on the alphabet for a base with a real conjugate greater than one.
\begin{thm}
\label{thm:lowerBoundAlphabet}

Let $\beta$ be an algebraic integer such that $|\beta|>1$. Let $0\in \A\subset \Zbeta$ be an alphabet such that $1\in \A[\beta]$. If addition in the numeration system $(\beta, \A)$ which uses the rewriting rule $x-\beta$ is computable in parallel, then
$$
\#\A \geq \max \{|m_\beta(0)|, |m_\beta(1)|\}\,.
$$
Moreover, if $\beta$ is such that it has a real conjugate greater than 1, then 
$$
\#\A \geq \max \{|m_\beta(0)|, |m_\beta(1)|+2\}\,.
$$
\end{thm}
\begin{proof}
By Theorem~\ref{thm:representativesInAlphabet}, there are all representatives modulo $\beta$ and modulo $\beta-1$ in the alphabet $\A$. The numbers of congruence classes are $|m_\beta(0)|$ and $|m_{\beta-1}(0)|$ by Theorem~\ref{thm:numbCongruenceClasses}. Obviously, $m_{\beta-1}(x) = m_\beta (x+1)$. Thus $m_{\beta-1}(0) = m_\beta (1)$.

Let $\phi$ be a mapping which defines the parallel addition. According to Lemma~\ref{lem:parAddAlgForConjugate}, we may assume that $\beta$ is real and greater than 1 in the proof of the second part. The assumption $1\in \A[\beta]$ implies that $\Lambda>0$. Thus, there are at least three elements in the alphabet $\A$, because $\A\ni\phi(\Lambda,\dots,\Lambda)\neq \lambda$ and $\A\ni\phi(\Lambda,\dots,\Lambda)\neq \Lambda$ by Lemma \ref{lem:alphabetRestrictions}. It also implies that there are at least two representatives modulo $\beta-1$ in the alphabet in the class which contains $\Lambda$, since $\phi(\Lambda,\dots,\Lambda)\equiv_{\beta-1} \Lambda$.  

If $\lambda\equiv_{\beta-1}\Lambda$, there must be one more element of the alphabet $\A$ in this class, since $\lambda \neq \Lambda$. Therefore, $\#\A\geq |m_\beta(1)|+2$. 

The case that $\lambda\not\equiv_{\beta-1}\Lambda$ is divided into two subcases. Suppose now that $\lambda\neq 0$. Then $\phi(\lambda,\dots,\lambda)\neq \lambda$ and hence there is one more element in the alphabet in the class containing $\lambda$. Thus, there are at least two congruence classes which contain at least two elements of the alphabet $\A$. Therefore, $\#\A\geq |m_\beta(1)|+2$.

%If $\lambda=0$, then all elements of $\A+\A$ are nonnegative and $\phi(b,\dots,b)\neq 0$ for all $b\in(\A+\A)\setminus 0$. Suppose for contradiction, that there is only $|m_\beta(1)|+1$ elements in $\A$ -- one in each congruence class modeulo $\beta-1$ and one more in the class which contains $\Lambda$. The set $\mathcal{D}=\{\phi(d,\dots,d)\colon d\in\Lambda+\A\}\subset\A$ has $|m_\beta(1)|+1$ elements, but none of them is congruent to 0 as there are nonzero and the class containing zero has only one element by the assumption. Therefore, the elements of the set $\mathcal{D}$ belong to only $|m_\beta(1)|-1$ congruence classes. Hence, there exists $e,f,g,h\in\A, e\neq f,g\neq h, h\neq f$ such that $\phi(e+\Lambda,\dots, e+\Lambda)\equiv_{\beta-1}\phi(f+\Lambda,\dots, f+\Lambda)$ and $\phi(g+\Lambda,\dots, g+\Lambda)=\phi(h+\Lambda,\dots, h+\Lambda)$. Since 
%$$
%e+\Lambda\equiv_{\beta-1}\phi(e+\Lambda,\dots, e+\Lambda)\equiv_{\beta-1}\phi(f+\Lambda,\dots, f+\Lambda)\equiv_{\beta-1}f+\Lambda
%$$
%and
%$$
%g+\Lambda\equiv_{\beta-1}\phi(g+\Lambda,\dots, g+\Lambda)=\phi(h+\Lambda,\dots, h+\Lambda)\equiv_{\beta-1}h+\Lambda\,,
%$$ 
%also $e\equiv_{\beta-1}f$ and $g\equiv_{\beta-1}h$ which is a contradiction. In the same manner, the assumption that there is no nonzero element of the alphabet $\A$ which is congruent to 0 leads to contradiction for arbitrarily large alphabet.

If $\lambda=0$, then all elements of $\A+\A$ are nonnegative and $\phi(b,\dots,b)\neq 0$ for all $b\in(\A+\A)\setminus 0$. Suppose for contradiction, that there is no nonzero element of the alphabet $\A$ congruent to 0. We know that there is at least one representative of each congruence class class modulo $\beta-1$ in $\A$ and at least two representatives in the congruence class which contains $\Lambda$. Let $k\in\NN$ denote the number of elements which are in $\A$ extra to one element in each congruence class, i.e., $\#A=|m_\beta(1)|+k$.  For $d\in\Lambda+\A$, the value $\phi(d,\dots,d)\in\A$ is not congruent to 0 as it is nonzero and the class containing zero has only one element by the assumption. Therefore, the values  $\phi(d,\dots,d)\in\A$ for $|m_\beta(1)|+k$ distinct letters $d\in\Lambda+\A$ belong to only $|m_\beta(1)|-1$ congruence classes. Hence, there exists $e_1,\dots e_k, e_{k+1}\in\A$, pairwise distinct, and $f_1,\dots f_k, f_{k+1}\in\A$ such that $e_i\neq f_i$ and $\phi(e_i+\Lambda,\dots, e_i+\Lambda)\equiv_{\beta-1}\phi(f_i+\Lambda,\dots, f_i+\Lambda)$ for all $i, 1\leq i\leq k+1$. Since 
$$
e_i+\Lambda\equiv_{\beta-1}\phi(e_i+\Lambda,\dots, e_i+\Lambda)\equiv_{\beta-1}\phi(f_i+\Lambda,\dots, f_i+\Lambda)\equiv_{\beta-1}f_i+\Lambda\,.
$$ 
also $e_i\equiv_{\beta-1}f_i$ for all $i, 1\leq i\leq k+1$. This is a contradiction since it implies that $\#\A=|m_\beta(1)|+k+1$. Hence, classes containing $\lambda$ and $\Lambda$ have both at least one more  element of the alphabet $\A$, i.e., $\#\A\geq |m_\beta(1)|+2$. 
\end{proof}



Since the proof is based on Theorem~\ref{thm:representativesInAlphabet}, it cannot be easily extended to alphabets which are subsets of  $\Zomega$.

\section{Alphabet generation}
\label{sec:alphabetGeneration}
Before we sketch how an alphabet $\A$ that allows parallel addition for a given base can be generated, we summarize its necessary properties. 

The alphabet $\A$ must contain representatives of all congruence classes modulo $\beta$ in order to guarantee convergence of Phase~1, see assumptions of Theorem~\ref{thm:suffCondPhase1}. According to Theorem~\ref{thm:reprBetaMinusOne}, there must be representatives of all congruence classes modulo $\beta-1$ which are contained in $\A+\A$. Moreover, if the base $\beta$ is real and greater than one, there must be another digit which is congruent to the minimal, resp., maximal digit of $\A$ (Lemma~\ref{lem:alphabetRestrictions}). If the base $\beta$ has a real conjugate $\sigma(\beta)>1$, then the isomorphism $\sigma$ is applied in the same way as in the Lemma~\ref{lem:parAddAlgForConjugate}.

For construction of an integer alphabet, start with $a=1$ and examine whether $\A=\{-a+1, \dots, -1,0,1,\dots, a-1,a\}$ or $\A=\{-a, -a+1, \dots, -1,0,1,\dots, a-1,a\}$ is suitable. If not, increment $a$ and check again. It might happen that there are not all representatives modulo $\beta$ in $\Zomega$ if $\Zomega \neq \Zbeta$. Therefore, stop without success when $a>m_\beta(0)$.

Generating of a non-integer alphabet is slightly more complicated. Start with $\A'=\{-1,0,1\}$ and $a=1$. Generate the set $$L_a=\left\{\sum_{i=0}^{d-1}a_i \omega^i \colon |a_i|\leq a\right\}\,.$$
Increment $a$ until representatives of all classes modulo $\beta-1$ are contained in $\A' \cup L_a$. Extend $\A'$ to $\A''$ by adding the smallest element in $\beta$-norm of each class of $\A' \cup L_a$ modulo $\beta -1$. Repeat in the same manner with $\A''$ and congruences modulo $\beta$ to obtain the alphabet $\A$.

If the base $\beta$ is real and greater than one, check if $\A$ contains another element congruent to the minimal, resp. maximal digit $\lambda$, resp. $\Lambda$. If not, add $\lambda + \beta-1$, resp. $\Lambda - (\beta-1)$ to the alphabet $\A$.

We remark that procedure does not guarantee that $\A$ has minimal size.