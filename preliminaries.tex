
\section{Numeration systems}
In the binary system, numbers are expressed as a sum of powers of 2 multiplied by 0 or 1. The following definition generalizes this concept.
\begin{defn}
  Let $\beta \in \CC, |\beta|>1$ and $\A \subset \CC$ be a finite set containing 0. A pair $(\beta, \A)$ is called a \emph{positional numeration system} with \emph{base} $\beta$ and \emph{digit set} $\A$, usually called \emph{alphabet}.
\end{defn}
Sometimes, the term \emph{radix} is used instead of base. So-called standard numeration systems have an integer base $\beta$ and an alphabet $\A$ which is a set of contiguous integers. We restrict ourselves to a base $\beta$ which is an algebraic integer and possibly non-integer alphabet $\A$. 

\begin{defn}
Let $(\beta, \A)$ be a positional numeration system. Let $x$ be a complex number and $x_n,x_{n-1}, x_{n-2},\ldots \in\A, n\geq 0$. We say that $^\omega0 x_n x_{n-1}\cdots x_1 x_0 \bullet x_{-1} x_{-2} \cdots $ is a \emph{$(\beta, \A)$-representation} of $x$ if~  $x=\sum_{j=-\infty}^n x_j \beta^j$, where $^\omega0$  denotes left-infinite sequence of zeros.
\end{defn}
 We write briefly a \emph{representation} instead of a $(\beta, \A)$-representation if the base $\beta$ and alphabet $\A$ follow from context. The assumption $|\beta|>1$ implies that the sum for a given representation converges.

\begin{defn}
Let $(\beta, \A)$ be a positional numeration system. The set of all complex numbers with a finite $(\beta, \A)$-representation is defined by
$$
    \fin{\A}:=\left\{\sum_{j=-m}^n x_j \beta^j\colon n, m \in \NN, x_j \in \A \right\}\,.
$$
\end{defn}
   
For  $x\in\fin{\A}$, we write 
$$
(x)_{\beta,\A}=\,  ^\omega0 x_n x_{n-1}\cdots x_1 x_0 \bullet x_{-1} x_{-2} \cdots x_{-m} 0^\omega\,,
$$ 
where $0^\omega$ denotes right-infinite sequence of zeros. From now on, we omit the starting $^\omega0$ and ending $0^\omega$ when we work with numbers in $\fin{\A}$, but we still consider a representation as a bi-infinite sequence of digits. Note that indices are decreasing from left to right as it is usual to write the most significant digits first. 

 We remark that the set $\fin{\A}$ is not necessarily closed under addition. Nevertheless, existence of a parallel addition algorithm in the sense of definitions in the next section implies, as we will see later, that the set $\fin{\A}$ is closed under addition, i.e.,
$$
\fin{\A} + \fin{\A} \subset \fin{\A}\,.
$$ 

Designing an algorithm for parallel addition requires some redundancy in numeration system \cite{kornerup}. According to \cite{redundant}, a numeration system $(\beta,\A)$ is called \emph{redundant} if there exists $x \in \fin{\A}$ which has two different $(\beta,\A)$-representations. For instance, the number 1 has $(2,\{-1,0,1\})$-representations $1\bullet$ and $1(-1)\bullet$.
Redundant numeration system may allow to avoid carry propagation in addition. On the other hand, redundancy brings some disadvantages. For example, comparison is problematic.  


\section{Parallel addition}
Informally, parallel algorithm consists of many threads or processes which run at the same time. Usually, the main task is to reduce communication among processes to minimum, since waiting for a result of another process slows down the whole computation. In the scope of parallel addition, parallelism is formalized by a  local function, which is also often called a sliding block code.

\begin{defn}
Let $\A$ and $\B$ be alphabets. A function $\varphi:\B^\ZZ \rightarrow \A^\ZZ$ is said to be \emph{$p$-local} if there exist $r,t\in\NN$ satisfying $p=r+t+1$ and a function $\phi: \B^p \rightarrow \A$ such that, for any $w=(w_j)_{j\in\ZZ}\in\B^\ZZ$ and its image $z=\varphi(w)=(z_j)_{j\in\ZZ}\in\A^\ZZ$, we have $z_j=\phi(w_{j+t},\cdots,w_{j-r})$ for every $j\in\ZZ$. The parameter $t$, resp. $r$, is called \emph{anticipation}, resp. \emph{memory}.
\end{defn}
%This means that each digit of the image $\varphi(w)$ is computed from $p$ digits of $w$ in a sliding window. 
This means that a sliding window of a length $p$ computes the digit $z_j$ of the image $\varphi(w)$ from digits $(w_{j+t},\cdots,w_{j-r})$, the digit $z_{j+1}$ from digits $(w_{j+t-1},\cdots,w_{j-r-1})$ etc.
  
  
Since two $(\beta,\A)$-representations may be easily summed up digit-wise in parallel, the crucial point of parallel addition is conversion of a $(\beta,\A+\A)$-representation of the sum to a $(\beta,\A)$-representation. The notion of a $p$-local function is applied to this conversion.
\begin{defn}
\label{def:digitSetConversion}
Let $\beta$ be a base and let $\A$ and $\B$ be alphabets containing 0. A function $\varphi:\B^\ZZ\rightarrow \A^\ZZ$ such that
  \begin{enumerate}[i)]
      \item for any $w=(w_j)_{j\in\ZZ}\in\B^\ZZ$ with finitely many non-zero digits, $z=\varphi(w)=(z_j)_{j\in\ZZ}\in\A^\ZZ$ has only finite number of non-zero digits, and
      \item $\sum_{j\in\ZZ} w_j \beta^j= \sum_{j\in\ZZ} z_j \beta^j$
  \end{enumerate}
  is called \emph{digit set conversion} in the base $\beta$ from $\B$ to $\A$. Such a conversion $\varphi$ is said to be \emph{computable in parallel} if $\varphi$ is a $p$-local function for some $p\in\NN$. 
\end{defn}

Suppose that there is a processing unit on  each position of an input $w=(w_j)_{j\in\ZZ}$ with access to $t$ input digits on the left and $r$ input digits on the right. If $w$ has only finitely many non-zero digits, then computation of $\varphi(w)$ can be done in a constant time independent on the number of non-zeros in the sequence $w$.

We recall some results on parallel addition in a numeration system with an integer alphabet. C. Frougny, E. Pelantov\'a and M. Svobodov\'a proved  the following sufficient condition of existence of an algorithm for parallel addition in \cite{parAddNS}.
  \begin{thm}
  \label{thm:suffConjugates}
  Let $\beta\in\CC$ be an algebraic number such that $|\beta|>1$ and all its conjugates in modulus differ from 1. There exists an alphabet $\A$ of contiguous integers containing 0 such that addition on $\fin{\A}$ can be performed in parallel.
  \end{thm}
An algorithm for an alphabet of the form $\{-a,-a+1, \dots,0,\dots,a-1,a\}$ is provided in the proof, but in general, $a$ is not minimal.
    
The same authors showed in \cite{kBlock} that the condition on the conjugates of the base $\beta$ is also necessary:
  \begin{thm}
  Let the base $\beta\in\CC, |\beta|>1,$ be an algebraic number with a conjugate $\beta'$ such that $|\beta'|=1$. Let $\A\subset\ZZ$ be an alphabet of contiguous integers containing 0. Then addition on $\fin{\A}$ cannot be computable in parallel.
  \end{thm}
We will see later that the construction of a parallel addition algorithm by the method from Chapter~\ref{chap:ewm} is also significantly influenced by the absolute value of conjugates of a base.
  
A lower bound on the size of an integer alphabet is provided in \cite{minAlph}. 
  \begin{thm}
  \label{thm:lowerBoundAlphabet}
Let $\beta\in\CC, |\beta|>1,$  be an algebraic integer with the minimal polynomial $m_\beta$. Let $\A\subset\ZZ$ be an alphabet of contiguous integers containing 0 and 1. If addition on $\fin{\A}$ is computable in parallel, then $\#\A \geq |m_\beta(1)|$. Moreover, if $\beta$ is a real number, $\beta>1$, then $\#\A \geq  |m_\beta(1)|+2$.
  \end{thm}
  
In Section~\ref{sec:minimalAlphabet}, we prove the same bound for a larger class of alphabets. The most general alphabets, which are considered in this thesis, are finite subsets of the set from the next definition. For our purposes, a ring is associative under multiplication and there is a multiplicative identity.
\begin{defn}
Let $\omega$ be a complex number. The smallest ring which contains integers $\ZZ$ and $\omega$ is denoted by
$$
    \ZZ[\omega] =\left\{\sum_{i=0}^n a_i \omega^i\colon n\in\NN, a_i\in\ZZ \right\} \,.
$$
\end{defn}
In other words, the set $\Zomega$ are all polynomials with integer coefficients evaluated in $\omega$. Obviously, it is a subset of the field extension $\QQ(\omega)$ and commutative ring.
    
From now on, let $\omega$ be an algebraic integer  which generates the set $\Zomega$ and let $\beta\in\Zomega$ be a base, i.e., $|\beta|>1$. We remark that $\beta$ is also an algebraic integer as all elements of $\Zomega$ are algebraic integers. Finally, let $\A\subset\Zomega$ be an alphabet. By definition, it means that it is a finite set containing 0 .

A few parallel addition algorithms for such numeration system with a non-integer alphabet were found manually \cite{milena}. We introduce the method for construction of a parallel addition algorithm for a given numeration system $(\beta,\A)$ in Chapter \ref{chap:ewm}. 
