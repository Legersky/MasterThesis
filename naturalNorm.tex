

\begin{lem}
\label{lem:vectNorm}
Let $\nu$ be a norm of the vector space $\CC^d$ and $P$ be a nonsingular matrix in $\CC^d$. Then the mapping $\mu:\CC^d\rightarrow \RR^+_0$ defined by $\mu(x)=\nu(Px)$ is also a norm of the vector space $\CC^d$.
\end{lem}
\begin{proof}
Let $x$ and $y$ be vectors in $\CC^d$ and $\alpha\in \CC$.  We use linearity of matrix multiplication, nonsingularity of matrix $P$ and the fact that $\nu$ is a norm to prove the following statements:
\begin{enumerate}
    \item $\mu(x)=\nu(Px)\geq 0\,,$
    \item $\mu(x)=0 \iff \nu(Px)=0 \iff Px=0 \iff x=0\,,$
    \item $\mu(\alpha x)=\nu(P(\alpha x))=\nu(\alpha Px)=|\alpha|\nu(Px)=|\alpha|\mu(x)\,,$
    \item $\mu(x+y)=\nu(P(x+y))=\nu(Px+Py)\leq \nu(Px)+\nu(Py)=\mu(Px)+\mu(Py)\,.$
\end{enumerate}
This  verifies that $\mu$ is a norm.
\end{proof}


\begin{theo}
\label{thm:norm}
Let $M\in\CC^{n\times n}$ be a diagonalizable matrix. There exists a vector norm $\norm{\cdot}{M}$ such that 
$$
\rho(M)=\Mnorm{M}{M}\,,
$$
where $\rho(M)$ is the spectral radius of the matrix $M$ and $\Mnorm{\cdot}{M}$ is the natural matrix norm induced by $\norm{\cdot}{M}$.
\end{theo}
\begin{proof}
First, we prove that $\Mnorm{M}{}\geq\rho(M)$ for every natural matrix norm induced by $\norm{\cdot}{}$. For all eigenvalues $\lambda$ in the spectrum $\sigma(M)$ with a respective eigenvector $u$ such that $\norm{u}{}=1$, we have
$$
\Mnorm{M}{}=\max_{\norm{x}{}=1} \norm{Mx}{}\geq \norm{Mu}{}=\norm{\lambda u}{}=|\lambda|\cdot\norm{u}{}=|\lambda|\,.
$$
Now, we construct the natural matrix norm $\norm{\cdot}{M}$ such that $\Mnorm{M}{M}\leq\rho(M)$. Since $M$ is diagonalizable, there exist nonsingular matrix $P\in\CC^{n\times n}$ and diagonal matrix $C\in\CC^{n\times n}$ with the eigenvalues of $M$ on the diagonal such that 
$$
PMP^{-1}=C\,.
$$ 

Using Lemma \ref{lem:vectNorm} and Euclidean norm $\norm{\cdot}{2}$, we define the vector norm $\norm{\cdot}{M}$ by 
\begin{equation}
\norm{x}{M}:=\norm{Px}{2}
\end{equation}
for all $x\in\CC^n$. The natural matrix norm $\norm{\cdot}{M}$ is induced by the vector norm $\norm{\cdot}{M}$, i.e.,
$$
\Mnorm{M}{M}=\max_{\norm{y}{M}=1} \norm{My}{M}\,.
$$
Let $y$ be a  vector such that $\norm{y}{M}=1$ and set $z=Py$. Notice that 
$$
\sqrt{z^*z}=\norm{z}{2}=\norm{Py}{2}=\norm{y}{M}=1\,.
$$
Consider
\begin{align*}
\norm{My}{M}&=\norm{PMy}{2}=\norm{PMy}{2}=\norm{CPy}{2}=\norm{Cz}{2}=\sqrt{z^*C^*Cz}\\
    &\leq \sqrt{\max_{\lambda\in\sigma(M)}|\lambda|^2 z^*z}=\max_{\lambda\in\sigma(M)}|\lambda|=\rho(M)\,\,.
\end{align*}
which implies the statement.
\end{proof}

\begin{lem}
Let $\omega$ be an algebraic integer of degree $d$ and let $S$ be the companion matrix of its minimal polynomial. Let $\beta=\sum_{i=0}^{d-1} b_i \omega^i$ be a nonzero element of $\Zomega$. Then the matrix $S_\beta=\multMat{b}$ is diagonalizable, its characteristic polynomial is $m_\beta^k$ with $k=d / \deg \beta$ and $|\det S_\beta|=|m_\beta(0)|^k$.  
\end{lem}
\begin{proof}
The characteristic polynomial of the companion matrix $S$ is the same as minimal polynomial of $\omega$ which has no multiple roots. Hence, $S$ is diagonalizable, i.e., $S=PDP^{-1}$ where $D$ is diagonal matrix with the conjugates of $\omega$ on the diagonal. The matrix $S_\beta$ is also diagonalized by $P$:
$$
S_\beta=\sum_{i=0}^{d-1} b_i S^i= \sum_{i=0}^{d-1} b_i \left(PDP^{-1}\right)^i=P\underbrace{\left(\sum_{i=0}^{d-1} b_i D^i\right)}_{D_\beta}P^{-1}\,.
$$
By Theorem CONJUGATES SE ZOBRAZUJI NA CONJUGATES, the diagonal elements of $D_\beta$ are conjugates of $\beta$. Since $S_\beta\in\ZZ^{d\times d}$, its characteristic polynomial has integer coefficients. Thus it is $k$-th power of minimal polynomial $m_\beta$. The value $k$ follows from the equality $d=\deg(m_\beta^k)=k \deg m_\beta$. 

The modulus of the determinant of $S_\beta$ equals the modulus of the absolute coefficient of the characteristic polynomial which is $|m_\beta(0)|^k$.

\end{proof}

\begin{defn}
Using the notation of the previous lemma, the vector norm $\norm{\cdot}{S_\beta}$ given by Theorem~\ref{thm:norm} for the matrix $S_\beta$ is called \emph{MRIZKOVA} norm and we denote it briefly $\normBeta{\cdot}$. Similarly, we write $\Mnorm{\cdot}{\beta}$  instead of $\Mnorm{\cdot}{S_\beta}$.
\end{defn}
