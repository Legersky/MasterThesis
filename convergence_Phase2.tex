\begin{lem}
\label{lem:equivalentStatementsForNonConvergenePhaseTwo}
The following statements are equivalent:
\begin{enumerate}[i)]
	\item Phase~2 does not converge,
	\item $\forall \,k\in \NN \,\exists\, \tupleo{k}\in\B^{k+1} \colon \#\Qwo{k}\geq 2$,
	\item $\exists \,(w_{-k})_{k\geq 0}, w_{-k}\in\B \,\exists\, k_0\forall k\geq k_0 \colon \#\Qwo{k}=\#\Qwo{(k-1)}\geq 2$.
\end{enumerate}
\end{lem}
\begin{proof}
\textit{i)}$\iff$\textit{ii):} The while loop in Algorithm~\ref{alg:weightFunction} ends if and only if there exist $k\in\NN$ such that $\#\Qwo{k}=1$ for all $\tupleo{k}\in\B^{k+1}$.

\textit{ii)}$\iff$\textit{iii):} There is an infinite sequence $(w_{-k})_{k\geq 0}$ such that $\#\Qwo{k}\geq 2$ for all $k\in\NN$ since $\Qwo{k}\supset\Qwo{(k+1)}$. Hence, the sequence of integers $(\#\Qwo{k})_{k\geq 0}$ is eventually constant. The opposite implication is trivial.
\end{proof}

\begin{defn}
Let $\B$ be an alphabet of input digits. We say that Phase~2 is \emph{stable}, if the following conditions on the sets of possible weight coefficients which are produced by the algorithm hold: 
\begin{enumerate}[i)]
	\item $\#\Qw{1}{k}=1 \implies \#\Qwo{k}=1$,
	\item $\Qw{1}{k}=\Qw{1}{(k-1)} \implies \Qwo{k}=\Qwo{(k-1)}$
\end{enumerate}
for all $k\in\NN, k\geq 2$ and for all $\tupleo{(k+1)} \in \B^{k+1}$.
\end{defn}

\komentar{okomentovat, co to znamena, i) mozna neni potreba pokud nebude opacna implikace}

\begin{upravit}
For shorter notation, set 
$$
\Qb{m}:=\Q_{[\underbrace{\scriptstyle b,\dots,b}_m]}
$$ for $m \in \NN$ and $b\in\B$.

Obviously, finiteness of Phase~2 implies that there exists a length of window $M$ such that the set $\Qb{m}$ contains only one element for all $b\in\B$. 
The following theorem is used for the construction of Algorithm \ref{alg:oneletterSets} which checks this necessary condition. 
\end{upravit}




\begin{thm}
\label{thm:bbbCondition}
Let $m_0 \in \NN$ and $b\in\B$ be such that sets $\Qb{m_0}$ and $\Qb{m_0-1}$ produced by deterministic Algorithm \ref{alg:minimalSet} within Phase~2 have the same size. Then
$$
    \#\Qb{m} = \#\Qb{m_0} \qquad \forall m\geq m_0-1\,.
$$ 
Particularly, if $\#\Qb{m_0}\geq 2$, Phase~2 does not converge.
\end{thm}
\begin{proof}
\komentar{plyne snadno pro phase 2 stable}
We prove the base case of induction with respect to $m$. For $m=m_0+1$, the set $\Qb{m_0+1}$ is found by Algorithm \ref{alg:minimalSet} such that 
$$
b + \Qb{m_0} \subset \A + \beta \Qb{m_0+1}\,
$$
and set $\Qb{m_0}$ is found by the same algorithm such that
$$
b + \Qb{m_0-1} \subset \A + \beta \Qb{m_0}\,.
$$
As $\Qb{m_0} \subset \Qb{m_0-1}$, the assumption of the same size implies
$$
    \Qb{m_0} = \Qb{m_0-1}\,.
$$
It means that Algorithm \ref{alg:minimalSet} runs with the same input and hence
$$
\Qb{m_0+1}=\Qb{m_0}\,.
$$
The inductive step of the proof for $m+1$ is analogous to the base case.

If $\#\Qb{m_0}\geq 2$, then the statement \textit{iii)} in Lemma~\ref{lem:equivalentStatementsForNonConvergenePhaseTwo} holds for the sequence $(b)_{k\geq 0}$.
%Phase~2 ends when there is only one element in $\Q_{[w_j,\dots, w_{j-m+1}]}$ for all $(w_j,\dots, w_{j-m+1}) \in \B^m$ for some fixed length of window $m$. But if $\#\Qb{m_0}\geq 2$, size of $\Q_{[b,\dots,b]}$ does not decrease despite of extending the length of window.
\end{proof}







\begin{defn}
Let $\B$ be an alphabet of input digits. Let Phase~2 is stable. Let $k\in\NN, k\geq 2$. We set
$$
V:=\left\{\tuple{1}{k}\in\B^k \colon \#\Qw{1}{k}=\#\Qw{1}{(k-1)}\right\}
$$
and
$$
E:=\left\{\tuple{1}{k}\rightarrow\tuple[w']{1}{k}\in V\times V \colon \tuple{2}{k}=\tuple[w']{1}{(k-1)}\right\}\,.
$$
The oriented graph $G_k=(V,E)$ is called \emph{Rauzy graph of Phase~2 (for the window of length~$k$)}.
\end{defn}

\begin{thm}
Let Phase~2 is stable.  If there exists $k_0\in\NN, k_0\geq 2$, and $\tupleo{k_0}\in\B^{k_0+1}$ such that
\begin{enumerate}[i)]
	\item $\#\Qwo{(k_0-1)}>1$ and
	\item there exist an infinite walk $(\tuple[w^{(i)}]{1}{k_0})_{i\geq 1}$ in $G_{k_0}$ which starts in the vertex  $$\tuple[w^{(1)}]{1}{k_0}=\tuple{1}{k_0}\,,$$ 
\end{enumerate}
then Phase~2 does not converge.
\end{thm}
\begin{proof}
Set
$$(w_k)_{k\geq 0}:=w_0, w_1^{(1)}, \dots , w^{(1)}_{k_0-1}, w^{(1)}_{k_0}, w_{k_0}^{(2)},w_{k_0}^{(3)},w_{k_0}^{(4)},\dots$$
We prove that $\#\Qwo{k}=\#\Qwo{(k_0-1)}>1$ for all $k\geq k_0-1$, i.e., the condition \textit{iii)} in Lemma~\ref{lem:equivalentStatementsForNonConvergenePhaseTwo} is satisfied.

%The general ideas of the proof are following: if $\tuple[w']{1}{k_0}$ is a vertex of $G_{k_0}$, then $$\Qw[w']{1}{(k_0-1)}=\Qw[w']{1}{k_0}\,.$$ And the assumption that Phase~2 is stable implies that $$\Qw[w']{1}{(k_0-1)}=\Qw[w']{1}{k_0}\implies \Qwo[w']{(k_0-1)}=\Qwo[w']{k_0}$$ for all $w'_0\in\B$.

%Now we show that $\Qwo{k}=\Qwo{(k+1)}$ for all $k\geq k_0-1$. 
Let $l\in\NN$. Since $\tuple{(1+l)}{(k_0+l)}$ is a vertex of $G_{k_0}$, the set $\Qw{l}{(k_0+l)}$ equals $\Qw{l}{(k_0+l-1)}$. As Phase~2 is stable, we have
\begin{align*}
\Qw{l}{(k_0+l)}&=\Qw{l}{(k_0+l-1)}\\
\implies \Qw{(l-1)}{(k_0+l)}&=\Qw{(l-1)}{(k_0+l-1)}\\
&\vdots \\
\implies \Qw{1}{(k_0+l)}&=\Qw{1}{(k_0+l-1)}\\
\implies \Qwo{(k_0+l)}&=\Qwo{(k_0+l-1)}\,.
\end{align*}
Hence, $\#\Qwo{k}=\#\Qwo{(k_0-1)}>1$ for all $k\geq k_0-1$.
\end{proof}
\komentar{bylo by fajn dokazat i opacny smer, jedine, pres co se neumim dostat je kdyby existovala jen aperiodicka posloupnoust, kvuli ktere to nekonverguje}













