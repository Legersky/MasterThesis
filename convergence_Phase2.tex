We have no simple sufficient or necessary condition of convergence of Phase~2 in the sense of properties of a base $\beta$ or an alphabet $\A$. Nevertheless, the convergence can be controled during a run of algorithm. An easy check of non-convergence can be done by searching $\Q_{[b, \dots, b]}$ for each $b\in\B$ separately. This was already described in \cite{vu}, but we recall it with a simpler proof. For its purposes, we introduce a notion of stable Phase~2, which is used also in the main result of this section -- the control of convergence during Phase 2 is transformed into searching for a cycle in an oriented graph.

Firstly, we menshion some equivalent conditions of non-convergence of Phase~2. It enables us to handle easier with non-convergence in later  proofs.
\begin{lem}
\label{lem:equivalentStatementsForNonConvergenePhaseTwo}
The following statements are equivalent:
\begin{enumerate}[i)]
	\item Phase~2 does not converge,
	\item $\forall \,k\in \NN \,\exists\, \tupleo{k}\in\B^{k+1} \colon \#\Qwo{k}\geq 2$,
	\item $\exists \,(w_{-k})_{k\geq 0}, w_{-k}\in\B \,\exists\, k_0\forall k\geq k_0 \colon \#\Qwo{k}=\#\Qwo{(k-1)}\geq 2$.
\end{enumerate}
\end{lem}
\begin{proof}
\textit{i)}$\iff$\textit{ii):} The while loop in Algorithm~\ref{alg:weightFunction} ends if and only if there exist $k\in\NN$ such that $\#\Qwo{k}=1$ for all $\tupleo{k}\in\B^{k+1}$.

\textit{ii)}$\iff$\textit{iii):} There is an infinite sequence $(w_{-k})_{k\geq 0}$ such that $\#\Qwo{k}\geq 2$ for all $k\in\NN$ since $\Qwo{k}\supset\Qwo{(k+1)}$. Hence, the sequence of integers $(\#\Qwo{k})_{k\geq 0}$ is eventually constant. The opposite implication is trivial.
\end{proof}

We need to ensure that choice of a possible weight coefficient set $\Qwo{k}\subset \Qwo{(k-1)}$ is determined by an input digit $w_0$ and a set $\Qw{1}{0}$, while the influence of the set $\Qwo{(k-1)}$ is limited. It is formalized in the following definition.

\begin{defn}
Let $\B$ be an alphabet of input digits. We say that Phase~2 is \emph{stable}, if the following condition on the sets of possible weight coefficients which are produced by the algorithm holds: 
$$
\Qw{1}{k}=\Qw{1}{(k-1)} \implies \Qwo{k}=\Qwo{(k-1)}
$$
for all $k\in\NN, k\geq 2$ and for all $\tupleo{(k+1)} \in \B^{k+1}$.

\komentar{ii)$\#\Qw{1}{k}=1 \implies \#\Qwo{k}=1$, neni potreba pokud nebude opacna implikace}
\end{defn}
The definition may seem too restrictive, but note that $\Qwo{k}$ if fully determined by $\Qw{1}{k}$, $\Qwo{(k-1)}$ and $w_0$ for a fixed deterministic way of choice of possible weight coefficients sets. Therefore, the assumption $\Qw{1}{k}=\Qw{1}{(k-1)}$ implies that the only difference in the choice of $\Qwo{k}$ an $\Qwo{(k-1)}$ is that $\Qwo{k}$ is a subset of  $\Qwo{(k-1)}$, while $\Qwo{(k-1)}$ is chosen as a subset of $\Qwo{(k-2)}$. At the same time, $\Qwo{(k-1)}$ is a subset of $\Qwo{(k-2)}$. Thus, the requirement that Phase~2 is stable means that the same possible weight coefficients set is found even if it is searched as a subset of smaller set. This is natural way how an algorithm of choice should be constructed -- the set $\Qwo{k}$ is searched such that
$$
\B+\Qw{1}{k} \subset \A+\beta \Qwo{k}\,,
$$
i.e., there is no reason to choose the set $\Qwo{k}\subsetneq\Qwo{(k-1)}$ as we know that
$$
\B+\underbrace{\Qw{1}{(k-1)}}_{=\Qw{1}{k}} \subset \A+\beta \Qwo{(k-1)}\,.
$$
In other word, if $\Qwo{k}\subsetneq\Qwo{(k-1)}$, the set $\Qwo{(k-1)}$ might have been chosen smaller.

Now we use that finiteness of Phase~2 implies that there exists a length of window $m$ such that the set $\Qb{m}$ contains only one element for all $b\in\B$, where $\Qb{m}$ is a shorter notation for
$$
\Q_{[\underbrace{\scriptstyle b,\dots,b}_m]}\,.
$$
The following theorem was proved in \cite{vu} with the assumption that Phase~2 is deterministic. Briefly, it says that $\#\Qb{m}$ must decrease every time we increase $m$, otherwise Phase~2 does not converge. When we consider only inputs of the form $bb\dots b$ for some $b\in\B$, determinism implies that Phase~2 is stable. The given proof with Phase~2 being stable is slightly shorter.

\begin{thm}
\label{thm:bbbCondition}
Let $m_0 \in \NN$ and $b\in\B$ be such that sets $\Qb{m_0}$ and $\Qb{m_0-1}$ produced by stable Phase~2 have the same size. Then
$$
    \#\Qb{m} = \#\Qb{m_0} \qquad \forall m\geq m_0-1\,.
$$ 
Particularly, if $\#\Qb{m_0}\geq 2$, Phase~2 does not converge.
\end{thm}
\begin{proof}
As $\Qb{m_0} \subset \Qb{m_0-1}$, the assumption of the same size implies
$$
    \Qb{m_0} = \Qb{m_0-1}\,.
$$
By the assumption that Phase~2 is stable, we have
\begin{align*}
 \Qb{m_0} = \Qb{m_0-1} &\implies  \Qb{m_0+1} = \Qb{m_0} \\
 						&\implies  \Qb{m_0+2} = \Qb{m_0+1} \\
 						&\vdots
\end{align*}
This implies the statement.

If $\#\Qb{m_0}\geq 2$, then the statement \textit{iii)} in Lemma~\ref{lem:equivalentStatementsForNonConvergenePhaseTwo} holds for the sequence $(b)_{k\geq 0}$.
\end{proof}



The condition of convergence during Phase~2 is formulated as a searching for an infinite path in a so-called Rauzy graph. This term comes from combinatorics on words. The vertices of our graph are combinations of input digits for which the size of their possible weight coefficients sets did not decreased with an increment of lenght of the window. Whereas in combinatorics on words, vertices are given as factors of some language. But the edges are placed identically -- if some combination of digits without the first one equals another one without the last digit.
\begin{defn}
Let $\B$ be an alphabet of input digits. Let Phase~2 is stable. Let $k\in\NN, k\geq 2$. We set
$$
V:=\left\{\tuple{1}{k}\in\B^k \colon \#\Qw{1}{k}=\#\Qw{1}{(k-1)}\right\}
$$
and
$$
E:=\left\{\tuple{1}{k}\rightarrow\tuple[w']{1}{k}\in V\times V \colon \tuple{2}{k}=\tuple[w']{1}{(k-1)}\right\}\,.
$$
The oriented graph $G_k=(V,E)$ is called \emph{Rauzy graph of Phase~2 (for the window of length~$k$)}.
\end{defn}

\begin{thm}
\label{thm:infinitePathInRauzyGraph}
Let Phase~2 is stable.  If there exists $k_0\in\NN, k_0\geq 2$, and $\tupleo{k_0}\in\B^{k_0+1}$ such that
\begin{enumerate}[i)]
	\item $\#\Qwo{(k_0-1)}>1$ and
	\item there exist an infinite walk $(\tuple[w^{(i)}]{1}{k_0})_{i\geq 1}$ in $G_{k_0}$ which starts in the vertex  $$\tuple[w^{(1)}]{1}{k_0}=\tuple{1}{k_0}\,,$$ 
\end{enumerate}
then Phase~2 does not converge.
\end{thm}
\begin{proof}
Set
$$(w_k)_{k\geq 0}:=w_0, w_1^{(1)}, \dots , w^{(1)}_{k_0-1}, w^{(1)}_{k_0}, w_{k_0}^{(2)},w_{k_0}^{(3)},w_{k_0}^{(4)},\dots$$
We prove that $\#\Qwo{k}=\#\Qwo{(k_0-1)}>1$ for all $k\geq k_0-1$, i.e., the condition \textit{iii)} in Lemma~\ref{lem:equivalentStatementsForNonConvergenePhaseTwo} is satisfied.

%The general ideas of the proof are following: if $\tuple[w']{1}{k_0}$ is a vertex of $G_{k_0}$, then $$\Qw[w']{1}{(k_0-1)}=\Qw[w']{1}{k_0}\,.$$ And the assumption that Phase~2 is stable implies that $$\Qw[w']{1}{(k_0-1)}=\Qw[w']{1}{k_0}\implies \Qwo[w']{(k_0-1)}=\Qwo[w']{k_0}$$ for all $w'_0\in\B$.

%Now we show that $\Qwo{k}=\Qwo{(k+1)}$ for all $k\geq k_0-1$. 
Let $l\in\NN$. Since $\tuple{(1+l)}{(k_0+l)}$ is a vertex of $G_{k_0}$, the set $\Qw{l}{(k_0+l)}$ equals $\Qw{l}{(k_0+l-1)}$. As Phase~2 is stable, we have
\begin{align*}
\Qw{l}{(k_0+l)}&=\Qw{l}{(k_0+l-1)}\\
\implies \Qw{(l-1)}{(k_0+l)}&=\Qw{(l-1)}{(k_0+l-1)}\\
&\vdots \\
\implies \Qw{1}{(k_0+l)}&=\Qw{1}{(k_0+l-1)}\\
\implies \Qwo{(k_0+l)}&=\Qwo{(k_0+l-1)}\,.
\end{align*}
Hence, $\#\Qwo{k}=\#\Qwo{(k_0-1)}>1$ for all $k\geq k_0-1$.
\end{proof}
\komentar{bylo by fajn dokazat i opacny smer, jedine, pres co se neumim dostat je kdyby existovala jen aperiodicka posloupnoust, kvuli ktere to nekonverguje}













