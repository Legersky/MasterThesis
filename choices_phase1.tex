\section{Different methods in Phase 1}
\label{sec:methodsOne}


We recall that the ambiguous part of Phase~1 is extending an intermediate weight coefficient set $\Q_k$ to $\Q_{k+1}$ so that 
$$
\B+\Q_k \subset \A+\beta \Q_{k+1}\,.
$$
It means that a weight coefficient $q$ such that $x=a+\beta q$ for some $a\in\A$ must be found for all $x\in \B+\Q_k$. Since the alphabet $\A$ is redundant, there might be more such weight coefficients. Let $C_x\subset \Zomega$ be a set of all these candidates for some $x\in \B+\Q_k$, i.e., $C_x=\{q\in\Zomega\colon x=a+\beta q, a\in\A\}$. Set $C:=\{C_x\colon x\in \B+\Q_k\}$. For a given $x\in \B+\Q_k$, the set $C_x$ is constructed so that all digits $a\in\A$ are tested whether $x-a$ is divisible by $\beta$ according to Theorem~\ref{thm:divisibility}. See Algorithm~\ref{alg:extendWeightCoefSet} which constructs the set $C$.

Now we extend the set $\Q_k$ to $\Q_{k+1}$ so that $C_x \cap \Q_{k+1} \neq \emptyset$ for all $C_x \in C$. We describe five methods how it can be done (1a, 1b, 1c, 1d and 1e).

As $Q_k\subset \Q_{k+1}$, set $\Q_{k+1}:=\Q_k$. For methods 1a, 1b and 1d, add  the element of all $C_x\in C$ such that $\#C_x=1$ to $\Q_{k+1}$. Next, select elements from all $C_x\in C$ such that $C_x \cap \Q_{k+1} = \emptyset$ according to a chosen method and add them to $\Q_{k+1}$. Selection for different methods is following:
\begin{enumerate}[ ]
	\item 1a -- all elements of $C_x$,
	\item 1b and 1c -- all smallest elements in absolute value of $C_x$,
	\item 1d and 1e -- all smallest elements in $\beta$-norm of $C_x$,
\end{enumerate}
Note that more elements may be added by method 1e than 1d, resp. 1c than 1b, since adding necessary elements before ($\#C_x=1$)  may cause that $C_{x'}\cap \Q_{k+1} \neq \emptyset$ for some $C_{x'}\in C$. 

The procedure is summarized in Algorithm~\ref{alg:extendWeightCoefSet}. 
Another methods may decrease the number of added elements for instance by picking only one of all smallest elements.
	


We can slightly improve performance by substituting the set $\B + \Q_{k}$ by $(\B + \Q_{k})\setminus (\B + \Q_{k-1})$ on the line~\ref{line:replaceBySmaller} in Algorithm~\ref{alg:searchCand}, because
$$
\B + \Q_{k-1} \subset \A + \beta \Q_{k}\,.
$$
Since $\Q_{k+1} \supset \Q_{k}$, $C_x \cap \Q_{k+1} \neq \emptyset$ for  $x\in \B + \Q_{k-1}$.

\begin{algorithm}
  \caption{Extending intermediate weight coefficients set}
    \label{alg:extendWeightCoefSet}
  \begin{algorithmic}[1]
    \REQUIRE{previous weight coefficients set $\Q_{k}$, method number $M\in\{$1a, 1b, 1c, 1d, 1e$\}$}
    \STATE $\widetilde{\Q}:=\Q_{k}$
    \IF {$M\in\{$1a,1b,1d$\}$}
    	\FORALL {$C_x \in C$}
    		\IF {$\# C_x =1$}
		    	\STATE Add the element of $C_x$ to $\widetilde{\Q}$
			\ENDIF
		\ENDFOR
	\ENDIF
    \STATE $\Q_{k+1}:=\widetilde{\Q}$
    \STATE By Algorithm~\ref{alg:searchCand}, find set of candidates $C$
    \FORALL{$C_x \in C$}
        \IF{$C_x \cap \widetilde{\Q}=\emptyset$}
	        \IF {$M=$ 1a}
	        	\STATE Add all elements of $C_x$ to $\Q_{k+1}$
	        \ELSIF {$M\in\{$1b, 1c$\}$}
	        	\STATE Add all smallest elements in absolute value of $C_x$ to $\Q_{k+1}$ 
	        \ELSIF {$M\in\{$1d, 1e$\}$}
	        	\STATE Add all smallest elements in $\beta$-norm of $C_x$ to $\Q_{k+1}$
	        \ENDIF
        \ENDIF
    \ENDFOR
    \RETURN $\Q_{k+1}$
  \end{algorithmic}
\end{algorithm}



\begin{algorithm}
  \caption{Search for set of candidates $C$}
    \label{alg:searchCand}
  \begin{algorithmic}[1]
    \REQUIRE{the previous weight coefficients set $\Q_{k}$, alternatively also the set $\Q_{k-1}$}
    \STATE $C:=\{\emptyset\}$
    \FORALL[Alternatively, $x \in (\B + \Q_{k})\setminus (\B + \Q_{k-1})$]{$x \in \B + \Q_{k}$} \label{line:replaceBySmaller}
      \STATE $C_x:=\emptyset$
      \FORALL{$a \in \A$}
          \IF{$(x-a)$ is divisible by $\beta$ in $\Zomega$ (using Theorem~\ref{thm:divisibility})}
              \STATE Add $\frac{x-a}{\beta}$ to $C_x$
            \ENDIF
      \ENDFOR 
      \STATE Add the set $C_x$ to $C$
  \ENDFOR
  \RETURN $C$
  \end{algorithmic}
\end{algorithm}  
