\section{Modified Phase 2}
\label{sec:modifiedPhase2}

In this section, we introduce algorithms based on the theorems proved in Section~\ref{sec:convergencePhase2}. The first one checks convergence of Phase 2 for $bb\dots b$ inputs. Next, we show that it is possible to make Phase 2 stable by wrapping the choice of a set of possible weight coefficients into a simple while loop. Finally, we present an algorithm for Phase~2 which includes all modifications -- the mentioned check for $bb\dots b$ inputs and control of non-convergence by searching for an infinite walk in Rauzy graph $G_k$.

\subsection*{Checking $bb\dots b$ inputs}

Algorithm~\ref{alg:oneletterSets} was proposed in \cite{vu}. It checks whether Phase~2 stops when it processes input digits $bb\dots b$.
Sets $\Qb{m}$  can be easily constructed separately for each $b\in\B$ for given $m$. We build the set $\Qb{m}$ for input digits $bb\dots b$ in the same way as in Phase 2. This means that we first search for $\Qb{1}$ such that 
$$
b + \Q \subset \A + \beta \Qb{1}\,.
$$
Until the set $\Qb{m}$ contains only one element, we increment the length of  window $m$ and, using Algorithm \ref{alg:possibleWeightCoefSetStable}, we find a subset $\Qb{m+1}$ of the set $\Qb{m}$ such that
$$
b + \Qb{m} \subset \A + \beta \Qb{m+1}\,.
$$
In each iteration, we check whether the set $\Qb{m+1}$ is strictly smaller than the set $\Qb{m}$. If not, we know by Theorem \ref{thm:bbbCondition} that Phase 2 does not converge because $\#\Qb{m}$ is eventually a constant greater than 2.

Hence, non-finiteness of Phase~2 can be revealed by running Algorithm \ref{alg:oneletterSets} for each input digit $b\in\B$.
\begin{algorithm}
  \caption{Check the input $bb\dots b$}
    \label{alg:oneletterSets}
  \begin{algorithmic}[1]
    \REQUIRE{Weight coefficient set $\Q$, digit $b\in\B$}
    \ENSURE{TRUE if there is a unique weight coefficient for input $bb\dots b$, FALSE otherwise}
    \STATE Find minimal set $\Qb{1} \subset \Q$ such that
      $$
      b + \Q \subset \A + \beta \Qb{1}\,.
      $$
      \vspace{-20pt}
    \STATE $m:=1$
    \WHILE{$\#\Qb{m} > 1$}
        \STATE $m:= m +1$
        \STATE By Algorithm \ref{alg:possibleWeightCoefSetStable}, find minimal set $\Qb{m} \subset \Qb{m-1}$ such that
          $$
          b + \Qb{m-1} \subset \A + \beta \Qb{m}\,.
          $$  
          \vspace{-20pt}
        \IF{$\#\Qb{m}=\#\Qb{m-1}$}
            \RETURN FALSE
        \ENDIF
    \ENDWHILE  
    \RETURN TRUE
  \end{algorithmic}
\end{algorithm}


\subsection*{Stable Phase 2}
Recall that in order to use Theorem~\ref{thm:infinitePathInRauzyGraph} to check non-convergence of Phase~2, we require Phase~2 to be stable. An algorithm of choice of possible weight coefficients set can be easily modified to ensure stability of Phase~2, i.e., 
$$\Qw{1}{k}=\Qw{1}{(k-1)} \implies \Qwo{k}=\Qwo{(k-1)}$$
for all $\tuple{1}{k}\in\B^k$. If the algorithm is deterministic, then the set $\Qwo{k}$ is determined by the set $\Qw{1}{k}$, $\Qwo{(k-1)}$ and $w_0$. Similarly, the set $\Qwo{(k-1)}$ is determined by  the set $\Qw{1}{(k-1)}$, $\Qwo{(k-2)}$ and $w_0$. 

Suppose that $\Qw{1}{k}=\Qw{1}{(k-1)}$. Now, the only difference between $\Qwo{k}$ and $\Qwo{(k-1)}$ is that $\Qwo{k}$ is searched as a subset of $\Qwo{(k-1)}$, whereas $\Qwo{(k-1)}$ as a subset of $\Qwo{(k-2)}$. In order to find  $\Qwo{(k-1)}$, we use Algorithm~\ref{alg:minimalSet} repeatedly instead of once. In each iteration, the input digit $w_0$ and the set $\Qw{1}{(k-1)}$ remains the same but we search for $\Qwo{(k-1)}$ as a subset of $\Q'_w$, where $\Q'_w$ is the output of the previous iteration or $\Qwo{(k-2)}$ in the first iteration. The algorithm stops when $\Qwo{(k-1)}=\Q'_w$. This loop is described in Algorithm~\ref{alg:possibleWeightCoefSetStable}.

Now, when the set $\Qwo{k}$ is searched as a subset of $\Qwo{(k-1)}$ in the same manner, it runs with the same inputs as the last iteration of the search for  $\Qwo{(k-1)}$. Hence, $\Qwo{k}=\Qwo{(k-1)}$.


\begin{algorithm}
  \caption{Stable search for possible weight coefficient set $\Qwo{k}$}
    \label{alg:possibleWeightCoefSetStable}
  \begin{algorithmic}[1]
    \REQUIRE{Input digit $w_0$, set of possible carries $\Qw{1}{k}$, previous set of possible weight coefficients $\Qwo{(k-1)}$}
	\STATE $\Q'_w:=\Qwo{(k-1)}$
	\WHILE{TRUE}
		\STATE By Algorithm \ref{alg:minimalSet}, find the set of possible weight coefficients  $\Qwo{k}$ as a subset of $\Q'_w$ instead of the previous set of weight coefficients $\Qwo{(k-1)}$.
		\IF{$\Qwo{k}=\Q'_w$}
			\RETURN $\Qwo{k}$
		\ENDIF
		\STATE $\Q'_w:=\Qwo{k}$
	\ENDWHILE
  \end{algorithmic}
\end{algorithm}


\subsection*{Infinite walk in Rauzy graph $G_k$}
As the Rauzy graph $G_k$ is finite, there exist an infinite walk in $G_k$ if and only if there exists an oriented cycle. Algorithm~\ref{alg:checkCycles}  checks whether some walk starting in $\tuple{1}{k}$ enters such a cycle eventually. First, it  checks if the vertex $\tuple{1}{k}$ is in the graph $G_k$. If yes, all vertices which are accessible by appropriately directed edge from $\tuple{1}{k}$ are entered by Algorithm~\ref{alg:enterVertices}. It is called recursively to enter all accessible vertices from the given one. The already traversed path is passed in each call and if an vertex is visited second time, then the cycle is found and algorithm ends. If all branches of the recursion ends up without visiting some vertex twice, then there is no cycle and thus no infinite path.

Note that saving of the traversed path requires only the label of the first vertex and last digits of the labels of next visited vertices due to the construction of Rauzy graph.
\begin{algorithm}
  \caption{Check if there is in an infinite walk in $G_k$ starting in $\tuple{1}{k}$}
    \label{alg:checkCycles}
  \begin{algorithmic}[1]
    \REQUIRE{Rauzy graph $G_k$, combination of input digits $\tuple{1}{k}$}
    \ENSURE{Return TRUE if TRUE is returned in any step of the recursion, that is when a walk $w_1, w_2, \dots$ enters a directed cycle in $G_k$. Otherwise return FALSE.}
	\IF {$\tuple{1}{k}\in G_k$}
		\STATE By Algorithm~\ref{alg:enterVertices}, enter next vertices from $\tuple{1}{k}$ with the traversed path  $\tuple{1}{k}$.
	\ELSE
		\RETURN FALSE
	\ENDIF
	\RETURN FALSE
  \end{algorithmic}
\end{algorithm}

\begin{algorithm}
  \caption{Enter vertices from $\tuple[w']{1}{k}$}
    \label{alg:enterVertices}
  \begin{algorithmic}[1]
  	\REQUIRE{Rauzy graph $G_k$, vertex $\tuple[w']{1}{k}$, traversed path $(w_1, \dots, w_l)$.}
	\FORALL {$w'_{k+1}\in \B$ such that $\tuple[w']{1}{k}\rightarrow \tuple[w']{2}{(k+1)}\in G_k$}
		\IF {$\tuple[w']{2}{(k+1)}$ is in the traversed path $(w_1, \dots, w_l)$}
			\RETURN TRUE
		\ELSE
			\STATE By Algorithm~\ref{alg:enterVertices}, enter next vertices from $\tuple[w']{2}{(k+1)}$ with the traversed path   $(w_1, \dots, w_l, w'_{(k+1)})$.
		\ENDIF
	\ENDFOR
  \end{algorithmic}
\end{algorithm}


\subsection*{Phase 2 with non-convergence control}
Algorithm~\ref{alg:weightFunction_modified} modifies the basic proposal of Phase~2 to reveal its possible non-convergence. First, the necessary condition given by Theorem~\ref{thm:bbbCondition} is checked, i.e., the convergence of Phase~2 for inputs $bb\dots b$ is verified by Algorithm~\ref{alg:oneletterSets} for all $b\in\B$. 

\begin{algorithm}
  \caption{Modified search for a weight function (Phase 2)}
    \label{alg:weightFunction_modified}
  \begin{algorithmic}[1]
    \REQUIRE{weight coefficients set $\Q$}
    \FORALL{$b\in\B$}
    	\IF  {\NOT Check the input $bb\dots b$ by Algorithm \ref{alg:oneletterSets}}
    		\RETURN Phase 2 does not converge for input $bb\dots b$.
    	\ENDIF
    \ENDFOR
    \FORALL{$w_0 \in \B$} 
        \STATE By Algorithm~\ref{alg:possibleWeightCoefSetStable}, find set $\Q_{[w_0]} \subset \Q$ such that
          $$
          w_0 + \Q \subset \A + \beta \Q_{[w_0]}
          $$\vspace{-20pt}
    \ENDFOR
    \STATE $k:=0$
    %\STATE $G_0:=$ empty graph
    \WHILE{$\max\{\#\Qwo{k}:\tupleo{k} \in \B^{k+1} \} > 1$}
        \STATE $k:= k +1$
        \FORALL{$\{\tupleo{k} \in \B^{k+1}\colon \#\Qwo{(k-1)}>1\}$}
            \STATE By Algorithm~\ref{alg:possibleWeightCoefSetStable}, find set $\Qwo{k} \subset \Qwo{(k-1)}$ such that
              $$
              w_0 + \Qw{1}{k} \subset \A + \beta \Qwo{k}\,.
              $$\vspace{-20pt}
              \IF {$\Qwo{k} = \Qwo{(k-1)}$ \AND $k\geq 2$}
              	\STATE Add the vertex $\tupleo{k}$ to the Rauzy graph $G_{k+1}$.
              	\IF{Check infinite walks in $G_k$ starting in $\tuple{1}{k}$ by Alg.~\ref{alg:checkCycles}}
              		\RETURN Phase 2 does not converge.
              	\ENDIF
              \ENDIF
     %    	\STATE $G_{k}:=G_{k+1}$
        \ENDFOR  
    \ENDWHILE  
    \STATE $r:= k+1$
    \FORALL {$l\in\NN, l\leq r$}
	    \FORALL{$\{\tupleo{l}\in\B^{l+1} \colon \Qwo{l} \text{ was found, } \#\Qwo{l}=1\}$}
	    	\FORALL{$\tuple{(l+1)}{r}\in\B^{r-l}$}
	    		\STATE $q\tupleo{r}:=$ only element of $\Qwo{l}$
	    	\ENDFOR
	    \ENDFOR
	\ENDFOR
    \RETURN $q$
  \end{algorithmic}
\end{algorithm}

Then we proceed in the same way as in Algorithm~\ref{alg:weightFunction} with the following modifications: it is sufficient to process only $\tupleo{k} \in \B^{k+1}$ such that $\#\Qwo{(k-1)}>1$ since if $\#\Qwo{(k-1)}=1$, then $\Qwo{k'}=\Qwo{(k-1)}$ for all $k'>k$. 

Moreover, we check possible non-convergence according to Theorem~\ref{thm:infinitePathInRauzyGraph}. If $\Qwo{k} = \Qwo{(k-1)}$, then the vertex $\Qwo{k}$ is added into the Rauzy graph $G_{k+1}$ and the Rauzy graph $G_k$ is examined by Algorithm~\ref{alg:checkCycles} whether it contains an infinite walk starting in $\tuple{1}{k}$. Note that $\Qwo{k} = \Qwo{(k-1)}$ is a necessary condition for $\tuple{1}{k}$ be a vertex of $G_k$. It also implies that $\#\Qwo{k}>1$.

We remark that  non-convergence caused by an input $bb\dots b$ for some $b\in\B$ would be revealed also as a cycle in a Rauzy graph. Nevertheless, the number of calls of Algorithm~\ref{alg:minimalSet} during Algorithm~\ref{alg:oneletterSets} is at most $\#Q$ for each $b\in\B$, while it grows exponentially with the length of window in search for a weight function. Thus, we save a lot of computational time in cases which fail already on $bb\dots b$ inputs. At the same time, the cost of this check is low in other cases.







