We recall the general concept of addition at the beginning of this chapter and then we describe a so-called \emph{extending window method} which is due to M. Svobodov\'a \cite{milena}. 

From now on, let $\omega$ be an algebraic integer and $(\beta,\A)$ be a numeration system such that the base $\beta \in \Zomega$ and the alphabet $\A \ni 0$ is a finite subset of $\Zomega$. 

The general concept of addition (standard or parallel) in any numeration system $(\beta,\A)$, such that $\fin{\A}$ is closed under addition, is the following: we add numbers digitwise and then we convert the result into the alphabet $\A$. Obviously, digitwise addition is computable in parallel, thus the crucial point is the digit set conversion of the obtained result. It can be easily done in a standard way but a parallel digit set conversion is nontrivial. However, formulas are basically the same but the choice of coefficients differs.

Now we go step by step more precisely. Let $x=\sum_{-m'}^{n'} x_i\beta^i,y=\sum_{-m'}^{n'} y_i\beta^i \in \fin{\A}$ with $(\beta,\A)$-representations padded by zeros to have the same length. We set 
  \begin{align*}
    w&=x+y =\sum_{-m'}^{n'} x_i\beta^i + \sum_{-m'}^{n'} y_i\beta^i = \sum_{-m'}^{n'} (x_i+y_i)\beta^i \\
    &=\sum_{-m'}^{n'} w_i\beta^i \,,
  \end{align*}
  where $w_j=x_j+y_j \in \A +\A$. Thus, $w_{n'} w_{{n'}-1}\cdots w_1 w_0 \bullet w_{-1} w_{-2} \cdots w_{-m'}$ is a  $(\beta, \A+\A)$-representation of $w\in \fin{\A+\A}$. 

We also use column notation of addition in what follows, e.g.,     
  \begin{align*}
  x_{n'} \;x_{{n'}-1}\cdots x_1 \;x_0 &\bullet x_{-1} \;x_{-2}\, \cdots x_{-m'} \\[-3pt]
  y_{n'} \;y_{{n'}-1}\cdots y_1 \,\;y_0 &\bullet y_{-1} \;y_{-2} \;\cdots y_{-m'} \\[-7pt]
    \line(1,0){90} & \line(1,0){100} \\[-7pt]
  w_{n'} w_{{n'}-1}\cdots w_1 w_0 &\bullet w_{-1} w_{-2} \cdots w_{-m'}\,.
  \end{align*}
  
As we want to obtain a $(\beta,\A)$-representation of $w$, we search for a sequence 
  $$z_{n} z_{n-1}\cdots z_1 z_0 z_{-1} z_{-2} \cdots z_{-m}$$ such that $z_j \in \A$ and
  $$
    z_{n} z_{n-1}\cdots z_1 z_0 \bullet z_{-1} z_{-2} \cdots z_{-m}=(w)_{\beta,\A}\,.
  $$
  From now on, we consider without lost of generality only $\beta$-integers since modification for representations with rational part is obvious:
  $$
  \beta^m \cdot z_{n} z_{n-1}\cdots z_1 z_0 \bullet z_{-1} z_{-2} \cdots z_{-m} = z_{n} z_{n-1}\cdots z_1 z_0 z_{-1} z_{-2} \cdots z_{-m} \bullet
  $$  
  Particularly, let $(w)_{\beta, \A+\A}=w_{n'} w_{{n'}-1}\cdots w_1 w_0 \bullet$. We search for a number $n \in \NN$ and \mbox{$z_{n}, z_{n-1},\dots, z_1, z_0 \in\A$} such that $(w)_{\beta, \A}=z_{n} z_{n-1}\cdots z_1 z_0 \bullet$.   
  
  We use a suitable representation of zero to convert digits $w_j$ into the alphabet $\A$. 
  For our purpose, we use the simplest possible representation deduced from the polynomial
  $$
    x-\beta \in \left(\Zomega\right)[x]\,.
  $$

We remark that any polynomial $R(x)=r_s x^s+r_{s-1}x^{s-1}+ \dots + r_1 x+r_0$ with coefficients $r_i \in \Zomega,$ such that $R(\beta)=0$ gives us a possible representation of zero. The polynomial $R$ is called a \emph{rewriting rule}. 

Within a digit set conversion with an arbitrary rewriting rule $R$, one of the coefficients of $R$ which is greatest in modulus (so-called \emph{core coefficient}) is used for the conversion of a digit $w_j$. But using of an arbitrary rewriting rule $R$ is out of scope of this thesis, so we focus on the simplest possible rewriting rule $R(x)=x-\beta$.
  
As $0=\beta^{j} \cdot R(\beta)=1\cdot \beta^{j+1} -\beta \cdot \beta^{j}$, we have a representation of zero 
$$1 (-\!\beta) \underbrace{0 \cdots 0}_{j}\bullet = (0)_\beta\,. $$
for all $j \in \NN$. We multiply this representation by $q_j \in \Zomega$, which is called a \emph{weight coefficient}, to obtain another  representation of zero 
$$q_j (-q_j\beta) \underbrace{0 \cdots 0}_{j}\bullet = (0)_\beta\,. $$ 
This is digitwise added to $w_{n} w_{n-1}\cdots w_1 w_0 \bullet$ to convert the digit $w_j$ into the alphabet $\A$. The conversion of $j$-th digit causes a \emph{carry} $q_{j}$ on the $(j+1)$-th position. The digit set conversion runs from the right ($j=0$) to the left until all digits and carries are converted into the alphabet $\A$:
        \begin{align}
        \label{eq:conversionScheme}
            \hspace{100pt}  w_n w_{n-1}&&&\cdots& &w_{j+1}&\!\! &\textcolor{red}{w_j}  & \!\!  &w_{j-1} &&\cdots &&w_1 w_0\bullet \hspace{100pt} \notag\\[-5pt]
                         &&&&       &       & &     &   &q_{j-2} &&\iddots  \notag \\[-3pt] 
                         &&&&       &       & &\textcolor{red}{q_{j-1}}& -&\beta q_{j-1} \notag \\[-3pt]
                         &&&&         &q_j&   \textcolor{red}{-}&\textcolor{red}{\beta q_j} &&\\[-8pt]
                         &&&  \iddots      &   -&\beta q_{j+1}&   &\ && \notag \\[-17pt]
          \intertext{\hspace{60pt}\line(1,0){300}}
          \notag \\[-30pt]
           z_{n+s} \cdots z_{n} z_{n-1}&&&\cdots& &z_{j+1}& &\textcolor{red}{z_j}& &z_{j-1} &&\cdots &&z_1 \; z_0\bullet \notag                  
        \end{align}
    Hence, the desired formula for conversion on the $j$-th position is 
    \begin{equation*}
        z_j=w_j + q_{j-1} - q_j \beta
    \end{equation*}
    for $j \in \NN$. We set $q_{-1}=0$ as there is no carry from the right on the 0-th position.
    
     Clearly, the value of $w$ is preserved:
\begin{align}
\label{eq:valuePreserving}
    \sum_{j\geq 0} z_j \beta^j &=w_0 - \beta q_0 + \sum_{j> 0} (w_j + q_{j-1} - q_j \beta) \beta^j \notag\\
    &=\sum_{j\geq 0} w_j \beta^j + \sum_{j>0} q_{j-1} \beta^j - \sum_{j\geq 0} q_j \cdot \beta^{j+1}  \\
    &=\sum_{j\geq 0} w_j \beta^j + \sum_{j>0} q_{j-1} \beta^j - \sum_{j> 0} q_{j-1} \cdot \beta^j \notag\\
    &=\sum_{j\geq 0} w_j \beta^j = w\,. \notag
\end{align}

     The weight coefficient $q_j$ must be chosen so that the converted digit is in the alphabet~$\A$, i.e., 
    \begin{equation}
    \label{eq:conversionFormula}
        z_j=w_j + q_{j-1} - q_j \beta \in \A\,.
    \end{equation} 
    The choice of weight coefficients is a crucial part of construction of addition algorithms which are computable in parallel. The extending window method determining weight coefficients for a given input is described in Section~\ref{sec:methodDescription}.
    
    
     On the other hand, the following example shows that determining weight coefficients is trivial for standard numeration systems.
     
     \begin{exmp}
        Assume now a standard numeration system $(\beta, \A)$, where
  $$
    \beta \in \NN\,,\beta  \geq 2\,, \A = \{0, 1, 2,\dots, \beta -1 \}\,.
  $$ 
       Notice that
    $$
        z_j \equiv w_j+q_{j-1} \mod \beta\,. 
    $$
  
  There is only one representative of each class modulo  $\beta$ in the standard numeration system $(\beta, \A)$. Therefore, the digit $z_j$ is uniquely determined for a given digit $w_j \in \A+\A$ and carry $q_{j-1}$ and thus so is the weight coefficient $q_j$. This means that $q_j=q_j(w_j,q_{j-1})$ for all $j\geq 0$. Generally,
  $$
  q_j=q_j(w_j,q_{j-1}(w_{j-1},q_{j-2}))=\dots =q_j(w_j ,\dots , w_1, w_0)
  $$
  and
  $$
  z_j=z_j(w_j ,\dots , w_1, w_0)\,,
  $$
  which implies that addition runs in linear time.
     
     \end{exmp}
  
  We require that the digit set conversion from $\A+\A$ into $\A$ is computable in parallel, i.e., there exist constants $r,t \in \NN_0$ such that for all $j\geq 0$ is $z_j=z_j(w_{j+r},\dots,w_{j-t})$. To avoid the dependency on all less, respectively more, significant digits, we need variety in the choice of weight coefficient $q_j$. This implies that the used numeration system must be redundant.
  

\section{Extending window method}
\label{sec:methodDescription}
In order to construct a digit set conversion in numeration system $(\beta,\A)$ which is computable in parallel, we consider a more general case of digit set conversion from an \emph{input alphabet} $\B$ such that $\A \subsetneq \B \subset \A +\A$ instead of the alphabet $\A+\A$.
As mentioned above, the key problem is to find for every $j\geq 0$ a weight coefficient $q_j$ such that 
    $$
        z_j=\underbrace{w_j}_{\in \B} + q_{j-1} - q_j \beta \in \A 
    $$  
    for any input $w_{n'}w_{n'-1}\dots w_1 w_0 \bullet=(w)_{\beta,\B}, w\in \fin{\B}$. We remark that the weight coefficient $q_{j-1}$ is determined by the input $w$. For a digit set conversion to be computable in parallel the digit $z_j$ is required to satisfy $z_j=z_j(w_{j+r},\dots,w_{j-t})$ for a fixed anticipation $r$ and memory $t$ in $\NN$.
    
    We introduce following definitions. 
    \begin{defn}
    \label{def:weightCoefficientsSet}
        Let $\B$ be a set such that $\A \subsetneq \B \subset \A +\A$. Then any finite set $\Q\subset\Zomega$ containing~0 such that 
        $$
            \B + \Q \subset \A + \beta \Q
        $$  
        is called a \emph{weight coefficients set}.
    \end{defn}
    We see that if $\Q$ is a weight coefficients set, then
        $$
        (\forall w_j \in \B)(\forall q_{j-1}\in\Q)(\exists q_j \in \Q )(w_j + q_{j-1} - q_j \beta \in \A )\,.
        $$
    In other words, there is a weight coefficient $q_j \in \Q$ for a carry from the right $q_{j-1}\in \Q$ and  digit $w_j$ in the input alphabet $\B$. I.e., we  satisfy the basic digit set conversion formula~\eqref{eq:conversionFormula}. Notice that $q_{-1}=0$ is in $\Q$ by the definition. Thus, all weight coefficients may be chosen from $\Q$.
    \begin{defn}
    Let $M$ be an integer and $q:\B^{M} \rightarrow \Q$ be a mapping such that 
    $$
    w_j+ q(w_{j-1}, \dots, w_{j-M}) - \beta q(w_j, \dots, w_{j-M+1}) \in \A
    $$
    for all $w_j,w_{j-1}, \dots, w_{j-M} \in \B$, and $q(0,0,\dots,0)=0$. Then $q$ is called a \emph{weight function} and $M$ is called a \emph{length of window}.    
    \end{defn}

 Having a weight function $q$, we define a function $\phi:\B^{M+1}\rightarrow \A$ by
    \begin{equation}
    \label{eq:localConversion}
        \phi(w_{j}, \dots, w_{j-M})=w_j+ \underbrace{q(w_{j-1}, \dots, w_{j-M})}_{=q_{j-1}} - \beta \underbrace{q(w_j, \dots, w_{j-M+1})}_{=q_j}=:z_j\,,
    \end{equation} 
    which verifies that the digit set conversion is indeed a $(M+1)$-local function with anticipation $r=0$ and memory $t=M$. The requirement of zero output of the weight function $q$ for the input of $M$ zeros guarantees that $\phi(0,0,\dots,0)=0$. Thus the first condition of Definition~\ref{def:digitSetConversion} is satisfied. The second one follows from the equation \eqref{eq:valuePreserving}. 
    
Let us summarize the construction of the digit set conversion by the rewriting rule \mbox{$x-\beta$}. We need to find weight coefficients for all possible combinations of digits of the input alphabet~$\B$. Their multiples of the rewriting rules are digitwise added to the input sequence. In fact, it means that the equation  \eqref{eq:conversionFormula} is applied on each position. If the digit set conversion is computable in parallel, the weight coefficients are determined as the outputs of the weight function $q$ with some fixed length of window $M$.  

We search for the weight function $q$ for a given base $\beta$ and input alphabet $\B$ by the extending window method. It consists of two phases. First, we find a minimal possible weight coefficients set $\Q$. We know that it is possible to convert an input sequence by choosing the weight coefficients from the set $\Q$. The set $\Q$ serves as the starting point for the second phase in which we increment the expected length of the window $M$ until the weight function $q$ is uniquely defined for each $(w_j,w_{j-1}, \dots , w_{j-M+1}) \in \B^{M}$. Then, the local conversion is determined -- we use the weight function outputs as weight coefficients in the formula \eqref{eq:conversionFormula}.    

Note that the convergence of both phases is discussed separately in Chapter~\ref{chap:convergence}.

\section{Phase 1 -- Weight coefficients set}
\label{subsec:phase1}
The goal of the first phase is to compute a weight coefficients set $\Q$, i.e., to find a set $\Q \ni 0$ such that 
$$
    \B + \Q \subset \A + \beta \Q\,.
$$  
We build the sequence $\Q_0, \Q_1, \Q_2,\dots$ iteratively so that we extend $\Q_k$ to $\Q_{k+1}$ in a way to cover all elements of the set $\B+\Q_k$ by elements of the extended set $\Q_{k+1}$, i.e.,
$$
\B+ \Q_k \subset \A + \beta \Q_{k+1}\,.
$$
This procedure is repeated until the extended weight coefficients set $\Q_{k+1}$ is the same as the original set $\Q_{k}$. We remark that the expression ``a weight coefficient $q$ covers an element $x$'' means that there is $a \in \A$ such that $x=a + \beta q$. 

In other words, we start with $\Q_0=\{0\}$ meaning that we search all weight coefficients $q_j$ necessary for digit set conversion for the case where there is no carry from the right, i.e., $q_{j-1}=0$. We add them to the weight coefficients set $\Q_0$ to obtain the set $\Q_1$. Assume now that we have the set $\Q_k$ for some $k\geq 1$. The weight coefficients in $\Q_k$ now may appear as a carry $q_{j-1}$. If there are no suitable weight coefficients $q_j$ in the weight coefficients set~$\Q_k$ to cover all sums of added coefficients and digits of the input alphabet $\B$, we extend $\Q_k$ to $\Q_{k+1}$ by  suitable coefficients using Algorithm~\ref{alg:extendWeightCoefSet}. And so on until there is no need to add more elements, i.e., the extended set $\Q_{k+1}$ equals $\Q_k$. Then the weight coefficients set $\Q:=\Q_{k+1}$ satisfies Definition~\ref{def:weightCoefficientsSet}. 
    
The precise description of the algorithm in a pseudocode is in Algorithm~\ref{alg:weightCoefSet}. For better understanding, see Figures~\ref{img:phase1img1}--\ref{img:phase1img13} in Appendix~\ref{app:phase1} which illustrate the construction of the weight coefficients set $\Q$ for the Eisenstein base and a complex alphabet (see Example~\ref{ex:Eisenstein1-blockcomplex} for its description). 

Section~\ref{sec:convergencePhase2} discusses the convergence of Phase 1, i.e. whether it happens that  $\Q_{k+1}=\Q_k$ for some  $k$.
    
\begin{algorithm}
  \caption{Search for weight coefficients set (Phase 1)}
    \label{alg:weightCoefSet}
  \begin{algorithmic}[1]
    \STATE $k:=0$ 
    \STATE $Q_0:=\{0\}$
    \REPEAT
     \STATE  By Algorithm~\ref{alg:extendWeightCoefSet}, extend $\Q_k$ to $\Q_{k+1}$ in a minimal possible way so that $$\B+ \Q_k \subset \A + \beta \Q_{k+1}$$
     \vspace{-20pt}
      \STATE  $k:=k+1$
      \UNTIL{$\Q_k = \Q_{k+1}$}      
      \STATE $\Q:=\Q_k$
    \RETURN $\Q$
  \end{algorithmic}
\end{algorithm}

There may be more possible weight coefficients which cover some element of the set $\B+ \Q_k$. Let us suppose that we have the list which contains the lists of these candidates for each element of the set $\B+ \Q_k$. This list of lists is saved in the variable \verb+candidates+ in Algorithm~\ref{alg:extendWeightCoefSet}. Now, for each element, we check the list of candidates which cover this element and if there is none of them contained in the set $\Q_k$, the smallest (in absolute value) weight coefficient from the list of candidates is added to the set $\Q_k$. If there are more elements with the same absolute value, we deterministically choose one of them. The extension $\Q_{k+1}$ of the set $\Q_k$ is obtained in this manner.   

We may slightly improve this procedure: for example we may first extend $\Q_{k}$ by all single-element lists of \verb+candidates+. These elements may be enough to cover also other elements of $\B+\Q_{k}$. It implies that the resulting $\Q$ is dependent on the way of selection from \verb+candidates+.

\begin{algorithm}
  \caption{Extending intermediate weight coefficients set}
    \label{alg:extendWeightCoefSet}
  \begin{algorithmic}[1]
    \REQUIRE{\verb+candidates+ from Algorithm~\ref{alg:searchCand}, previous weight coefficients set $\Q_{k}$}
    \STATE $\Q_{k+1}:=\Q_{k}$
    \FORALL{\texttt{cand\_for\_x} in \texttt{candidates}}
        \IF{no element of \texttt{cand\_for\_x} in $\Q_{k}$}
            \STATE Add the smallest element (in absolute value) of  \verb+cand_for_x+ to $\Q_{k+1}$  
        \ENDIF
    \ENDFOR
    \RETURN $\Q_{k+1}$
  \end{algorithmic}
\end{algorithm}

Algorithm~\ref{alg:searchCand} describes the search for the list of lists of candidates. For each element $x\in \B + \Q_{k}$ we build the list of candidates (in the variable \verb+cand_for_x+) so that we test the divisibility of $x-a$ by the base $\beta$ for all letters $a\in\A$. In the positive case, the result of division is appended to \verb+cand_for_x+ as a possible weight coefficient. We remark that Theorem~\ref{thm:divisibility} is used to check the divisibility.


\begin{algorithm}
  \caption{Search for candidates}
    \label{alg:searchCand}
  \begin{algorithmic}[1]
    \REQUIRE{the previous weight coefficients set $\Q_{k}$, alternatively also the set $\Q_{k-1}$}
    \STATE \verb+candidates+:= empty list of lists
    \FORALL[Alternatively, $x \in (\B + \Q_{k})\setminus (\B + \Q_{k-1})$]{$x \in \B + \Q_{k}$} \label{line:replaceBySmaller}
      \STATE \verb+cand_for_x+:= empty list
      \FORALL{$a \in \A$}
          \IF{$(x-a)$ is divisible by $\beta$ in $\Zomega$ (using Theorem~\ref{thm:divisibility})}
              \STATE Append $\frac{x-a}{\beta}$ to \verb+cand_for_x+
            \ENDIF
      \ENDFOR 
      \STATE Append \verb+cand_for_x+ to \verb+candidates+
  \ENDFOR
  \RETURN \verb+candidates+
  \end{algorithmic}
\end{algorithm}  

We can improve the performance of Algorithm~\ref{alg:searchCand} by substituting the set $\B + \Q_{k}$ by $(\B + \Q_{k})\setminus (\B + \Q_{k-1})$ on the line~\ref{line:replaceBySmaller} because
$$
\B + \Q_{k-1} \subset \A + \beta \Q_{k} \subset \A + \beta \Q_{k+1}
$$
for any $\Q_{k+1} \supset \Q_{k}$. Thus there is no need to check whether the elements of $\B + \Q_{k-1}$ are covered by some weight coefficient from $\Q_k$ in Algorithm~\ref{alg:extendWeightCoefSet}.

% An added element from each list of \verb+candidates+ is chosen as the smallest one unless there is already a covering element contained in $\Q_{k}$ (Algorithm~\ref{alg:extendWeightCoefSet}).  


    

  
    
    





\section{Phase 2 -- Weight function}
\label{subsec:phase2}
    We want to find a length of the window $M$ and a weight function $q:\B^{M} \to \Q$. We start with the weight coefficients set $\Q$ obtained in Phase 1. The idea is to reduce necessary weight coefficients for the conversion of a given digit up to single value. This is done by enlarging the number of considered input digits (extending the length of window) -- there might be  less possible carries from the right if we know which digits on the right are converted. 
     
    We introduce the following notation. 
        Let $\Q$ be a weight coefficients set and $w_j\in \B$. Denote by $\Q_{[w_j]}$ any set such that
        $$
            (\forall q_{j-1} \in \Q)(\exists q_j \in \Q_{[w_j]})(w_j + q_{j-1} - q_j \beta \in \A)\,.
        $$
        By induction with respect to $m \in \NN, m>1$, for all $(w_{j},\dots, w_{j-m+1})\in \B^{m}$ denote by $\Q_{[w_{j},\dots, w_{j-m+1}]}$ any subset of  $\Q_{[w_{j},\dots, w_{j-m+2}]}$ such that 
        $$
           (\forall q_{j-1} \in \Q_{[w_{j-1},\dots, w_{j-m+1}]})(\exists q_j \in \Q_{[w_{j},\dots, w_{j-m+1}]})(w_j + q_{j-1} - q_j \beta \in \A)\,.
        $$
        
    
  
    Recall the scheme \eqref{eq:conversionScheme} of the digit set conversion for better understanding of the notation and method:
    \begin{align*}
        \hspace{130pt}\cdots\; &w_{j+1}&\!\! &w_j  & \!\!  &w_{j-1}&\cdots w_{j-M+1} &w_{j-M}\cdots \hspace{130pt} \\[-3pt] 
                         & &       & & & q_{j-2} \\[-3pt]
                         & &       &q_{j-1}& -&\beta q_{j-1} \\[-3pt]
                           &q_j&   -&\beta q_j &&\\[-3pt]      
                           -&\beta q_{j+1}&   &  &&\\[-15pt]      
    \intertext{\hspace{120pt}\line(1,0){250}} 
          \vspace{-15pt}
          \\[-30pt]
     \cdots\; &z_{j+1}& &z_j& &z_{j-1}& \cdots z_{j-M+1}\; &z_{j-M}\cdots                     
    \end{align*}     
    The idea is to check all possible right carries $q_{j-1}\in\Q$ and determine values $q_j\in\Q$ such that 
    $$
    z_j=w_j + q_{j-1} - q_j \beta \in \A \,.
    $$  
    
    So we obtain a set $\Q_{[w_j]}\subset\Q$ of weight coefficients which are necessary to convert the digit~$w_j$ with any carry $q_{j-1}\in\Q$. Assuming that we know the input digit $w_{j-1}$, the set of possible carries from the right is also reduced to $\Q_{[w_{j-1}]}$. Thus we may reduce the set $\Q_{[w_j]}$ to a set $\Q_{[w_j,w_{j-1}]}\subset \Q_{[w_j]}$ which is necessary to cover all elements of $w_j + \Q_{[w_{j-1}]}$. Prolonging the length of window in this manner may lead to a unique weight coefficient $q_j$ for enough given input digits.  
    
    Accordingly, the weight function $q$ is found if there is $M\in\NN$ such that 
    $$
    \#\Q_{[w_{j},\dots, w_{j-M+1}]}=1
    $$
    for all $w_{j},\dots, w_{j-M+1} \in \B^M$. The precise description of the construction of the weight function is in  Algorithm~\ref{alg:weightFunction}.  
 Figures~\ref{img:phase2img1}--\ref{img:phase2img7} in Appendix~\ref{app:phase2} illustrate the construction of the set $\Q_{[\omega,1,2]}$ for the Eisenstein numeration system.   
    
\begin{algorithm}
  \caption{Search for weight function (Phase 2)}
    \label{alg:weightFunction}
  \begin{algorithmic}[1]
    \REQUIRE{weight coefficients set $\Q$}
    \STATE $m:=1$
    \FORALL{$w_j \in \B$} 
        \STATE By Algorithm~\ref{alg:minimalSet}, find set $\Q_{[w_j]} \subset \Q$ such that
          $$
          w_j + \Q \subset \A + \beta \Q_{[w_j]}
          $$
    \ENDFOR
    \WHILE{$\max\{\#\Q_{[w_j,\dots, w_{j-m+1}]}:(w_j,\dots, w_{j-m+1}) \in \B^m \} > 1$}
        \STATE $m:= m +1$
        \FORALL{$(w_j,\dots, w_{j-m+1}) \in \B^{m}$}
            \STATE By Algorithm~\ref{alg:minimalSet}, find set $\Q_{[w_j,\dots, w_{j-m+1}]} \subset \Q_{[w_j,\dots, w_{j-m+2}]}$ such that
              $$
              w_j + \Q_{[w_{j-1},\dots, w_{j-m+1}]} \subset \A + \beta \Q_{[w_j,\dots, w_{j-m+1}]}\,,
              $$
        \ENDFOR  
    \ENDWHILE  
    \STATE $M:= m$ 
    \FORALL{$(w_j,\dots, w_{j-M+1}) \in \B^{M}$}  
        \STATE $q(w_j,\dots, w_{j-M+1}):=$ only element of $\Q_{[w_j,\dots, w_{j-M+1}]}$
    \ENDFOR
    \RETURN $q$
  \end{algorithmic}
\end{algorithm}
    
For construction of the set $\Q_{[w_j,\dots, w_{j-m+1}]}$ we first choose such elements of $\Q_{[w_j,\dots, w_{j-m+2}]}$ which are the only possible to cover some value $x \in w_0 + \Q_{[w_{j-1},\dots, w_{j-m+1}]}$. Other elements from $\Q_{[w_j,\dots, w_{j-m+2}]}$ which cover an uncovered value are added one by one to $\Q_{[w_j,\dots, w_{j-m+1}]}$ until each $x$ equals $a+\beta q_j$ for some $q_j$ in $\Q_{[w_j,\dots, w_{j-m+1}]}$ and $a\in\A$. The pseudocode is in Algorithm~\ref{alg:minimalSet}. 

\begin{algorithm}
  \caption{Search for set $\Q_{[w_j,\dots, w_{j-m+1}]}$ }
    \label{alg:minimalSet}
  \begin{algorithmic}[1]
    \REQUIRE{Input digit $w_j$, set of possible carries $\Q_{[w_{j-1},\dots, w_{j-m+1}]}$, previous set of possible weight coefficients $\Q_{[w_j,\dots, w_{j-m+2}]}$}
    \STATE \verb+list_of_coverings+:=empty list of lists
    \FORALL{$x \in w_j + \Q_{[w_{j-1},\dots, w_{j-m+1}]}$}
        \STATE Build a list \verb+x_covered_by+ of weight coefficients $q_j \in \Q_{[w_j,\dots, w_{j-m+2}]}$ such that 
        $$
        x=a+ \beta q_j \qquad\text{for some } a\in\A\,.
        $$ 
        \vspace{-20pt}
        \STATE Append \verb+x_covered_by+ to \verb+list_of_coverings+
    \ENDFOR
    \STATE $\Q_{[w_j,\dots, w_{j-m+1}]}$:= empty set
    \WHILE{\texttt{list\_of\_coverings} is nonempty}
        \STATE Pick an element $q$ of one of the shortest lists of \verb+list_of_coverings+ 
            \label{line:pickElement}
        \STATE Add the element $q$ to $\Q_{[w_j,\dots, w_{j-m+1}]}$
        \STATE Remove lists of \verb+list_of_coverings+ containing the element $q$ from \verb+list_of_coverings+
    \ENDWHILE
    \RETURN $\Q_{[w_j,\dots, w_{j-m+1}]}$
  \end{algorithmic}
\end{algorithm}


Notice that the result of Algorithm~\ref{alg:minimalSet} is influenced by the way how we pick an element on line~\ref{line:pickElement}. It can be done deterministically or non-deterministically. We use the following deterministic choice -- suppose that we want to choose from elements $x_1=\sum_{i=0}^{d-1}x_{1,i}\omega^i, x_2=\sum_{i=0}^{d-1}x_{2,i}\omega^i, \dots,x_n=\sum_{i=0}^{d-1}x_{n,i}\omega^i\in\Zomega$, where $d$ is the degree of $\omega$. Let $a_0,\dots,a_{d-1}\in\ZZ$ be such that 
$$
\sum_{i=0}^{d-1}a_i\omega^i=\sum_{j=1}^n x_j\,.
$$ 
Set $c:=\sum_{i=0}^{d-1}c_i\omega^i \in\Zomega$ with $c_i=[\frac{a_i}{n}]$ where $[\cdot]$ denotes rounding to the nearest integer. Let the index set $I_0\subset\{1,\dots,n\}$ be such that 
$$
|x_{j,0}-c_0|=\min\{|x_{k,0}-c_0|\colon k\in{1,\dots,n}\}
$$
for all $j\in I_0$. For all $i\in\{1,\dots,d-1\}$, let the index set $I_i\subset I_{i-1}$ be such that
$$
|x_{j,i}-c_i|=\min\{|x_{k,i}-c_i|\colon k\in I_{i-1}\}
$$
for all $j\in I_i$. If there is only one element in the index set $I_{d-1}=\{j_0\}$, choose the element~$x_{j_0}$. Otherwise choose $j_0\in I_{d-1}$ such that $x_{j_0,0}$ is the smallest from all $x_{j,0}$ such that $j\in I_{d-1}$. If there are more such elements, then choose from them according to the value $x_{j,1}$ etc. 

In other words, we take the elements which are the ``closest'' ones to the rounded center of gravity~$c$ of the values $x_1,\dots,x_n$ where ``closest'' is measured by absolute value of the first coordinate of $\pi(x_j)-\pi(c)$. In case of equality, according to the second coordinate etc. If there is more than one such element, we choose the element $x_{j_0}$ with the smallest first, resp. second, etc. coordinate of $\pi(x_{j_0})$.  

There is space to improve Phase 2 by a modification of Algorithm~\ref{alg:minimalSet}. It is possible that the effort to reduce the size of  $\Q_{[w_j,\dots, w_{j-m+1}]}$ as much as possible is not the best for convergence of Phase~2.

Unfortunately, we do not know when Phase 2 converges. But  we may reveal the nonconvergence of Phase 2 with deterministic Algorithm~\ref{alg:minimalSet} for some numeration systems by Algorithm~\ref{alg:oneletterSets}, which is described in Section~\ref{sec:convergencePhase2}.



Notice that for a given length of window $M$, the number of calls of Algorithm~\ref{alg:minimalSet} within Algorithm~\ref{alg:weightFunction} is
$$
\sum_{m=1}^M  \#\B^m = \#\B \sum_{m=0}^{M-1}  \#\B^m = \#\B \frac{\#\B^M-1}{\#\B-1}\,.
$$    
It implies that the time complexity grows exponentially as about $\#\B^M$. The required memory is also exponential because we have to store sets $\Q_{[w_j,\dots, w_{j-m+1}]}$ at least for $m\in\{M-1,M\}$  for all $w_j,\dots, w_{j-m+1} \in \B$.

We may reduce the number of the combinations of the input digits so that if for some $(w_j,\dots, w_{j-m+1})\in\B^m, m<M$ we have   $\#\Q_{[w_j,\dots, w_{j-m+1}]}=1$, we do not extend the window for these digits but we set the ouput of $q(w_j,\dots, w_{j-m+1}, w_{j-m}, \dots w_{j-M+1})$ to the single element of $\Q_{[w_j,\dots, w_{j-m+1}]}$ for all $(w_{j-m}, \dots w_{j-M+1})\in \B^{M-m}$. 

















