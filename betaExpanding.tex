\begin{thm}
\label{thm:betaExpanding}
\komentar{A[beta] neni definovane}    Let $\omega$ be a complex number and $\beta \in\Zomega$ be such that $|\beta|>1$. Let $\A\subset \Zomega$ be an alphabet. If $\NN\subset \A[\beta]$, then the number $\beta$ is expanding.
\end{thm}
\begin{proof}
For all $n\in\NN$ we may write 
    $$
    n=\sum_{i=0}^{N}a_i\beta^i\,,
    $$
    where $N\in\NN$, $a_i\in\A$ and $a_N\neq 0$.
    
    Set $m:= \max\{|a|\colon a\in\A\}$. We take $n\in\NN$ such that $n>m$. 
    Since $|a_0|\leq m<n$, we have  $N\geq 1$ and there is $i_0 \in \{1,2,\dots,N\}$ such that $a_{i_0}\neq 0$. Thus, $\omega$ is an algebraic number as $a_i\in\A\subset\Zomega$ and $\beta$ can be expressed as an integer combination of powers of $\omega$. Therefore, $\beta$ is also an algebraic number.
    
    Let $\beta'$ be an algebraic conjugate of $\beta$.  
    Since $\beta\in\Zomega\subset\QQ(\omega)$, there is an algebraic conjugate $\omega'$ of $\omega$ and an isomorphism $\sigma: \QQ(\omega)\rightarrow \QQ(\omega')$ such that $\sigma(\beta)=\beta'$. Now 
    $$
    n=\sigma(n)=\sum_{i=0}^{N}\sigma(a_i)(\beta')^i\,.
    $$
    Set $\tilde m:= \max\{|\sigma(a)|\colon a\in\A\}$.  For all $n\in\NN$, we have 
    $$
    n=|n|\leq\sum_{i=0}^{N}|\sigma(a_i)|\cdot|\beta'|^i \leq \sum_{i=0}^{\infty}|\sigma(a_i)|\cdot|\beta'|^i \leq \tilde m\sum_{i=0}^{\infty}|\beta'|^i\,.  
    $$
    Hence, the sum on the right  side diverges which implies that $|\beta'|\geq 1$. Thus, all conjugates of $\beta$ are at least one in modulus.
    
    If the degree of $\beta$ is one, the statement is obvious.  Therefore, we may assume that $\deg \beta \geq 2$. 
    
    Suppose  for contradiction that $|\beta'|=1$ for an algebraic conjugate $\beta'$  of $\beta$. The complex conjugate $\overline{\beta'}$ is also an algebraic conjugate of $\beta$. Take any algebraic conjugate $\gamma$ of $\beta$ and the isomorphism $\sigma': \QQ(\beta')\rightarrow \QQ(\gamma)$ given by $\sigma'(\beta')=\gamma$.
    Now
    $$
    \frac{1}{\gamma}=\frac{1}{\sigma'(\beta')}=\sigma'\left(\frac{1}{\beta'}\right)=\sigma'\left(\frac{\overline{\beta'}}{\beta'\overline{\beta'}}\right)=\sigma'\left(\frac{\overline{\beta'}}{|\beta'|^2}\right)=\sigma'(\overline{\beta'})\,.
    $$
    Hence, $\frac{1}{\gamma}$ is also an algebraic conjugate of $\beta$. Moreover, $\left|\frac{1}{\gamma}\right|\geq 1$ and $|\gamma|\geq 1$ which implies that $|\gamma|=1$. We may set $\gamma=\beta$ which contradicts $|\beta|>1$. Thus all conjugates of $\beta$ are greater than one in modulus, i.e., $\beta$ is an expanding algebraic number.
\end{proof}



\begin{thm}
Let $\A\subset \Zomega$ be an alphabet such that $1\in \A[\beta]$. If the extending window method with the rewriting rule $x-\beta$ converges for the numeration system $(\beta, \A)$, then the base $\beta$ is expanding. 
\end{thm}
\begin{proof}
The existence of an algorithm for addition which is computable in parallel implies that the set $\fin{\A}$ is closed under addition. Moreover, the set $\A[\beta]$ is closed under addition since there is no carry to the right when the rewriting rule $x-\beta$ is used. For any $n\in\NN$, the sum $1+1+\cdots +1=n$ is in $\A[\beta]$ by the assumption $1\in \A[\beta]$. Therefore, $\NN\subset \A[\beta]$ and thus the base $\beta$ is expanding by Theorem~\ref{thm:betaExpanding}.
\end{proof}
