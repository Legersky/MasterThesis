\documentclass{article}    

\usepackage[english]{babel}
\usepackage[utf8]{inputenc}
 
\usepackage{amsmath, amsthm,amssymb}
\newtheorem{theo}{Theorem}
\newtheorem{lem}[theo]{Lemma}


\usepackage{amsmath, amsthm, amssymb, units, dsfont}
\usepackage{sidecap}
\usepackage{enumerate}
\usepackage{xcolor}

\usepackage{mathtools, mathdots}
\usepackage{pgffor}
\usepackage{pdflscape}
\usepackage{afterpage}
\usepackage{chngcntr}
\usepackage{multirow}
\usepackage{tabulary}

\usepackage{listings}
\usepackage{color}

\usepackage{algorithm}
\usepackage{algorithmic}

\usepackage{breqn}
\usepackage{hyperref}

\usepackage{pgfkeys}

\usepackage{changepage}

\definecolor{darkgreen}{rgb}{0,0.6,0}
\definecolor{darkred}{rgb}{0.6,0,0}
\definecolor{darkblue}{rgb}{0,0,0.6}
\definecolor{darkgrey}{rgb}{0.3,0.3,0.3}
\definecolor{grey}{rgb}{0.6,0.6,0.6} %comment
\definecolor{lightgrey}{rgb}{0.92,0.92,0.92}
\definecolor{terminal}{rgb}{0.9,0.9,0.6}
\definecolor{cmd}{rgb}{0.8,0.8,0.98}
\lstset{ 
        basicstyle=\ttfamily,
%         language=Matlab,                                % choose the language of the code
%       basicstyle=10pt,                                % the size of the fonts that are used for the code
%         numbers=left,                                   % where to put the line-numbers
        keywordstyle=\color{darkblue},
        commentstyle=\color{darkgreen},
        stringstyle=\color{darkred},
%         numberstyle=\footnotesize,                      % the size of the fonts that are used for the line-numbers
        stepnumber=1,                                           % the step between two line-numbers. If it's 1 each line will be numbered
        numbersep=5pt,                                  % how far the line-numbers are from the code
%       backgroundcolor=\color{white},          % choose the background color. You must add \usepackage{color}
        showspaces=false,                               % show spaces adding particular underscores
        showstringspaces=false,                         % underline spaces within strings
        showtabs=false,                                         % show tabs within strings adding particular underscores
%       frame=single,                                           % adds a frame around the code
%       tabsize=2,                                              % sets default tabsize to 2 spaces
%       captionpos=b,                                           % sets the caption-position to bottom
        breaklines=true,                                        % sets automatic line breaking
        breakatwhitespace=false,                        % sets if automatic breaks should only happen at whitespace
        escapeinside={\%*}{*)},                          % if you want to add a comment within your code
        emph={%  
    False, True%
    },emphstyle={\color{darkblue}}
}




%\newcommand{\komentar}[1]{\textcolor{red}{\MakeUppercase{#1}} \newline}
%\newenvironment{upravit}{\color{blue}}{}

\newcommand{\Zomega}{\mathbb{Z}[\omega]}
\newcommand{\Zbeta}{\mathbb{Z}[\beta]}

\newcommand{\ZZ}{\mathbb{Z}}
\newcommand{\QQ}{\mathbb{Q}}
\newcommand{\CC}{\mathbb{C}}
\newcommand{\NN}{\mathbb{N}}
\newcommand{\RR}{\mathbb{R}}


\newcommand{\A}{\mathcal{A}}
\newcommand{\B}{\mathcal{B}}
\newcommand{\Q}{\mathcal{Q}}

\newcommand{\Qw}[3][w]{\Q_{[#1_{-#2}, \dots, #1_{-#3}]}}
\newcommand{\Qwo}[2][w]{\Q_{[#1_{0}, \dots, #1_{-#2}]}}

\newcommand{\tuple}[3][w]{(#1_{-#2}, \dots, #1_{-#3})}
\newcommand{\tupleo}[2][w]{(#1_{0}, \dots, #1_{-#2})}

%\newcommand{\Qb}[1]{\mathcal{Q}_{[b^{#1}]}}
\newcommand{\Qb}[1]{\mathcal{Q}_{[\scriptstyle b]}^{\scriptstyle #1}}

\newcommand{\fin}[1]{\text{Fin}_{#1}(\beta)}

\newcommand{\multMat}[1]{\sum_{i=0}^{d-1} {#1}_i S^i}



\newcommand{\vect}[1]{\begin{pmatrix}
             {#1}_0 \\
             {#1}_1 \\
             \vdots \\
             {#1}_{d-1} 
             \end{pmatrix}}
             
\newcommand{\enum}[1]{({#1}_0,\ldots,{#1}_{d-1})}             

\newcommand{\vertiii}[1]{{\left\vert\kern-0.25ex\left\vert\kern-0.25ex\left\vert #1\right\vert\kern-0.25ex\right\vert\kern-0.25ex\right\vert}}
    
\newcommand{\norm}[2]{\left\lVert#1\right\rVert_{#2}}
\newcommand{\Mnorm}[2]{\vertiii{#1}_{#2}}
\newcommand{\normBeta}[1]{\norm{#1}{\beta}}
\newcommand{\MnormBeta}[1]{\Mnorm{#1}{\beta}}

\renewcommand\Re{\operatorname{Re}}
\renewcommand\Im{\operatorname{Im}}


\renewcommand{\algorithmicrequire}{\textbf{Input:}}
\renewcommand{\algorithmicensure}{\textbf{Ouput:}}
\algsetup{indent=2em}

 \usepackage{pifont}
 \renewcommand\checkmark{\ding{51}}
 \newcommand\xmark{\ding{55}}

 \newcommand{\var}[1]{\textit{#1}}
 \newcommand{\fun}[2]{\textbf{#1}\allowbreak{}(\var{#2})}

 \def\changemargin#1#2{\list{}{\rightmargin#2\leftmargin#1}\item[]}
 \let\endchangemargin=\endlist 


 \newenvironment{method}[2]{
 \noindent \textbf{#1(}\textit{#2}\textbf{)}
 \vspace{-5pt}
 \begin{changemargin}{3em}{0em}}
 {\end{changemargin}}

\def\Cpp{{C\nolinebreak[4]\hspace{-.05em}\raisebox{.4ex}{\tiny\bf ++}}}


 \pgfkeys{
  /phaseOnecaptions array/.is family, /phaseOnecaptions array,
  .unknown/.style = {\pgfkeyscurrentname/.initial = #1},
 }
 
 \newcommand\figurehascaptionOne[1]{\pgfkeys{/phaseOnecaptions array, #1}}
 \newcommand\getcaptionOne[1]{\pgfkeysvalueof{/phaseOnecaptions array/#1}}
 
 \pgfkeys{
  /phase2captions array/.is family, /phase2captions array,
  .unknown/.style = {\pgfkeyscurrentname/.initial = #1},
 }
 
 \newcommand\figurehascaptionTwo[1]{\pgfkeys{/phase2captions array, #1}}
 \newcommand\getcaptionTwo[1]{\pgfkeysvalueof{/phase2captions array/#1}}


% \hyphenation{coef-fi-cient}
% \hyphenation{Algorithm-For-Parallel-Addition}
% \hyphenation{Polynomial-Quotient-Ring}

\hyphenation{con-ver-gen-ce}
\hyphenation{non-con-ver-gen-ce}





\begin{document}

% \begin{lem}
% \label{lem:NsubsetAbeta}
% Let $\A\subset \Zbeta$ be an alphabet such that $1\in \A[\beta]$. If the set $\A[\beta]$ is closed under addition, than $\NN\subset \A[\beta]$.
% \end{lem}
% \begin{proof}
% For any $n\in\NN$, the sum $1+1+\cdots +1=n$ is in $\A[\beta]$ by the assumptions.
% \end{proof}




\begin{theo}
\label{thm:betaExpanding}
    Let $\omega$ be a complex number and $\beta \in\Zomega$ be such that $|\beta|>1$. Let $\A\subset \Zomega$ be an alphabet. If $\NN\subset \A[\beta]$, number $\beta$ is expanding.
\end{theo}
\begin{proof}
For all $n\in\NN$ we may write 
    $$
    n=\sum_{i=0}^{N}a_i\beta^i\,,
    $$
    where $N\in\NN$ and $a_i\in\A$.
    
    Set $m:= \max\{|a|\colon a\in\A\}$. We take $n\in\NN$ such that $n>m$. 
    Since $|a_0|\leq m<n$, there is $i_0 \in \{1,2,\dots,N\}$ such that $a_{i_0}\neq 0$. Thus, $\omega$ is an algebraic number as $a_i\in\A\subset\Zomega$ and $\beta$ can be expressed as an integer combination of powers of $\omega$. Therefore, $\beta$ is also an algebraic integer.
    
    Let $\beta'$ be a conjugate of $\beta$.  
    Since $\beta\in\Zomega\subset\QQ(\omega)$, there is an embbeding $\sigma: \QQ(\omega)\rightarrow \CC$ such that $\sigma(\beta)=\beta'$. Now 
    $$
    n=\sigma(n)=\sum_{i=0}^{N}\sigma(a_i)\beta'^i\,.
    $$
    Set $m':= \max\{|\sigma(a)|\colon a\in\A\}$.  Assume that $|\beta'|<1$. For all $n\in\NN$, we have 
    $$
    n=|n|\leq\sum_{i=0}^{N}|\sigma(a_i)|\cdot|\beta'|^i \leq \sum_{i=0}^{\infty}|\sigma(a_i)|\cdot|\beta'|^i \leq m'\frac{1}{1-|\beta'|}\,,  
    $$
    which leads to contradiction. I.e., all conjugates of $\beta$ are at least one in modulus.
    
    If the degree of $\beta$ is one, the statement is obvious.  Therefore, we may assume that $\deg \beta \geq 2$. 
    
    Suppose  for contradiction that $|\beta'|=1$ for the conjugate $\beta'$  of $\beta$ such that $\beta\neq\beta'$. The complex conjugate $\overline{\beta'}$ is also conjugate of $\beta$. Take any conjugate $\gamma$ of $\beta$ and the embbeding $\sigma': \QQ(\beta')\rightarrow \CC$ given by $\sigma'(\beta')=\gamma$.
    Now
    $$
    \frac{1}{\gamma}=\frac{1}{\sigma'(\beta')}=\sigma'\left(\frac{1}{\beta'}\right)=\sigma'\left(\frac{\overline{\beta'}}{\beta'\overline{\beta'}}\right)=\sigma'\left(\frac{\overline{\beta'}}{|\beta'|^2}\right)=\sigma'(\overline{\beta'})\,.
    $$
    Hence, $\frac{1}{\gamma}$ is also conjugate of $\beta$. From the previous, $\left|\frac{1}{\gamma}\right|\geq 1$ and $|\gamma|\geq 1$ which implies that $|\gamma|=1$. We may set $\gamma=\beta$ which contradicts $|\beta|>1$. Thus all conjugates of $\beta$ are greater then one in modulus, i.e., $\beta$ is an expanding algebraic number.
\end{proof}



\begin{theo}
Let $\A\subset \Zbeta$ be an alphabet such that $1\in \A[\beta]$. If the extending window method with the rewriting rule $x-\beta$ converges for the numeration system $(\beta, \A)$, than the base $\beta$ is expanding and the alphabet $\A$ contains at least one representative of each congruence class modulo $\beta$ in $\Zbeta$. 
\end{theo}
\begin{proof}
The existence of an algorithm for addition which is computable in parallel implies that the set $\fin{\A}$ is closed under addition. Moreover, the set $\A[\beta]$ is closed under addition since there is no carry to the right when the rewriting rule $x-\beta$ is used. For any $n\in\NN$, the sum $1+1+\cdots +1=n$ is in $\A[\beta]$ by the assumption $1\in \A[\beta]$. I.e., $\NN\subset \A[\beta]$ and thus the base $\beta$ is expanding by Theorem~\ref{thm:betaExpanding}.

In order to prove the second part, we have to show that for every $x=\sum_{i=0}^N x_i \beta^i \in\Zbeta$ there exists $q\in\Zbeta$ and $a\in\A$ such that $x=a+\beta q$. We can suppose that $x_0> 0$ because $x=x+k\cdot p(\beta)$ where $p$ is minimal polynomial of $\beta$ and  $k\in\ZZ$ is such that $x_0+k\cdot p(0)>0$. Since $\NN\subset \A[\beta]$, we have
$$
x_0=\sum_{i=0}^{n}a_i\beta^i\,,
    $$
    where $n\in\NN$ and $a_i\in\A$. Now
    $$
    x=\underbrace{\sum_{i=0}^{n}a_i\beta^i}_{=x_0} + \sum_{i=1}^N x_i \beta^i=\underbrace{a_0}_{\in\A} + \beta \underbrace{\left(\sum_{i=0}^{n-1}a_{i+1}\beta^i + \sum_{i=0}^{N-1} x_{i+1} \beta^i\right)}_{=q\in\Zbeta}\,.
    $$
\end{proof}
\end {document}