Class \emph{WeightCoefficientsSetSearch} implements Phase 1 of the extending window method described in Section \ref{subsec:phase1} with different methods how an intermediate weight coefficients set $\Q_k$ is extended to $\Q_{k+1}$. Algorithm~\ref{alg:extendWeightCoefSet} explains methods 1a, 1b, 1c, 1d and 1e. There are also implemented other experimental methods denoted by numbers. Method 1a corresponds to 14, 1b to 12, 1c to 16, 1d to 13 and 1e to 15. 

The constructor of the class is 

\begin{method}{\_\_init\_\_}{algForParallelAdd, method}
The ring generator $\omega$, base $\beta$, an alphabet $\A$ and input alphabet $\B$ are initialize by values obtained from \var{algForParallelAdd}. The parameter \var{method} is a number of an experimental method or \verb+'1a'+, \verb+'1b'+, etc. The chosen method determines how an intermediate weight coefficients set $\Q_k$ is extended to $\Q_{k+1}$. If \var{None}, then the method 1d from Algorithm~\ref{alg:extendWeightCoefSet} is used as default.
\end{method}

Class methods implementing Phase 1 are the following:

\begin{method}{\_findCandidates}{to\_cover}
 The method returns the list of lists \var{candidates}, which corresponds to a covering set $C$ in Algorithm~\ref{alg:searchCand}, such that each element in \var{to\_cover} is covered by all values of the appropriate list in \var{candidates}.  
\end{method}


\begin{method}{\_chooseQk\_FromCandidates}{candidates}
The method takes the previous intermediate weight coefficients set $\Q_{k}$ and constructs a new intermediate weight coefficients set $\Q_{k+1}$ from \var{candidates} by Algorithm~\ref{alg:extendWeightCoefSet}.
\end{method}


\begin{method}{\_getQk}{to\_cover}
Methods \fun{\_chooseQk\_FromCandidates}{} and \fun{\_findCandidates}{to\_cover} are linked together to return an intermediate weight coefficients set $\Q_{k+1}$.
\end{method}

\begin{method}{findWeightCoefficientsSet}{maxIterations}
According to  Algorithm~\ref{alg:weightCoefSet}, a weight coefficients set $\Q$ is constructed by iterative using \fun{\_getQk}{}. A computation is aborted if the number of iterations exceeds \var{maxIterations}. 
\end{method}

