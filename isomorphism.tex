


\section{\texorpdfstring{Isomorphism of $\Zomega$ and $\ZZ^{d}$}{Isomorphism of Z[omega] and Zd}}
In this section, we recall that  the ring $\Zomega$ is isomorphic to the set $\ZZ^d$ equiped with multiplication, where $d$ is the degree of an algebraic integer $\omega$. This structure is useful as it allows to handle elements of $\Zomega$ as vectors. For example, division in $\Zomega$ can be replaced by searching for an integer solution of a linear system (Theorem \ref{thm:divisibility}) which is used in our implementation of the extending window method.


For defining multiplication in $\ZZ^d$, we recall the notion of companion matrix. 
\begin{defn}
Let  $p(x)=x^d +p_{d-1}x^{d-1}+ \cdots + p_1 x+p_0 \in \ZZ[x]$ be a monic polynomial with integer coefficients, $d\geq 1$. The matrix 
$$
S := \begin{pmatrix}
            0 & 0 & \cdots & 0 & -p_0 \\
            1 & 0 & \cdots & 0 & -p_1 \\
            0 & 1 & \cdots & 0 & -p_2 \\
            \vdots &   & \ddots & & \vdots \\
            0 & 0 & \cdots & 1 & -p_{d-1} 
            \end{pmatrix} \in \ZZ^{d\times d}
$$
is called \emph{companion matrix} of the polynomial $p$.
\end{defn}
Let $S$ be the companion matrix of a polynomial $p$. It is well known (see for instance \cite{horn}) that the characteristic polynomial of the companion matrix $S$ is $p$. The matrix $S$ is also root of the polynomial $p$.


%In what follows, the standard basis vectors of $\ZZ^d$  are denoted by 
%$$
%e_0=\begin{pmatrix}
%              1 \\
%              0 \\
%              0 \\
%              \vdots \\
%              0
%              \end{pmatrix}, \\
%e_1=\begin{pmatrix}
%              0 \\
%              1 \\
%              0 \\
%              \vdots \\
%              0
%              \end{pmatrix}, \dots ,\\
%e_{d-1}=\begin{pmatrix}
%              0 \\        
%              \vdots \\
%              0 \\
%              0\\
%              1
%              \end{pmatrix}\,.             
%$$
% We remark that 1 in $e_i$ is in the $(i+1)$-st row because the index corresponds to the power of a companion matrix in the following definition. 
We remark that the minimal polynomial of an algebraic integer $\omega$ is denoted by $m_\omega$ and is always meant to be monic.  
Multiplication in $\ZZ^d$ is defined in the following way.
\begin{defn}
Let $\omega$ be an algebraic integer of degree $d\geq 1$ and let $S$ be the companion matrix of $m_\omega$. We define a mapping $\odot_\omega: \ZZ^d \times \ZZ^d \rightarrow \ZZ^d$ by 
$$
u \odot_\omega v := \left(\multMat{u}\right)\cdot \vect{v} \quad \text{ for all } u=\vect{u}, v=\vect{v} \in \ZZ^d\,.
$$ 
%and we define powers of $u \in \ZZ^d$ by
%\begin{align*}
%    u^0&=e_0, \\
%    u^{i}&= u^{i-1} \odot_\omega u \text{ for } i\in\NN\,.
%\end{align*}
\end{defn}

The technical proof of Theorem~\ref{thm:isomorphismWithZd}, which shows that $\ZZ^d$ with multiplication $\odot_\omega$ is a ring isomorphic to $\Zomega$, can be found in \cite{vu}.

\begin{thm}
\label{thm:isomorphismWithZd}
Let  $\omega$ be an algebraic integer of degree $d$. Then 
$$
\Zomega =\left\{\sum_{i=0}^{d-1} a_i \omega^i \colon a_i\in\ZZ \right\},
$$ 
$(\ZZ^d,+,\odot_\omega)$ is a commutative ring and the mapping $\pi:\Zomega \rightarrow \ZZ^{d}$ defined by 
$$
\pi(u)=\vect{u} \quad \text{ for every } u=\sum_{i=0}^{d-1} u_i \omega^i \in \Zomega
$$
is a ring isomorphism.
\end{thm}

This theorem provides simple representation of elements of $\Zomega$ in computer -- they are represented by integer vectors and multiplication in $\Zomega$ is replaced by multiplying by an appropriate matrix. 

Divisibility in $\Zomega$ can be also transformed into $\ZZ^d$. To check whether an element of $\Zomega$ is divisible by another element, we look for an integer solution of a linear system given by Theorem~\ref{thm:divisibility}. Moreover, this solution provides the result of  division in the positive case. 
\begin{thm}
\label{thm:divisibility}
Let $\omega$ be an algebraic integer of degree $d$ and let $S$ be the companion matrix of its minimal polynomial. Let $\beta=\sum_{i=0}^{d-1} b_i \omega^i$ be a nonzero element of $\Zomega$. Then for every $u\in\Zomega$
$$
u\in\beta\Zomega \iff S_\beta^{-1}\cdot \pi(u) \in \ZZ^d\,,
$$
where $S_\beta=\multMat{b}$.
\end{thm}
The proof can be found in \cite{vu}.
