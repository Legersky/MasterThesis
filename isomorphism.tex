\begin{upravit}


\section{\texorpdfstring{Isomorphism of $\Zomega$ and $\ZZ^{d}$}{Isomorphism of Z[omega] and Zd}}
The goal of this section is to show a connection between the ring $\Zomega$ and the set $\ZZ^d$. Using Theorem \ref{thm:divisibility}, division in $\Zomega$ can be replaced by searching for an integer solution of a linear system. This is used for the implementation of the extending window method.

First we recall the notion of companion matrix which we use to define multiplication in $\ZZ^d$. By the minimal polynomial of an algebraic integer, we always mean the monic minimal polynomial.  
\begin{defn}
Let $\omega$ be an algebraic integer of degree $d\geq 1$ with the  minimal polynomial $p(x)=x^d +p_{d-1}x^{d-1}+ \cdots + p_1 x+p_0 \in \ZZ[x]$. The matrix 
$$
S := \begin{pmatrix}
            0 & 0 & \cdots & 0 & -p_0 \\
            1 & 0 & \cdots & 0 & -p_1 \\
            0 & 1 & \cdots & 0 & -p_2 \\
            \vdots &   & \ddots & & \vdots \\
            0 & 0 & \cdots & 1 & -p_{d-1} 
            \end{pmatrix} \in \ZZ^{d\times d}
$$
is called \emph{companion matrix} of the minimal polynomial of $\omega$.
\end{defn}
\komentar{zminit jak vypada charakteristicky polynom companion matrix}
In what follows, the standard basis vectors of $\ZZ^d$  are denoted by 
$$
e_0=\begin{pmatrix}
              1 \\
              0 \\
              0 \\
              \vdots \\
              0
              \end{pmatrix}, \\
e_1=\begin{pmatrix}
              0 \\
              1 \\
              0 \\
              \vdots \\
              0
              \end{pmatrix}, \dots ,\\
e_{d-1}=\begin{pmatrix}
              0 \\        
              \vdots \\
              0 \\
              0\\
              1
              \end{pmatrix}\,.             
$$
% We remark that 1 in $e_i$ is in the $(i+1)$-st row because the index corresponds to the power of a companion matrix in the following definition. 

\begin{defn}
Let $\omega$ be an algebraic integer of degree $d\geq 1$, let $p$ be its minimal polynomial and let $S$ be its companion matrix. We define the mapping $\odot_\omega: \ZZ^d \times \ZZ^d \rightarrow \ZZ^d$ by 
$$
u \odot_\omega v := \left(\multMat{u}\right)\cdot \vect{v} \quad \text{ for all } u=\vect{u}, v=\vect{v} \in \ZZ^d\,.
$$ 
and we define powers of $u \in \ZZ^d$ by
\begin{align*}
    u^0&=e_0, \\
    u^{i}&= u^{i-1} \odot_\omega u \text{ for } i\in\NN\,.
\end{align*}
\end{defn}

We will see later that $\ZZ^d$ equipped with elementwise addition and multiplication $\odot_\omega$ builds a commutative ring. 
% It will follow from the isomorphism with $\Zomega$. 
Let us first recall an important property of a companion matrix  -- it is a root of its defining polynomial.
\begin{lem}
\label{lem:compMatrixIsRoot}
Let $\omega$ be an algebraic integer with a minimal polynomial $p$ and let $S$ be its companion matrix. Then
$$
p(S)=0\,.
$$
\end{lem}






Now we can prove that there is a correspondence between elements of $\Zomega$ and $\ZZ^d$.

\begin{thm}
Let  $\omega$ be an algebraic integer of degree $d$. Then 
$$
\Zomega =\left\{\sum_{i=0}^{d-1} a_i \omega^i \colon a_i\in\ZZ \right\},
$$ 
$(\ZZ^d,+,\odot_\omega)$ is a commutative ring and the mapping $\pi:\Zomega \rightarrow \ZZ^{d}$ defined by 
$$
\pi(u)=\vect{u} \quad \text{ for every } u=\sum_{i=0}^{d-1} u_i \omega^i \in \Zomega
$$
is a ring isomorphism.
\end{thm}


Due to this theorem we may work with integer vectors instead of elements of $\Zomega$ and multiplication in $\Zomega$ is replaced by multiplying by an appropriate matrix. 

The last theorem of this section is a practical tool for divisibility in $\Zomega$. To check whether an element of $\Zomega$ is divisible by another element, we look for an integer solution of a linear system. Moreover, this solution provides the result of  division in the positive case. 
\begin{thm}
\label{thm:divisibility}
Let $\omega$ be an algebraic integer of degree $d$ and let $S$ be the companion matrix of its minimal polynomial. Let $\beta=\sum_{i=0}^{d-1} b_i \omega^i$ be a nonzero element of $\Zomega$. Then for every $u\in\Zomega$
$$
u\in\beta\Zomega \iff S_\beta^{-1}\cdot \pi(u) \in \ZZ^d\,,
$$
where $S_\beta=\multMat{b}$.
\end{thm}


\end{upravit}