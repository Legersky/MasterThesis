\documentclass[a4paper, 11pt]{report}

\usepackage{a4wide,cite}
\usepackage[english]{babel}
\usepackage[utf8]{inputenc}
% \usepackage[IL2]{fontenc}  
% \usepackage{amsmath,amsthm} 
% \usepackage{amsfonts}


\usepackage[fixlanguage]{babelbib}
\selectbiblanguage{english}



\usepackage{amsmath, amsthm, amssymb, units, dsfont}
\usepackage{sidecap}
\usepackage{enumerate}
\usepackage{xcolor}

\usepackage{mathtools, mathdots}
\usepackage{pgffor}
\usepackage{pdflscape}
\usepackage{afterpage}
\usepackage{chngcntr}
\usepackage{multirow}
\usepackage{tabulary}

\usepackage{listings}
\usepackage{color}

\usepackage{algorithm}
\usepackage{algorithmic}

\usepackage{breqn}
\usepackage{hyperref}

\usepackage{pgfkeys}

\usepackage{changepage}

\definecolor{darkgreen}{rgb}{0,0.6,0}
\definecolor{darkred}{rgb}{0.6,0,0}
\definecolor{darkblue}{rgb}{0,0,0.6}
\definecolor{darkgrey}{rgb}{0.3,0.3,0.3}
\definecolor{grey}{rgb}{0.6,0.6,0.6} %comment
\definecolor{lightgrey}{rgb}{0.92,0.92,0.92}
\definecolor{terminal}{rgb}{0.9,0.9,0.6}
\definecolor{cmd}{rgb}{0.8,0.8,0.98}
\lstset{ 
        basicstyle=\ttfamily,
%         language=Matlab,                                % choose the language of the code
%       basicstyle=10pt,                                % the size of the fonts that are used for the code
%         numbers=left,                                   % where to put the line-numbers
        keywordstyle=\color{darkblue},
        commentstyle=\color{darkgreen},
        stringstyle=\color{darkred},
%         numberstyle=\footnotesize,                      % the size of the fonts that are used for the line-numbers
        stepnumber=1,                                           % the step between two line-numbers. If it's 1 each line will be numbered
        numbersep=5pt,                                  % how far the line-numbers are from the code
%       backgroundcolor=\color{white},          % choose the background color. You must add \usepackage{color}
        showspaces=false,                               % show spaces adding particular underscores
        showstringspaces=false,                         % underline spaces within strings
        showtabs=false,                                         % show tabs within strings adding particular underscores
%       frame=single,                                           % adds a frame around the code
%       tabsize=2,                                              % sets default tabsize to 2 spaces
%       captionpos=b,                                           % sets the caption-position to bottom
        breaklines=true,                                        % sets automatic line breaking
        breakatwhitespace=false,                        % sets if automatic breaks should only happen at whitespace
        escapeinside={\%*}{*)},                          % if you want to add a comment within your code
        emph={%  
    False, True%
    },emphstyle={\color{darkblue}}
}




%\newcommand{\komentar}[1]{\textcolor{red}{\MakeUppercase{#1}} \newline}
%\newenvironment{upravit}{\color{blue}}{}

\newcommand{\Zomega}{\mathbb{Z}[\omega]}
\newcommand{\Zbeta}{\mathbb{Z}[\beta]}

\newcommand{\ZZ}{\mathbb{Z}}
\newcommand{\QQ}{\mathbb{Q}}
\newcommand{\CC}{\mathbb{C}}
\newcommand{\NN}{\mathbb{N}}
\newcommand{\RR}{\mathbb{R}}


\newcommand{\A}{\mathcal{A}}
\newcommand{\B}{\mathcal{B}}
\newcommand{\Q}{\mathcal{Q}}

\newcommand{\Qw}[3][w]{\Q_{[#1_{-#2}, \dots, #1_{-#3}]}}
\newcommand{\Qwo}[2][w]{\Q_{[#1_{0}, \dots, #1_{-#2}]}}

\newcommand{\tuple}[3][w]{(#1_{-#2}, \dots, #1_{-#3})}
\newcommand{\tupleo}[2][w]{(#1_{0}, \dots, #1_{-#2})}

%\newcommand{\Qb}[1]{\mathcal{Q}_{[b^{#1}]}}
\newcommand{\Qb}[1]{\mathcal{Q}_{[\scriptstyle b]}^{\scriptstyle #1}}

\newcommand{\fin}[1]{\text{Fin}_{#1}(\beta)}

\newcommand{\multMat}[1]{\sum_{i=0}^{d-1} {#1}_i S^i}



\newcommand{\vect}[1]{\begin{pmatrix}
             {#1}_0 \\
             {#1}_1 \\
             \vdots \\
             {#1}_{d-1} 
             \end{pmatrix}}
             
\newcommand{\enum}[1]{({#1}_0,\ldots,{#1}_{d-1})}             

\newcommand{\vertiii}[1]{{\left\vert\kern-0.25ex\left\vert\kern-0.25ex\left\vert #1\right\vert\kern-0.25ex\right\vert\kern-0.25ex\right\vert}}
    
\newcommand{\norm}[2]{\left\lVert#1\right\rVert_{#2}}
\newcommand{\Mnorm}[2]{\vertiii{#1}_{#2}}
\newcommand{\normBeta}[1]{\norm{#1}{\beta}}
\newcommand{\MnormBeta}[1]{\Mnorm{#1}{\beta}}

\renewcommand\Re{\operatorname{Re}}
\renewcommand\Im{\operatorname{Im}}


\renewcommand{\algorithmicrequire}{\textbf{Input:}}
\renewcommand{\algorithmicensure}{\textbf{Ouput:}}
\algsetup{indent=2em}

 \usepackage{pifont}
 \renewcommand\checkmark{\ding{51}}
 \newcommand\xmark{\ding{55}}

 \newcommand{\var}[1]{\textit{#1}}
 \newcommand{\fun}[2]{\textbf{#1}\allowbreak{}(\var{#2})}

 \def\changemargin#1#2{\list{}{\rightmargin#2\leftmargin#1}\item[]}
 \let\endchangemargin=\endlist 


 \newenvironment{method}[2]{
 \noindent \textbf{#1(}\textit{#2}\textbf{)}
 \vspace{-5pt}
 \begin{changemargin}{3em}{0em}}
 {\end{changemargin}}

\def\Cpp{{C\nolinebreak[4]\hspace{-.05em}\raisebox{.4ex}{\tiny\bf ++}}}


 \pgfkeys{
  /phaseOnecaptions array/.is family, /phaseOnecaptions array,
  .unknown/.style = {\pgfkeyscurrentname/.initial = #1},
 }
 
 \newcommand\figurehascaptionOne[1]{\pgfkeys{/phaseOnecaptions array, #1}}
 \newcommand\getcaptionOne[1]{\pgfkeysvalueof{/phaseOnecaptions array/#1}}
 
 \pgfkeys{
  /phase2captions array/.is family, /phase2captions array,
  .unknown/.style = {\pgfkeyscurrentname/.initial = #1},
 }
 
 \newcommand\figurehascaptionTwo[1]{\pgfkeys{/phase2captions array, #1}}
 \newcommand\getcaptionTwo[1]{\pgfkeysvalueof{/phase2captions array/#1}}


% \hyphenation{coef-fi-cient}
% \hyphenation{Algorithm-For-Parallel-Addition}
% \hyphenation{Polynomial-Quotient-Ring}

\hyphenation{con-ver-gen-ce}
\hyphenation{non-con-ver-gen-ce}



    
\newtheorem{thm}{Theorem}[chapter]
\newtheorem{lem}[thm]{Lemma}

\theoremstyle{definition}
\newtheorem{defn}{Definition}[chapter]
\newtheorem{exmp}{Example}[chapter]

\begin{document}

\chapter*{List of symbols}
% \centering
\begin{tabular}{ll}
Symbol        & Description \\ \hline
$\NN$         & set of nonnegative integers $\{0,1,2,3,\dots\}$   \\
$\ZZ$         & set of integers $\{\dots,-2,-1,0,1,2,\dots\}$ \\
$\RR$           & set of real numbers \\
$\CC$           & set of complex numbers \\
$\QQ$           &set of rational numbers \\
$\QQ(\beta)$    &the smallest field containing  $\QQ$ and algebraic number  $\beta$ \\
$\#S$          & number of elements of the finite set $S$ \\
$C^*$            & complex conjugation and transposition of the complex matrix $C$ \\
\rule{0cm}{0cm}& \\
$m_\beta$       &monic minimal polynomial of the algebraic number $\beta$ \\
$\deg \beta$    &degree of the algebraic number $\beta$ (over $\QQ$)\\
\rule{0cm}{0cm}& \\
$(\beta,\A)$            & numeration system with the base $\beta$ and the alphabet $\A$\\
$(x)_{\beta,\A}$    &$(\beta,\A)$-representation of the number $x$\\
$\fin{\A}$          &set of all complex numbers with a finite $(\beta,\A)$-representation \\
$\A^\ZZ$        &set of all bi-infinite sequences of digits in $\A$\\
$\Zomega$       &the smallest ring containing $\ZZ$ and $\omega$ \\%&set of values of all polynomials with integer coefficients evaluated in $\omega$\\
$\pi$           &isomorphism from $\Zomega$ to $\ZZ^d$ ($d=\deg \omega$)\\
\rule{0cm}{0cm}& \\
$\B$            &alphabet of input digits\\
$q_j$           &weight coefficient for the $j$-th position \\
$\Q$            &weight coefficients set\\
$\Qwo{k}$ 		&set of possible weight coefficients for digits $\tupleo{k}\in\B^{k+1}$ \\
%\rule{0cm}{0cm}& \\
%$\lfloor x \rfloor$ & floor function of the number $x$ \\  
%$\Re x$           & real part of the complex number $x$ \\
%$\Im x$           & imaginary part of the complex number $x$
\end{tabular}

%\chapter*{Introduction}

\chapter{Preliminaries}
\komentar{vlozeno z vyzkumaku, potrEBA UPRAVIT}
In this chapter, we recall few definitions and results connected to numeration systems and parallelism. We define the set $\Zomega$ for an algebraic integer $\omega$ and we prove that $\Zomega$ is isomorphic to $\ZZ^d$. This property is used in Theorem \ref{thm:divisibility} which is an important tool for divisibility in $\Zomega$. Division in $\Zomega$ is necessary for the extending window method described in Chapter \ref{chap:methodDescription}.

\section{Numeration systems}
Firstly, we give a general definition of numeration system.
\begin{defn}
  Let $\beta \in \CC, |\beta|>1$ and $\A \subset \CC$ be a finite set containing 0. A pair $(\beta, \A)$ is called a \emph{positional numeration system} with \emph{base} $\beta$ and \emph{digit set} $\A$, usually called \emph{alphabet}.
\end{defn}
So-called standard numeration systems have an integer base $\beta$ and an alphabet $\A$ which is a set of contiguous integers. We restrict ourselves to the base $\beta$ which is an algebraic integer and possibly non-integer alphabet $\A$. 
\begin{defn}
Let $(\beta, \A)$ be a positional numeration system.  We say that a complex number $x$ has a \emph{$(\beta, \A)$-representation} if~ there exist digits $x_n,x_{n-1}, x_{n-2},\dots \in\A, n\geq 0$ such that $x=\sum_{j=-\infty}^n x_j \beta^j$.
\end{defn}
 We write briefly a \emph{representation} instead of a $(\beta, \A)$-representation if the base $\beta$ and the alphabet $\A$ follow from context. 

\begin{defn}
Let $(\beta, \A)$ be a positional numeration system. The set of all complex numbers with a finite $(\beta, \A)$-representation is defined by
$$
    \fin{\A}:=\left\{\sum_{j=-m}^n x_j \beta^j\colon n, m \in \NN, x_j \in \A \right\}\,.
$$
\end{defn}
   
For  $x\in\fin{\A}$, we write 
$$
(x)_{\beta,\A}= 0^\omega x_n x_{n-1}\cdots x_1 x_0 \bullet x_{-1} x_{-2} \cdots x_{-m} 0^\omega\,,
$$ 
where $0^\omega$ denotes right, respectively left-infinite sequence of zeros. Notice that indices are decreasing from left to right as it is usual to write the most significant digits first. In what follows, we omit the starting and ending $0^\omega$ when we work with numbers in $\fin{\A}$. We remark that existence of an algorithm (standard or parallel) producing a finite $(\beta,\A)$-representation of $x+y$ where $x,y\in\fin{\A}$ implies that the set $\fin{\A}$ is closed under addition, i.e.,
$$
\fin{\A} + \fin{\A} \subset \fin{\A}\,.
$$ 

Designing an algorithm for parallel addition requires some redundancy in numeration system. According to \cite{redundant}, a numeration system $(\beta,\A)$ is called \emph{redundant} if there exists $x \in \fin{\A}$ which has two different $(\beta,\A)$-representations. For instance, the number 1 has $(2,\{-1,0,1\})$-representations $1\bullet$ and $1(-1)\bullet$.
Redundant numeration system can enable us to avoid carry propagation in addition. On the other hand, there are some disadvantages. For example, comparison is problematic.  


\section{Parallel addition}
A local function, which is also often called a sliding block code, is used to mathematically formalize parallelism. 
\begin{defn}
Let $\A$ and $\B$ be alphabets. A function $\varphi:\B^\ZZ \rightarrow \A^\ZZ$ is said to be \emph{$p$-local} if there exist $r,t\in\NN$ satisfying $p=r+t+1$ and a function $\phi: \B^p \rightarrow \A$ such that, for any $w=(w_j)_{j\in\ZZ}\in\B^\ZZ$ and its image $z=\varphi(w)=(z_j)_{j\in\ZZ}\in\A^\ZZ$, we have $z_j=\phi(w_{j+t},\cdots,w_{j-r})$ for every $j\in\ZZ$. The parameter $t$, resp. $r$, is called \emph{anticipation}, resp. \emph{memory}.
\end{defn}
This means that each digit of the image $\varphi(w)$ is computed from $p$ digits of $w$ in a sliding window. Suppose that there is a processor on  each position with access to $t$ input digits on the left and $r$ input digits on the right. Then computation of $\varphi(w)$, where $w$ is a finite sequence, can be done in constant time independent on the length of $w$.   
  
\begin{defn}
\label{def:digitSetConversion}
Let $\beta$ be a base and $\A$ and $\B$ two alphabets containing 0. A function $\varphi:\B^\ZZ\rightarrow \A^\ZZ$ such that
  \begin{enumerate}
      \item for any $w=(w_j)_{j\in\ZZ}\in\B^\ZZ$ with finitely many non-zero digits, $z=\varphi(w)=(z_j)_{j\in\ZZ}\in\A^\ZZ$ has only finite number of non-zero digits, and
      \item $\sum_{j\in\ZZ} w_j \beta^j= \sum_{j\in\ZZ} z_j \beta^j$
  \end{enumerate}
  is called \emph{digit set conversion} in base $\beta$ from $\B$ to $\A$. Such a conversion $\varphi$ is said to be \emph{computable in parallel} if $\varphi$ is a $p$-local function for some $p\in\NN$. 
\end{defn}
In fact, addition on $\fin{\A}$ can be performed in parallel if there is a digit set conversion from $\A+\A$ to $\A$ computable in  parallel as we can easily output digitwise sum of two $(\beta,\A)$-representations in parallel.   


We recall few results about parallel addition in a numeration system with an integer alphabet. C. Frougny, E. Pelantov\'a and M. Svobodov\'a proved  the following sufficient condition of existence of an algorithm for parallel addition in \cite{parAddNS}.
  \begin{theo}
  \label{thm:suffConjugates}
  Let $\beta\in\CC$ be an algebraic number such that $|\beta|>1$ and all its conjugates in modulus differ from 1. There exists an alphabet $\A$ of contiguous integers containing 0 such that addition on $\fin{\A}$ can be performed in parallel.
  \end{theo}
  The proof of the theorem provides the algorithm for the alphabet of the form $\{-a,-a+1, \dots,0,\dots,a-1,a\}$. But in general, $a$ is not minimal.
    
The same authors showed in \cite{kBlock} that the condition on the conjugates of the base $\beta$ is also necessary:
  \begin{theo}
  Let the base $\beta\in\CC, |\beta|>1,$ be an algebraic number with a conjugate $\beta'$ such that $|\beta'|=1$. Let $\A\subset\ZZ$ be an alphabet of contiguous integers containing 0. Then addition on $\fin{\A}$ cannot be computable in parallel.
  \end{theo}
  
The question of minimality of the alphabet is studied in \cite{minAlph}. The following lower bound for the size of the alphabet is provided:
  \begin{theo}
  \label{thm:lowerBoundAlphabet}
  Let $\beta\in\CC, |\beta|>1,$  be an algebraic integer with the minimal polynomial $p$. Let $\A\subset\ZZ$ be an alphabet of contiguous integers containing 0 and 1. If addition on $\fin{\A}$ is computable in parallel, then $\#\A \geq |p(1)|$. Moreover, if $\beta$ is a positive real number, $\beta>1$, then $\#\A \geq  |p(1)|+2$.
  \end{theo}
  

In this thesis, we work in a more general concept as we consider also non-integer alphabets. First, we recall the following definition.
\begin{defn}
Let $\omega$ be a complex number. The set of values of all polynomials with integer coefficients evaluated in $\omega$ is denoted by
$$
    \ZZ[\omega] =\left\{\sum_{i=0}^n a_i \omega^i\colon n\in\NN, a_i\in\ZZ \right\} \subset \QQ(\omega)\,.
$$
\end{defn}
 Notice that $\ZZ[\omega]$ is a commutative ring (for our purposes, a ring is associative under multiplication and there is a multiplicative identity).     
    
From now on, let $\omega$ be an algebraic integer  which generates the set $\Zomega$ and let the base $\beta\in\Zomega$ be such that $|\beta|>1$. We remark that $\beta$ is also an algebraic integer as all elements of $\Zomega$ are algebraic integers. Finally, let the alphabet $\A$ be a finite subset of $\Zomega$ such that $0\in\A$.

Few parallel addition algorithms for such numeration system with a non-integer alphabet were found ad hoc. We introduce the method for construction of the parallel addition algorithm for a given numeration system $(\beta,\A)$ in Chapter \ref{chap:methodDescription}. 
  


\section{\texorpdfstring{Isomorphism of $\Zomega$ and $\ZZ^{d}$}{Isomorphism of Z[omega] and Zd}}
The goal of this section is to show a connection between the ring $\Zomega$ and the set $\ZZ^d$. Using Theorem \ref{thm:divisibility}, division in $\Zomega$ can be replaced by searching for an integer solution of a linear system. This is used for the implementation of the extending window method.

First we recall the notion of companion matrix which we use to define multiplication in $\ZZ^d$. By the minimal polynomial of an algebraic integer, we always mean the monic minimal polynomial.  
\begin{defn}
Let $\omega$ be an algebraic integer of degree $d\geq 1$ with the  minimal polynomial $p(x)=x^d +p_{d-1}x^{d-1}+ \cdots + p_1 x+p_0 \in \ZZ[x]$. The matrix 
$$
S := \begin{pmatrix}
            0 & 0 & \cdots & 0 & -p_0 \\
            1 & 0 & \cdots & 0 & -p_1 \\
            0 & 1 & \cdots & 0 & -p_2 \\
            \vdots &   & \ddots & & \vdots \\
            0 & 0 & \cdots & 1 & -p_{d-1} 
            \end{pmatrix} \in \ZZ^{d\times d}
$$
is called \emph{companion matrix} of the minimal polynomial of $\omega$.
\end{defn}
In what follows, the standard basis vectors of $\ZZ^d$  are denoted by 
$$
e_0=\begin{pmatrix}
              1 \\
              0 \\
              0 \\
              \vdots \\
              0
              \end{pmatrix}, \\
e_1=\begin{pmatrix}
              0 \\
              1 \\
              0 \\
              \vdots \\
              0
              \end{pmatrix}, \dots ,\\
e_{d-1}=\begin{pmatrix}
              0 \\        
              \vdots \\
              0 \\
              0\\
              1
              \end{pmatrix}\,.             
$$
% We remark that 1 in $e_i$ is in the $(i+1)$-st row because the index corresponds to the power of a companion matrix in the following definition. 

\begin{defn}
Let $\omega$ be an algebraic integer of degree $d\geq 1$, let $p$ be its minimal polynomial and let $S$ be its companion matrix. We define the mapping $\odot_\omega: \ZZ^d \times \ZZ^d \rightarrow \ZZ^d$ by 
$$
u \odot_\omega v := \left(\multMat{u}\right)\cdot \vect{v} \quad \text{ for all } u=\vect{u}, v=\vect{v} \in \ZZ^d\,.
$$ 
and we define powers of $u \in \ZZ^d$ by
\begin{align*}
    u^0&=e_0, \\
    u^{i}&= u^{i-1} \odot_\omega u \text{ for } i\in\NN\,.
\end{align*}
\end{defn}

We will see later that $\ZZ^d$ equipped with elementwise addition and multiplication $\odot_\omega$ builds a commutative ring. 
% It will follow from the isomorphism with $\Zomega$. 
Let us first recall an important property of a companion matrix  -- it is a root of its defining polynomial.
\begin{lem}
\label{lem:compMatrixIsRoot}
Let $\omega$ be an algebraic integer with a minimal polynomial $p$ and let $S$ be its companion matrix. Then
$$
p(S)=0\,.
$$
\end{lem}






Now we can prove that there is a correspondence between elements of $\Zomega$ and $\ZZ^d$.

\begin{theo}
Let  $\omega$ be an algebraic integer of degree $d$. Then 
$$
\Zomega =\left\{\sum_{i=0}^{d-1} a_i \omega^i \colon a_i\in\ZZ \right\},
$$ 
$(\ZZ^d,+,\odot_\omega)$ is a commutative ring and the mapping $\pi:\Zomega \rightarrow \ZZ^{d}$ defined by 
$$
\pi(u)=\vect{u} \quad \text{ for every } u=\sum_{i=0}^{d-1} u_i \omega^i \in \Zomega
$$
is a ring isomorphism.
\end{theo}


Due to this theorem we may work with integer vectors instead of elements of $\Zomega$ and multiplication in $\Zomega$ is replaced by multiplying by an appropriate matrix. 

The last theorem of this section is a practical tool for divisibility in $\Zomega$. To check whether an element of $\Zomega$ is divisible by another element, we look for an integer solution of a linear system. Moreover, this solution provides the result of  division in the positive case. 
\begin{theo}
\label{thm:divisibility}
Let $\omega$ be an algebraic integer of degree $d$ and let $S$ be the companion matrix of its minimal polynomial. Let $\beta=\sum_{i=0}^{d-1} b_i \omega^i$ be a nonzero element of $\Zomega$. Then for every $u\in\Zomega$
$$
u\in\beta\Zomega \iff S_\beta^{-1}\cdot \pi(u) \in \ZZ^d\,,
$$
where $S_\beta=\multMat{b}$.
\end{theo}
























  
   
\komentar{konec vlozeni z vyzkumaku}



\chapter{Design of extending window method}
\komentar{vlozeno z vyzkumaku, potrEBA UPRAVIT}

We recall the general concept of addition at the beginning of this chapter and then we describe a so-called \emph{extending window method} which is due to M. Svobodov\'a \cite{milena}. The method attempts to construct a conversion algorithm computable in parallel for a given numeration system $(\beta,\A)$. We recall that $\omega$ is an algebraic integer, $\beta \in \Zomega$ is a base and $0\in\A\subset\Zomega$ is an alphabet. 

\section{Addition}

The general idea of addition (standard or parallel) in any numeration system $(\beta,\A)$ is the following: we sum up two numbers digit-wise and then we convert the result with digits in $\A+\A$ into the alphabet $\A$. Obviously, digit-wise addition is computable in parallel, thus the problematic part is the digit set conversion of the obtained result. It can be easily done in a standard way but a parallel digit set conversion is non-trivial. A parallel conversion is based on the same  formulas as the standard one but the choice of so-called \emph{weight coefficients} differs.

Now, we go step by step more precisely. Let $x,y \in \fin{\A}$ with $(\beta,\A)$-representations $x_{n'}x_{{n'}-1}\cdots x_1 x_0\bullet x_{-1} x_{-2} \cdots x_{-m'}$ and $y_{n'}y_{{n'}-1}\cdots y_1 y_0\bullet y_{-1} y_{-2} \cdots y_{-m'}$ padded by zeros to have the same length $n'+m'+1$. We set 
  \begin{align*}
    w&=x+y =\sum_{i=-m'}^{n'} x_i\beta^i + \sum_{i=-m'}^{n'} y_i\beta^i = \sum_{i=-m'}^{n'} (x_i+y_i)\beta^i \\
    &=\sum_{i=-m'}^{n'} w_i\beta^i \,,
  \end{align*}
  where $w_i=x_i+y_i \in \A +\A$. Thus, $w_{n'} w_{{n'}-1}\cdots w_1 w_0 \bullet w_{-1} w_{-2} \cdots w_{-m'}$ is a  $(\beta, \A+\A)$-representation of $w\in \fin{\A+\A}$. 

We also use column notation for digit-wise addition in what follows, e.g.,    
%	\begin{center}
%	\begin{tabulary}{0.7\textwidth}{CCCCCCCCCCCL}
%	$x_{n'}$& $x_{{n'}-1}$ & $\cdots$ & $x_1$ &$x_0$ &$\bullet$  &$x_{-1}$ & $x_{-2}$ & $\cdots$ & $x_{-m'}$ & $=$& $(x)_{\beta,\A}$\\
%	$y_{n'}$& $y_{{n'}-1}$ & $\cdots$ & $y_1$ &$y_0$ &$\bullet$  &$y_{-1}$ & $y_{-2}$ & $\cdots$ & $y_{-m'}$ & $=$& $(y)_{\beta,\A}$\\ \hline
%	$w_{n'}$& $w_{{n'}-1}$ & $\cdots$ & $w_1$ &$w_0$ &$\bullet$  &$w_{-1}$ & $w_{-2}$ & $\cdots$ & $w_{-m'}$ & $=$& $(x+y)_{\beta,\A+\A}$\\\,.
%	\end{tabulary}	
%	\end{center}
 
  \begin{align*}
  x_{n'} \;x_{{n'}-1}\cdots x_1 \;x_0 &\bullet x_{-1} \;x_{-2}\, \cdots x_{-m'} \\[-3pt]
  y_{n'} \;y_{{n'}-1}\cdots y_1 \,\;y_0 &\bullet y_{-1} \;y_{-2} \;\cdots y_{-m'} \\[-10pt]
    \line(1,0){90} & \line(1,0){100} \\[-7pt]
  w_{n'} w_{{n'}-1}\cdots w_1 w_0 &\bullet w_{-1} w_{-2} \cdots w_{-m'}\,.
  \end{align*}
  
We search for a $(\beta,\A)$-representation of $w$, i.e., a  sequence 
  $$z_{n} z_{n-1}\cdots z_1 z_0 z_{-1} z_{-2} \cdots z_{-m}$$ such that $z_j \in \A$ and
  $$
    z_{n} z_{n-1}\cdots z_1 z_0 \bullet z_{-1} z_{-2} \cdots z_{-m}=(w)_{\beta,\A}\,.
  $$
  Note that the index of the first, resp. last, non-zero digit of the converted representation $z_{n} z_{n-1}\cdots z_1 z_0 \bullet z_{-1} z_{-2} \cdots z_{-m}=(w)_{\beta,\A}$ may differ from the original representation $w_{n'} w_{{n'}-1}\cdots w_1 w_0 \bullet w_{-1} w_{-2} \cdots w_{-m'}$. We assume that $ n\geq n'$ and $m\geq m'$, otherwise we pad the converted representation by zeros.
  
   Multiplication of a representation $w_{n'} w_{n-1}\cdots w_1 w_0 \bullet w_{-1} w_{-2} \cdots w_{-m'}$ by a power of $\beta$ is obvious:
  $$
  \beta^m \cdot w_{n'} w_{n'-1}\cdots w_1 w_0 \bullet w_{-1} w_{-2} \cdots w_{-m'} = w_{n} w_{n'-1}\cdots w_1 w_0 w_{-1} w_{-2} \cdots w_{-m'} \bullet
  $$  
  and after conversion
  $$
  z_{n} z_{n-1}\cdots z_1 z_0 z_{-1} z_{-2} \cdots z_{-m'}\bullet\cdots z_{-m} = \beta^m \cdot z_{n} z_{n-1}\cdots z_1 z_0 \bullet z_{-1} z_{-2} \cdots z_{-m'}\,. 
  $$  
Hence, without lost of generality, we consider only conversion of so-called $\beta$-integers -- numbers from $\fin{\A+\A}$ whose representations have all digits with negative indices equal to zero.
%  Particularly, let $(w)_{\beta, \A+\A}=w_{n'} w_{{n'}-1}\cdots w_1 w_0 \bullet$. We search for a number $n \in \NN$ and \mbox{$z_{n}, z_{n-1},\dots, z_1, z_0 \in\A$} such that $(w)_{\beta, \A}=z_{n} z_{n-1}\cdots z_1 z_0 \bullet$.   
  
  Digits $w_j$ are converted into the alphabet $\A$ by adding a suitable representation of zero digit-wise.
  For our purpose, we use the simplest possible representation which is deduced from the polynomial
  $$
    x-\beta \in \left(\Zomega\right)[x]\,.
  $$

We remark that any polynomial $R(x)=r_s x^s+r_{s-1}x^{s-1}+ \dots + r_1 x+r_0$ with coefficients $r_i \in \Zomega$ such that $R(\beta)=0$ gives us a possible representation of zero. The polynomial $R$ is called a \emph{rewriting rule}. One of the coefficients of $R$ which is greatest in modulus (so-called \emph{core coefficient}) is used for the conversion of a digit $w_j$. Nevertheless, the extending window method is strongly dependent on the rewriting rule, so we focus only on the simplest possible rewriting rule $R(x)=x-\beta$. Usage of an arbitrary rewriting rule $R$ is out of scope of this thesis.

 
As $0=\beta^{j} \cdot R(\beta)=1\cdot \beta^{j+1} -\beta \cdot \beta^{j}$, we have a representation of zero 
$$1 (-\!\beta) \underbrace{0 \cdots 0}_{j}\bullet = (0)_\beta\,. $$
for all $j \in \NN$. We multiply this representation by $q_j \in \Zomega$, which is called a \emph{weight coefficient}, to obtain another  representation of zero 
%$$q_j (-q_j\beta) \underbrace{0 \cdots 0}_{j}\bullet = (0)_\beta\,. $$ 
$q_j (-q_j\beta) 0 \cdots 0\bullet = (0)_\beta\,. $
This is digit-wise added to $w_{n} w_{n-1}\cdots w_1 w_0 \bullet$ to convert the digit $w_j$ into the alphabet $\A$. The conversion of $j$-th digit causes a \emph{carry} $q_{j}$ on the $(j+1)$-th position. 

In standard addition, the digit set conversion runs from the right ($j=0$) to the left until all non-zero digits and carries are converted into the alphabet $\A$:

	\begin{tabular}{rcccccccclcl}
	$w_{n'}$ & $w_{{n'}-1}$ & $\cdots$ & $w_{j+1}$ &\textcolor{red}{$w_{j}$} &$w_{j-1}$ & $\cdots$ & $w_1$ &$w_0$ &$\bullet$ & $=$&$(w)_{\beta,{\A+\A}}$\\
	  &   & &   &  &$q_{j-2}$ & $\iddots$ &  &  &\\
	   &   & &   & \textcolor{red}{$q_{j-1}$}  & $-\beta q_{j-1}$ &  &  &  &\\
	  &   &  $\iddots$&  $q_{j}$ &  \textcolor{red}{$-\beta q_{j}$} & & &  &  &\\ \cline{1-10}
	$z_{n} \cdots z_{n'}$ & $z_{{n'}-1}$ & $\cdots$ & $z_{j+1}$ &\textcolor{red}{$z_{j}$} &$z_{j-1}$ & $\cdots$ & $z_1$ &$z_0$ &$\bullet$  & $=$&$(w)_{\beta,{\A}}$\\
	\end{tabular}

%        \begin{align}
%        \label{eq:conversionScheme}
%            \hspace{100pt}  w_n w_{n-1}&&&\cdots& &w_{j+1}&\!\! &\textcolor{red}{w_j}  & \!\!  &w_{j-1} &&\cdots &&w_1 w_0\bullet \hspace{100pt} \notag\\[-5pt]
%                         &&&&       &       & &     &   &q_{j-2} &&\iddots  \notag \\[-3pt] 
%                         &&&&       &       & &\textcolor{red}{q_{j-1}}& -&\beta q_{j-1} \notag \\[-3pt]
%                         &&&&         &q_j&   \textcolor{red}{-}&\textcolor{red}{\beta q_j} &&\\[-8pt]
%                         &&&  \iddots      &   -&\beta q_{j+1}&   &\ && \notag \\[-17pt]
%          \intertext{\hspace{60pt}\line(1,0){300}}
%          \notag \\[-30pt]
%           z_{n'} \cdots z_{n} z_{n-1}&&&\cdots& &z_{j+1}& &\textcolor{red}{z_j}& &z_{j-1} &&\cdots &&z_1 \; z_0\bullet \notag                  
%        \end{align}
    Hence, the desired formula for conversion on the $j$-th position is 
    \begin{equation*}
        z_j=w_j + q_{j-1} - q_j \beta
    \end{equation*}
    for $j \in \NN$. We set $q_{-1}=0$ as there is no carry from the right on the 0-th position.
    
     The terms carry and weight coefficient are related to a position: while $q_{j-1}$ is a carry from the right  and $q_j$ is a chosen weight coefficient on the $j$-th position, $q_j$ is a carry from the right on the $(j+1)$-th position etc.

We remark that the conversion with the rewriting rule $x-\beta$ prolongs the part of non-zero digits only to the left as there is no carry from the left. Thus, all digits with negative indices of the converted sequence are zero.


     The fact that the conversion preserves the value of $w$ follows from adding a representation of zero:
\begin{align}
\label{eq:valuePreserving}
    \sum_{j\geq 0} z_j \beta^j &=w_0 - \beta q_0 + \sum_{j> 0} (w_j + q_{j-1} - q_j \beta) \beta^j \notag\\
    &=\sum_{j\geq 0} w_j \beta^j + \sum_{j>0} q_{j-1} \beta^j - \sum_{j\geq 0} q_j \cdot \beta^{j+1}  \\
    &=\sum_{j\geq 0} w_j \beta^j + \sum_{j>0} q_{j-1} \beta^j - \sum_{j> 0} q_{j-1} \cdot \beta^j \notag\\
    &=\sum_{j\geq 0} w_j \beta^j = w\,. \notag
\end{align}

    The weight coefficient $q_j$ must be chosen so that the converted digit is in the alphabet~$\A$, i.e., 
    \begin{equation}
    \label{eq:conversionFormula}
        z_j=w_j + q_{j-1} - q_j \beta \in \A\,.
    \end{equation} 
    The choice of weight coefficients is a crucial part of the construction of addition algorithms which are computable in parallel. The extending window method determining weight coefficients for a given input is described in Section~\ref{sec:methodDescription}.
    
    
     On the other hand, the following example shows that determining weight coefficients is trivial for  numeration systems such that an alphabet contains right one representative of each class modulo $\beta$.
     
     \begin{exmp}
        Assume now a numeration system $(\beta, \A)$, where
  $$
    \beta \in \NN\,,\beta  \geq 2\,, \A = \{0, 1, 2,\dots, \beta -1 \}\,.
  $$ 
       Notice that
    $$
        z_j \equiv w_j+q_{j-1} \mod \beta\,. 
    $$
  
  There is only one representative of each class modulo  $\beta$ in the standard numeration system $(\beta, \A)$. Therefore, the digit $z_j$ is uniquely determined for a given digit $w_j \in \A+\A$ and carry $q_{j-1}$ and thus so is the weight coefficient $q_j$. This means that $q_j=q_j(w_j,q_{j-1})$ for all $j\geq 0$. Generally,
  $$
  q_j=q_j(w_j,q_{j-1}(w_{j-1},q_{j-2}))=\dots =q_j(w_j ,\dots , w_1, w_0)
  $$
  and
  $$
  z_j=z_j(w_j ,\dots , w_1, w_0)\,,
  $$
  which implies that addition runs in linear time. For instance, the carry $q_{j-1}=1$ propagates through the whole result when we sum up $(\beta-1)(\beta-1)\dots(\beta-1)\bullet$ and $1\bullet$.
     
     \end{exmp}
  
  We require that the digit set conversion from $\A+\A$ into $\A$ is computable in parallel, i.e., there exist constants $r,t \in \NN_0$ such that for all $j\geq 0$ is $z_j=z_j(w_{j+t},\dots,w_{j-r})$. Anticipation $t$ equals zero since we use the rewriting rule $x-\beta$. To avoid the dependency on all less significant digits, we need variety in the choice of the weight coefficient $q_j$. This implies that the used numeration system must be redundant.
  

\section{Extending window method}
\label{sec:methodDescription}
In order to construct a digit set conversion in numeration system $(\beta,\A)$ from $\A+\A$ to $\A$ which is computable in parallel, we consider a more general case of digit set conversion from an \emph{input alphabet} $\B$ such that $\A \subsetneq \B$ instead of the alphabet $\A+\A$.
As mentioned above, the key problem is to find for every $j\geq 0$ a weight coefficient $q_j$ such that 
    $$
        z_j=\underbrace{w_j}_{\in \B} + q_{j-1} - q_j \beta \in \A 
    $$  
    for any input $w_{n'}w_{n'-1}\dots w_1 w_0 \bullet=(w)_{\beta,\B}, w\in \fin{\B}$. We remark that the weight coefficient $q_{j-1}$ is determined by the input $w_{j-1}\dots w_1 w_0 \bullet$. For a digit set conversion with the rewriting rule $x-\beta$ to be computable in parallel, the digit $z_j$ is required to satisfy $z_j=z_j(w_{j},\dots,w_{j-r})$ for a fixed memory $r$ in $\NN$.
    
    Note that the digit $z_j$ is given by the input digit $w_j$ and carry $q_{j-1}$ which is determined by input digits  $w_{j-1} w_{j-2}\dots$. Thus, if we find a weight coefficient $q_j$ for all possible combinations of input digits $w_j w_{j-1} w_{j-2}\dots$, then the position $j$ is not important. Therefore, we may strongly simplify our notation if we omit $j$ in subscripts. From now on, $w_0\in\B$ is a converted digit, $w_{-1} w_{-2}\dots\in\B$ are digits on right, $q_{-1}\in\Zomega$ is a carry from the right and we search for a weight coefficient $q_0\in\Zomega$ such that 
    $$
    z_0=w_0 + q_{-1} - q_0 \beta \in \A\,.
    $$
    
   
    We introduce two definitions before we describe the extending window method.
    \begin{defn}
    \label{def:weightCoefficientsSet}
        Let $\B$ be a set such that $\A \subsetneq \B$. Then any finite set $\Q\subset\Zomega$ containing~0 such that 
        $$
            \B + \Q \subset \A + \beta \Q
        $$  
        is called a \emph{weight coefficients set}.
    \end{defn}
    We see that if $\Q$ is a weight coefficients set, then
        $$
        (\forall w_0 \in \B)(\forall q_{-1}\in\Q)(\exists q_0 \in \Q )(\underbrace{w_0 + q_{-1} - q_0 \beta}_{z_0} \in \A )\,.
        $$
    In other words, there is a weight coefficient $q_0 \in \Q$ for a carry from the right $q_{-1}\in \Q$ and a digit $w_0$ in the input alphabet $\B$ such that $z_0$ is in the alphabet $\A$.  Notice that  the  carry from the right for the rightmost non-zero digit of the converted sequence which is $0$ is in $\Q$ by the definition.
    \begin{defn}
    Let $r$ be an integer and $q:\B^{r} \rightarrow \Q$ be a mapping such that 
    $$
    w_0+ q(w_{-1}, \dots, w_{-r}) - \beta q\tupleo{(r-1)} \in \A
    $$
    for all $w_0,w_{-1}, \dots, w_{-r} \in \B$, and $q(0,0,\dots,0)=0$. Then $q$ is called \emph{weight function} and $r$~is called \emph{length of window}.    
    \end{defn}

 Having a weight function $q$, we define a function $\phi:\B^{r+1}\rightarrow \A$ by
    \begin{equation}
    \label{eq:localConversion}
        \phi(w_{0}, \dots, w_{-r})=w_0+ \underbrace{q(w_{-1}, \dots, w_{-r})}_{=q_{-1}} - \beta \underbrace{q\tupleo{(r-1)}}_{=q_0}=:z_0\,,
    \end{equation} 
    which verifies that the digit set conversion is indeed a $(r+1)$-local function with anticipation $0$ and memory $r$. The requirement of zero output of the weight function $q$ for the input of $r$ zeros guarantees that $\phi(0,0,\dots,0)=0$. Thus, the first condition of Definition~\ref{def:digitSetConversion} is satisfied. The second one follows from the equation \eqref{eq:valuePreserving}. 
    
Let us summarize the construction of the digit set conversion by the rewriting rule \mbox{$x-\beta$}. We need to find weight coefficients for all possible combinations of digits of the input alphabet~$\B$. The rewriting rules multiplied by the weight coefficients are digit-wise added to an input sequence. In fact, it means that the equation  \eqref{eq:conversionFormula} is applied on each position. If the digit set conversion is computable in parallel, the weight coefficients are determined as the outputs of the weight function $q$ with some fixed length of window $r$.  

We search for a weight function $q$ for a given base $\beta$ and input alphabet $\B$ by the extending window method. It consists of two phases. First, we find some weight coefficients set $\Q$. We know that it is possible to convert an input sequence by choosing the weight coefficients from the set $\Q$. The set $\Q$ serves as the starting point for the second phase in which we increment the expected length of the window $r$ until the weight function $q$ is uniquely defined for each $\tupleo{(r-1)} \in \B^{r}$. Then, the local conversion is determined -- we use the weight function outputs as weight coefficients in the formula \eqref{eq:localConversion}.    

We describe the general concept of the extending window method in this chapter, while various possibilities of construction of sets during both phases are discussed in Chapter~\ref{chap:diffChoices}.
Note that  convergence of both phases is studied in Chapter~\ref{chap:convergence}.
      
\section{Phase 1 -- Weight coefficients set}
\label{subsec:phase1}
The goal of the first phase is to compute a weight coefficients set $\Q$, i.e., to find a set $\Q \ni 0$ such that 
$$
    \B + \Q \subset \A + \beta \Q\,.
$$  
We build a sequence $\Q_0, \Q_1, \Q_2,\dots$ iteratively so that we extend $\Q_k$ to $\Q_{k+1}$ in a way to cover all elements of the set $\B+\Q_k$ by elements of the extended set $\Q_{k+1}$, i.e.,
$$
\B+ \Q_k \subset \A + \beta \Q_{k+1}\,.
$$
This procedure is repeated until the extended weight coefficients set $\Q_{k+1}$ is the same as the previous set $\Q_{k}$. We remark that the expression ``a weight coefficient $q$ covers an element $x$'' means that there is a digit $a \in \A$ such that $x=a + \beta q$. 

In other words, we start with $\Q_0=\{0\}$ meaning that we search all weight coefficients $q_0$ necessary for digit set conversion for the case where there is no carry from the right, i.e., $q_{-1}=0$. We add them to the weight coefficients set $\Q_0$ to obtain the set $\Q_1$. Assume now that we have a set $\Q_k$ for some $k\geq 1$. The weight coefficients in $\Q_k$ now may appear as a carry $q_{-1}$. If there are no suitable weight coefficients $q_0$ in the weight coefficients set~$\Q_k$ to cover all sums of coefficients from $\Q_k$ and digits of the input alphabet $\B$, we extend $\Q_k$ to $\Q_{k+1}$ by  suitable coefficients. And so on until there is no need to add more elements, i.e., the extended set $\Q_{k+1}$ equals $\Q_k$. Then the weight coefficients set $\Q:=\Q_{k+1}$ satisfies Definition~\ref{def:weightCoefficientsSet}. 

\komentar{For better understanding, see Figures~\ref{img:phase1img1}--\ref{img:phase1img13} in Appendix~\ref{app:phase1} which illustrate the construction of the weight coefficients set $\Q$ for the Eisenstein base and a complex alphabet (see Example~\ref{ex:Eisenstein1-blockcomplex} for its description). }

The precise description of the algorithm in a pseudocode is in Algorithm~\ref{alg:weightCoefSet}. Observe that extending $\Q_k$ to $\Q_{k+1}$ is not unique. Various methods of choice are described in Section~\ref{sec:methodsOne} in Algorithm~\ref{alg:extendWeightCoefSet}.
    



Section~\ref{sec:convergencePhase2} discusses the convergence of Phase 1, i.e. whether it happens that  $\Q_{k+1}=\Q_k$ for some  $k$.
    
\begin{algorithm}
  \caption{Search for weight coefficients set (Phase 1)}
    \label{alg:weightCoefSet}
  \begin{algorithmic}[1]
    \STATE $k:=0$ 
    \STATE $Q_0:=\{0\}$
    \REPEAT
     \STATE Extend $\Q_k$ to $\Q_{k+1}$ (by Algorithm~\ref{alg:extendWeightCoefSet}) so that $$\B+ \Q_k \subset \A + \beta \Q_{k+1}$$
     \vspace{-20pt}
      \STATE  $k:=k+1$
      \UNTIL{$\Q_k = \Q_{k+1}$}      
      \STATE $\Q:=\Q_k$
    \RETURN $\Q$
  \end{algorithmic}
\end{algorithm}


% An added element from each list of \verb+candidates+ is chosen as the smallest one unless there is already a covering element contained in $\Q_{k}$ (Algorithm~\ref{alg:extendWeightCoefSet}).  


    

  
    
    





\section{Phase 2 -- Weight function}
\label{subsec:phase2}
    We want to find a length of the window $r$ and a weight function $q:\B^{r} \to \Q$. We start with the weight coefficients set $\Q$ obtained in Phase 1. The idea is to reduce necessary weight coefficients which are used to convert a given input digit up to a single value. This is done by enlarging the number of considered input digits, i.e. incrementing $r$.  If the window is extended to the right, we know more digits that cause a carry form the right. This may decrease the number of  possible carries from the right and hence, less weight coefficients to convert the input digit may be necessary.
     
    We introduce notation for sets of possible weight coefficients for given input digits.
        Let $\Q$ be a weight coefficients set and $w_0\in \B$. Denote by $\Q_{[w_0]}$ any set such that
        $$
            (\forall q_{-1} \in \Q)(\exists q_0 \in \Q_{[w_0]})(w_0 + q_{-1} - q_0 \beta \in \A)\,.
        $$
It means that we do not know any input digits on the right, therefore there might be any carry from the set $\Q$. However, we may determine a set $\Q_{[w_0]}$ of  weight coefficients which allow the conversion of $w_0$ to $\A$ since we know the input digit $w_0$.
        
        By induction with respect to $k \in \NN, k\geq 1$, for all $\tupleo{k}\in \B^{k+1}$ denote by $\Qwo{k}$ any subset of  $\Qwo{(k-1)}$ such that 
        $$
           (\forall q_{-1} \in \Qw{1}{k})(\exists q_0 \in \Qwo{k})(w_0 + q_{-1} - q_0 \beta \in \A)\,.
        $$
        
    
 
%    Recall the scheme \eqref{eq:conversionScheme} of the digit set conversion for better understanding of the notation and method:
%    \begin{align*}
%        \hspace{130pt}\cdots\; &w_{j+1}&\!\! &w_j  & \!\!  &w_{j-1}&\cdots w_{j-M+1} &w_{j-M}\cdots \hspace{130pt} \\[-3pt] 
%                         & &       & & & q_{j-2} \\[-3pt]
%                         & &       &q_{j-1}& -&\beta q_{j-1} \\[-3pt]
%                           &q_j&   -&\beta q_j &&\\[-3pt]      
%                           -&\beta q_{j+1}&   &  &&\\[-15pt]      
%    \intertext{\hspace{120pt}\line(1,0){250}} 
%          \vspace{-15pt}
%          \\[-30pt]
%     \cdots\; &z_{j+1}& &z_j& &z_{j-1}& \cdots z_{j-M+1}\; &z_{j-M}\cdots                     
%    \end{align*}     

Sets of possible weight coefficients and a weight function $q$ are constructed by Algorithm~\ref{alg:weightFunction}. The idea is to check all possible right carries $q_{-1}\in\Q$ and determine values $q_0\in\Q$ such that 
    $$
    z_0=w_0 + q_{-1} - q_0 \beta \in \A \,.
    $$  
    
    So we obtain a set $\Q_{[w_0]}\subset\Q$ of weight coefficients which are necessary to convert the digit~$w_0$ with any carry $q_{-1}\in\Q$. Assuming that we know the input digit $w_{-1}$, the set of possible carries from the right is also reduced to $\Q_{[w_{-1}]}$. Thus we may reduce the set $\Q_{[w_0]}$ to a set $\Qwo{1}\subset \Q_{[w_0]}$ which is necessary to cover all elements of $w_0 + \Q_{[w_{-1}]}$. 

In the $k$-th step, we search for a set $\Qwo{k}\subset\Qwo{(k-1)}$ such that 
               $$
              w_0 + \Qw{1}{k} \subset \A + \beta \Qwo{k}\,.
              $$
              The length of window is $k+1$, i.e., we know $k$ digits on the right. To  construct the set $\Qwo{k}$, we select from $\Qwo{(k-1)}$ such weight coefficients which are necessary to convert digit $w_0$ to the alphabet $\A$ with all possible carries from the set $\Qw{1}{k}$.
                 
    Proceeding in this manner may lead to a unique weight coefficient $q_0$ for enough long window.     
    If there is $r\in\NN, r\geq 1$ such that 
    $$
    \#\Qwo{(r-1)}=1
    $$
    for all $\tupleo{(r-1)} \in \B^r$, then the output $q\tupleo{(r-1)}$ is defined as the element of $\Qwo{(r-1)}$. 
    
    Similarly to Phase 1, the choice of $\Qwo{k}$ is not unique. We list different methods of choice in Section~\ref{sec:methodsTwo}, Algorithm~\ref{alg:minimalSet}.
    \komentar{ Figures~\ref{img:phase2img1}--\ref{img:phase2img7} in Appendix~\ref{app:phase2} illustrate the construction of the set $\Q_{[\omega,1,2]}$ for the Eisenstein numeration system.   }
    
        
    To verify that 
$$
	z_0=\phi(w_{0}, \dots, w_{-r})=w_0+ \underbrace{q\tuple{1}{r}}_{=q_{-1}} - \beta \underbrace{q\tupleo{(r-1)}}_{=q_0}
$$    
is in the alphabet $\A$, consider that $q_0=q\tupleo{(r-1)}$ is the only element of $\Qwo{(r-1)}$ which was constructed such that 
$$
w_0 + \Qw{1}{(r-1)} \subset \A +\beta \Qwo{(r-1)}\,.
$$
At the same time, $q_{-1}=q\tupleo{(r-1)}$ is the only element of $\Qw{1}{r}$ which is a subset of $\Qw{1}{(r-1)}$.

    

    
\begin{algorithm}
  \caption{Search for weight function $q$ (Phase 2)}
    \label{alg:weightFunction}
  \begin{algorithmic}[1]
    \REQUIRE{weight coefficients set $\Q$}
    \FORALL{$w_0 \in \B$} 
        \STATE Find set $\Q_{[w_0]} \subset \Q$ (by Algorithm~\ref{alg:minimalSet}) such that
          $$
          w_0 + \Q \subset \A + \beta \Q_{[w_0]}
          $$
    \ENDFOR
    \STATE $k:=0$
    \WHILE{$\max\{\#\Qwo{k}\colon \tupleo{k}\in \B^{k+1} \} > 1$}
        \STATE $k:= k +1$
        \FORALL{$\tupleo{k}\in \B^{k+1}$}
            \STATE Find set $\Qwo{k} \subset \Qwo{(k-1)}$ (by Algorithm~\ref{alg:minimalSet}) such that
              $$
              w_0 + \Qw{1}{k} \subset \A + \beta \Qwo{k}\,,
              $$
        \ENDFOR  
    \ENDWHILE  
    \STATE $r:= k+1$ 
    \FORALL{$\tupleo{(r-1)} \in \B^{r}$}  
        \STATE $q\tupleo{(r-1)}\in \B^{r}:=$ only element of $\Qwo{(r-1)}$
    \ENDFOR
    \RETURN $q$
  \end{algorithmic}
\end{algorithm}

Unfortunately, finiteness of Phase 2 is not guaranteed. But the non-convergence of Phase 2 with a specific property may be revealed by Algorithm~\ref{alg:oneletterSets} before the run of Phase 2 or by Algorithm~\ref{alg:checkCycles} during it. These algorithms are based on theorems in Chapter ~\ref{chap:convergence}. Modified Phase 2 which includes these algorithms can be found in Section~\ref{sec:modifiedPhase2}.



Notice that for a given length of window $r$, the number of calls of Algorithm~\ref{alg:minimalSet} within Algorithm~\ref{alg:weightFunction} is
$$
\sum_{k=0}^{r-1}  \#\B^{k+1} = \#\B \frac{\#\B^r-1}{\#\B-1}\,.
$$    
It implies that the time complexity grows exponentially. The required memory is also exponential because we have to store sets $\Qwo{k}$ at least for $k\in\{r-2, r-1\}$  for all $w_0,\dots, w_{-k} \in \B$.

\komentar{konec vlozeni z vyzkumaku}

\chapter{Convergence}
\section{Number of congruence classes}

\begin{defn}
Let $M\in\ZZ^{d\times d}$ be a nonsingular integer matrix. Vectors $x,y \in \ZZ^d$ are \emph{congruent modulo $M$ in $\ZZ^d$}, if $x-y \in M\ZZ^d$.
\end{defn}

\begin{lem}
Let $M\in\ZZ^{d\times d}$ be a nonsingular integer matrix. The number of congruence classes modulo $M$ in $\ZZ^d$ is $|\det M|$.
\end{lem}
\begin{proof}
Set $y_i:=M e_i$ for $i\in\{0, \dots, d-1 \}$ and 
$$
B_{\enum{\alpha}}:=\left\{\sum_{i=0}^{d-1} (\alpha_i + t_i) y_i \colon t_i \in [0,1)\right\}
$$
for $\enum{\alpha} \in \ZZ^d$. Obviously,
$$
\RR^d=\bigcup_{\enum{\alpha} \in \ZZ^d} B_{\enum{\alpha}}\,.
$$
For fixed $\enum{\alpha} \in \ZZ^d$, the number of points of $\ZZ^d$ in $B_{\enum{\alpha}}$  is volume of $B_{\enum{\alpha}}$  which is $|\det M|$. Hence, it is enough to prove that there is exactly one representative of each congruence class in $B_{\enum{\alpha}}$. 

To show that there are representatives of all classes, assume $x\in\ZZ^d$. Since $\enum{y}$ is a basis of $\RR^d$, there are scalars $s_0, \dots, s_{d-1}\in \RR$ such that $x= \sum_{i=0}^{d-1} s_i y_i$. Set $\gamma_i:=\lfloor s_i \rfloor$ and $t_i:=s_i-\gamma_i$. Now
\begin{align*}
 x=\sum_{i=0}^{d-1} (\gamma_i+t_i) y_i=\sum_{i=0}^{d-1}t_i y_i +\sum_{i=0}^{d-1} (\gamma_i-\alpha_i)y_i + \sum_{i=0}^{d-1} \alpha_i y_i=\underbrace{\sum_{i=0}^{d-1} (\alpha_i+t_i)y_i}_{\in B_{\enum{\alpha}} } +M\underbrace{(\gamma-\alpha)}_{\in\ZZ^d}\,,
\end{align*}
where
$$
\alpha=\enum{\alpha}^T \quad \text{ and }\quad \gamma=\enum{\gamma}^T\,.
$$
Let $x=\sum_{i=0}^{d-1} s_i y_i \in \ZZ^d$ and $x'=\sum_{i=0}^{d-1}s'_i y_i \in \ZZ^d$ be distinct elements of $B_{\enum{\alpha}}$ which are congruent modulo $M$, i.e., there exists $z=\enum{z}^T\in\ZZ^d$ such that $x=x'+M z$. There is $i_0\in\{0, \dots , d-1\}$ such that $|z_{i_0}|\geq 1$ as $x\neq x'$. Thus, $|s_{i_0}-s'_{i_0}|=|z_{i_0}|\geq 1$ which contradicts that  $x, x'\in B_{\enum{\alpha}}$.
\end{proof}

\begin{thm}
Let $\omega$ be an algebraic integer of degree $d$ and  $\beta$ be an element of $\Zomega$ such that $\deg \omega = \deg\beta$. The number of congruence classes modulo $\beta$ in $\Zomega$ is $|m_\beta(0)|$.
\end{thm}
\begin{proof}
\komentar{dopsat dukaz}
\end{proof}
\section{Minimal alphabet $\A$}
\label{sec:minimalAlphabet}	

Frougny, Pelantov\'a and Svodov\'a \cite{minAlph} proved a lower bound on the size of an alphabet $0\in\A\subset\ZZ$ of consecutive integers which enables parallel addition. In this section, we prove the same bound for an arbitrary alphabet $\A\in\Zbeta$.  We recall their auxiliary results in Theorem~\ref{thm:reprBetaMinusOne} an Lemma~\ref{lem:alphabetRestrictions}, but only for a parallel digit set conversion without anticipation as our rewriting rule $x-\beta$ does not require memory. 

We remark that we indicate by the assumption $\A\in\Zbeta$  that we work in $\Zbeta$ instead of $\Zomega$. Notice that congruence classes modulo $\beta$ in $\Zomega$ and $\Zbeta$ are generally different. It implies that even an integer alphabet behaves differently if $\beta\in\Zomega$ and $\Zomega\neq\Zbeta$. On the other hand, if $\beta=\pm \omega +c$, where $c\in\ZZ$, then $\Zomega=\Zbeta$

The following theorem says that all classes modulo $\beta$ in $\Zomega$ which are contained in $\A+\A$ must have their representatives in $\A$.
\begin{thm}
\label{thm:reprBetaMinusOne}
Let $\omega$ be an algebraic integer. Let the base $\beta\in\Zomega$ be such that $|\beta|>1$ and the alphabet $\A\subset\Zomega$ be such that $0\in\A$. If there exists a $p$-local digit set conversion defined by the function $\phi\colon (\A+\A)^p\rightarrow \A$ and $p=r+1$, then the number $\phi(b,\dots,b)-b$ belongs to the set $(\beta-1)\Zomega$ for any $b\in\A+\A$. 
\end{thm}
\begin{proof}
Let $b\in\A+\A$ and $a=\phi(b, \dots,b)$. For $n\in\NN, n\geq 1$, we denote $S_n$ the number represented by
$$
^{\omega}\!0 \underbrace{b\dots b}_{n}\bullet \underbrace{b\dots b}_{r}0^\omega\,.
$$
The representation of $S_n$ after the digit set conversion is of the form
$$
^{\omega}\!0 \underbrace{w_{r}\dots w_{1}}_{\beta^n W}\underbrace{a\dots a}_{n}\bullet \underbrace{\widetilde{w_1}\dots \widetilde{w_r}}_{\beta^{-r}\widetilde{W}}0^\omega\,,
$$
where 
$$W=\sum_{j=1}^r w_j \beta^{j-1} \qquad \text{and} \qquad \widetilde{W}=\sum_{j=1}^r\widetilde{w_j} \beta^{r-j}\,.$$
Since both representations have same value, we have
\begin{align}
\label{eq:reprBetaMinusOne}
b \sum_{j=-r}^{n-1} \beta^j &= W \beta^n + a \sum_{j=0}^{n-1} \beta^j + \beta^{-r}\widetilde{W} \notag \\
b \sum_{j=-r}^{-1} \beta^j +b\frac{\beta^n-1}{\beta-1} &= W \beta^n + a \frac{\beta^n-1}{\beta-1} + \beta^{-r}\widetilde{W}\,,
\end{align}
for all $n\geq 1$. We subtract this equation for $n$ and $n-1$ to obtain
$$
b\frac{\beta^n-\beta^{n-1}}{\beta-1}=W(\beta^n-\beta^{n-1}) + a\frac{\beta^n-\beta^{n-1}}{\beta-1}\,.
$$
We simplify it to
\begin{equation}
\label{eq:reprBetaMinusOneFinal}
b=W(\beta-1) + a\,.
\end{equation}
Hence, $a=\phi(b, \dots,b)\equiv b$ modulo $\beta-1$.
\end{proof}

If a base $\beta$ has a real conjugate greater than one, then there are some extra requirements on the alphabet $\A$. For simplicity, we assume that the base $\beta$ itself is real and greater than one. We show later that this assumption is without loss of generality.
\begin{lem}
\label{lem:alphabetRestrictions}
Let $\omega$ be a real algebraic integer and the base $\beta\in\Zomega$ be such that $\beta>1$. Let the alphabet $\A\subset\Zomega$ be such that $0\in\A$ and denote $\lambda=\min \A$ and $\Lambda=\max \A$. If there exists a $p$-local digit set conversion defined by the function $\phi\colon (\A+\A)^p\rightarrow \A$ and $p=r+1$, then:
\begin{enumerate}[i)]
	\item $\phi(b,\dots,b)\neq \lambda$ for all $b\in\A+\A$ such that $b>\lambda$.
	\item $\phi(b,\dots,b)\neq \Lambda$ for all $b\in\A+\A$ such that $b<\Lambda$.
	\item If $\Lambda\neq 0$, then $\phi(\Lambda,\dots,\Lambda)\neq \Lambda$.
	\item If $\lambda\neq 0$, then $\phi(\lambda,\dots,\lambda)\neq \lambda$.
\end{enumerate}
\end{lem}
\begin{proof}
To prove \textit{i)}, assume in contradiction that $\phi(b,\dots,b)= \lambda$. We proceed in the same manner as in Theorem \ref{thm:reprBetaMinusOne}, the equation \eqref{eq:reprBetaMinusOne} implies
$$
b \sum_{j=-r}^{-1} \beta^j +b\frac{\beta^n-1}{\beta-1} =  \beta^n W + \lambda \frac{\beta^n-1}{\beta-1} + \beta^{-r}\widetilde{W}\,.
$$
We apply also the equation c to obtain
\begin{align*}
b \sum_{j=-r}^{-1} \beta^j +b\frac{\beta^n-1}{\beta-1} &=  \beta^n \frac{b-\lambda}{\beta-1} + \lambda \frac{\beta^n-1}{\beta-1} + \beta^{-r}\widetilde{W}\,.
\end{align*}
Now we may simplify and estimate
\begin{align*}
b \sum_{j=-r}^{-1} \beta^j +\frac{-b}{\beta-1} &=  \frac{-\lambda}{\beta-1} + \beta^{-r}\sum_{j=1}^r\widetilde{w_j} \beta^{r-j} \\
b \underbrace{\left(\sum_{j=1}^{r} \frac{1}{\beta^j} -\frac{1}{\beta-1}\right)}_{-\sum_{j=r+1}^{\infty} \frac{1}{\beta^j}} &=  -\lambda\frac{1}{\beta-1} + \sum_{j=1}^r\widetilde{w_j} \beta^{-j}\geq  \lambda\underbrace{\left(-\frac{1}{\beta-1}+\sum_{j=1}^{r} \frac{1}{\beta^j} \right)}_{-\sum_{j=r+1}^{\infty}\frac{1}{\beta^j}}\,.
\end{align*}
Hence $b\leq\lambda$ which is a contradiction. The proof of \textit{ii)} can be done in the same way.

For \textit{iii)}, assume that $\phi(\Lambda,\dots,\Lambda)= \Lambda$. Now consider a number $T_q$ represented by
$$
^{\omega}\!0 \bullet \underbrace{\Lambda\dots\Lambda}_{r} \underbrace{(2\Lambda)\dots(2\Lambda)}_{q} 0^\omega\,.
$$
After the digit set conversion, a representation is
$$
^{\omega}\!0  \underbrace{w_r\dots w_1}_{W} \bullet z_1\dots z_{r+q} 0^\omega\,.
$$
The value $T_q$  preserves, thus,
$$
\Lambda\sum_{j=1}^r \beta^{-j} +2\Lambda \sum_{j=r+1}^{r+q} \beta^{-j}=W+\sum_{j=1}^{r+q}z_j\beta^{-j}\,.
$$
But $W=0$ from the equation \eqref{eq:reprBetaMinusOneFinal}. We estimate
\begin{align*}
\Lambda\sum_{j=1}^{r+q} \beta^{-j} +\Lambda \sum_{j=r+1}^{r+q} \beta^{-j}&=\sum_{j=1}^{r+q}z_j\beta^{-j}\leq \Lambda\sum_{j=1}^{r+q}\beta^{-j}\\
\Lambda \sum_{j=r+1}^{r+q} \beta^{-j}&\leq 0\,.
\end{align*}
This contradicts that $\Lambda$ is positive as it is a nonzero, maximal element of the alphabet $\A$ which contains 0. The proof of \textit{iv)} is analogous.
\end{proof}

In order to prove the lower bound, we need to show that the alphabet $\A$ must contain all representatives modulo $\beta$ and $\beta-1$. 
\begin{thm}
Let $\beta$ be an algebraic integer such that $|\beta|>1$. Let $0\in \A\subset \Zbeta$ be an alphabet such that $1\in \A[\beta]$. If addition in the numeration system $(\beta, \A)$ which uses the rewriting rule $x-\beta$ is computable in parallel, then the alphabet $\A$ contains at least one representative of each congruence class modulo $\beta$ and $\beta-1$ in $\Zbeta$. 
\label{thm:representativesInAlphabet}
\end{thm}
\begin{proof}
The existence of an algorithm for addition with the rewriting rule $x-\beta$ implies that the set $\A[\beta]$ is closed under addition. By the assumption $1\in \A[\beta]$, the set $\NN$ is subset of  $\A[\beta]$. Since $0\in\A$, we have $\beta \cdot \A[\beta] \subset \A[\beta]$. Hence, $\NN[\beta] \subset \A[\beta]$.

For any element  $x=\sum_{i=0}^N x_i \beta^i\in \Zbeta$ there is an element $x'=\sum_{i=0}^N x'_i \beta^i\in \NN[\beta]$ such that $x\equiv_\beta x'$  since $m_\beta (0)\equiv_\beta 0$ and $\beta^i\equiv_\beta 0$. As $x'\in \NN[\beta] \subset \A[\beta]$, we have
$$
x\equiv_\beta x'=\sum_{i=0}^{n}a_i\beta^i \equiv_\beta a_0\,,
$$
where $a_i\in \A$. Hence, for any element $x\in\Zomega$, there is a letter $a_0\in\A$ such that $x\equiv_\beta a_0$.

In order to prove that there is at least one representative of each congruence class modulo $\beta-1$ in the alphabet $\A$, we consider again an element $x=\sum_{i=0}^N x_i \beta^i\in \Zbeta$. Similarly, there is an element $x'=\sum_{i=0}^N x'_i \beta^i\in \NN[\beta]$ such that $x\equiv_{\beta-1} x'$  since $m_{\beta-1} (0)\equiv_{\beta-1} 0$ and $(\beta-1)^i\equiv_{\beta-1} 0$.

Since $x'\in \NN[\beta]\subset \A[\beta]$,
$$
x'=\sum_{i=0}^{n}a_i\beta^i\,,
$$
where $a_i\in \A$. We prove by induction with respect to $n$ that $x'\equiv_{\beta-1} a$ for some $a\in\A$.
If $n=0$, $x'=a_0$. Now we use the fact that if there is a parallel addition algorithm, for each letter $b \in\A+\A$, there is $a\in\A$ such that $b \equiv_{\beta-1} a$ (Theorem~\ref{thm:reprBetaMinusOne}). For $n+1$, we have
\begin{align*}
x'&=\sum_{i=0}^{n+1}a_i\beta^i =a_0 + \sum_{i=1}^{n+1}a_{i}\beta^i\\
    &=a_0 + \beta \sum_{i=0}^{n}a_{i+1}\beta^i - \sum_{i=0}^{n}a_{i+1}\beta^i+ \sum_{i=0}^{n}a_{i+1}\beta^i \\
    &\equiv_{\beta-1} a_0 + (\beta-1)\sum_{i=0}^{n}a_{i+1}\beta^i + a \equiv_{\beta-1}a_0 +a \equiv_{\beta-1}a' \in\A\,,
\end{align*}
where we use the induction assumption
$$
\sum_{i=0}^{n}a_{i+1}\beta^i\equiv_{\beta-1} a\,.
$$
\end{proof}
Unfortunately, the claim cannot be generalized to modulo in $\Zomega$ -- there are numeration systems with integer alphabets which allow parallel addition, but these alphabets does not contain all representatives modulo $\beta-1$ in $\Zomega$, see Table~\ref{tab:resultsQuadrInt} and Examples \ref{ex:integerAB},  \ref{ex:integerAJ}, \ref{ex:integerAO} and \ref{ex:integerAR}. Nevertheless, each class modulo $\beta -1$ in $\Zomega$ which is contained in $\A+\A$ must still has its representative in $\A$ according to Theorem~\ref{thm:reprBetaMinusOne}.

The following lemma summarizes that if we have a parallel addition algorithm for a base $\beta$, then we easily obtain an algorithm also for conjugates of $\beta$.
\begin{lem}
\label{lem:parAddAlgForConjugate}
Let $\omega$ be an algebraic integer with a conjugate $\omega'$. Let $\beta\in\Zomega, |\beta|>1$ and let $\sigma:\QQ(\omega)\rightarrow \QQ(\omega')$ be an isomorphism such that $|\sigma(\beta)|>1$. Let $\varphi$ be a digit set conversion  in the base $\beta$ from $\A+\A$ to $\A$. There exists  is a digit set conversion $\varphi'$ in the base $\beta'$ from $\A'+\A'$ to $\A'$ where $\beta'=\sigma(\beta)$ and $\A'=\{\sigma(a) \colon a\in\A\}$.
\end{lem}
\begin{proof}
Let $\phi:\A^p\rightarrow\A$ be a mapping which defines $\varphi$ with $p=r+t+1$. We define a mapping $\phi':\A^p\rightarrow \A$ by 
$$
\phi'(w'_{j+t}, \dots, w'_{j-r})=\sigma\left(\phi\left(\sigma^{-1}(w'_{j+t}), \dots, \sigma^{-1}(w'_{j-r})\right)\right)\,.
$$
Next, we define a digit set conversion  $\varphi':(\A'+\A')\rightarrow\A'$ by $\varphi'(w')=(z'_j)_{j\in\ZZ}$ where $w'=(w'_j)_{j\in\ZZ}$ and $z'_j=\phi'(w'_{j+t}, \dots, w'_{j-r})$. Obviously, if $w'$ has only finitely many nonzero entries, then there is only finitely many nonzeros in $(z'_j)_{j\in\ZZ}$   since
$$
\phi'(0, \dots, 0)=\sigma\left(\phi\left(\sigma^{-1}(0), \dots, \sigma^{-1}(0)\right)\right)=\sigma\left(\phi\left(0, \dots, 0\right)\right)=\sigma\left(0\right)=0\,.
$$
The value of the number represented by $w'$ is also preserved:
\begin{align*}
\sum_{j\in\ZZ}w'_j {\beta'}^j&=\sum_{j\in\ZZ}\sigma(w_j) \sigma(\beta)^j=\sigma\left(\sum_{j\in\ZZ}w_j\beta^j\right) \\
&=\sigma\left(\sum_{j\in\ZZ}z_j\beta^j \right)=\sigma\left(\sum_{j\in\ZZ}\phi\left(w_{j+t}, \dots,w_{j-r}\right)\beta^j\right) \\
&=\sum_{j\in\ZZ}\sigma(\phi\left(w_{j+t}, \dots,w_{j-r}\right)){\beta'}^j=\sum_{j\in\ZZ}z'_j {\beta'}^j
\end{align*}
where $w_j=\sigma^{-1}(w'_j)$ for $j\in\ZZ$ and $\varphi((w_j)_{j\in\ZZ})=(z_j)_{j\in\ZZ}$.
\end{proof}

Finally, we put together that the alphabet $\A$ contains all representative modulo $\beta$ and $\beta-1$, number of congruence classes and restrictions on the alphabet for a base with a real conjugate greater than one.
\begin{thm}
Let $\beta$ be an algebraic integer such that $|\beta|>1$. Let $0\in \A\subset \Zbeta$ be an alphabet such that $1\in \A[\beta]$. If addition in the numeration system $(\beta, \A)$ which uses the rewriting rule $x-\beta$ is computable in parallel, then
$$
\#\A \geq \max \{|m_\beta(0)|, |m_\beta(1)|\}\,.
$$
Moreover, if $\beta$ is such that it has a real conjugate greater than 1, then 
$$
\#\A \geq \max \{|m_\beta(0)|, |m_\beta(1)|+2\}\,.
$$
\end{thm}
\begin{proof}
By Theorem~\ref{thm:representativesInAlphabet}, there are all representatives modulo $\beta$ and modulo $\beta-1$ in the alphabet $\A$. The numbers of congruence classes are $|m_\beta(0)|$ and $|m_{\beta-1}(0)|$ by Theorem~\ref{thm:numbCongruenceClasses}. Obviously, $m_{\beta-1}(x) = m_\beta (x+1)$. Thus $m_{\beta-1}(0) = m_\beta (1)$.

Let $\phi$ be a mapping which defines the parallel addition. According to Lemma~\ref{lem:parAddAlgForConjugate}, we may assume that $\beta$ is real and greater than 1 in the proof of the second part. The assumption $1\in \A[\beta]$ implies that $\Lambda>0$. Thus, there are at least three elements in the alphabet $\A$, because $\A\ni\phi(\Lambda,\dots,\Lambda)\neq \lambda$ and $\A\ni\phi(\Lambda,\dots,\Lambda)\neq \Lambda$ by Lemma \ref{lem:alphabetRestrictions}. It also implies that there are at least two representatives modulo $\beta-1$ in the alphabet in the class which contains $\Lambda$, since $\phi(\Lambda,\dots,\Lambda)\equiv_{\beta-1} \Lambda$.  

If $\lambda\equiv_{\beta-1}\Lambda$, there must be one more element of the alphabet $\A$ in this class, since $\lambda \neq \Lambda$. Therefore, $\#\A\geq |m_\beta(1)|+2$. 

The case that $\lambda\not\equiv_{\beta-1}\Lambda$ is divided into two subcases. Suppose now that $\lambda\neq 0$. Then $\phi(\lambda,\dots,\lambda)\neq \lambda$ and hence there is one more element in the alphabet in the class containing $\lambda$. Thus, there are at least two congruence classes which contain at least two elements of the alphabet $\A$. Therefore, $\#\A\geq |m_\beta(1)|+2$.

%If $\lambda=0$, then all elements of $\A+\A$ are nonnegative and $\phi(b,\dots,b)\neq 0$ for all $b\in(\A+\A)\setminus 0$. Suppose for contradiction, that there is only $|m_\beta(1)|+1$ elements in $\A$ -- one in each congruence class modeulo $\beta-1$ and one more in the class which contains $\Lambda$. The set $\mathcal{D}=\{\phi(d,\dots,d)\colon d\in\Lambda+\A\}\subset\A$ has $|m_\beta(1)|+1$ elements, but none of them is congruent to 0 as there are nonzero and the class containing zero has only one element by the assumption. Therefore, the elements of the set $\mathcal{D}$ belong to only $|m_\beta(1)|-1$ congruence classes. Hence, there exists $e,f,g,h\in\A, e\neq f,g\neq h, h\neq f$ such that $\phi(e+\Lambda,\dots, e+\Lambda)\equiv_{\beta-1}\phi(f+\Lambda,\dots, f+\Lambda)$ and $\phi(g+\Lambda,\dots, g+\Lambda)=\phi(h+\Lambda,\dots, h+\Lambda)$. Since 
%$$
%e+\Lambda\equiv_{\beta-1}\phi(e+\Lambda,\dots, e+\Lambda)\equiv_{\beta-1}\phi(f+\Lambda,\dots, f+\Lambda)\equiv_{\beta-1}f+\Lambda
%$$
%and
%$$
%g+\Lambda\equiv_{\beta-1}\phi(g+\Lambda,\dots, g+\Lambda)=\phi(h+\Lambda,\dots, h+\Lambda)\equiv_{\beta-1}h+\Lambda\,,
%$$ 
%also $e\equiv_{\beta-1}f$ and $g\equiv_{\beta-1}h$ which is a contradiction. In the same manner, the assumption that there is no nonzero element of the alphabet $\A$ which is congruent to 0 leads to contradiction for arbitrarily large alphabet.

If $\lambda=0$, then all elements of $\A+\A$ are nonnegative and $\phi(b,\dots,b)\neq 0$ for all $b\in(\A+\A)\setminus 0$. Suppose for contradiction, that there is no nonzero element of the alphabet $\A$ congruent to 0. We know that there is at least one representative of each congruence class class modulo $\beta-1$ in $\A$ and at least two representatives in the congruence class which contains $\Lambda$. Let $k\in\NN$ denote the number of elements which are in $\A$ extra to one element in each congruence class, i.e., $\#A=|m_\beta(1)|+k$.  For $d\in\Lambda+\A$, the value $\phi(d,\dots,d)\in\A$ is not congruent to 0 as it is nonzero and the class containing zero has only one element by the assumption. Therefore, the values  $\phi(d,\dots,d)\in\A$ for $|m_\beta(1)|+k$ distinct letters $d\in\Lambda+\A$ belong to only $|m_\beta(1)|-1$ congruence classes. Hence, there exists $e_1,\dots e_k, e_{k+1}\in\A$, pairwise distinct, and $f_1,\dots f_k, f_{k+1}\in\A$ such that $e_i\neq f_i$ and $\phi(e_i+\Lambda,\dots, e_i+\Lambda)\equiv_{\beta-1}\phi(f_i+\Lambda,\dots, f_i+\Lambda)$ for all $i, 1\leq i\leq k+1$. Since 
$$
e_i+\Lambda\equiv_{\beta-1}\phi(e_i+\Lambda,\dots, e_i+\Lambda)\equiv_{\beta-1}\phi(f_i+\Lambda,\dots, f_i+\Lambda)\equiv_{\beta-1}f_i+\Lambda\,.
$$ 
also $e_i\equiv_{\beta-1}f_i$ for all $i, 1\leq i\leq k+1$. This is a contradiction since it implies that $\#\A=|m_\beta(1)|+k+1$. Hence, classes containing $\lambda$ and $\Lambda$ have both at least one more  element of the alphabet $\A$, i.e., $\#\A\geq |m_\beta(1)|+2$. 
\end{proof}

Note that the larger necessary alphabet for a base with a real conjugate greater than one is caused by the fact that minimal and maximal element of $\A$ both require another element in $\A$ which is in the same congruence class. This should be considered when an alphabet for such a base is generated.

Since the proof is based on Theorem~\ref{thm:representativesInAlphabet}, it cannot be easily extended to alphabets which are subsets of  $\Zomega$.




\section{$\beta$-norm}
The goal of this section is to construct a norm in $\Zomega$. We use the isomorphism with $\ZZ^d$ and some results from matrix theory. This norm is used for the proof of convergence of Phase~1 in Chapter~\ref{chap:convergence}.

First, we recall a simple way how a new norm can be constructed from another one.

\begin{lem}
\label{lem:vectNorm}
Let $\nu$ be a norm of the vector space $\CC^d$ and $P$ be a nonsingular matrix in $\CC^d$. Then the mapping $\mu:\CC^d\rightarrow \RR^+_0$ defined by $\mu(x)=\nu(Px)$ is also a norm of the vector space $\CC^d$.
\end{lem}
\begin{proof}
Let $x$ and $y$ be vectors in $\CC^d$ and $\alpha\in \CC$.  We use linearity of matrix multiplication, nonsingularity of matrix $P$ and the fact that $\nu$ is a norm to prove the following statements:
\begin{enumerate}
    \item $\mu(x)=\nu(Px)\geq 0\,,$
    \item $\mu(x)=0 \iff \nu(Px)=0 \iff Px=0 \iff x=0\,,$
    \item $\mu(\alpha x)=\nu(P(\alpha x))=\nu(\alpha Px)=|\alpha|\nu(Px)=|\alpha|\mu(x)\,,$
    \item $\mu(x+y)=\nu(P(x+y))=\nu(Px+Py)\leq \nu(Px)+\nu(Py)=\mu(x)+\mu(y)\,.$
\end{enumerate}
This  verifies that $\mu$ is a norm.
\end{proof}


Now we use Lemma \ref{lem:vectNorm} to define a new norm for a given diagonalizable matrix.
\begin{defn}
\label{def:newNorm}
Let $M\in\CC^{n\times n}$ be a diagonalizable matrix and $P\in\CC^{n\times n}$ be a nonsingular matrix which diagonalizes $M$, i.e., $M=P^{-1}DP$ for some diagonal matrix $D\in\CC^{n\times n}$. We define a vector norm $\norm{\cdot}{M}$ by  
\begin{equation}
\norm{x}{M}:=\norm{Px}{2}
\end{equation}
for all $x\in\CC^n$, where $\norm{\cdot}{2}$ is Euclidean norm. A matrix norm $\Mnorm{\cdot}{M}$ is induced by the norm $\norm{\cdot}{M}$, i.e, 
$$
\Mnorm{A}{M}=\sup_{\norm{x}{M}=1} \norm{Ax}{M}\,
$$
for all $A\in\CC^{n\times n}$.
\end{defn}

The following theorem is a known result from matrix theory -- for a given diagonalizable matrix, there is a matrix norm such that the norm of the matrix equals its spectral radius.
\komentar{je potreba nejakou citaci?}

\begin{thm}
\label{thm:norm}
Let $M\in\CC^{n\times n}$ be a diagonalizable matrix. Then %re exists a vector norm $\norm{\cdot}{M}$ such that 
$$
\rho(M)=\Mnorm{M}{M}\,,
$$
where $\rho(M)$ is the spectral radius of the matrix $M$. % and $\Mnorm{\cdot}{M}$ is the natural matrix norm induced by $\norm{\cdot}{M}$.
\end{thm}
\begin{proof}
First, we prove that $\Mnorm{M}{M}\geq\rho(M)$. For all eigenvalues $\lambda$ in the spectrum $\sigma(M)$ with a respective eigenvector $u$ such that $\norm{u}{M}=1$, we have
$$
\Mnorm{M}{M}=\sup_{\norm{x}{M}=1} \norm{Mx}{M}\geq \norm{Mu}{M}=\norm{\lambda u}{M}=|\lambda|\cdot\norm{u}{M}=|\lambda|\,.
$$
Secondly, we show that $\Mnorm{M}{M}\leq\rho(M)$. Following Definition~\ref{def:newNorm}, let $P\in\CC^{n\times n}$ be a  nonsingular matrix  and $D\in\CC^{n\times n}$ a diagonal matrix  with the eigenvalues of $M$ on the diagonal such that $PMP^{-1}=D$.

Let $y$ be a  vector such that $\norm{y}{M}=1$ and set $z=Py$. Notice that 
$$
\sqrt{z^*z}=\norm{z}{2}=\norm{Py}{2}=\norm{y}{M}=1\,.
$$
Consider
\begin{align*}
\norm{My}{M}&=\norm{PMy}{2}=\norm{DPy}{2}=\norm{Dz}{2}=\sqrt{z^*D^*Dz}\\
    &\leq \sqrt{\max_{\lambda\in\sigma(M)}|\lambda|^2 z^*z}=\max_{\lambda\in\sigma(M)}|\lambda|=\rho(M)\,\,.
\end{align*}
which implies the statement.
\end{proof}


Before we define a norm in $\Zomega$, we verify that a specific matrix given by an algebraic number $\beta\in\Zomega$ is diagonalizable. Lemma~\ref{lem:propertiesSbeta} summarizes also some other properties of this matrix and the norm which it induces according to Theorem~\ref{thm:norm}.
\begin{lem}
\label{lem:propertiesSbeta}
Let $\omega$ be an algebraic integer of degree $d$ and let $S$ be the companion matrix of its minimal polynomial $m_\omega$. Let $\beta=\sum_{i=0}^{d-1} b_i \omega^i$, where $b_i \in \ZZ$, be a nonzero element of $\Zomega$. Set $S_\beta=\multMat{b}$. Then
\begin{enumerate}[i)]
   \item The matrix $S_\beta$ is diagonalizable.
   \item The characteristic polynomial of $S_\beta$ is $m_\beta^k$ with $k=d / \deg \beta$.
   \item $|\det S_\beta|=|m_\beta(0)|^k$.
   \item $\norm{x}{S_\beta}=\norm{x}{S_\beta^{-1}}$ for all $x \in \CC^d$ and $\Mnorm{X}{S_\beta}=\Mnorm{X}{S_\beta^{-1}}$ for all $X \in \CC^{d\times d}$.
   \item $\Mnorm{S_\beta}{S_\beta}=\max \{|\beta'| \colon \beta' \text{ is conjugate of } \beta\}$ and $ \Mnorm{S_\beta^{-1}}{S_\beta}=\max \{\frac{1}{|\beta'|} \colon \beta' \text{ is conjugate of } \beta\}$.
\end{enumerate}  
\end{lem}
\begin{proof}
The characteristic polynomial of the companion matrix $S$ is the same as minimal polynomial of $\omega$ which has no multiple roots. Hence, $S$ is diagonalizable, i.e., $S=P^{-1}DP$ where $D$ is diagonal matrix with the conjugates of $\omega$ on the diagonal and $P$ is a nonsingular complex matrix. The matrix $S_\beta$ is also diagonalized by $P$:
$$
S_\beta=\sum_{i=0}^{d-1} b_i S^i= \sum_{i=0}^{d-1} b_i \left(P^{-1}DP\right)^i=P^{-1}\underbrace{\left(\sum_{i=0}^{d-1} b_i D^i\right)}_{D_\beta}P\,.
$$
There is a result in theory of algebraic numbers about conjugates under a field isomorphism -- let $\alpha$ be an algebraic number with a conjugate $\alpha'$ and $\sigma:\QQ(\alpha)\rightarrow\QQ(\alpha')$ be a field isomorphism. If $\gamma\in\QQ(\alpha)$, then $\sigma(\gamma)$ is a conjugate of $\gamma$. \komentar{je potreba nejakou citaci?}

Therefore, the diagonal elements of the diagonal matrix $D_\beta$ are conjugates of $\beta$. Since $S_\beta\in\ZZ^{d\times d}$, its characteristic polynomial $p_{S_\beta}$ has integer coefficients. There exists $k\in\NN, k\geq 1$ such that $p_{S_\beta}=m^k_\beta$ as all roots of $p_{S_\beta}$ must be conjugates of $\beta$. The value $k$ follows from the equality $d=\deg(m_\beta^k)=k \deg m_\beta$. 

The modulus of the determinant of $S_\beta$ equals the modulus of the absolute coefficient of the characteristic polynomial $p_{S_\beta}$ which is $|m_\beta(0)|^k$.

The matrix $S_\beta^{-1}$ is also diagonalized by $P$ since $S_\beta^{-1}=(P^{-1}D_\beta P)^{-1}=P^{-1}D_\beta^{-1}P$. Thus, the norms $\norm{\cdot}{S_\beta}$ and $\norm{\cdot}{S_\beta^{-1}}$ are the same and so are the induced matrix norms $\Mnorm{\cdot}{S_\beta}$ and $\Mnorm{\cdot}{S_\beta^{-1}}$.

The matrix $S_\beta$ is diagonalizable and its eigenvalues are the conjugates of $\beta$. Theorem~\ref{thm:norm} implies that 
$$
\Mnorm{S_\beta}{S_\beta}=\rho(S_\beta)= \max \{|\beta'| \colon \beta' \text{ is conjugate of } \beta\}\,. 
$$
For the second part of the last statement, we use the part \textit{iv)}, Theorem~\ref{thm:norm} and the fact that the eigenvalues of $S_\beta^{-1}$ are  reciprocal of the conjugates of $\beta$.
\end{proof}

Finally, we may define a norm in $\Zomega$.
\begin{defn}
Let $\pi$ be the isomorphism between $\Zomega$ and $(\ZZ^d,+,\odot_\omega)$. Using the notation of the previous lemma, we define \emph{$\beta$-norm}  $\normBeta{\cdot}:\Zomega \rightarrow \RR^+_0$ by 
$$
\normBeta{x}=\norm{\pi(x)}{S_\beta}
$$
for all $x\in\Zomega$.
\end{defn}
We can easily verify that $\beta$-norm is a norm in $\Zomega$:
\begin{enumerate}
    \item $\normBeta{x}=\norm{\pi(x)}{S_\beta}\geq 0\,,$
    \item $\normBeta{x}=0 \iff \norm{\pi(x)}{S_\beta}=0 \iff \pi(x)=0 \iff x=0\,,$
    \item $\normBeta{\alpha x}=\norm{\pi(\alpha x)}{S_\beta}=|\alpha|\norm{\pi(x)}{S_\beta}=|\alpha|\normBeta{x}\,,$
    \item $\normBeta{x+y}=\norm{\pi(x+y)}{S_\beta}=\norm{\pi(x)+\pi(y)}{S_\beta}\leq \norm{\pi(x)}{S_\beta}+\norm{\pi(y)}{S_\beta}=\normBeta{x}+\normBeta{y}\,,$
\end{enumerate}
for all $x,y \in \Zomega$ and $\alpha \in \Zomega$.

The important property of $\beta$-norm is that there is only finitely many elements of $\Zomega$ which are bounded in this norm by a given constant. The explanation is following -- images of elements of $\Zomega$ under the isomorphism $\pi$ are integer vector and there are only finitely many integer vectors in any finite dimensional vector space bounded by any norm. It follows from equivalence of all norms a finite dimensional vector space.




\section{FAZE 1 IFF BETA EXPANDING}
\begin{theo}
\label{thm:betaExpanding}
    Let $\omega$ be a complex number and $\beta \in\Zomega$ be such that $|\beta|>1$. Let $\A\subset \Zomega$ be an alphabet. If $\NN\subset \A[\beta]$, number $\beta$ is expanding.
\end{theo}
\begin{proof}
For all $n\in\NN$ we may write 
    $$
    n=\sum_{i=0}^{N}a_i\beta^i\,,
    $$
    where $N\in\NN$, $a_i\in\A$ and $a_N\neq 0$.
    
    Set $m:= \max\{|a|\colon a\in\A\}$. We take $n\in\NN$ such that $n>m$. 
    Since $|a_0|\leq m<n$, we have  $N\geq 1$ and there is $i_0 \in \{1,2,\dots,N\}$ such that $a_{i_0}\neq 0$. Thus, $\omega$ is an algebraic number as $a_i\in\A\subset\Zomega$ and $\beta$ can be expressed as an integer combination of powers of $\omega$. Therefore, $\beta$ is also an algebraic number.
    
    Let $\beta'$ be an algebraic conjugate of $\beta$.  
    Since $\beta\in\Zomega\subset\QQ(\omega)$, there is an algebraic conjugate $\omega'$ of $\omega$ and an isomorphism $\sigma: \QQ(\omega)\rightarrow \QQ(\omega')$ such that $\sigma(\beta)=\beta'$. Now 
    $$
    n=\sigma(n)=\sum_{i=0}^{N}\sigma(a_i)(\beta')^i\,.
    $$
    Set $\tilde m:= \max\{|\sigma(a)|\colon a\in\A\}$.  For all $n\in\NN$, we have 
    $$
    n=|n|\leq\sum_{i=0}^{N}|\sigma(a_i)|\cdot|\beta'|^i \leq \sum_{i=0}^{\infty}|\sigma(a_i)|\cdot|\beta'|^i \leq \tilde m\sum_{i=0}^{\infty}|\beta'|^i\,.  
    $$
    Hence, the sum on the right  side diverges which implies that $|\beta'|\geq 1$. Thus, all conjugates of $\beta$ are at least one in modulus.
    
    If the degree of $\beta$ is one, the statement is obvious.  Therefore, we may assume that $\deg \beta \geq 2$. 
    
    Suppose  for contradiction that $|\beta'|=1$ for an algebraic conjugate $\beta'$  of $\beta$. The complex conjugate $\overline{\beta'}$ is also an algebraic conjugate of $\beta$. Take any algebraic conjugate $\gamma$ of $\beta$ and the isomorphism $\sigma': \QQ(\beta')\rightarrow \QQ(\gamma)$ given by $\sigma'(\beta')=\gamma$.
    Now
    $$
    \frac{1}{\gamma}=\frac{1}{\sigma'(\beta')}=\sigma'\left(\frac{1}{\beta'}\right)=\sigma'\left(\frac{\overline{\beta'}}{\beta'\overline{\beta'}}\right)=\sigma'\left(\frac{\overline{\beta'}}{|\beta'|^2}\right)=\sigma'(\overline{\beta'})\,.
    $$
    Hence, $\frac{1}{\gamma}$ is also an algebraic conjugate of $\beta$. From the previous, $\left|\frac{1}{\gamma}\right|\geq 1$ and $|\gamma|\geq 1$ which implies that $|\gamma|=1$. We may set $\gamma=\beta$ which contradicts $|\beta|>1$. Thus all conjugates of $\beta$ are greater than one in modulus, i.e., $\beta$ is an expanding algebraic number.
\end{proof}



\begin{theo}
Let $\A\subset \Zbeta$ be an alphabet such that $1\in \A[\beta]$. If the extending window method with the rewriting rule $x-\beta$ converges for the numeration system $(\beta, \A)$, then the base $\beta$ is expanding and the alphabet $\A$ contains at least one representative of each congruence class modulo $\beta$ in $\Zbeta$. 
\end{theo}
\begin{proof}
The existence of an algorithm for addition which is computable in parallel implies that the set $\fin{\A}$ is closed under addition. Moreover, the set $\A[\beta]$ is closed under addition since there is no carry to the right when the rewriting rule $x-\beta$ is used. For any $n\in\NN$, the sum $1+1+\cdots +1=n$ is in $\A[\beta]$ by the assumption $1\in \A[\beta]$. Therefore, $\NN\subset \A[\beta]$ and thus the base $\beta$ is expanding by Theorem~\ref{thm:betaExpanding}.

TOHLE ASI STEJNE NEMA MOC SMYSL TAM DAVAT POKUD TO NEPUJDE ZOBECNIT NA MOD V ZOMEGA (I NA ZBETA SE MUSI PRIDAT, ZE BETA JE ALG INTEGER):

In order to prove the second part, we have to show that for every $x=\sum_{i=0}^N x_i \beta^i \in\Zbeta$ there exists $q\in\Zbeta$ and $a\in\A$ such that $x=a+\beta q$. A representation of $x=\sum_{i=0}^N x'_i \beta^i=\sum_{i=0}^N x_i \beta^i+k\cdot m_\beta(\beta)$ such that $x'_0> 0$ can be found by adding an integer multiple of the minimal polynomial $m_\beta$ evaluated in $\beta$. Since $\NN\subset \A[\beta]$, we have
$$
x_0=\sum_{i=0}^{n}a_i\beta^i\,
    $$
    for some $n\in\NN$ and $a_i\in\A$. Now
    $$
    x=\underbrace{\sum_{i=0}^{n}a_i\beta^i}_{=x_0} + \sum_{i=1}^N x_i \beta^i=\underbrace{a_0}_{\in\A} + \beta \underbrace{\left(\sum_{i=0}^{n-1}a_{i+1}\beta^i + \sum_{i=0}^{N-1} x_{i+1} \beta^i\right)}_{=q\in\Zbeta}\,.
    $$
\end{proof}

\begin{lem}
\label{lem:suffCondPhase1}
    Let $\omega$ be an algebraic integer, $\deg \omega=d$, and $\beta$ be an expanding algebraic integer in $\Zomega$. Let $\A$ and $\B$ be finite subsets of $\Zomega$ such that $\A$ contains at least one representative of each congruence class modulo $\beta$ in $\Zomega$. Then there exists a finite set $\Q\subset\Zomega$  such that $ \B + \Q \subset \A + \beta \Q$.
\end{lem}

\begin{proof}
We use the isomorphism $\pi:\Zomega \rightarrow \ZZ^{d}$ and $\beta$-norm $\normBeta{\cdot}$ to bound the elements of $\Zomega$.
Let $\gamma$ be the smallest conjugate of $\beta$ in modulus. 
 Denote $C:=\max\{\normBeta{b-a}\colon a \in \A, b \in \B\}$. Consequently, set $R:=\frac{C}{|\gamma|-1}$ and $\Q:=\{q\in\Zomega\colon \normBeta{q}\leq R\}$. By Lemma~\ref{lem:propertiesSbeta}, we have 
 $$
 \Mnorm{S_\beta^{-1}}{S_\beta}=\max \{\frac{1}{|\beta'|} \colon \beta' \text{ is conjugate of } \beta\}=\frac{1}{|\gamma|}\,.
 $$ 
 Also, $|\gamma|>1$ as $\beta$ is an expanding integer.  Since $C>0$, the set $\Q$ is nonempty. Any element $x=b+q \in \Zomega$ with $b\in\B$ and $q\in\Q$ can be written as $x=a+\beta q'$ for some $a\in\A$  and $q'\in\Zomega$ due to existence of a representative of each congruence class in $\A$. Using the isomorphism $\pi$, we may write $\pi(q')=S^{-1}_\beta \cdot \pi(b-a+q)$. We prove that $q'$ is in $Q$:
\begin{align*}
    \normBeta{q'}&=\norm{\pi(q')}{S_\beta}=\norm{S^{-1}_\beta \cdot \pi(b-a+q)}{S_\beta}\leq \Mnorm{S^{-1}_\beta}{S_\beta}  \normBeta{b-a+q} \\
    &\leq  \frac{1}{|\gamma|} (\normBeta{b-a} +\normBeta{q})=\frac{1}{|\gamma|} (C+R)=\frac{C}{|\gamma|} (1+\frac{1}{|\gamma|-1}) =R\,.
\end{align*}
 
 Hence $q'\in\Q$ and thus  $x=b+q \in \A + \beta \Q$. 
 
 Since there are only finitely many elements of $\ZZ^{d}$ bounded by the constant $R$, the set $Q$ is finite.
\end{proof}

\begin{theo}
\label{thm:suffCondPhase1}
Let $\omega$ be an algebraic integer and $\beta\in\Zomega$. Let the alphabet $\A\subset\Zomega$ be such that $\A$ contains at least one representative of each congruence class modulo $\beta$ in $\Zomega$. Let $\B\subset\Zomega$ be the input alphabet. 

If $\beta$ is expanding, Phase 1 of the extending window method converges.
\end{theo}
\begin{proof}
We have the constant $R$ and finite set $\Q$ from Lemma \ref{lem:suffCondPhase1} for the alphabet $\A$ and input alphabet $\B$. We prove by induction that  all intermediate weight coefficient sets $\Q_k$ in Algorithm \ref{alg:weightCoefSet} are subsets of the finite set $\Q$. 

We start with $\Q_0=\{0\}$ which is bounded by any positive constant. Suppose that the intermediate weight coefficients set $\Q_k$ has elements bounded by the constant $R$. We see from the previous  proof that the candidates obtained by Algorithm \ref{alg:searchCand} for the set $\Q_k$ are also bounded by $R$. Thus, the next intermediate weight coefficients set $\Q_{k+1}$ has elements bounded by the constant $R$, i.e., $\Q_{k+1}\subset\Q$. 

Since $\#\Q$ is finite and $\Q_0\subset\Q_1\subset\Q_2\subset\cdots\Q$,  Phase 1 successfully ends. 
\end{proof}




\section{Convergence Phase 2}
sdv
\komentar{vstupy bbb -- odkaz na vyzkumak}
\komentar{tady je potreba zabrat!!!!}




\chapter{Design and implementation}
\komentar{castecne vyuzit z vyzkumaku??}
\komentar{pridat poznamku o google tabulce, hromadne testovani}

\chapter{Testing}

%\chapter*{Summary}
\newpage
\bibliography{literatura}
\addcontentsline{toc}{chapter}{References}
\bibliographystyle{amsplain}

\end{document}