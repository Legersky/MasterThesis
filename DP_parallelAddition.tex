\documentclass[11pt,a4paper]{report}	%,twoside,openright
%git-latexdiff 17807bf54f868 --main DP_parallelAddition.tex --cleanup none --latex



\usepackage{a4wide,cite}
\usepackage[english]{babel}
\usepackage[utf8]{inputenc}
% \usepackage[IL2]{fontenc}  
% \usepackage{amsmath,amsthm} 
% \usepackage{amsfonts}


\usepackage[fixlanguage]{babelbib}
\selectbiblanguage{english}



% fix left/right margins
%\let\tmp\oddsidemargin
%\let\oddsidemargin\evensidemargin
%\let\evensidemargin\tmp
%\reversemarginpar

\usepackage{amsmath, amsthm, amssymb, units, dsfont}
% \usepackage[nomessages]{fp}
\usepackage{sidecap}
\usepackage{enumerate}
\usepackage{xcolor}

\usepackage{mathtools, mathdots}
\usepackage{pgffor}
\usepackage{pdflscape}
\usepackage{afterpage}
\usepackage{chngcntr}
\usepackage{multirow}
\usepackage{tabulary}
\usepackage{listings}


\usepackage{breqn}
\usepackage{hyperref}

\newcommand{\komentar}[1]{\textcolor{red}{\MakeUppercase{#1}} \newline}
\newenvironment{upravit}{\color{blue}}{}

\newcommand{\Zomega}{\mathbb{Z}[\omega]}
\newcommand{\Zbeta}{\mathbb{Z}[\beta]}

\newcommand{\ZZ}{\mathbb{Z}}
\newcommand{\QQ}{\mathbb{Q}}
\newcommand{\CC}{\mathbb{C}}
\newcommand{\NN}{\mathbb{N}}
\newcommand{\RR}{\mathbb{R}}

% \newcommand{\OO}{\mathbb{O}}
\newcommand{\II}{\mathbb{I}}

\newcommand{\A}{\mathcal{A}}
\newcommand{\B}{\mathcal{B}}
\newcommand{\Q}{\mathcal{Q}}

\newcommand{\Qw}[3][w]{\Q_{[#1_{-#2}, \dots, #1_{-#3}]}}
\newcommand{\Qwo}[2][w]{\Q_{[#1_{0}, \dots, #1_{-#2}]}}

\newcommand{\tuple}[3][w]{(#1_{-#2}, \dots, #1_{-#3})}
\newcommand{\tupleo}[2][w]{(#1_{0}, \dots, #1_{-#2})}

%\newcommand{\Qb}[1]{\mathcal{Q}_{[b^{#1}]}}
\newcommand{\Qb}[1]{\mathcal{Q}_{[\scriptstyle b]}^{\scriptstyle #1}}

\newcommand{\fin}[1]{\text{Fin}_{#1}(\beta)}

\newcommand{\multMat}[1]{\sum_{i=0}^{d-1} {#1}_i S^i}



\newcommand{\vect}[1]{\begin{pmatrix}
             {#1}_0 \\
             {#1}_1 \\
             \vdots \\
             {#1}_{d-1} 
             \end{pmatrix}}
             
\newcommand{\enum}[1]{({#1}_0,\ldots,{#1}_{d-1})}             

\newcommand{\vertiii}[1]{{\left\vert\kern-0.25ex\left\vert\kern-0.25ex\left\vert #1\right\vert\kern-0.25ex\right\vert\kern-0.25ex\right\vert}}
    
\newcommand{\norm}[2]{\left\lVert#1\right\rVert_{#2}}
\newcommand{\Mnorm}[2]{\vertiii{#1}_{#2}}
\newcommand{\normBeta}[1]{\norm{#1}{\beta}}
\newcommand{\MnormBeta}[1]{\Mnorm{#1}{\beta}}

\renewcommand\Re{\operatorname{Re}}
\renewcommand\Im{\operatorname{Im}}

\usepackage{algorithm}
\usepackage{algorithmic}
\renewcommand{\algorithmicrequire}{\textbf{Input:}}
\renewcommand{\algorithmicensure}{\textbf{Ouput:}}
\algsetup{indent=2em}

 \usepackage{pifont}
 \renewcommand\checkmark{\ding{51}}
 \newcommand\xmark{\ding{55}}

 \newcommand{\var}[1]{\textit{#1}}
 \newcommand{\fun}[2]{\textbf{#1}(\var{#2})}

 \def\changemargin#1#2{\list{}{\rightmargin#2\leftmargin#1}\item[]}
 \let\endchangemargin=\endlist 

 \newenvironment{method}[2]{
 \noindent \textbf{#1(}\textit{#2}\textbf{)}
 \vspace{-5pt}
 \begin{changemargin}{3em}{0em}}
 {\end{changemargin}}

\def\Cpp{{C\nolinebreak[4]\hspace{-.05em}\raisebox{.4ex}{\tiny\bf ++}}}

 \usepackage{pgfkeys}
 \pgfkeys{
  /phaseOnecaptions array/.is family, /phaseOnecaptions array,
  .unknown/.style = {\pgfkeyscurrentname/.initial = #1},
 }
 
 \newcommand\figurehascaptionOne[1]{\pgfkeys{/phaseOnecaptions array, #1}}
 \newcommand\getcaptionOne[1]{\pgfkeysvalueof{/phaseOnecaptions array/#1}}
 
 \pgfkeys{
  /phase2captions array/.is family, /phase2captions array,
  .unknown/.style = {\pgfkeyscurrentname/.initial = #1},
 }
 
 \newcommand\figurehascaptionTwo[1]{\pgfkeys{/phase2captions array, #1}}
 \newcommand\getcaptionTwo[1]{\pgfkeysvalueof{/phase2captions array/#1}}


% \hyphenation{coef-fi-cient}
% \hyphenation{Algorithm-For-Parallel-Addition}
% \hyphenation{Polynomial-Quotient-Ring}





    
\newtheorem{thm}{Theorem}[chapter]
\newtheorem{lem}[thm]{Lemma}

\theoremstyle{definition}
\newtheorem{defn}{Definition}[chapter]
\newtheorem{exmp}{Example}[chapter]

\begin{document}
\tableofcontents
\chapter*{List of symbols}
% \centering
\begin{tabular}{ll}
Symbol        & Description \\ \hline
$\NN$         & set of nonnegative integers $\{0,1,2,3,\dots\}$   \\
$\ZZ$         & set of integers $\{\dots,-2,-1,0,1,2,\dots\}$ \\
$\RR$           & set of real numbers \\
$\CC$           & set of complex numbers \\
$\QQ$           &set of rational numbers \\
$\QQ(\beta)$    &the smallest field containing  $\QQ$ and algebraic number  $\beta$ \\
$\#S$          & number of elements of the finite set $S$ \\
$C^*$            & complex conjugation and transposition of the complex matrix $C$ \\
\rule{0cm}{0cm}& \\
$m_\beta$       &monic minimal polynomial of the algebraic number $\beta$ \\
$\deg \beta$    &degree of the algebraic number $\beta$ (over $\QQ$)\\
\rule{0cm}{0cm}& \\
$(\beta,\A)$            & numeration system with the base $\beta$ and the alphabet $\A$\\
$(x)_{\beta,\A}$    &$(\beta,\A)$-representation of the number $x$\\
$\fin{\A}$          &set of all complex numbers with a finite $(\beta,\A)$-representation \\
$\A^\ZZ$        &set of all bi-infinite sequences of digits in $\A$\\
$\Zomega$       &the smallest ring containing $\ZZ$ and $\omega$ \\%&set of values of all polynomials with integer coefficients evaluated in $\omega$\\
$\pi$           &isomorphism from $\Zomega$ to $\ZZ^d$ ($d=\deg \omega$)\\
\rule{0cm}{0cm}& \\
$\B$            &alphabet of input digits\\
$q_j$           &weight coefficient for the $j$-th position \\
$\Q$            &weight coefficients set\\
$\Qwo{k}$ 		&set of possible weight coefficients \\
				&for the input digits $\tupleo{k}\in\B^{k+1}$ \\
%\rule{0cm}{0cm}& \\
%$\lfloor x \rfloor$ & floor function of the number $x$ \\  
%$\Re x$           & real part of the complex number $x$ \\
%$\Im x$           & imaginary part of the complex number $x$
\end{tabular}

\chapter*{Introduction}
Parallel addition have been studied as it is an important part of algorithms for multiplication and division. 
While carry propagation in the standard algorithms requires to compute digits of the sum of $x_n x_{n-1}\cdots x_1 x_0 \bullet$ and $y_n y_{n-1}\cdots y_1 y_0 \bullet$  one by one, a parallel algorithm determines the $j$-th digit of the sum just from the knowledge of fixed number of digits around $x_j$ and $y_j$. Thus it avoids the carry propagation and all digits digits of the result can be outputted at the same time. Besides theoretical reasons, parallel addition is used for instance in fast division algorithm in processors Pentium which is implemented in a numeration system with base 4 and digits $\{-2,-1,0,1,2\}$.

A parallel addition algorithm  was introduced by A. Avizienis in \cite{avizienis} in 1961. The algorithm works with an integer base $\beta\geq3$ and an alphabet $\{-a, \dots, 0, \dots a\}$ where $a\in\NN$ is such that $\beta/2 <a \leq \beta-1$. Later, C. Y. Chow and J. E. Robertson presented a parallel addition algorithm for  base 2 and alphabet $\{-1,0,1\}$ in \cite{chow}.

So-called non-standard numeration systems have been extensively studied. Non-standard means that a base is not a positive integer. An example of such system is the Penney numeration system with the base $i-1$ and alphabet $\{0,1\}$. Complex numbers can be represented by this system without separating real and imaginary part. As division algorithms are also developed for non-standard numeration systems, we focus on construction of a parallel addition algorithm.

The numeration system which allow parallel addition must be redundant, i.e, some numbers may have more than one represenation. P.~Kornerup studied how  redundancy is related to parallel addition \cite{kornerup}. 

C. Frougny, E. Pelantov\'a and M. Svobodov\'a provided parallel addition algorithms with an integer alphabet for all bases $\beta$ such that $|\beta|>1$ and no conjugate of $\beta$ equals 1 in modulus, see \cite{parAddNS}. Nevertheless, the alphabet is not minimal in general.
 The parallel addition algorithms for several bases (negative integers, complex numbers $-1+\imath, 2\imath$ and $\sqrt{2}\imath$, quadratic Pisot unit and  non-integer rationals) with minimal integer alphabet are presented in \cite{minAlph}.




\chapter{Preliminaries}

This chapter summarizes definitions related to numeration systems and parallel addition. We recall some known results on properties of a numeration system which allows parallel addition. At the end of the chapter, we define set $\Zomega$ and determine numeration systems within the scope of this thesis.
	\begin{upravit}
In this chapter, we recall few definitions and results connected to numeration systems and parallelism. We define the set $\Zomega$ for an algebraic integer $\omega$ and we prove that $\Zomega$ is isomorphic to $\ZZ^d$. This property is used in Theorem \ref{thm:divisibility} which is an important tool for divisibility in $\Zomega$. Division in $\Zomega$ is necessary for the extending window method described in Chapter \ref{chap:methodDescription}.

\section{Numeration systems}
Firstly, we give a general definition of numeration system.
\begin{defn}
  Let $\beta \in \CC, |\beta|>1$ and $\A \subset \CC$ be a finite set containing 0. A pair $(\beta, \A)$ is called a \emph{positional numeration system} with \emph{base} $\beta$ and \emph{digit set} $\A$, usually called \emph{alphabet}.
\end{defn}
So-called standard numeration systems have an integer base $\beta$ and an alphabet $\A$ which is a set of contiguous integers. We restrict ourselves to the base $\beta$ which is an algebraic integer and possibly non-integer alphabet $\A$. 
\begin{defn}
Let $(\beta, \A)$ be a positional numeration system.  We say that a complex number $x$ has a \emph{$(\beta, \A)$-representation} if~ there exist digits $x_n,x_{n-1}, x_{n-2},\dots \in\A, n\geq 0$ such that $x=\sum_{j=-\infty}^n x_j \beta^j$.
\end{defn}
 We write briefly a \emph{representation} instead of a $(\beta, \A)$-representation if the base $\beta$ and the alphabet $\A$ follow from context. 

\begin{defn}
Let $(\beta, \A)$ be a positional numeration system. The set of all complex numbers with a finite $(\beta, \A)$-representation is defined by
$$
    \fin{\A}:=\left\{\sum_{j=-m}^n x_j \beta^j\colon n, m \in \NN, x_j \in \A \right\}\,.
$$
\end{defn}
   
For  $x\in\fin{\A}$, we write 
$$
(x)_{\beta,\A}= 0^\omega x_n x_{n-1}\cdots x_1 x_0 \bullet x_{-1} x_{-2} \cdots x_{-m} 0^\omega\,,
$$ 
where $0^\omega$ denotes right, respectively left-infinite sequence of zeros. Notice that indices are decreasing from left to right as it is usual to write the most significant digits first. In what follows, we omit the starting and ending $0^\omega$ when we work with numbers in $\fin{\A}$. We remark that existence of an algorithm (standard or parallel) producing a finite $(\beta,\A)$-representation of $x+y$ where $x,y\in\fin{\A}$ implies that the set $\fin{\A}$ is closed under addition, i.e.,
$$
\fin{\A} + \fin{\A} \subset \fin{\A}\,.
$$ 

Designing an algorithm for parallel addition requires some redundancy in numeration system. According to \cite{redundant}, a numeration system $(\beta,\A)$ is called \emph{redundant} if there exists $x \in \fin{\A}$ which has two different $(\beta,\A)$-representations. For instance, the number 1 has $(2,\{-1,0,1\})$-representations $1\bullet$ and $1(-1)\bullet$.
Redundant numeration system can enable us to avoid carry propagation in addition. On the other hand, there are some disadvantages. For example, comparison is problematic.  


\section{Parallel addition}
A local function, which is also often called a sliding block code, is used to mathematically formalize parallelism. 
\begin{defn}
Let $\A$ and $\B$ be alphabets. A function $\varphi:\B^\ZZ \rightarrow \A^\ZZ$ is said to be \emph{$p$-local} if there exist $r,t\in\NN$ satisfying $p=r+t+1$ and a function $\phi: \B^p \rightarrow \A$ such that, for any $w=(w_j)_{j\in\ZZ}\in\B^\ZZ$ and its image $z=\varphi(w)=(z_j)_{j\in\ZZ}\in\A^\ZZ$, we have $z_j=\phi(w_{j+t},\cdots,w_{j-r})$ for every $j\in\ZZ$. The parameter $t$, resp. $r$, is called \emph{anticipation}, resp. \emph{memory}.
\end{defn}
This means that each digit of the image $\varphi(w)$ is computed from $p$ digits of $w$ in a sliding window. Suppose that there is a processor on  each position with access to $t$ input digits on the left and $r$ input digits on the right. Then computation of $\varphi(w)$, where $w$ is a finite sequence, can be done in constant time independent on the length of $w$.   
  
\begin{defn}
\label{def:digitSetConversion}
Let $\beta$ be a base and $\A$ and $\B$ two alphabets containing 0. A function $\varphi:\B^\ZZ\rightarrow \A^\ZZ$ such that
  \begin{enumerate}
      \item for any $w=(w_j)_{j\in\ZZ}\in\B^\ZZ$ with finitely many non-zero digits, $z=\varphi(w)=(z_j)_{j\in\ZZ}\in\A^\ZZ$ has only finite number of non-zero digits, and
      \item $\sum_{j\in\ZZ} w_j \beta^j= \sum_{j\in\ZZ} z_j \beta^j$
  \end{enumerate}
  is called \emph{digit set conversion} in base $\beta$ from $\B$ to $\A$. Such a conversion $\varphi$ is said to be \emph{computable in parallel} if $\varphi$ is a $p$-local function for some $p\in\NN$. 
\end{defn}
In fact, addition on $\fin{\A}$ can be performed in parallel if there is a digit set conversion from $\A+\A$ to $\A$ computable in  parallel as we can easily output digitwise sum of two $(\beta,\A)$-representations in parallel.   


We recall few results about parallel addition in a numeration system with an integer alphabet. C. Frougny, E. Pelantov\'a and M. Svobodov\'a proved  the following sufficient condition of existence of an algorithm for parallel addition in \cite{parAddNS}.
  \begin{thm}
  \label{thm:suffConjugates}
  Let $\beta\in\CC$ be an algebraic number such that $|\beta|>1$ and all its conjugates in modulus differ from 1. There exists an alphabet $\A$ of contiguous integers containing 0 such that addition on $\fin{\A}$ can be performed in parallel.
  \end{thm}
  The proof of the theorem provides the algorithm for the alphabet of the form $\{-a,-a+1, \dots,0,\dots,a-1,a\}$. But in general, $a$ is not minimal.
    
The same authors showed in \cite{kBlock} that the condition on the conjugates of the base $\beta$ is also necessary:
  \begin{thm}
  Let the base $\beta\in\CC, |\beta|>1,$ be an algebraic number with a conjugate $\beta'$ such that $|\beta'|=1$. Let $\A\subset\ZZ$ be an alphabet of contiguous integers containing 0. Then addition on $\fin{\A}$ cannot be computable in parallel.
  \end{thm}
  
The question of minimality of the alphabet is studied in \cite{minAlph}. The following lower bound for the size of the alphabet is provided:
  \begin{thm}
  \label{thm:lowerBoundAlphabet}
  Let $\beta\in\CC, |\beta|>1,$  be an algebraic integer with the minimal polynomial $p$. Let $\A\subset\ZZ$ be an alphabet of contiguous integers containing 0 and 1. If addition on $\fin{\A}$ is computable in parallel, then $\#\A \geq |p(1)|$. Moreover, if $\beta$ is a positive real number, $\beta>1$, then $\#\A \geq  |p(1)|+2$.
  \end{thm}
  

In this thesis, we work in a more general concept as we consider also non-integer alphabets. First, we recall the following definition.
\begin{defn}
Let $\omega$ be a complex number. The set of values of all polynomials with integer coefficients evaluated in $\omega$ is denoted by
$$
    \ZZ[\omega] =\left\{\sum_{i=0}^n a_i \omega^i\colon n\in\NN, a_i\in\ZZ \right\} \subset \QQ(\omega)\,.
$$
\end{defn}
 Notice that $\ZZ[\omega]$ is a commutative ring (for our purposes, a ring is associative under multiplication and there is a multiplicative identity).     
    
From now on, let $\omega$ be an algebraic integer  which generates the set $\Zomega$ and let the base $\beta\in\Zomega$ be such that $|\beta|>1$. We remark that $\beta$ is also an algebraic integer as all elements of $\Zomega$ are algebraic integers. Finally, let the alphabet $\A$ be a finite subset of $\Zomega$ such that $0\in\A$.

Few parallel addition algorithms for such numeration system with a non-integer alphabet were found ad hoc. We introduce the method for construction of the parallel addition algorithm for a given numeration system $(\beta,\A)$ in Chapter \ref{chap:methodDescription}. 
  


\section{\texorpdfstring{Isomorphism of $\Zomega$ and $\ZZ^{d}$}{Isomorphism of Z[omega] and Zd}}
The goal of this section is to show a connection between the ring $\Zomega$ and the set $\ZZ^d$. Using Theorem \ref{thm:divisibility}, division in $\Zomega$ can be replaced by searching for an integer solution of a linear system. This is used for the implementation of the extending window method.

First we recall the notion of companion matrix which we use to define multiplication in $\ZZ^d$. By the minimal polynomial of an algebraic integer, we always mean the monic minimal polynomial.  
\begin{defn}
Let $\omega$ be an algebraic integer of degree $d\geq 1$ with the  minimal polynomial $p(x)=x^d +p_{d-1}x^{d-1}+ \cdots + p_1 x+p_0 \in \ZZ[x]$. The matrix 
$$
S := \begin{pmatrix}
            0 & 0 & \cdots & 0 & -p_0 \\
            1 & 0 & \cdots & 0 & -p_1 \\
            0 & 1 & \cdots & 0 & -p_2 \\
            \vdots &   & \ddots & & \vdots \\
            0 & 0 & \cdots & 1 & -p_{d-1} 
            \end{pmatrix} \in \ZZ^{d\times d}
$$
is called \emph{companion matrix} of the minimal polynomial of $\omega$.
\end{defn}
In what follows, the standard basis vectors of $\ZZ^d$  are denoted by 
$$
e_0=\begin{pmatrix}
              1 \\
              0 \\
              0 \\
              \vdots \\
              0
              \end{pmatrix}, \\
e_1=\begin{pmatrix}
              0 \\
              1 \\
              0 \\
              \vdots \\
              0
              \end{pmatrix}, \dots ,\\
e_{d-1}=\begin{pmatrix}
              0 \\        
              \vdots \\
              0 \\
              0\\
              1
              \end{pmatrix}\,.             
$$
% We remark that 1 in $e_i$ is in the $(i+1)$-st row because the index corresponds to the power of a companion matrix in the following definition. 

\begin{defn}
Let $\omega$ be an algebraic integer of degree $d\geq 1$, let $p$ be its minimal polynomial and let $S$ be its companion matrix. We define the mapping $\odot_\omega: \ZZ^d \times \ZZ^d \rightarrow \ZZ^d$ by 
$$
u \odot_\omega v := \left(\multMat{u}\right)\cdot \vect{v} \quad \text{ for all } u=\vect{u}, v=\vect{v} \in \ZZ^d\,.
$$ 
and we define powers of $u \in \ZZ^d$ by
\begin{align*}
    u^0&=e_0, \\
    u^{i}&= u^{i-1} \odot_\omega u \text{ for } i\in\NN\,.
\end{align*}
\end{defn}

We will see later that $\ZZ^d$ equipped with elementwise addition and multiplication $\odot_\omega$ builds a commutative ring. 
% It will follow from the isomorphism with $\Zomega$. 
Let us first recall an important property of a companion matrix  -- it is a root of its defining polynomial.
\begin{lem}
\label{lem:compMatrixIsRoot}
Let $\omega$ be an algebraic integer with a minimal polynomial $p$ and let $S$ be its companion matrix. Then
$$
p(S)=0\,.
$$
\end{lem}






Now we can prove that there is a correspondence between elements of $\Zomega$ and $\ZZ^d$.

\begin{thm}
Let  $\omega$ be an algebraic integer of degree $d$. Then 
$$
\Zomega =\left\{\sum_{i=0}^{d-1} a_i \omega^i \colon a_i\in\ZZ \right\},
$$ 
$(\ZZ^d,+,\odot_\omega)$ is a commutative ring and the mapping $\pi:\Zomega \rightarrow \ZZ^{d}$ defined by 
$$
\pi(u)=\vect{u} \quad \text{ for every } u=\sum_{i=0}^{d-1} u_i \omega^i \in \Zomega
$$
is a ring isomorphism.
\end{thm}


Due to this theorem we may work with integer vectors instead of elements of $\Zomega$ and multiplication in $\Zomega$ is replaced by multiplying by an appropriate matrix. 

The last theorem of this section is a practical tool for divisibility in $\Zomega$. To check whether an element of $\Zomega$ is divisible by another element, we look for an integer solution of a linear system. Moreover, this solution provides the result of  division in the positive case. 
\begin{thm}
\label{thm:divisibility}
Let $\omega$ be an algebraic integer of degree $d$ and let $S$ be the companion matrix of its minimal polynomial. Let $\beta=\sum_{i=0}^{d-1} b_i \omega^i$ be a nonzero element of $\Zomega$. Then for every $u\in\Zomega$
$$
u\in\beta\Zomega \iff S_\beta^{-1}\cdot \pi(u) \in \ZZ^d\,,
$$
where $S_\beta=\multMat{b}$.
\end{thm}

\end{upravit}




\chapter{Design of extending window method}
\label{chap:ewm}
In this chapter, the general concept of addition and digit set conversion is recalled. We outline a so-called \emph{extending window method} which is due to M. Svobodov\'a \cite{milena}. The method consists of two phases. For a given numeration system $(\beta,\A)$, it attempts to construct a digit set conversion algorithm which is computable in parallel. We recall that $\omega$ is an algebraic integer, $\beta \in \Zomega$ is a base and $0\in\A\subset\Zomega$ is an alphabet. 


	\section{Addition}

The general idea of addition (standard or parallel) in any numeration system $(\beta,\A)$ is the following: we sum up two numbers digitwise and then we convert the result with digits in $\A+\A$ into the alphabet $\A$. Obviously, digitwise addition is computable in parallel, thus the problematic part is the digit set conversion of the obtained result. It can be easily done in a standard way but a parallel digit set conversion is non-trivial. A parallel conversion is based on the same  formulas as the standard one but the choice of so-called \emph{weight coefficients} differs.

Now, we go step by step more precisely. Let $x,y \in \fin{\A}$ with $(\beta,\A)$-representations $x_{n'}x_{{n'}-1}\cdots x_1 x_0\bullet x_{-1} x_{-2} \cdots x_{-m'}$ and $y_{n'}y_{{n'}-1}\cdots y_1 y_0\bullet y_{-1} y_{-2} \cdots y_{-m'}$ padded by zeros to have the same length $n'+m'+1$. We set 
  \begin{align*}
    w&=x+y =\sum_{i=-m'}^{n'} x_i\beta^i + \sum_{i=-m'}^{n'} y_i\beta^i = \sum_{i=-m'}^{n'} (x_i+y_i)\beta^i \\
    &=\sum_{i=-m'}^{n'} w_i\beta^i \,,
  \end{align*}
  where $w_i=x_i+y_i \in \A +\A$. Thus, $w_{n'} w_{{n'}-1}\cdots w_1 w_0 \bullet w_{-1} w_{-2} \cdots w_{-m'}$ is a  $(\beta, \A+\A)$-representation of $w\in \fin{\A+\A}$. 

We also use column notation for digitwise addition in what follows, e.g.,    
%	\begin{center}
%	\begin{tabulary}{0.7\textwidth}{CCCCCCCCCCCL}
%	$x_{n'}$& $x_{{n'}-1}$ & $\cdots$ & $x_1$ &$x_0$ &$\bullet$  &$x_{-1}$ & $x_{-2}$ & $\cdots$ & $x_{-m'}$ & $=$& $(x)_{\beta,\A}$\\
%	$y_{n'}$& $y_{{n'}-1}$ & $\cdots$ & $y_1$ &$y_0$ &$\bullet$  &$y_{-1}$ & $y_{-2}$ & $\cdots$ & $y_{-m'}$ & $=$& $(y)_{\beta,\A}$\\ \hline
%	$w_{n'}$& $w_{{n'}-1}$ & $\cdots$ & $w_1$ &$w_0$ &$\bullet$  &$w_{-1}$ & $w_{-2}$ & $\cdots$ & $w_{-m'}$ & $=$& $(x+y)_{\beta,\A+\A}$\\\,.
%	\end{tabulary}	
%	\end{center}
 
  \begin{align*}
  x_{n'} \;x_{{n'}-1}\cdots x_1 \;x_0 &\bullet x_{-1} \;x_{-2}\, \cdots x_{-m'} \\[-3pt]
  y_{n'} \;y_{{n'}-1}\cdots y_1 \,\;y_0 &\bullet y_{-1} \;y_{-2} \;\cdots y_{-m'} \\[-10pt]
    \line(1,0){90} & \line(1,0){100} \\[-7pt]
  w_{n'} w_{{n'}-1}\cdots w_1 w_0 &\bullet w_{-1} w_{-2} \cdots w_{-m'}\,.
  \end{align*}
  
We search for a $(\beta,\A)$-representation of $w$, i.e., a  sequence 
  $$z_{n} z_{n-1}\cdots z_1 z_0 z_{-1} z_{-2} \cdots z_{-m}$$ such that $z_j \in \A$ and
  $$
    z_{n} z_{n-1}\cdots z_1 z_0 \bullet z_{-1} z_{-2} \cdots z_{-m}=(w)_{\beta,\A}\,.
  $$
  Note that the index of the first, resp. last, non-zero digit of the converted representation $z_{n} z_{n-1}\cdots z_1 z_0 \bullet z_{-1} z_{-2} \cdots z_{-m}=(w)_{\beta,\A}$ may differ from the original representation $w_{n'} w_{{n'}-1}\cdots w_1 w_0 \bullet w_{-1} w_{-2} \cdots w_{-m'}$. We assume that $ n\geq n'$ and $m\geq m'$, otherwise we pad the converted representation by zeros.
  
   Multiplication of a representation $w_{n'} w_{n-1}\cdots w_1 w_0 \bullet w_{-1} w_{-2} \cdots w_{-m'}$ by a power of $\beta$ is obvious:
  $$
  \beta^m \cdot w_{n'} w_{n'-1}\cdots w_1 w_0 \bullet w_{-1} w_{-2} \cdots w_{-m'} = w_{n} w_{n'-1}\cdots w_1 w_0 w_{-1} w_{-2} \cdots w_{-m'} \bullet
  $$  
  and after conversion
  $$
  z_{n} z_{n-1}\cdots z_1 z_0 z_{-1} z_{-2} \cdots z_{-m'}\bullet\cdots z_{-m} = \beta^m \cdot z_{n} z_{n-1}\cdots z_1 z_0 \bullet z_{-1} z_{-2} \cdots z_{-m'}\,. 
  $$  
Hence, without lost of generality, we consider only conversion of so-called $\beta$-integers -- numbers from $\fin{\A+\A}$ whose representations have all digits with negative indices equal to zero.
%  Particularly, let $(w)_{\beta, \A+\A}=w_{n'} w_{{n'}-1}\cdots w_1 w_0 \bullet$. We search for a number $n \in \NN$ and \mbox{$z_{n}, z_{n-1},\dots, z_1, z_0 \in\A$} such that $(w)_{\beta, \A}=z_{n} z_{n-1}\cdots z_1 z_0 \bullet$.   
  
  Digits $w_j$ are converted into the alphabet $\A$ by adding a suitable representation of zero digitwise.
  For our purpose, we use the simplest possible representation which is deduced from the polynomial
  $$
    x-\beta \in \left(\Zomega\right)[x]\,.
  $$

We remark that any polynomial $R(x)=r_s x^s+r_{s-1}x^{s-1}+ \dots + r_1 x+r_0$ with coefficients $r_i \in \Zomega$ such that $R(\beta)=0$ gives us a possible representation of zero. The polynomial $R$ is called a \emph{rewriting rule}. One of the coefficients of $R$ which is greatest in modulus (so-called \emph{core coefficient}) is used for the conversion of a digit $w_j$. Nevertheless, the extending window method is strongly dependent on the rewriting rule, so we focus only on the simplest possible rewriting rule $R(x)=x-\beta$. Usage of an arbitrary rewriting rule $R$ is out of scope of this thesis.

 
As $0=\beta^{j} \cdot R(\beta)=1\cdot \beta^{j+1} -\beta \cdot \beta^{j}$, we have a representation of zero 
$$1 (-\!\beta) \underbrace{0 \cdots 0}_{j}\bullet = (0)_\beta\,. $$
for all $j \in \NN$. We multiply this representation by $q_j \in \Zomega$, which is called a \emph{weight coefficient}, to obtain another  representation of zero 
%$$q_j (-q_j\beta) \underbrace{0 \cdots 0}_{j}\bullet = (0)_\beta\,. $$ 
$q_j (-q_j\beta) 0 \cdots 0\bullet = (0)_\beta\,. $
This is digitwise added to $w_{n} w_{n-1}\cdots w_1 w_0 \bullet$ to convert the digit $w_j$ into the alphabet $\A$. The conversion of $j$-th digit causes a \emph{carry} $q_{j}$ on the $(j+1)$-th position. 

In standard addition, the digit set conversion runs from the right ($j=0$) to the left until all non-zero digits and carries are converted into the alphabet $\A$:

	\begin{tabular}{rcccccccclcl}
	$w_{n'}$ & $w_{{n'}-1}$ & $\cdots$ & $w_{j+1}$ &\textcolor{red}{$w_{j}$} &$w_{j-1}$ & $\cdots$ & $w_1$ &$w_0$ &$\bullet$ & $=$&$(w)_{\beta,{\A+\A}}$\\
	  &   & &   &  &$q_{j-2}$ & $\iddots$ &  &  &\\
	   &   & &   & \textcolor{red}{$q_{j-1}$}  & $-\beta q_{j-1}$ &  &  &  &\\
	  &   &  $\iddots$&  $q_{j}$ &  \textcolor{red}{$-\beta q_{j}$} & & &  &  &\\ \cline{1-10}
	$z_{n} \cdots z_{n'}$ & $z_{{n'}-1}$ & $\cdots$ & $z_{j+1}$ &\textcolor{red}{$z_{j}$} &$z_{j-1}$ & $\cdots$ & $z_1$ &$z_0$ &$\bullet$  & $=$&$(w)_{\beta,{\A}}$\\
	\end{tabular}

%        \begin{align}
%        \label{eq:conversionScheme}
%            \hspace{100pt}  w_n w_{n-1}&&&\cdots& &w_{j+1}&\!\! &\textcolor{red}{w_j}  & \!\!  &w_{j-1} &&\cdots &&w_1 w_0\bullet \hspace{100pt} \notag\\[-5pt]
%                         &&&&       &       & &     &   &q_{j-2} &&\iddots  \notag \\[-3pt] 
%                         &&&&       &       & &\textcolor{red}{q_{j-1}}& -&\beta q_{j-1} \notag \\[-3pt]
%                         &&&&         &q_j&   \textcolor{red}{-}&\textcolor{red}{\beta q_j} &&\\[-8pt]
%                         &&&  \iddots      &   -&\beta q_{j+1}&   &\ && \notag \\[-17pt]
%          \intertext{\hspace{60pt}\line(1,0){300}}
%          \notag \\[-30pt]
%           z_{n'} \cdots z_{n} z_{n-1}&&&\cdots& &z_{j+1}& &\textcolor{red}{z_j}& &z_{j-1} &&\cdots &&z_1 \; z_0\bullet \notag                  
%        \end{align}
    Hence, the desired formula for conversion on the $j$-th position is 
    \begin{equation*}
        z_j=w_j + q_{j-1} - q_j \beta
    \end{equation*}
    for $j \in \NN$. We set $q_{-1}=0$ as there is no carry from the right on the 0-th position.
    
     The terms carry and weight coefficient are related to a position: while $q_{j-1}$ is a carry from the right  and $q_j$ is a chosen weight coefficient on the $j$-th position, $q_j$ is a carry from the right on the $(j+1)$-th position etc.

We remark that the conversion with the rewriting rule $x-\beta$ prolongs the part of non-zero digits only to the left as there is no carry from the left. Thus, all digits with negative indices of the converted sequence are zero.


     The fact that the conversion preserves the value of $w$ follows from adding a representation of zero:
\begin{align}
\label{eq:valuePreserving}
    \sum_{j\geq 0} z_j \beta^j &=w_0 - \beta q_0 + \sum_{j> 0} (w_j + q_{j-1} - q_j \beta) \beta^j \notag\\
    &=\sum_{j\geq 0} w_j \beta^j + \sum_{j>0} q_{j-1} \beta^j - \sum_{j\geq 0} q_j \cdot \beta^{j+1}  \\
    &=\sum_{j\geq 0} w_j \beta^j + \sum_{j>0} q_{j-1} \beta^j - \sum_{j> 0} q_{j-1} \cdot \beta^j \notag\\
    &=\sum_{j\geq 0} w_j \beta^j = w\,. \notag
\end{align}

    The weight coefficient $q_j$ must be chosen so that the converted digit is in the alphabet~$\A$, i.e., 
    \begin{equation}
    \label{eq:conversionFormula}
        z_j=w_j + q_{j-1} - q_j \beta \in \A\,.
    \end{equation} 
    The choice of weight coefficients is a crucial part of the construction of addition algorithms which are computable in parallel. The extending window method determining weight coefficients for a given input is described in Section~\ref{sec:methodDescription}.
    
    
     On the other hand, the following example shows that determining weight coefficients is trivial for  numeration systems such that an alphabet contains right one representative of each class modulo $\beta$.
     
     \begin{exmp}
        Assume now a numeration system $(\beta, \A)$, where
  $$
    \beta \in \NN\,,\beta  \geq 2\,, \A = \{0, 1, 2,\dots, \beta -1 \}\,.
  $$ 
       Notice that
    $$
        z_j \equiv w_j+q_{j-1} \mod \beta\,. 
    $$
  
  There is only one representative of each class modulo  $\beta$ in the standard numeration system $(\beta, \A)$. Therefore, the digit $z_j$ is uniquely determined for a given digit $w_j \in \A+\A$ and carry $q_{j-1}$ and thus so is the weight coefficient $q_j$. This means that $q_j=q_j(w_j,q_{j-1})$ for all $j\geq 0$. Generally,
  $$
  q_j=q_j(w_j,q_{j-1}(w_{j-1},q_{j-2}))=\dots =q_j(w_j ,\dots , w_1, w_0)
  $$
  and
  $$
  z_j=z_j(w_j ,\dots , w_1, w_0)\,,
  $$
  which implies that addition runs in linear time. For instance, the carry $q_{j-1}=1$ propagates through the whole result when we sum up $(\beta-1)(\beta-1)\dots(\beta-1)\bullet$ and $1\bullet$.
     
     \end{exmp}
  
  We require that the digit set conversion from $\A+\A$ into $\A$ is computable in parallel, i.e., there exist constants $r,t \in \NN_0$ such that for all $j\geq 0$ is $z_j=z_j(w_{j+t},\dots,w_{j-r})$. Anticipation $t$ equals zero since we use the rewriting rule $x-\beta$. To avoid the dependency on all less significant digits, we need variety in the choice of the weight coefficient $q_j$. This implies that the used numeration system must be redundant.
  

\section{Extending window method}
\label{sec:methodDescription}
In order to construct a digit set conversion in numeration system $(\beta,\A)$ from $\A+\A$ to $\A$ which is computable in parallel, we consider a digit set conversion from an \emph{input alphabet} $\B$ such that $\A \subsetneq \B \subset \A+\A$ instead of the alphabet $\A+\A$.
As mentioned above, the key problem is to find for every $j\geq 0$ a weight coefficient $q_j$ such that 
    $$
        z_j=\underbrace{w_j}_{\in \B} + q_{j-1} - q_j \beta \in \A 
    $$  
    for any input $w_{n'}w_{n'-1}\dots w_1 w_0 \bullet=(w)_{\beta,\B}, w\in \fin{\B}$. We remark that the weight coefficient $q_{j-1}$ is determined by the input $w_{j-1}\dots w_1 w_0 \bullet$. For a digit set conversion with the rewriting rule $x-\beta$ to be computable in parallel, the digit $z_j$ is required to satisfy $z_j=z_j(w_{j},\dots,w_{j-r})$ for a fixed memory $r$ in $\NN$.
    
    Note that the digit $z_j$ is given by the input digit $w_j$ and carry $q_{j-1}$ which is determined by input digits  $w_{j-1} w_{j-2}\dots$. Thus, if we find a weight coefficient $q_j$ for all possible combinations of input digits $w_j w_{j-1} w_{j-2}\dots$, then the position $j$ is not important. Therefore, we may strongly simplify our notation if we omit $j$ in subscripts. From now on, $w_0\in\B$ is a converted digit, $w_{-1} w_{-2}\dots\in\B$ are digits on right, $q_{-1}\in\Zomega$ is a carry from the right and we search for a weight coefficient $q_0\in\Zomega$ such that 
    $$
    z_0=w_0 + q_{-1} - q_0 \beta \in \A\,.
    $$
    
   
    We introduce two definitions before we describe the extending window method.
    \begin{defn}
    \label{def:weightCoefficientsSet}
        Let $\B$ be a set such that $\A \subsetneq \B \subset \A+\A$. Then any finite set $\Q\subset\Zomega$ containing~0 such that 
        $$
            \B + \Q \subset \A + \beta \Q
        $$  
        is called a \emph{weight coefficients set}.
    \end{defn}
    We see that if $\Q$ is a weight coefficients set, then
        $$
        (\forall w_0 \in \B)(\forall q_{-1}\in\Q)(\exists q_0 \in \Q )(\underbrace{w_0 + q_{-1} - q_0 \beta}_{z_0} \in \A )\,.
        $$
    In other words, there is a weight coefficient $q_0 \in \Q$ for a carry from the right $q_{-1}\in \Q$ and a digit $w_0$ in the input alphabet $\B$ such that $z_0$ is in the alphabet $\A$.  Notice that  the  carry from the right for the rightmost non-zero digit of the converted sequence which is $0$ is in $\Q$ by the definition.
    \begin{defn}
    Let $r$ be an integer and $q:\B^{r} \rightarrow \Q$ be a mapping such that 
    $$
    w_0+ q(w_{-1}, \dots, w_{-r}) - \beta q\tupleo{(r-1)} \in \A
    $$
    for all $w_0,w_{-1}, \dots, w_{-r} \in \B$, and $q(0,0,\dots,0)=0$. Then $q$ is called \emph{weight function} and $r$~is called \emph{length of window}.    
    \end{defn}

 Having a weight function $q$, we define a function $\phi:\B^{r+1}\rightarrow \A$ by
    \begin{equation}
    \label{eq:localConversion}
        \phi(w_{0}, \dots, w_{-r})=w_0+ \underbrace{q(w_{-1}, \dots, w_{-r})}_{=q_{-1}} - \beta \underbrace{q\tupleo{(r-1)}}_{=q_0}=:z_0\,,
    \end{equation} 
    which verifies that the digit set conversion is indeed a $(r+1)$-local function with anticipation $0$ and memory $r$. The requirement of zero output of the weight function $q$ for the input of $r$ zeros guarantees that $\phi(0,0,\dots,0)=0$. Thus, the first condition of Definition~\ref{def:digitSetConversion} is satisfied. The second one follows from the equation \eqref{eq:valuePreserving}. 
    
Let us summarize the construction of the digit set conversion by the rewriting rule \mbox{$x-\beta$}. We need to find weight coefficients for all possible combinations of digits of the input alphabet~$\B$. The rewriting rules multiplied by the weight coefficients are digitwise added to an input sequence. In fact, it means that the equation  \eqref{eq:conversionFormula} is applied on each position. If the digit set conversion is computable in parallel, the weight coefficients are determined as the outputs of the weight function $q$ with some fixed length of window $r$.  

We search for a weight function $q$ for a given base $\beta$ and input alphabet $\B$ by the extending window method. It consists of two phases. First, we find some weight coefficients set $\Q$. We know that it is possible to convert an input sequence by choosing the weight coefficients from the set $\Q$. The set $\Q$ serves as the starting point for the second phase in which we increment the expected length of the window $r$ until the weight function $q$ is uniquely defined for each $\tupleo{(r-1)} \in \B^{r}$. Then, the local conversion is determined -- we use the weight function outputs as weight coefficients in the formula \eqref{eq:localConversion}.    

We describe the general concept of the extending window method in this chapter, while various possibilities of construction of sets during both phases are discussed in Chapter~\ref{chap:diffChoices}.
Note that  convergence of both phases is studied in Chapter~\ref{chap:convergence}.
      
\section{Phase 1 -- Weight coefficients set}
\label{subsec:phase1}
The goal of the first phase is to compute a weight coefficients set $\Q$, i.e., to find a set $\Q \ni 0$ such that 
$$
    \B + \Q \subset \A + \beta \Q\,.
$$  
We build a sequence $\Q_0, \Q_1, \Q_2,\dots$ iteratively so that we extend $\Q_k$ to $\Q_{k+1}$ in a way to cover all elements of the set $\B+\Q_k$ by elements of the extended set $\Q_{k+1}$, i.e.,
$$
\B+ \Q_k \subset \A + \beta \Q_{k+1}\,.
$$
This procedure is repeated until the extended weight coefficients set $\Q_{k+1}$ is the same as the previous set $\Q_{k}$. We remark that the expression ``a weight coefficient $q$ covers an element $x$'' means that there is a digit $a \in \A$ such that $x=a + \beta q$. 

In other words, we start with $\Q_0=\{0\}$ meaning that we search all weight coefficients $q_0$ necessary for digit set conversion for the case where there is no carry from the right, i.e., $q_{-1}=0$. We add them to the weight coefficients set $\Q_0$ to obtain the set $\Q_1$. Assume now that we have a set $\Q_k$ for some $k\geq 1$. The weight coefficients in $\Q_k$ now may appear as a carry $q_{-1}$. If there are no suitable weight coefficients $q_0$ in the weight coefficients set~$\Q_k$ to cover all sums of coefficients from $\Q_k$ and digits of the input alphabet $\B$, we extend $\Q_k$ to $\Q_{k+1}$ by  suitable coefficients. And so on until there is no need to add more elements, i.e., the extended set $\Q_{k+1}$ equals $\Q_k$. Then the weight coefficients set $\Q:=\Q_{k+1}$ satisfies Definition~\ref{def:weightCoefficientsSet}. 

For better understanding, see Figures~\ref{img:phase1img3}--\ref{img:phase1img12} in Appendix~\ref{app:phase1} which illustrate the construction of the weight coefficients set $\Q$ for the Eisenstein base and a complex alphabet. 

The precise description of the algorithm in a pseudocode is in Algorithm~\ref{alg:weightCoefSet}. Observe that extending $\Q_k$ to $\Q_{k+1}$ is not unique. Various methods of choice are described in Section~\ref{sec:methodsOne} in Algorithm~\ref{alg:extendWeightCoefSet}.
    



Section~\ref{sec:convergencePhase2} discusses the convergence of Phase 1, i.e. whether it happens that  $\Q_{k+1}=\Q_k$ for some  $k$.
    
\begin{algorithm}
  \caption{Search for weight coefficients set (Phase 1)}
    \label{alg:weightCoefSet}
  \begin{algorithmic}[1]
    \STATE $k:=0$ 
    \STATE $Q_0:=\{0\}$
    \REPEAT
     \STATE Extend $\Q_k$ to $\Q_{k+1}$ (by Algorithm~\ref{alg:extendWeightCoefSet}) so that $$\B+ \Q_k \subset \A + \beta \Q_{k+1}$$
     \vspace{-20pt}
      \STATE  $k:=k+1$
      \UNTIL{$\Q_k = \Q_{k+1}$}      
      \STATE $\Q:=\Q_k$
    \RETURN $\Q$
  \end{algorithmic}
\end{algorithm}


    

  
    
    





\section{Phase 2 -- Weight function}
\label{subsec:phase2}
    We want to find a length of the window $r$ and a weight function $q:\B^{r} \to \Q$. We start with the weight coefficients set $\Q$ obtained in Phase 1. The idea is to reduce necessary weight coefficients which are used to convert a given input digit up to a single value. This is done by enlarging the number of considered input digits, i.e. incrementing $r$.  If the window is extended to the right, we know more digits that cause a carry form the right. This may decrease the number of  possible carries from the right and hence, less weight coefficients to convert the input digit may be necessary.
     
    We introduce notation for sets of possible weight coefficients for given input digits.
        Let $\Q$ be a weight coefficients set and $w_0\in \B$. Denote by $\Q_{[w_0]}$ any set such that
        $$
            (\forall q_{-1} \in \Q)(\exists q_0 \in \Q_{[w_0]})(w_0 + q_{-1} - q_0 \beta \in \A)\,.
        $$
It means that we do not know any input digits on the right, therefore there might be any carry from the set $\Q$. However, we may determine a set $\Q_{[w_0]}$ of  weight coefficients which allow the conversion of $w_0$ to $\A$ since we know the input digit $w_0$.
        
        By induction with respect to $k \in \NN, k\geq 1$, for all $\tupleo{k}\in \B^{k+1}$ denote by $\Qwo{k}$ any subset of  $\Qwo{(k-1)}$ such that 
        $$
           (\forall q_{-1} \in \Qw{1}{k})(\exists q_0 \in \Qwo{k})(w_0 + q_{-1} - q_0 \beta \in \A)\,.
        $$
        
    
 
%    Recall the scheme \eqref{eq:conversionScheme} of the digit set conversion for better understanding of the notation and method:
%    \begin{align*}
%        \hspace{130pt}\cdots\; &w_{j+1}&\!\! &w_j  & \!\!  &w_{j-1}&\cdots w_{j-M+1} &w_{j-M}\cdots \hspace{130pt} \\[-3pt] 
%                         & &       & & & q_{j-2} \\[-3pt]
%                         & &       &q_{j-1}& -&\beta q_{j-1} \\[-3pt]
%                           &q_j&   -&\beta q_j &&\\[-3pt]      
%                           -&\beta q_{j+1}&   &  &&\\[-15pt]      
%    \intertext{\hspace{120pt}\line(1,0){250}} 
%          \vspace{-15pt}
%          \\[-30pt]
%     \cdots\; &z_{j+1}& &z_j& &z_{j-1}& \cdots z_{j-M+1}\; &z_{j-M}\cdots                     
%    \end{align*}     

Sets of possible weight coefficients and a weight function $q$ are constructed by Algorithm~\ref{alg:weightFunction}. The idea is to check all possible right carries $q_{-1}\in\Q$ and determine values $q_0\in\Q$ such that 
    $$
    z_0=w_0 + q_{-1} - q_0 \beta \in \A \,.
    $$  
    
    So we obtain a set $\Q_{[w_0]}\subset\Q$ of weight coefficients which are necessary to convert the digit~$w_0$ with any carry $q_{-1}\in\Q$. Assuming that we know the input digit $w_{-1}$, the set of possible carries from the right is also reduced to $\Q_{[w_{-1}]}$. Thus we may reduce the set $\Q_{[w_0]}$ to a set $\Qwo{1}\subset \Q_{[w_0]}$ which is necessary to cover all elements of $w_0 + \Q_{[w_{-1}]}$. 

In the $k$-th step, we search for a set $\Qwo{k}\subset\Qwo{(k-1)}$ such that 
               $$
              w_0 + \Qw{1}{k} \subset \A + \beta \Qwo{k}\,.
              $$
              The length of window is $k+1$, i.e., we know $k$ digits on the right. To  construct the set $\Qwo{k}$, we select from $\Qwo{(k-1)}$ such weight coefficients which are necessary to convert digit $w_0$ to the alphabet $\A$ with all possible carries from the set $\Qw{1}{k}$.
                 
    Proceeding in this manner may lead to a unique weight coefficient $q_0$ for enough long window.     
    If there is $r\in\NN, r\geq 1$ such that 
    $$
    \#\Qwo{(r-1)}=1
    $$
    for all $\tupleo{(r-1)} \in \B^r$, then the output $q\tupleo{(r-1)}$ is defined as the element of $\Qwo{(r-1)}$. 
    
    Similarly to Phase 1, the choice of $\Qwo{k}$ is not unique. We list different methods of choice in Section~\ref{sec:methodsTwo}, Algorithm~\ref{alg:minimalSet}.
    
    Figures~\ref{img:phase2img3}--\ref{img:phase2img7} in Appendix~\ref{app:phase2} illustrate the construction of the set $\Q_{[\omega,1,2]}$ for the Eisenstein numeration system.
    
        
    To verify that 
$$
	z_0=\phi(w_{0}, \dots, w_{-r})=w_0+ \underbrace{q\tuple{1}{r}}_{=q_{-1}} - \beta \underbrace{q\tupleo{(r-1)}}_{=q_0}
$$    
is in the alphabet $\A$, consider that $q_0=q\tupleo{(r-1)}$ is the only element of $\Qwo{(r-1)}$ which was constructed such that 
$$
w_0 + \Qw{1}{(r-1)} \subset \A +\beta \Qwo{(r-1)}\,.
$$
At the same time, $q_{-1}=q\tupleo{(r-1)}$ is the only element of $\Qw{1}{r}$ which is a subset of $\Qw{1}{(r-1)}$.

    

    
\begin{algorithm}
  \caption{Search for weight function $q$ (Phase 2)}
    \label{alg:weightFunction}
  \begin{algorithmic}[1]
    \REQUIRE{weight coefficients set $\Q$}
    \FORALL{$w_0 \in \B$} 
        \STATE Find set $\Q_{[w_0]} \subset \Q$ (by Algorithm~\ref{alg:minimalSet}) such that
          $$
          w_0 + \Q \subset \A + \beta \Q_{[w_0]}
          $$
    \ENDFOR
    \STATE $k:=0$
    \WHILE{$\max\{\#\Qwo{k}\colon \tupleo{k}\in \B^{k+1} \} > 1$}
        \STATE $k:= k +1$
        \FORALL{$\tupleo{k}\in \B^{k+1}$}
            \STATE Find set $\Qwo{k} \subset \Qwo{(k-1)}$ (by Algorithm~\ref{alg:minimalSet}) such that
              $$
              w_0 + \Qw{1}{k} \subset \A + \beta \Qwo{k}\,,
              $$
        \ENDFOR  
    \ENDWHILE  
    \STATE $r:= k+1$ 
    \FORALL{$\tupleo{(r-1)} \in \B^{r}$}  
        \STATE $q\tupleo{(r-1)}\in \B^{r}:=$ only element of $\Qwo{(r-1)}$
    \ENDFOR
    \RETURN $q$
  \end{algorithmic}
\end{algorithm}

Unfortunately, finiteness of Phase 2 is not guaranteed. But the non-convergence of Phase 2 with a specific property may be revealed by Algorithm~\ref{alg:oneletterSets} before the run of Phase 2 or by Algorithm~\ref{alg:checkCycles} during it. These algorithms are based on theorems in Chapter ~\ref{chap:convergence}. Modified Phase 2 which includes these algorithms can be found in Section~\ref{sec:modifiedPhase2}.



Notice that for a given length of window $r$, the number of calls of Algorithm~\ref{alg:minimalSet} within Algorithm~\ref{alg:weightFunction} is
$$
\sum_{k=0}^{r-1}  \#\B^{k+1} = \#\B \frac{\#\B^r-1}{\#\B-1}\,.
$$    
It implies that the time complexity grows exponentially. The required memory is also exponential because we have to store sets $\Qwo{k}$ at least for $k\in\{r-2, r-1\}$  for all $w_0,\dots, w_{-k} \in \B$.


\chapter{\texorpdfstring{Properties of $\Zomega$}{Properties of Z[omega]}}
We compile some properties  of the ring $\Zomega$ in this chapter. The isomorphism to $\ZZ^d$ is indicated and used for check of divisibility.  We review some results from matrix theory to introduce a norm in $\Zomega$. This norm is used in the proof of convergence of Phase~1 in Chapter~\ref{chap:convergence}. In the last section of this chapter, we show how the number of congruence classes modulo $\beta$ in $\Zomega$ is determined. That is used in the proof of a lower bound on the size of an alphabet which allows parallel addition in Section~\ref{sec:minimalAlphabet}.
	\begin{upravit}


\section{\texorpdfstring{Isomorphism of $\Zomega$ and $\ZZ^{d}$}{Isomorphism of Z[omega] and Zd}}
The goal of this section is to show a connection between the ring $\Zomega$ and the set $\ZZ^d$. Using Theorem \ref{thm:divisibility}, division in $\Zomega$ can be replaced by searching for an integer solution of a linear system. This is used for the implementation of the extending window method.

First we recall the notion of companion matrix which we use to define multiplication in $\ZZ^d$. By the minimal polynomial of an algebraic integer, we always mean the monic minimal polynomial.  
\begin{defn}
Let $\omega$ be an algebraic integer of degree $d\geq 1$ with the  minimal polynomial $p(x)=x^d +p_{d-1}x^{d-1}+ \cdots + p_1 x+p_0 \in \ZZ[x]$. The matrix 
$$
S := \begin{pmatrix}
            0 & 0 & \cdots & 0 & -p_0 \\
            1 & 0 & \cdots & 0 & -p_1 \\
            0 & 1 & \cdots & 0 & -p_2 \\
            \vdots &   & \ddots & & \vdots \\
            0 & 0 & \cdots & 1 & -p_{d-1} 
            \end{pmatrix} \in \ZZ^{d\times d}
$$
is called \emph{companion matrix} of the minimal polynomial of $\omega$.
\end{defn}
\komentar{zminit jak vypada charakteristicky polynom companion matrix}
In what follows, the standard basis vectors of $\ZZ^d$  are denoted by 
$$
e_0=\begin{pmatrix}
              1 \\
              0 \\
              0 \\
              \vdots \\
              0
              \end{pmatrix}, \\
e_1=\begin{pmatrix}
              0 \\
              1 \\
              0 \\
              \vdots \\
              0
              \end{pmatrix}, \dots ,\\
e_{d-1}=\begin{pmatrix}
              0 \\        
              \vdots \\
              0 \\
              0\\
              1
              \end{pmatrix}\,.             
$$
% We remark that 1 in $e_i$ is in the $(i+1)$-st row because the index corresponds to the power of a companion matrix in the following definition. 

\begin{defn}
Let $\omega$ be an algebraic integer of degree $d\geq 1$, let $p$ be its minimal polynomial and let $S$ be its companion matrix. We define the mapping $\odot_\omega: \ZZ^d \times \ZZ^d \rightarrow \ZZ^d$ by 
$$
u \odot_\omega v := \left(\multMat{u}\right)\cdot \vect{v} \quad \text{ for all } u=\vect{u}, v=\vect{v} \in \ZZ^d\,.
$$ 
and we define powers of $u \in \ZZ^d$ by
\begin{align*}
    u^0&=e_0, \\
    u^{i}&= u^{i-1} \odot_\omega u \text{ for } i\in\NN\,.
\end{align*}
\end{defn}

We will see later that $\ZZ^d$ equipped with elementwise addition and multiplication $\odot_\omega$ builds a commutative ring. 
% It will follow from the isomorphism with $\Zomega$. 
Let us first recall an important property of a companion matrix  -- it is a root of its defining polynomial.
\begin{lem}
\label{lem:compMatrixIsRoot}
Let $\omega$ be an algebraic integer with a minimal polynomial $p$ and let $S$ be its companion matrix. Then
$$
p(S)=0\,.
$$
\end{lem}






Now we can prove that there is a correspondence between elements of $\Zomega$ and $\ZZ^d$.

\begin{thm}
Let  $\omega$ be an algebraic integer of degree $d$. Then 
$$
\Zomega =\left\{\sum_{i=0}^{d-1} a_i \omega^i \colon a_i\in\ZZ \right\},
$$ 
$(\ZZ^d,+,\odot_\omega)$ is a commutative ring and the mapping $\pi:\Zomega \rightarrow \ZZ^{d}$ defined by 
$$
\pi(u)=\vect{u} \quad \text{ for every } u=\sum_{i=0}^{d-1} u_i \omega^i \in \Zomega
$$
is a ring isomorphism.
\end{thm}


Due to this theorem we may work with integer vectors instead of elements of $\Zomega$ and multiplication in $\Zomega$ is replaced by multiplying by an appropriate matrix. 

The last theorem of this section is a practical tool for divisibility in $\Zomega$. To check whether an element of $\Zomega$ is divisible by another element, we look for an integer solution of a linear system. Moreover, this solution provides the result of  division in the positive case. 
\begin{thm}
\label{thm:divisibility}
Let $\omega$ be an algebraic integer of degree $d$ and let $S$ be the companion matrix of its minimal polynomial. Let $\beta=\sum_{i=0}^{d-1} b_i \omega^i$ be a nonzero element of $\Zomega$. Then for every $u\in\Zomega$
$$
u\in\beta\Zomega \iff S_\beta^{-1}\cdot \pi(u) \in \ZZ^d\,,
$$
where $S_\beta=\multMat{b}$.
\end{thm}


\end{upravit}
	\section{$\beta$-norm}

\begin{lem}
\label{lem:vectNorm}
Let $\nu$ be a norm of the vector space $\CC^d$ and $P$ be a nonsingular matrix in $\CC^d$. Then the mapping $\mu:\CC^d\rightarrow \RR^+_0$ defined by $\mu(x)=\nu(Px)$ is also a norm of the vector space $\CC^d$.
\end{lem}
\begin{proof}
Let $x$ and $y$ be vectors in $\CC^d$ and $\alpha\in \CC$.  We use linearity of matrix multiplication, nonsingularity of matrix $P$ and the fact that $\nu$ is a norm to prove the following statements:
\begin{enumerate}
    \item $\mu(x)=\nu(Px)\geq 0\,,$
    \item $\mu(x)=0 \iff \nu(Px)=0 \iff Px=0 \iff x=0\,,$
    \item $\mu(\alpha x)=\nu(P(\alpha x))=\nu(\alpha Px)=|\alpha|\nu(Px)=|\alpha|\mu(x)\,,$
    \item $\mu(x+y)=\nu(P(x+y))=\nu(Px+Py)\leq \nu(Px)+\nu(Py)=\mu(x)+\mu(y)\,.$
\end{enumerate}
This  verifies that $\mu$ is a norm.
\end{proof}


Lemma \ref{lem:vectNorm} enables us to define a new norm.
\begin{defn}
\label{def:newNorm}
Let $M\in\CC^{n\times n}$ be a diagonalizable matrix and $P\in\CC^{n\times n}$ be a nonsingular matrix which diagonalizes $M$, i.e., $M=P^{-1}DP$ for some diagonal matrix $D\in\CC^{n\times n}$. We define a vector norm $\norm{\cdot}{M}$ by  
\begin{equation}
\norm{x}{M}:=\norm{Px}{2}
\end{equation}
for all $x\in\CC^n$, where $\norm{\cdot}{2}$ is Euclidean norm. A matrix norm $\Mnorm{\cdot}{M}$ is induced by the norm $\norm{\cdot}{M}$.
\end{defn}

 

\begin{thm}
\label{thm:norm}
Let $M\in\CC^{n\times n}$ be a diagonalizable matrix. Then %re exists a vector norm $\norm{\cdot}{M}$ such that 
$$
\rho(M)=\Mnorm{M}{M}\,,
$$
where $\rho(M)$ is the spectral radius of the matrix $M$. % and $\Mnorm{\cdot}{M}$ is the natural matrix norm induced by $\norm{\cdot}{M}$.
\end{thm}
\begin{proof}
First, we prove that $\Mnorm{M}{M}\geq\rho(M)$. For all eigenvalues $\lambda$ in the spectrum $\sigma(M)$ with a respective eigenvector $u$ such that $\norm{u}{M}=1$, we have
$$
\Mnorm{M}{M}=\sup_{\norm{x}{M}=1} \norm{Mx}{M}\geq \norm{Mu}{M}=\norm{\lambda u}{M}=|\lambda|\cdot\norm{u}{M}=|\lambda|\,.
$$
Secondly, we show that $\Mnorm{M}{M}\leq\rho(M)$. Following Definition~\ref{def:newNorm}, let $P\in\CC^{n\times n}$ be a  nonsingular matrix  and $D\in\CC^{n\times n}$ a diagonal matrix  with the eigenvalues of $M$ on the diagonal such that $PMP^{-1}=D$.

Let $y$ be a  vector such that $\norm{y}{M}=1$ and set $z=Py$. Notice that 
$$
\sqrt{z^*z}=\norm{z}{2}=\norm{Py}{2}=\norm{y}{M}=1\,.
$$
Consider
\begin{align*}
\norm{My}{M}&=\norm{PMy}{2}=\norm{DPy}{2}=\norm{Dz}{2}=\sqrt{z^*D^*Dz}\\
    &\leq \sqrt{\max_{\lambda\in\sigma(M)}|\lambda|^2 z^*z}=\max_{\lambda\in\sigma(M)}|\lambda|=\rho(M)\,\,.
\end{align*}
which implies the statement.
\end{proof}

\begin{lem}
\label{lem:propertiesSbeta}
Let $\omega$ be an algebraic integer of degree $d$ and let $S$ be the companion matrix of its minimal polynomial $m_\omega$. Let $\beta=\sum_{i=0}^{d-1} b_i \omega^i$, where $b_i \in \ZZ$, be a nonzero element of $\Zomega$. Set $S_\beta=\multMat{b}$. Then
\begin{enumerate}[i)]
   \item The matrix $S_\beta$ is diagonalizable.
   \item The characteristic polynomial of $S_\beta$ is $m_\beta^k$ with $k=d / \deg \beta$.
   \item $|\det S_\beta|=|m_\beta(0)|^k$.
   \item $\norm{x}{S_\beta}=\norm{x}{S_\beta^{-1}}$ for all $x \in \CC^d$ and $\Mnorm{X}{S_\beta}=\Mnorm{X}{S_\beta^{-1}}$ for all $X \in \CC^{d\times d}$.
   \item $\Mnorm{S_\beta}{S_\beta}=\max \{|\beta'| \colon \beta' \text{ is conjugate of } \beta\}$ and $ \Mnorm{S_\beta^{-1}}{S_\beta}=\max \{\frac{1}{|\beta'|} \colon \beta' \text{ is conjugate of } \beta\}$.
\end{enumerate}  
\end{lem}
\begin{proof}
The characteristic polynomial of the companion matrix $S$ is the same as minimal polynomial of $\omega$ which has no multiple roots. Hence, $S$ is diagonalizable, i.e., $S=P^{-1}DP$ where $D$ is diagonal matrix with the conjugates of $\omega$ on the diagonal and $P$ is a nonsingular complex matrix. The matrix $S_\beta$ is also diagonalized by $P$:
$$
S_\beta=\sum_{i=0}^{d-1} b_i S^i= \sum_{i=0}^{d-1} b_i \left(P^{-1}DP\right)^i=P^{-1}\underbrace{\left(\sum_{i=0}^{d-1} b_i D^i\right)}_{D_\beta}P\,.
$$
By Theorem CONJUGATES SE ZOBRAZUJI NA CONJUGATES, the diagonal elements of the diagonal matrix $D_\beta$ are conjugates of $\beta$. Since $S_\beta\in\ZZ^{d\times d}$, its characteristic polynomial $p_{S_\beta}$ has integer coefficients. There exits $k\in\NN, k\geq 1$ such that $p_{S_\beta}=m^k_\beta$ as all roots of $p_{S_\beta}$ must be conjugates of $\beta$. The value $k$ follows from the equality $d=\deg(m_\beta^k)=k \deg m_\beta$. 

The modulus of the determinant of $S_\beta$ equals the modulus of the absolute coefficient of the characteristic polynomial $p_{S_\beta}$ which is $|m_\beta(0)|^k$.

The matrix $S_\beta^{-1}$ is also diagonalized by $P$ since $S_\beta^{-1}=(P^{-1}D_\beta P)^{-1}=P^{-1}D_\beta^{-1}P$. Thus, the norms $\norm{\cdot}{S_\beta}$ and $\norm{\cdot}{S_\beta^{-1}}$ are the same and so are the induced matrix norms $\Mnorm{\cdot}{S_\beta}$ and $\Mnorm{\cdot}{S_\beta^{-1}}$.

The matrix $S_\beta$ is diagonalizable and its eigenvalues are the conjugates of $\beta$. Theorem~\ref{thm:norm} implies that 
$$
\Mnorm{S_\beta}{S_\beta}=\rho(S_\beta)= \max \{|\beta'| \colon \beta' \text{ is conjugate of } \beta\}\,. 
$$
For the second part of the last statement, we use the part \textit{iv)}, Theorem~\ref{thm:norm} and the fact that the eigenvalues of $S_\beta^{-1}$ are  reciprocal of the conjugates of $\beta$.
\end{proof}

\begin{defn}
Let $\pi$ be the isomorphism between $\Zomega$ and $(\ZZ^d,+,\odot_\omega)$. Using the notation of the previous lemma, we define \emph{$\beta$-norm}  $\normBeta{\cdot}:\Zomega \rightarrow \RR^+_0$ by 
$$
\normBeta{x}=\norm{\pi(x)}{S_\beta}
$$
for all $x\in\Zomega$.
\end{defn}
We can easily verify that $\beta$-norm is a norm in $\Zomega$:
\begin{enumerate}
    \item $\normBeta{x}=\norm{\pi(x)}{S_\beta}\geq 0\,,$
    \item $\normBeta{x}=0 \iff \norm{\pi(x)}{S_\beta}=0 \iff \pi(x)=0 \iff x=0\,,$
    \item $\normBeta{\alpha x}=\norm{\pi(\alpha x)}{S_\beta}=|\alpha|\norm{\pi(x)}{S_\beta}=|\alpha|\normBeta{x}\,,$
    \item $\normBeta{x+y}=\norm{\pi(x+y)}{S_\beta}=\norm{\pi(x)+\pi(y)}{S_\beta}\leq \norm{\pi(x)}{S_\beta}+\norm{\pi(y)}{S_\beta}=\normBeta{x}+\normBeta{y}\,,$
\end{enumerate}
for all $x,y \in \Zomega$ and $\alpha \in \Zomega$.


	\section{Number of congruence classes}
Congruence classes play important role in the structure of an alphabet which allows parallel addition, as we will see later. We have seen that the isomorphism with $\ZZ^d$ is an efficient tool for handling elements of $\Zomega$. We use it also for counting number of congruence classes. The definition of congruence in $\ZZ^d$ is following.
\begin{defn}
Let $M\in\ZZ^{d\times d}$ be a nonsingular integer matrix. Vectors $x,y \in \ZZ^d$ are \emph{congruent modulo $M$ in $\ZZ^d$} if $x-y \in M\ZZ^d$.
\end{defn}
Let $x,y,z\in\ZZ^d$. We verify that congruence modulo $M$ is an equivalence. 
\begin{enumerate}[i)]
	\item reflexivity: $x-x=0=M\cdot 0$,
	\item symmetry: if $\exists\, v \in\ZZ^d$ such that $x-y=M\cdot v$, then $y-x=M\cdot (-v)$,
	\item transitivity: if $\exists\, v,v'\in \ZZ^d$ such that $x-y=M\cdot v$ and $y-z=M\cdot v'$, then $z-x=(z-y)+(y-x)=M\cdot (v'+v)$.
\end{enumerate}
In Theorem~\ref{thm:numbCongruenceClasses}, we will see that a congruence class modulo $\beta$ in $\Zomega$ corresponds to a congruence class modulo $S_\beta$ in $\ZZ^d$, where we use the notation from the previous section. Therefore, we count number of congruence classes modulo a matrix $M$ in Lemma~\ref{lem:numCongruenceClasses}.
\begin{lem}
Let $M\in\ZZ^{d\times d}$ be a nonsingular integer matrix. The number of congruence classes modulo $M$ in $\ZZ^d$ is $|\det M|$.
\label{lem:numCongruenceClasses}
\end{lem}
\begin{proof}
Set $y_i:=M e_i$ for $i\in\{0, \dots, d-1 \}$ and 
$$
B_{\enum{\alpha}}:=\left\{\sum_{i=0}^{d-1} (\alpha_i + t_i) y_i \colon t_i \in [0,1)\right\}
$$
for $\enum{\alpha} \in \ZZ^d$. Obviously,
$$
\RR^d=\bigcup_{\enum{\alpha} \in \ZZ^d} B_{\enum{\alpha}}\,.
$$
For fixed $\enum{\alpha} \in \ZZ^d$, the number of points of $\ZZ^d$ in $B_{\enum{\alpha}}$  is the volume of $B_{\enum{\alpha}}$  which is $|\det M|$. Hence, it is enough to prove that there is exactly one representative of each congruence class in $B_{\enum{\alpha}}$. 

To show that there are representatives of all classes, assume an arbitrary vector $x\in\ZZ^d$. Since $\enum{y}$ is a basis of $\RR^d$, there are scalars $s_0, \dots, s_{d-1}\in \RR$ such that $x= \sum_{i=0}^{d-1} s_i y_i$. Set $\gamma_i:=\lfloor s_i \rfloor$ and $t_i:=s_i-\gamma_i$. Now
\begin{align*}
 x=\sum_{i=0}^{d-1} (\gamma_i+t_i) y_i=\sum_{i=0}^{d-1}t_i y_i +\sum_{i=0}^{d-1} (\gamma_i-\alpha_i)y_i + \sum_{i=0}^{d-1} \alpha_i y_i=\underbrace{\sum_{i=0}^{d-1} (\alpha_i+t_i)y_i}_{\in B_{\enum{\alpha}} } +M\underbrace{(\gamma-\alpha)}_{\in\ZZ^d}\,,
\end{align*}
where $\alpha=\enum{\alpha}^T \quad \text{ and }\quad \gamma=\enum{\gamma}^T$. Hence, there is an integer vector $\sum_{i=0}^{d-1} (\alpha_i+t_i)y_i$ in $B_{\enum{\alpha}}$ which is congruent to $x$ modulo $M$.

Let $x'=\sum_{i=0}^{d-1} s'_i y_i \in \ZZ^d$ and $x''=\sum_{i=0}^{d-1}s''_i y_i \in \ZZ^d$ be distinct elements of $B_{\enum{\alpha}}$ which are congruent modulo $M$, i.e., there exists $z=\enum{z}^T\in\ZZ^d$ such that $x'=x''+M z$. There is $i_0\in\{0, \dots , d-1\}$ such that $|z_{i_0}|\geq 1$ as $x'\neq x''$. Thus, $|s'_{i_0}-s''_{i_0}|=|z_{i_0}|\geq 1$ which contradicts that  $x', x''\in B_{\enum{\alpha}}$.
\end{proof}
Now we compute number of congruence classes modulo $\beta$ in $\Zomega$ since two elements of $\Zomega$ are congruent modulo $\beta$ if and only if the corresponding vectors in $\ZZ^d$ are congruent modulo $S_\beta$.
\begin{thm}
Let $\omega$ be an algebraic integer of degree $d$ and  $\beta=\sum_{i=0}^d b_i \omega^i$, where $b_i\in\ZZ$, be such that $\deg \omega = \deg\beta$. The number of congruence classes modulo $\beta$ in $\Zomega$ is $|m_\beta(0)|$.
\label{thm:numbCongruenceClasses}
\end{thm}
\begin{proof}
Let $x, y\in\Zomega$ and let $S$ be the companion matrix of the minimal polynomial $m_\omega$. Set $S_\beta=\multMat{b}$. Then
\begin{align*}
x\equiv y \mod \beta &\iff \exists z \in \Zomega \colon x-y= \beta z \\
&\iff \exists z \in \Zomega \colon \pi(x-y)= S_\beta \pi(z) \\
&\iff \pi(x) \equiv \pi(y) \mod S_\beta\,.
\end{align*}
Thus, the number of congruence classes modulo $\beta$ is $|\det S_\beta|$ by Lemma~\ref{lem:numCongruenceClasses}. The statement follows from Lemma~\ref{lem:propertiesSbeta}. 
\end{proof}	
	
\chapter{Convergence}
We discuss convergence of the extending window method designed in Chapter~\ref{chap:ewm}. We prove a necessary condition of convergence of the whole method and we show that it is also sufficient condition of convergence of Phase~1, using the tools from the previous chapter. Next, we establishes a condition which implies non-convergence of Phase~2. The condition is formulated by existence of an infinite walk in a specific directed graph. We also prove that there is the same lower bound on the size of an alphabet from $\Zbeta$ as for integer alphabets when the extending method converges.
\label{chap:convergence}

	\section{Convergence of Phase 1}
	\begin{lem}
\label{lem:suffCondPhase1}
    Let $\omega$ be an algebraic integer, $\deg \omega=d$, and $\beta$ be an expanding algebraic integer in $\Zomega$. Let $\A$ and $\B$ be finite subsets of $\Zomega$ such that $\A$ contains at least one representative of each congruence class modulo $\beta$ in $\Zomega$. Then there exists a finite set $\Q\subset\Zomega$  such that $ \B + \Q \subset \A + \beta \Q$.
\end{lem}

\begin{proof}
We use the isomorphism $\pi:\Zomega \rightarrow \ZZ^{d}$ and $\beta$-norm $\normBeta{\cdot}$ to bound the elements of $\Zomega$.
Let $\gamma$ be the smallest conjugate of $\beta$ in modulus. 
 Denote $C:=\max\{\normBeta{b-a}\colon a \in \A, b \in \B\}$. Consequently, set $R:=\frac{C}{|\gamma|-1}$ and $\Q:=\{q\in\Zomega\colon \normBeta{q}\leq R\}$. By Lemma~\ref{lem:propertiesSbeta}, we have 
 $$
 \Mnorm{S_\beta^{-1}}{S_\beta}=\max \{\frac{1}{|\beta'|} \colon \beta' \text{ is conjugate of } \beta\}=\frac{1}{|\gamma|}\,.
 $$ 
 Also, $|\gamma|>1$ as $\beta$ is an expanding integer.  Since $C>0$, the set $\Q$ is nonempty. Any element $x=b+q \in \Zomega$ with $b\in\B$ and $q\in\Q$ can be written as $x=a+\beta q'$ for some $a\in\A$  and $q'\in\Zomega$ due to existence of a representative of each congruence class in $\A$. Using the isomorphism $\pi$, we may write $\pi(q')=S^{-1}_\beta \cdot \pi(b-a+q)$. We prove that $q'$ is in $Q$:
\begin{align*}
    \normBeta{q'}&=\norm{\pi(q')}{S_\beta}=\norm{S^{-1}_\beta \cdot \pi(b-a+q)}{S_\beta}\leq \Mnorm{S^{-1}_\beta}{S_\beta}  \normBeta{b-a+q} \\
    &\leq  \frac{1}{|\gamma|} (\normBeta{b-a} +\normBeta{q})=\frac{1}{|\gamma|} (C+R)=\frac{C}{|\gamma|} (1+\frac{1}{|\gamma|-1}) =R\,.
\end{align*}
 
 Hence $q'\in\Q$ and thus  $x=b+q \in \A + \beta \Q$. 
 
 Since there are only finitely many elements of $\ZZ^{d}$ bounded by the constant $R$, the set $Q$ is finite.
\end{proof}

\begin{theo}
\label{thm:suffCondPhase1}
Let $\omega$ be an algebraic integer and $\beta\in\Zomega$. Let the alphabet $\A\subset\Zomega$ be such that $\A$ contains at least one representative of each congruence class modulo $\beta$ in $\Zomega$. Let $\B\subset\Zomega$ be the input alphabet. 

If $\beta$ is expanding, Phase 1 of the extending window method converges.
\end{theo}
\begin{proof}
We have the constant $R$ and finite set $\Q$ from Lemma \ref{lem:suffCondPhase1} for the alphabet $\A$ and input alphabet $\B$. We prove by induction that  all intermediate weight coefficient sets $\Q_k$ in Algorithm \ref{alg:weightCoefSet} are subsets of the finite set $\Q$. 

We start with $\Q_0=\{0\}$ which is bounded by any positive constant. Suppose that the intermediate weight coefficients set $\Q_k$ has elements bounded by the constant $R$. We see from the previous  proof that the candidates obtained by Algorithm \ref{alg:searchCand} for the set $\Q_k$ are also bounded by $R$. Thus, the next intermediate weight coefficients set $\Q_{k+1}$ has elements bounded by the constant $R$, i.e., $\Q_{k+1}\subset\Q$. 

Since $\#\Q$ is finite and $\Q_0\subset\Q_1\subset\Q_2\subset\cdots\Q$,  Phase 1 successfully ends. 
\end{proof}




	\section{Convergence of Phase 2}
	\label{sec:convergencePhase2}

We have no simple sufficient or necessary condition for convergence of Phase~2 on properties of a base $\beta$ or an alphabet $\A$. Nevertheless, the non-convergence can be controled during a run of algorithm. An easy check of non-convergence can be done by finding $\Q_{[b, \dots, b]}$ for each $b\in\B$ separately. This was already described in \cite{vu}, but we recall it in this section with a simpler proof. For its purposes, we introduce a notion of stable Phase~2, which is used also in the main result of this section -- the control of non-convergence during Phase 2 is transformed into searching for a cycle in a directed graph.

Firstly, we mention some equivalent conditions of non-convergence of Phase~2. 
\begin{lem}
\label{lem:equivalentStatementsForNonConvergenePhaseTwo}
The following statements are equivalent:
\begin{enumerate}[i)]
	\item Phase~2 does not converge,
	\item $\forall \,k\in \NN \,\exists\, \tupleo{k}\in\B^{k+1} \colon \#\Qwo{k}\geq 2$,
	\item $\exists \,(w_{-j})_{j\geq 0}, w_{-j}\in\B \,\exists\, k_0\forall k\geq k_0 \colon \#\Qwo{k}=\#\Qwo{(k-1)}\geq 2$.
\end{enumerate}
\end{lem}
\begin{proof}
\textit{i)}$\iff$\textit{ii):} The while loop in Algorithm~\ref{alg:weightFunction} ends if and only if there exist $k\in\NN$ such that $\#\Qwo{k}=1$ for all $\tupleo{k}\in\B^{k+1}$.

\textit{ii)}$\iff$\textit{iii):} There is an infinite sequence $(w_{-k})_{k\geq 0}$ such that $\#\Qwo{k}\geq 2$ for all $k\in\NN$ since $\Qwo{k}\supset\Qwo{(k+1)}$. Hence, the sequence of integers $(\#\Qwo{k})_{k\geq 0}$ is eventually constant. The opposite implication is trivial.
\end{proof}

We need to ensure that choice of a possible weight coefficient set $\Qwo{k}\subset \Qwo{(k-1)}$ is determined by an input digit $w_0$ and a set $\Qw{1}{0}$, while the influence of the set $\Qwo{(k-1)}$ is limited. It is formalized in the following definition.

\begin{defn}
Let $\B$ be an alphabet of input digits. We say that Phase~2 is \emph{stable} if 
$$
\Qw{1}{k}=\Qw{1}{(k-1)} \implies \Qwo{k}=\Qwo{(k-1)}
$$
for all $k\in\NN, k\geq 2$ and for all $\tupleo{(k+1)} \in \B^{k+1}$.

%\komentar{ii)$\#\Qw{1}{k}=1 \implies \#\Qwo{k}=1$, neni potreba pokud nebude opacna implikace}
\end{defn}
The definition may seem too restrictive, but note that $\Qwo{k}$ if fully determined by $\Qw{1}{k}$, $\Qwo{(k-1)}$ and $w_0$ for a fixed deterministic way of choice of possible weight coefficients sets. Therefore, the assumption $\Qw{1}{k}=\Qw{1}{(k-1)}$, i.e. carries from the right are same, implies that the only difference in the choice of $\Qwo{k}$ an $\Qwo{(k-1)}$ is that $\Qwo{k}$ is a subset of  $\Qwo{(k-1)}$, while $\Qwo{(k-1)}$ is chosen as a subset of $\Qwo{(k-2)}$. At the same time, $\Qwo{(k-1)}$ is a subset of $\Qwo{(k-2)}$. Thus, the property that Phase~2 is stable means that the same possible weight coefficients set is found even if it is constructed as a subset of smaller set. This is natural way how an algorithm of choice should be constructed -- the set $\Qwo{k}$ is constructed such that
$$
\B+\Qw{1}{k} \subset \A+\beta \Qwo{k}\,,
$$
i.e., there is no reason to choose the set $\Qwo{k}$ to be a proper subset of $\Qwo{(k-1)}$ as we know that
$$
\B+\underbrace{\Qw{1}{(k-1)}}_{=\Qw{1}{k}} \subset \A+\beta \Qwo{(k-1)}\,,
$$
and $\Qw{1}{(k-1)}$ was chosen as sufficient.
In other words, if $\Qwo{k}\subsetneq\Qwo{(k-1)}$, the set $\Qwo{(k-1)}$ might have been chosen better. 

We may guarantee that Phase~2 is stable by wrapping the choice of the set $\Qwo{k}$ into a simple while loop, see Algorithm~\ref{alg:possibleWeightCoefSetStable} in Chapter~\ref{chap:design}.

Now we use that finiteness of Phase~2 implies that there exists a length of window $m$ such that the set $\Qb{m}$ contains only one element for all $b\in\B$, where $\Qb{m}$ is a shorter notation for
$$
\Q_{[\underbrace{\scriptstyle b,\dots,b}_m]}\,.
$$
The following theorem was proved in \cite{vu} with the assumption that Phase~2 is deterministic. Briefly, it says that $\#\Qb{m}$ must decrease every time we increase $m$, otherwise Phase~2 does not converge. When we consider only inputs of the form $bb\dots b$ for some $b\in\B$, determinism implies that Phase~2 is stable. The given proof with Phase~2 being stable is slightly shorter.

\begin{thm}
\label{thm:bbbCondition}
Let $m_0 \in \NN$ and $b\in\B$ be such that sets $\Qb{m_0}$ and $\Qb{m_0-1}$ produced by stable Phase~2 have the same size. Then
$$
    \#\Qb{m} = \#\Qb{m_0} \qquad \forall m\geq m_0-1\,.
$$ 
Particularly, if $\#\Qb{m_0}\geq 2$, Phase~2 does not converge.
\end{thm}
\begin{proof}
As $\Qb{m_0} \subset \Qb{m_0-1}$, the assumption of the same size implies
$$
    \Qb{m_0} = \Qb{m_0-1}\,.
$$
By the assumption that Phase~2 is stable, we have
\begin{align*}
 \Qb{m_0} = \Qb{m_0-1} &\implies  \Qb{m_0+1} = \Qb{m_0} \\
 						&\implies  \Qb{m_0+2} = \Qb{m_0+1} \\
 						&\vdots
\end{align*}
This implies the statement.

If $\#\Qb{m_0}\geq 2$, then the statement \textit{iii)} in Lemma~\ref{lem:equivalentStatementsForNonConvergenePhaseTwo} holds for the sequence $(b)_{k\geq 0}$.
\end{proof}



The condition of non-convergence during Phase~2 is formulated as a searching for an infinite path in a so-called Rauzy graph.
\begin{defn}
Let $\B$ be an alphabet of input digits. Let Phase~2 be stable. Let $k\in\NN, k\geq 2$. We set
$$
V:=\left\{\tuple{1}{k}\in\B^k \colon \#\Qw{1}{k}=\#\Qw{1}{(k-1)}\right\}
$$
and
$$
E:=\left\{\tuple{1}{k}\rightarrow\tuple[w']{1}{k}\in V\times V \colon \tuple{2}{k}=\tuple[w']{1}{(k-1)}\right\}\,.
$$
The directed graph $G_k=(V,E)$ is called \emph{Rauzy graph of Phase~2 (for the window of length~$k$)}.
\end{defn}
 This term comes from combinatorics on words. The vertices of our graph are combinations of input digits for which the size of their possible weight coefficients sets did not decrease with an increment of length of the window. Whereas in combinatorics on words, vertices are given as factors of some language. But the directed edges are placed in the same manner -- if some combination of digits without the first digit equals another combination without the last digit.

The structure of Rauzy graph $G_k$ signifies whether the non-decreasing combinations are such that they cause non-convergence of Phase~2. Existence of an infinite walk in $G_k$ implies that Phase~2 does not converge:

\begin{thm}
\label{thm:infinitePathInRauzyGraph}
Let Phase~2 is stable.  If there exists $k_0\in\NN, k_0\geq 2$, and $\tupleo{k_0}\in\B^{k_0+1}$ such that
\begin{enumerate}[i)]
	\item $\#\Qwo{(k_0-1)}>1$ and
	\item there exists an infinite walk $(\tuple[w^{(i)}]{1}{k_0})_{i\geq 1}$ in $G_{k_0}$ which starts in the vertex  $$\tuple[w^{(1)}]{1}{k_0}=\tuple{1}{k_0}\,,$$ 
\end{enumerate}
then Phase~2 does not converge.
\end{thm}
\begin{proof}
Set
$$(w_k)_{k\geq 0}:=w_0, w_1^{(1)}, \dots , w^{(1)}_{k_0-1}, w^{(1)}_{k_0}, w_{k_0}^{(2)},w_{k_0}^{(3)},w_{k_0}^{(4)},\dots$$
We prove that $\#\Qwo{k}=\#\Qwo{(k_0-1)}>1$ for all $k\geq k_0-1$, i.e., the condition \textit{iii)} in Lemma~\ref{lem:equivalentStatementsForNonConvergenePhaseTwo} is satisfied.

%The general ideas of the proof are following: if $\tuple[w']{1}{k_0}$ is a vertex of $G_{k_0}$, then $$\Qw[w']{1}{(k_0-1)}=\Qw[w']{1}{k_0}\,.$$ And the assumption that Phase~2 is stable implies that $$\Qw[w']{1}{(k_0-1)}=\Qw[w']{1}{k_0}\implies \Qwo[w']{(k_0-1)}=\Qwo[w']{k_0}$$ for all $w'_0\in\B$.

%Now we show that $\Qwo{k}=\Qwo{(k+1)}$ for all $k\geq k_0-1$. 
Let $\ell\in\NN$. Since $\tuple{(1+\ell)}{(k_0+\ell)}$ is a vertex of $G_{k_0}$, the set $\Qw{\ell}{(k_0+\ell)}$ equals $\Qw{\ell}{(k_0+\ell-1)}$. As Phase~2 is stable, we have
\begin{align*}
\Qw{\ell}{(k_0+\ell)}&=\Qw{\ell}{(k_0+\ell-1)}\\
\implies \Qw{(\ell-1)}{(k_0+\ell)}&=\Qw{(\ell-1)}{(k_0+\ell-1)}\\
&\vdots \\
\implies \Qw{1}{(k_0+\ell)}&=\Qw{1}{(k_0+\ell-1)}\\
\implies \Qwo{(k_0+\ell)}&=\Qwo{(k_0+\ell-1)}\,.
\end{align*}
Hence, $\#\Qwo{k}=\#\Qwo{(k_0-1)}>1$ for all $k\geq k_0-1$.
\end{proof}
%\komentar{bylo by fajn dokazat i opacny smer, jedine, pres co se neumim dostat je kdyby existovala jen aperiodicka posloupnoust, kvuli ktere to nekonverguje}



We remark that existence of an infinite walk in a finite graph is equivalent to existence of a cycle in the graph. Thus, if there is an infinite walk, we may find another one whose sequence of vertices is eventually periodic. We use this fact in Section~\ref{sec:modifiedPhase2} which describes modified algorithm for Phase~2 which implements the result of Theorem~\ref{thm:infinitePathInRauzyGraph}.










	

	For our purposes, we recall the proof of the following theorem for a parallel digit set conversion without anticipation. The full version can be found in \cite{minAlph}.
\begin{thm}
\label{thm:reprBetaMinusOne}
Let $\omega$ be an algebraic integer. Let the base $\beta\in\Zomega$ be such that $|\beta|>1$ and the alphabet $\A\subset\Zomega$ be such that $0\in\A$. If there exists a $p$-local digit set conversion defined by the function $\phi\colon (\A+\A)^p\rightarrow \A$ and $p=r+1$, then the number $\phi(b,\dots,b)-b$ belongs to the set $(\beta-1)\Zomega$ for any $b\in\A+\A$. 
\end{thm}
\begin{proof}
Let $b\in\A+\A$ and $a=\phi(b, \dots,b)$. For $n\in\NN, n\geq 1$, we denote $S_n$ the number represented by
$$
^{\omega}\!0 \underbrace{b\dots b}_{n}\bullet \underbrace{b\dots b}_{r}0^\omega\,.
$$
The representation of $S_n$ after the digit set conversion is of the form
$$
^{\omega}\!0 \underbrace{w_{r}\dots w_{1}}_{\beta^n W}\underbrace{a\dots a}_{n}\bullet \underbrace{\widetilde{w_1}\dots \widetilde{w_r}}_{\beta^{-r}\widetilde{W}}0^\omega\,,
$$
where 
$$W=\sum_{j=1}^r w_j \beta^{j-1} \qquad \text{and} \qquad \widetilde{W}=\sum_{j=1}^r\widetilde{w_j} \beta^{r-j}\,.$$
Since both representations have same value, we have
\begin{align}
\label{eq:reprBetaMinusOne}
b \sum_{j=-r}^{n-1} \beta^j &= W \beta^n + a \sum_{j=0}^{n-1} \beta^j + \beta^{-r}\widetilde{W} \notag \\
b \sum_{j=-r}^{-1} \beta^j +b\frac{\beta^n-1}{\beta-1} &= W \beta^n + a \frac{\beta^n-1}{\beta-1} + \beta^{-r}\widetilde{W}\,,
\end{align}
for all $n\geq 1$. We substract this equation for $n$ and $n-1$ to obtain
$$
b\frac{\beta^n-\beta^{n-1}}{\beta-1}=W(\beta^n-\beta^{n-1}) + a\frac{\beta^n-\beta^{n-1}}{\beta-1}\,.
$$
We simplify it to
\begin{equation}
\label{eq:reprBetaMinusOneFinal}
b=W(\beta-1) + a\,.
\end{equation}
Hence, $a=\phi(b, \dots,b)\equiv b$ modulo $\beta-1$.
\end{proof}

The following lemma is a modification of lemma in \cite{minAlph}.
\begin{lem}
\label{lem:alphabetRestrictions}
Let $\omega$ be a real algebraic integer and the base $\beta\in\Zomega$ be such that $\beta>1$. Let the alphabet $\A\subset\Zomega$ be such that $0\in\A$ and denote $\lambda=\min \A$ and $\Lambda=\max \A$. If there exists a $p$-local digit set conversion defined by the function $\phi\colon (\A+\A)^p\rightarrow \A$ and $p=r+1$, then:
\begin{enumerate}[i)]
	\item $\phi(b,\dots,b)\neq \lambda$ for all $b\in\A+\A$ such that $b>\lambda$.
	\item $\phi(b,\dots,b)\neq \Lambda$ for all $b\in\A+\A$ such that $b<\Lambda$.
	\item If $\Lambda\neq 0$, then $\phi(\Lambda,\dots,\Lambda)\neq \Lambda$.
	\item If $\lambda\neq 0$, then $\phi(\lambda,\dots,\lambda)\neq \lambda$.
\end{enumerate}
\end{lem}
\begin{proof}
To prove \textit{i)}, assume in contradiction that $\phi(b,\dots,b)= \lambda$. We proceed in the same manner as in Theorem \ref{thm:reprBetaMinusOne}, the equation \eqref{eq:reprBetaMinusOne} implies
$$
b \sum_{j=-r}^{-1} \beta^j +b\frac{\beta^n-1}{\beta-1} =  \beta^n W + \lambda \frac{\beta^n-1}{\beta-1} + \beta^{-r}\widetilde{W}\,.
$$
We apply also the equation c to obtain
\begin{align*}
b \sum_{j=-r}^{-1} \beta^j +b\frac{\beta^n-1}{\beta-1} &=  \beta^n \frac{b-\lambda}{\beta-1} + \lambda \frac{\beta^n-1}{\beta-1} + \beta^{-r}\widetilde{W}\,.
\end{align*}
Now we may simplify and estimate
\begin{align*}
b \sum_{j=-r}^{-1} \beta^j +\frac{-b}{\beta-1} &=  \frac{-\lambda}{\beta-1} + \beta^{-r}\sum_{j=1}^r\widetilde{w_j} \beta^{r-j} \\
b \underbrace{\left(\sum_{j=1}^{r} \frac{1}{\beta^j} -\frac{1}{\beta-1}\right)}_{-\sum_{j=r+1}^{\infty} \frac{1}{\beta^j}} &=  -\lambda\frac{1}{\beta-1} + \sum_{j=1}^r\widetilde{w_j} \beta^{-j}\geq  \lambda\underbrace{\left(-\frac{1}{\beta-1}+\sum_{j=1}^{r} \frac{1}{\beta^j} \right)}_{-\sum_{j=r+1}^{\infty}\frac{1}{\beta^j}}\,.
\end{align*}
Hence $b\leq\lambda$ which is a contradiction. The proof of \textit{ii)} can be done in the same way.

For \textit{iii)}, assume that $\phi(\Lambda,\dots,\Lambda)= \Lambda$. Now consider a number $T_q$ represented by
$$
^{\omega}\!0 \bullet \underbrace{\Lambda\dots\Lambda}_{r} \underbrace{(2\Lambda)\dots(2\Lambda)}_{q} 0^\omega\,.
$$
After the digit set conversion, a representation is
$$
^{\omega}\!0  \underbrace{w_r\dots w_1}_{W} \bullet z_1\dots z_{r+q} 0^\omega\,.
$$
The value $T_q$  preserves, thus,
$$
\Lambda\sum_{j=1}^r \beta^{-j} +2\Lambda \sum_{j=r+1}^{r+q} \beta^{-j}=W+\sum_{j=1}^{r+q}z_j\beta^{-j}\,.
$$
But $W=0$ from the equation \eqref{eq:reprBetaMinusOneFinal}. We estimate
\begin{align*}
\Lambda\sum_{j=1}^{r+q} \beta^{-j} +\Lambda \sum_{j=r+1}^{r+q} \beta^{-j}&=\sum_{j=1}^{r+q}z_j\beta^{-j}\leq \Lambda\sum_{j=1}^{r+q}\beta^{-j}\\
\Lambda \sum_{j=r+1}^{r+q} \beta^{-j}&\leq 0\,.
\end{align*}
This contradicts that $\Lambda$ positive as it is a nonzero, maximal element of the alphabet $\A$ which contains 0. The proof of \textit{iv)} is analogous.
\end{proof}


\begin{thm}
Let $\beta$ be an algebraic integer such that $|\beta|>1$. Let $0\in \A\subset \Zbeta$ be an alphabet such that $1\in \A[\beta]$. If addition in the numeration system $(\beta, \A)$ which uses the rewriting rule $x-\beta$ is computable in parallel, then the alphabet $\A$ contains at least one representative of each congruence class modulo $\beta$ and $\beta-1$ in $\Zbeta$. 
\label{thm:representativesInAlphabet}
\end{thm}
\begin{proof}
The existence of an algorithm for addition with the rewriting rule $x-\beta$ implies that the set $\A[\beta]$ is closed under addition. By the assumption $1\in \A[\beta]$, the set $\NN$ is subset of  $\A[\beta]$. Since $0\in\A$, we have $\beta \cdot \A[\beta] \subset \A[\beta]$. Hence, $\NN[\beta] \subset \A[\beta]$.

For any element  $x=\sum_{i=0}^N x_i \beta^i\in \Zbeta$ there is an element $x'=\sum_{i=0}^N x'_i \beta^i\in \NN[\beta]$ such that $x\equiv_\beta x'$  since $m_\beta (0)\equiv_\beta 0$ and $\beta^i\equiv_\beta 0$. As $x'\in \NN[\beta] \subset \A[\beta]$, we have
$$
x\equiv_\beta x'=\sum_{i=0}^{n}a_i\beta^i \equiv_\beta a_0\,,
$$
where $a_i\in \A$. Hence, for any element $x\in\Zomega$, there is a letter $a_0\in\A$ such that $x\equiv_\beta a_0$.

In order to prove that there is at least one representative of each congruence class modulo $\beta-1$ in the alphabet $\A$, we consider again an element $x=\sum_{i=0}^N x_i \beta^i\in \Zbeta$. Similarly, there is an element $x'=\sum_{i=0}^N x'_i \beta^i\in \NN[\beta]$ such that $x\equiv_{\beta-1} x'$  since $m_{\beta-1} (0)\equiv_{\beta-1} 0$ and $(\beta-1)^i\equiv_{\beta-1} 0$.

Since $x'\in \NN[\beta]\subset \A[\beta]$,
$$
x'=\sum_{i=0}^{n}a_i\beta^i\,,
$$
where $a_i\in \A$. We prove by induction with respect to $n$ that $x'\equiv_{\beta-1} a$ for some $a\in\A$.
If $n=0$, $x'=a_0$. Now we use the fact that if there is a parallel addition algorithm, for each letter $b \in\A+\A$, there is $a\in\A$ such that $b \equiv_{\beta-1} a$ (Theorem~\ref{thm:reprBetaMinusOne}). For $n+1$, we have
\begin{align*}
x'&=\sum_{i=0}^{n+1}a_i\beta^i =a_0 + \sum_{i=1}^{n+1}a_{i}\beta^i\\
    &=a_0 + \beta \sum_{i=0}^{n}a_{i+1}\beta^i - \sum_{i=0}^{n}a_{i+1}\beta^i+ \sum_{i=0}^{n}a_{i+1}\beta^i \\
    &\equiv_{\beta-1} a_0 + (\beta-1)\sum_{i=0}^{n}a_{i+1}\beta^i + a \equiv_{\beta-1}a_0 +a \equiv_{\beta-1}a' \in\A\,,
\end{align*}
where we use the induction assumption
$$
\sum_{i=0}^{n}a_{i+1}\beta^i\equiv_{\beta-1} a\,.
$$
\end{proof}

\begin{lem}
\label{lem:parAddAlgForConjugate}
Let $\omega$ be an algebraic integer with a conjugate $\omega'$. Let $\beta\in\Zomega, |\beta|>1$ and let $\sigma:\QQ(\omega)\rightarrow \QQ(\omega')$ be an isomorphism such that $|\sigma(\beta)|>1$. Let $\varphi$ is a digit set conversion  in the base $\beta$ from $\A+\A$ to $\A$. There exists  is a digit set conversion $\varphi'$ in the base $\beta'$ from $\A'+\A'$ to $\A'$ where $\beta'=\sigma(\beta)$ and $\A'=\{\sigma(a) \colon a\in\A\}$.
\end{lem}
\begin{proof}
Let $\phi:\A^p\rightarrow\A$ be a mapping which defines $\varphi$ with $p=r+t+1$. We define a mapping $\phi':\A^p\rightarrow \A$ by 
$$
\phi'(w'_{j+t}, \dots, w'_{j-r})=\sigma\left(\phi\left(\sigma^{-1}(w'_{j+t}), \dots, \sigma^{-1}(w'_{j-r})\right)\right)\,.
$$
Next, we define a digit set conversion  $\varphi':(\A'+\A')\rightarrow\A'$ by $\varphi'(w')=(z'_j)_{j\in\ZZ}$ where $w'=(w'_j)_{j\in\ZZ}$ and $z'_j=\phi'(w'_{j+t}, \dots, w'_{j-r})$. Obviously, if $w'$ has only finitely many nonzero entries, then there is only finitely many nonzeros in $(z'_j)_{j\in\ZZ}$   since
$$
\phi'(0, \dots, 0)=\sigma\left(\phi\left(\sigma^{-1}(0), \dots, \sigma^{-1}(0)\right)\right)=\sigma\left(\phi\left(0, \dots, 0\right)\right)=\sigma\left(0\right)=0\,.
$$
The value of the number represented by $w'$ is also preserved:
\begin{align*}
\sum_{j\in\ZZ}w'_j \beta'^j&=\sum_{j\in\ZZ}\sigma(w_j) \sigma(\beta)^j=\sigma(\sum_{j\in\ZZ}w_j\beta^j)\\
&=\sigma(\sum_{j\in\ZZ}z_j\beta^j)=\sigma(\sum_{j\in\ZZ}\phi\left(w_{j+t}, \dots,w_{j-r}\right)\beta^j)\\
&=\sum_{j\in\ZZ}\sigma(\phi\left(w_{j+t}, \dots,w_{j-r}\right))\beta'^j=\sum_{j\in\ZZ}z'_j \beta'^j
\end{align*}
where $w_j=\sigma^{-1}(w'_j)$ for $j\in\ZZ$ and $\varphi((w_j)_{j\in\ZZ})=(z_j)_{j\in\ZZ}$.
\end{proof}

\begin{thm}
Let $\beta$ be an algebraic integer such that $|\beta|>1$. Let $0\in \A\subset \Zbeta$ be an alphabet such that $1\in \A[\beta]$. If addition in the numeration system $(\beta, \A)$ which uses the rewriting rule $x-\beta$ is computable in parallel, then
$$
\#\A \geq \max \{|m_\beta(0)|, |m_\beta(1)|\}\,.
$$
Moreover, if $\beta$ is such that it has a real conjugate greater than 1, then 
$$
\#\A \geq \max \{|m_\beta(0)|, |m_\beta(1)|+2\}\,.
$$
\end{thm}
\begin{proof}
By Theorem~\ref{thm:representativesInAlphabet}, there are all representatives modulo $\beta$ and modulo $\beta-1$ in the alphabet $\A$. The numbers of congruence classes are $|m_\beta(0)|$ and $|m_{\beta-1}(0)|$ by Theorem~\ref{thm:numbCongruenceClasses}. Obviously, $m_{\beta-1}(x) = m_\beta (x+1)$. Thus $m_{\beta-1}(0) = m_\beta (1)$.

Let $\phi$ be a mapping which defines the parallel addition. According to Lemma~\ref{lem:parAddAlgForConjugate}, we may asssume that $\beta$ is real and greater than 1 in the proof of the second part. The assumption $1\in \A[\beta]$ implies that $\Lambda>0$. Thus, there are at least three elements in the alphabet $\A$, because $\A\ni\phi(\Lambda,\dots,\Lambda)\neq \lambda$ and $\A\ni\phi(\Lambda,\dots,\Lambda)\neq \Lambda$ by Lemma \ref{lem:alphabetRestrictions}. It also implies that there are at least two representatives modulo $\beta-1$ in the alphabet in the class which contains $\Lambda$, since $\phi(\Lambda,\dots,\Lambda)\equiv_{\beta-1} \Lambda$.  

If $\lambda\equiv_{\beta-1}\Lambda$, there must be one more element of the alphabet $\A$ in this class, since $\lambda \neq \Lambda$. Therefore, $\#\A\geq |m_\beta(1)|+2$. 

The case that $\lambda\not\equiv_{\beta-1}\Lambda$ is divided into two subcases. Suppose now that $\lambda\neq 0$. Then $\phi(\lambda,\dots,\lambda)\neq \lambda$ and hence there is one more element in the alphabet in the class containing $\lambda$. Thus, there are at least two congruence classes which contain at least two elements of the alphabet $\A$. Therefore, $\#\A\geq |m_\beta(1)|+2$.

%If $\lambda=0$, then all elements of $\A+\A$ are nonnegative and $\phi(b,\dots,b)\neq 0$ for all $b\in(\A+\A)\setminus 0$. Suppose for contradiction, that there is only $|m_\beta(1)|+1$ elements in $\A$ -- one in each congruence class modeulo $\beta-1$ and one more in the class which contains $\Lambda$. The set $\mathcal{D}=\{\phi(d,\dots,d)\colon d\in\Lambda+\A\}\subset\A$ has $|m_\beta(1)|+1$ elements, but none of them is congruent to 0 as there are nonzero and the class containing zero has only one element by the assumption. Therefore, the elements of the set $\mathcal{D}$ belong to only $|m_\beta(1)|-1$ congruence classes. Hence, there exists $e,f,g,h\in\A, e\neq f,g\neq h, h\neq f$ such that $\phi(e+\Lambda,\dots, e+\Lambda)\equiv_{\beta-1}\phi(f+\Lambda,\dots, f+\Lambda)$ and $\phi(g+\Lambda,\dots, g+\Lambda)=\phi(h+\Lambda,\dots, h+\Lambda)$. Since 
%$$
%e+\Lambda\equiv_{\beta-1}\phi(e+\Lambda,\dots, e+\Lambda)\equiv_{\beta-1}\phi(f+\Lambda,\dots, f+\Lambda)\equiv_{\beta-1}f+\Lambda
%$$
%and
%$$
%g+\Lambda\equiv_{\beta-1}\phi(g+\Lambda,\dots, g+\Lambda)=\phi(h+\Lambda,\dots, h+\Lambda)\equiv_{\beta-1}h+\Lambda\,,
%$$ 
%also $e\equiv_{\beta-1}f$ and $g\equiv_{\beta-1}h$ which is a contradiction. In the same manner, the assumption that there is no nonzero element of the alphabet $\A$ which is congruent to 0 leads to contradiction for arbitrarily large alphabet.

If $\lambda=0$, then all elements of $\A+\A$ are nonnegative and $\phi(b,\dots,b)\neq 0$ for all $b\in(\A+\A)\setminus 0$. Suppose for contradiction, that there is no nonzero element of the alphabet $\A$ congruent to 0. We know that there is at least one represenative of each congruence class class modulo $\beta-1$ in $\A$ and at least two representatives in the congruence class which contains $\Lambda$. Let $k\in\NN$ denote the number of elements which are in $\A$ extra to one element in each congruence class, i.e., $\#A=|m_\beta(1)|+k$.  For $d\in\Lambda+\A$, the value $\phi(d,\dots,d)\in\A$ is not congruent to 0 as it is nonzero and the class containing zero has only one element by the assumption. Therefore, the values  $\phi(d,\dots,d)\in\A$ for $|m_\beta(1)|+k$ distinct letters $d\in\Lambda+\A$ belong to only $|m_\beta(1)|-1$ congruence classes. Hence, there exists $e_1,\dots e_k, e_{k+1}\in\A$, pairwise distinct, and $f_1,\dots f_k, f_{k+1}\in\A$ such that $e_i\neq f_i$ and $\phi(e_i+\Lambda,\dots, e_i+\Lambda)\equiv_{\beta-1}\phi(f_i+\Lambda,\dots, f_i+\Lambda)$ for all $i, 1\leq i\leq k+1$. Since 
$$
e_i+\Lambda\equiv_{\beta-1}\phi(e_i+\Lambda,\dots, e_i+\Lambda)\equiv_{\beta-1}\phi(f_i+\Lambda,\dots, f_i+\Lambda)\equiv_{\beta-1}f_i+\Lambda\,.
$$ 
also $e_i\equiv_{\beta-1}f_i$ for all $i, 1\leq i\leq k+1$. This is a contradiction since it implies that $\#\A=|m_\beta(1)|+k+1$. Hence, classes containing $\lambda$ and $\Lambda$ have both at least one more  element of the alphabet $\A$, i.e., $\#\A\geq |m_\beta(1)|+2$. 
\end{proof}
\komentar{BYLO BY FAJN TO JESTE ZOBECNIT NA abecedu ze Z[OMEGA]}



\chapter{Different methods in the extending window method}
	\label{chap:diffChoices}
	We mentioned in Chapter~\ref{chap:ewm} that there is a lot of variability in both phases of the extending window method. Within this chapter, we propose various methods how an intermediate weight coefficients set $\Q_k$ can be extended to $\Q_{k+1}$ in Phase~2 and how the set $\Qwo{k}$ can be constructed in Phase~2. The selection of methods is based on testing many experimental ones. 

	\section{Different methods in Phase 1}
\label{sec:methodsOne}


We recall that the ambiguous part of Phase~1 is extending an intermediate weight coefficient set $\Q_k$ to $\Q_{k+1}$ so that 
$$
\B+\Q_k \subset \A+\beta \Q_{k+1}\,.
$$
It means that a weight coefficient $q$ such that $x=a+\beta q$ for some $a\in\A$ must be found for all $x\in \B+\Q_k$. Since the alphabet $\A$ is redundant, there might be more such weight coefficients. Let $C_x\subset \Zomega$ be a set of all these candidates for some $x\in \B+\Q_k$, i.e., $C_x=\{q\in\Zomega\colon x=a+\beta q, a\in\A\}$. Set $C:=\{C_x\colon x\in \B+\Q_k\}$. For a given $x\in \B+\Q_k$, the set $C_x$ is constructed so that all digits $a\in\A$ are tested whether $x-a$ is divisible by $\beta$ according to Theorem~\ref{thm:divisibility}. See Algorithm~\ref{alg:extendWeightCoefSet} which constructs the set $C$.

Now we extend the set $\Q_k$ to $\Q_{k+1}$ so that $C_x \cap \Q_{k+1} \neq \emptyset$ for all $C_x \in C$. We describe five methods how it can be done (1a, 1b, 1c, 1d and 1e).

As $Q_k\subset \Q_{k+1}$, set $\Q_{k+1}:=\Q_k$. For methods 1a, 1b and 1d, add  the element of all $C_x\in C$ such that $\#C_x=1$ to $\Q_{k+1}$. Next, select elements from all $C_x\in C$ such that $C_x \cap \Q_{k+1} = \emptyset$ according to a chosen method and add them to $\Q_{k+1}$. Selection for different methods is following:
\begin{enumerate}[ ]
	\item 1a -- all elements of $C_x$,
	\item 1b and 1c -- all smallest elements in absolute value of $C_x$,
	\item 1d and 1e -- all smallest elements in $\beta$-norm of $C_x$,
\end{enumerate}
Note that more elements may be added by method 1e than 1d, resp. 1c than 1b, since adding necessary elements before ($\#C_x=1$)  may cause that $C_{x'}\cap \Q_{k+1} \neq \emptyset$ for some $C_{x'}\in C$. 

The procedure is summarized in Algorithm~\ref{alg:extendWeightCoefSet}. 
Another methods may decrease the number of added elements for instance by picking only one of all smallest elements.
	


We can slightly improve performance by substituting the set $\B + \Q_{k}$ by $(\B + \Q_{k})\setminus (\B + \Q_{k-1})$ on the line~\ref{line:replaceBySmaller} in Algorithm~\ref{alg:searchCand}, because
$$
\B + \Q_{k-1} \subset \A + \beta \Q_{k}\,.
$$
Since $\Q_{k+1} \supset \Q_{k}$, $C_x \cap \Q_{k+1} \neq \emptyset$ for  $x\in \B + \Q_{k-1}$.

\begin{algorithm}
  \caption{Extending intermediate weight coefficients set}
    \label{alg:extendWeightCoefSet}
  \begin{algorithmic}[1]
    \REQUIRE{previous weight coefficients set $\Q_{k}$, method number $M\in\{$1a, 1b, 1c, 1d, 1e$\}$}
    \STATE $\widetilde{\Q}:=\Q_{k}$
    \IF {$M\in\{$1a,1b,1d$\}$}
    	\FORALL {$C_x \in C$}
    		\IF {$\# C_x =1$}
		    	\STATE Add the element of $C_x$ to $\widetilde{\Q}$
			\ENDIF
		\ENDFOR
	\ENDIF
    \STATE $\Q_{k+1}:=\widetilde{\Q}$
    \STATE By Algorithm~\ref{alg:searchCand}, find set of candidates $C$
    \FORALL{$C_x \in C$}
        \IF{$C_x \cap \widetilde{\Q}=\emptyset$}
	        \IF {$M=$ 1a}
	        	\STATE Add all elements of $C_x$ to $\Q_{k+1}$
	        \ELSIF {$M\in\{$1b, 1c$\}$}
	        	\STATE Add all smallest elements in absolute value of $C_x$ to $\Q_{k+1}$ 
	        \ELSIF {$M\in\{$1d, 1e$\}$}
	        	\STATE Add all smallest elements in $\beta$-norm of $C_x$ to $\Q_{k+1}$
	        \ENDIF
        \ENDIF
    \ENDFOR
    \RETURN $\Q_{k+1}$
  \end{algorithmic}
\end{algorithm}



\begin{algorithm}
  \caption{Search for set of candidates $C$}
    \label{alg:searchCand}
  \begin{algorithmic}[1]
    \REQUIRE{the previous weight coefficients set $\Q_{k}$, alternatively also the set $\Q_{k-1}$}
    \STATE $C:=\{\emptyset\}$
    \FORALL[Alternatively, $x \in (\B + \Q_{k})\setminus (\B + \Q_{k-1})$]{$x \in \B + \Q_{k}$} \label{line:replaceBySmaller}
      \STATE $C_x:=\emptyset$
      \FORALL{$a \in \A$}
          \IF{$(x-a)$ is divisible by $\beta$ in $\Zomega$ (using Theorem~\ref{thm:divisibility})}
              \STATE Add $\frac{x-a}{\beta}$ to $C_x$
            \ENDIF
      \ENDFOR 
      \STATE Add the set $C_x$ to $C$
  \ENDFOR
  \RETURN $C$
  \end{algorithmic}
\end{algorithm}  


	\begin{upravit}
For construction of the set $\Q_{[w_j,\dots, w_{j-m+1}]}$ we first choose such elements of $\Q_{[w_j,\dots, w_{j-m+2}]}$ which are the only possible to cover some value $x \in w_0 + \Q_{[w_{j-1},\dots, w_{j-m+1}]}$. Other elements from $\Q_{[w_j,\dots, w_{j-m+2}]}$ which cover an uncovered value are added one by one to $\Q_{[w_j,\dots, w_{j-m+1}]}$ until each $x$ equals $a+\beta q_j$ for some $q_j$ in $\Q_{[w_j,\dots, w_{j-m+1}]}$ and $a\in\A$. The pseudocode is in Algorithm~\ref{alg:minimalSet}. 

\begin{algorithm}
  \caption{Search for set $\Q_{[w_j,\dots, w_{j-m+1}]}$ }
    \label{alg:minimalSet}
  \begin{algorithmic}[1]
    \REQUIRE{Input digit $w_j$, set of possible carries $\Q_{[w_{j-1},\dots, w_{j-m+1}]}$, previous set of possible weight coefficients $\Q_{[w_j,\dots, w_{j-m+2}]}$}
    \STATE \verb+list_of_coverings+:=empty list of lists
    \FORALL{$x \in w_j + \Q_{[w_{j-1},\dots, w_{j-m+1}]}$}
        \STATE Build a list \verb+x_covered_by+ of weight coefficients $q_j \in \Q_{[w_j,\dots, w_{j-m+2}]}$ such that 
        $$
        x=a+ \beta q_j \qquad\text{for some } a\in\A\,.
        $$ 
        \vspace{-20pt}
        \STATE Append \verb+x_covered_by+ to \verb+list_of_coverings+
    \ENDFOR
    \STATE $\Q_{[w_j,\dots, w_{j-m+1}]}$:= empty set
    \WHILE{\texttt{list\_of\_coverings} is nonempty}
        \STATE Pick an element $q$ of one of the shortest lists of \verb+list_of_coverings+ 
            \label{line:pickElement}
        \STATE Add the element $q$ to $\Q_{[w_j,\dots, w_{j-m+1}]}$
        \STATE Remove lists of \verb+list_of_coverings+ containing the element $q$ from \verb+list_of_coverings+
    \ENDWHILE
    \RETURN $\Q_{[w_j,\dots, w_{j-m+1}]}$
  \end{algorithmic}
\end{algorithm}


Notice that the result of Algorithm~\ref{alg:minimalSet} is influenced by the way how we pick an element on line~\ref{line:pickElement}. It can be done deterministically or non-deterministically. We use the following deterministic choice -- suppose that we want to choose from elements $x_1=\sum_{i=0}^{d-1}x_{1,i}\omega^i, x_2=\sum_{i=0}^{d-1}x_{2,i}\omega^i, \dots,x_n=\sum_{i=0}^{d-1}x_{n,i}\omega^i\in\Zomega$, where $d$ is the degree of $\omega$. Let $a_0,\dots,a_{d-1}\in\ZZ$ be such that 
$$
\sum_{i=0}^{d-1}a_i\omega^i=\sum_{j=1}^n x_j\,.
$$ 
Set $c:=\sum_{i=0}^{d-1}c_i\omega^i \in\Zomega$ with $c_i=[\frac{a_i}{n}]$ where $[\cdot]$ denotes rounding to the nearest integer. Let the index set $I_0\subset\{1,\dots,n\}$ be such that 
$$
|x_{j,0}-c_0|=\min\{|x_{k,0}-c_0|\colon k\in{1,\dots,n}\}
$$
for all $j\in I_0$. For all $i\in\{1,\dots,d-1\}$, let the index set $I_i\subset I_{i-1}$ be such that
$$
|x_{j,i}-c_i|=\min\{|x_{k,i}-c_i|\colon k\in I_{i-1}\}
$$
for all $j\in I_i$. If there is only one element in the index set $I_{d-1}=\{j_0\}$, choose the element~$x_{j_0}$. Otherwise choose $j_0\in I_{d-1}$ such that $x_{j_0,0}$ is the smallest from all $x_{j,0}$ such that $j\in I_{d-1}$. If there are more such elements, then choose from them according to the value $x_{j,1}$ etc. 

In other words, we take the elements which are the ``closest'' ones to the rounded center of gravity~$c$ of the values $x_1,\dots,x_n$ where ``closest'' is measured by absolute value of the first coordinate of $\pi(x_j)-\pi(c)$. In case of equality, according to the second coordinate etc. If there is more than one such element, we choose the element $x_{j_0}$ with the smallest first, resp. second, etc. coordinate of $\pi(x_{j_0})$.  
\end{upravit}

	
	


\chapter{Design and implementation}

In the first section of this chapter, we propose algorithms which may reveal non-convergence of Phase~2. They are based on the theorems from Chapter~\ref{chap:convergence}. We present a simple algorithm which makes Phase~2 stable. We develop Algorithm~\ref{alg:weightFunction_modified} that summarizes  Phase~2 with all modifications. All designed algorithms are implemented in SageMath. The program is described in Section~\ref{sec:implementation}.
\label{chap:design}
	

\begin{upravit}
 Now we describe Algorithm \ref{alg:oneletterSets} which checks whether Phase~2 stops when it processes input digits $bb\dots b$.
For arbitrary $m$, sets $\Qb{m}$  can be easily constructed separately for each $b\in\B$. We build the set $\Qb{m}$ for input digits $bb\dots b$ in the same way as in Phase 2. This means that we first search for $\Qb{}$ such that 
$$
b + \Q \subset \A + \beta \Qb{}\,.
$$
Until the set $\Qb{m}$ contains only one element, we increment the length of  window $m$ and, using Algorithm \ref{alg:minimalSet}, we build the subset $\Qb{m+1}$ of the set $\Qb{m}$ such that
$$
b + \Qb{m} \subset \A + \beta \Qb{m+1}\,.
$$
In addition, we check whether the set $\Qb{m+1}$ is strictly smaller than the set $\Qb{m}$. If not, we know by Theorem \ref{thm:bbbCondition} that Phase 2 does not converge because of the input digits $bb\dots b$.

Thus, running of Algorithm \ref{alg:oneletterSets} for each input digit $b\in\B$ can reveal non-finiteness of Phase~2.

\end{upravit}

\begin{algorithm}
  \caption{Check the input $bb\dots b$}
    \label{alg:oneletterSets}
  \begin{algorithmic}[1]
    \REQUIRE{Weight coefficient set $\Q$, digit $b\in\B$}
    \ENSURE{TRUE if there is a unique weight coefficient for input $bb\dots b$, False otherwise}
    \STATE Find minimal set $\Qb{1} \subset \Q$ such that
      $$
      b + \Q \subset \A + \beta \Qb{1}\,.
      $$
      \vspace{-20pt}
    \STATE $m:=1$
    \WHILE{$\#\Qb{m} > 1$}
        \STATE $m:= m +1$
        \STATE By Algorithm \ref{alg:minimalSet}, find minimal set $\Qb{m} \subset \Qb{m-1}$ such that
          $$
          b + \Qb{m-1} \subset \A + \beta \Qb{m}\,.
          $$  
          \vspace{-20pt}
        \IF{$\#\Qb{m}=\#\Qb{m-1}$}
            \RETURN FALSE
        \ENDIF
    \ENDWHILE  
    \RETURN TRUE
  \end{algorithmic}
\end{algorithm}


\begin{algorithm}
  \caption{Modified search for a weight function (Phase 2)}
    \label{alg:weightFunction_modified}
  \begin{algorithmic}[1]
    \REQUIRE{weight coefficients set $\Q$}
    \FORALL{$b\in\B$}
    	\IF  {\NOT Check the input $bb\dots b$ by Algorithm \ref{alg:oneletterSets}}
    		\RETURN Phase 2 does not converge for input $bb\dots b$
    	\ENDIF
    \ENDFOR
    \FORALL{$w_0 \in \B$} 
        \STATE By Algorithm~\ref{alg:minimalSet}, find set $\Q_{[w_0]} \subset \Q$ such that
          $$
          w_0 + \Q \subset \A + \beta \Q_{[w_0]}
          $$\vspace{-20pt}
    \ENDFOR
    \STATE $k:=0$
    \WHILE{$\max\{\#\Qwo{k}:\tupleo{k} \in \B^{k+1} \} > 1$}
        \STATE $k:= k +1$
        \FORALL{$\tupleo{k} \in \B^{k+1}$}
            \STATE By Algorithm~\ref{alg:minimalSet}, find set $\Qwo{k} \subset \Qwo{(k-1)}$ such that
              $$
              w_j + \Qw{1}{k} \subset \A + \beta \Qwo{k}\,,
              $$\vspace{-20pt}
              \IF {ygtv0}
              	\STATE dkfjgn
              \ENDIF
        \ENDFOR  
    \ENDWHILE  
    \STATE $r:= k$ 
    \FORALL{$\tupleo{r} \in \B^{r+1}$}  
        \STATE $q\tupleo{r}:=$ only element of $\Qwo{r}$
    \ENDFOR
    \RETURN $q$
  \end{algorithmic}
\end{algorithm}
	\section{Implementation}

Our implementation of the design is based on the program attached to \cite{vu}. The chosen programming language is SageMath. It is Python-based language with numerous implemented mathematical structures. That is the main reason of our choice -- the extending window method requires to handle elements of $\Zomega$ and arithmetic operations in this set. Moreover, SageMath provides various data structures and plotting tools.  The code is simpler and more similar to pseudocode than if we implemented in pure \Cpp. Due to it, we may concern ourselves with the algorithmic part of the problem instead of difficult programming. Unfortunately, SageMath is much slower than \Cpp.


The implementation is object-oriented. It consists of five classes. Class \emph{AlgorithmForParallelAddition} contains structures for computations in $\Zomega$. The build-in classes \emph{PolynomialQuotientRing} and \emph{NumberField} are used to represent elements of $\Zomega$ as an algebraic and complex numbers. The class also links together all functions and instances of other classes which are necessary to construct an algorithm for digit set conversion from $\B$ to $\A$ by the extending window method. The input parameters are an algebraic integer $\omega$ given by its minimal polynomial $m_\omega$ and an approximate complex value, a base $\beta\in\Zomega$, an alphabet $\A\subset\Zomega$ and input alphabet $\B\subset\Zomega$. 

Phase 1 of the extending window method is implemented in class \emph{WeightCoefficientsSetSearch} and Phase 2 in \emph{WeightFunctionSearch}. Class \emph{WeightFunction} serves to save a function $q$ computed in Phase 2. The last class \emph{ExceptionParAdd} is inherited from build-in class \emph{Exception} to distinguish between errors which are raised by the algorithm  of the extending window method and other ones.

We use notation from previous chapters in descriptions of the classes. We list only the most important methods of each class, see commented source code for all of them.  

For basic use, load \verb+AlgorithmForParallelAddition.sage+, create an instance of \emph{AlgorithmForParallelAddition} and call \textbf{findWeightFunction()}.

We also provide a shell interface and a graphic user interface which runs as an interact in SageMathCloud. If the network access is enabled and the modul \verb+gspread+ is installed (see \cite{gspread}), then results from both interfaces are automatically saved to Google spreadsheet \href{https://docs.google.com/spreadsheets/d/1TnhrHdefHfHa0WSeVs4q6XVj3epjPlPlnoekE0E1xeM/edit?usp=sharing}{ParallelAddition\_results}. The interfaces are described at the end of this section. The whole implementation is on the attached DVD or it can be downloaded from  \url{https://github.com/Legersky/ParallelAddition}.



\subsection*{Class AlgorithmForParallelAddition}
The necessary structures for computation in $\Zomega$ are constructed when an instance of this class is created. 
The set $\Zomega$ is represented by \emph{PolynomialQuotientRing} which is obtained by factorization of \emph{PolynomialRing} over integers by polynomial $m_\omega$. 
We remark that arithmetic operations in $\Zomega$ are independent on the choice of root of the  minimal polynomial $m_\omega$. Since comparison of absolute value of numbers in~$\Zomega$ is required, we specify $\omega$ by its approximate complex value to form a factor ring of rational polynomials by using class \emph{NumberField}. Elements of this class can be coerced to complex numbers and their absolute value is computed.

%Method \textbf{findWeightFunction()} links together both phases of the extending window method to find a weight function $q$ for given parametres. That is used in the methods for addition and digit set conversion to process them as local functions. There are also verification methods.
%
%Moreover, many methods for printing, plotting and saving outputs are implemented.

The constructor of class \emph{AlgorithmForParallelAddition} is 

\begin{method}{\_\_init\_\_}{minPol\_str, embd, alphabet, base, name='NumerationSystem', inputAlphabet=' ',\\
 printLog=True, printLogLatex=False, verbose=0, maxInputs=Infinity}
Take \var{minPol\_str} which is a string of symbolic expression in the variable $x$ of an irreducible polynomial~$m_\omega$. The closest root of  \var{minPol\_str} to \var{embd} is used as the ring generator $\omega$ (see more in the documentation of \emph{NumberField} in SageMath \cite{sage}). The structures for $\Zomega$ are constructed as described above. 

A base $\beta$ is given by a symbolic expression \var{base} where \verb+omega+ is used for $\omega$.
Setter \fun{setBase}{base} inspects whether the base $\beta$ has a real conjugate greater than one.

The parameter \var{alphabet} and \var{inputAlphabet} expect a string with a list of symbolic expressions (use \verb+omega+ for $\omega$) to define an alphabet $\A$ and input alphabet $\B$. If the string \var{alphabet} is empty, then an alphabet $\A\subset\Zomega$ is generated such that it contains all representants modulo $\beta$ and $\beta-1$ in $\Zomega$ according to Theorem~\ref{thm:lowerBoundAlphabet}. If \var{alphabet} is \verb+oneMore+, then a bigger alphabet is generated.  If \var{alphabet} is \verb+integer+, \verb+integer2+ or \verb+integer3+, an algorithm attempts to find an integer alphabet of different size. An exception is raised when alphabet generating fails. If  \var{inputAlphabet} is empty, then $\B$ is set to $\A+\A$.

Messages saved to logfile during existence of an instance are printed (using \LaTeX) on standard output depending on \var{printLog} and \var{printLogLatex}. The level of messages for a development is set by \var{verbose}. The number of entries in a constructed weight function may be limited by \var{maxInputs}.

Example:
\verb|alg= AlgorithmForParallelAddition('x^2+x+1',-0.5+0.8*I,|

\verb| '[0,1,-1,omega,-omega,-omega-1,omega+1]','omega-1','Eisenstein')|
\end{method}

Methods for the construction of a weight function for a digit set conversion from $\B$ to $\A$:
\begin{method}{checkAlphabet}{}
It is verified that the alphabet $\A$ contains all representatives mod $\beta$ in  and that all elements of the input alphabet $\B$ have their representatives mod $\beta-1$ in the alphabet $\A$ according to Theorem~\ref{thm:lowerBoundAlphabet}, including the case when the base $\beta$ has a real conjugate greater than one.
\end{method}

\begin{method}{\_findWeightCoefSet}{ max\_iterations, method\_number}
Create an instance of \emph{WeightCoefficientsSetSearch(self,method\_number)} and call its method \fun{findWeightCoefficientsSet}{max\_iterations} to obtain a weight coefficients set $\Q$.
\end{method}

\begin{method}{\_findWeightFunction}{ max\_input\_length,method\_number}
Create an instance of \emph{WeightFunctionSearch(self, $\Q$, method\_number, maxInputs)} and call its methods \fun{check\_one\_letter\_inputs}{max\_input\_length} and \\ \fun{findWeightFunction}{max\_input\_length} to obtain a weight function $q$.
\end{method}


\begin{method}{findWeightFunction}{ max\_iterations, max\_input\_length, method\_weightCoefSet=None,\\ method\_weightFunSearch=None}
Call \fun{checkAlphabet}{} and return the weight function $q$ obtained by calling \\ \fun{\_findWeightCoefSet}{max\_iterations, method\_weightCoefSet} and \\ \fun{\_findWeightFunction}{max\_input\_length, method\_weightFunSearch}.
\end{method}

%The important function for the searching for possible weight coefficients is
%
%\begin{method}{divideByBase}{divided\_number}
%Using Theorem \ref{thm:divisibility}, check if \var{divided\_number} is divisible by the base $\beta$. If it is so, return the result of division, else return \var{None}.
%\end{method}


Methods for the addition and the digit set conversion computable in parallel are following:

\begin{method}{addParallel}{a,b}
Sum up numbers represented by the lists of digits \var{a} and \var{b} digitwise and convert the result by \fun{parallelConversion}{}. 
\end{method}


\begin{method}{parallelConversion}{\_w}
Return $(\beta,\A)$-representation of the number represented by the list \var{\_w} of digits from the input alphabet $\B$. According to the equation \eqref{eq:conversionFormula}, it is computed locally by using the weight function $q$. An exception is raised when an outputed digit is not in the alphabet $\A$.
\end{method}


%\begin{method}{localConversion}{w}
%Return converted digit according to the equation \eqref{eq:conversionFormula} for the list of input digits \var{w}.
%\end{method}


The correctness of the implementation of the extending window for a given numeration system can be verified by
 
\begin{method}{sanityCheck\_conversion}{num\_digits}
Check whether the values of all possible numbers of the length \var{num\_digits} with digits in the input alphabet $\B$ are the same as their $(\beta, \A)$-representation obtained by \fun{parallelConversion}{}.   
\end{method}

\begin{upravit}


\subsection*{Class WeightCoefficientsSetSearch}
Class \emph{WeightCoefficientsSetSearch} implements Phase 1 of the extending window method described in Section \ref{subsec:phase1} with different methods how an intermediate weight coefficients set $\Q_k$ is extended to $\Q_{k+1}$. Algorithm~\ref{alg:extendWeightCoefSet} explains methods 1a, 1b, 1c, 1d and 1e. There are also implemented other experimental methods denoted by numbers. Method 1a corresponds to 14, 1b to 12, 1c to 16, 1d to 13 and 1e to 15. 

The constructor of the class is 

\begin{method}{\_\_init\_\_}{algForParallelAdd, method}
Initialize a ring generator $\omega$, base $\beta$, an alphabet $\A$ and input alphabet $\B$ by values obtained from \var{algForParallelAdd}. The parameter \var{method} is a number of an experimental method or \verb+'1a'+, \verb+'1b'+, etc. The chosen method determines how an intermediate weight coefficients set $\Q_k$ is extended to $\Q_{k+1}$. If \var{None}, then the method 1d from Algorithm~\ref{alg:extendWeightCoefSet} is used as default.
\end{method}

Class methods implementing Phase 1 are the following:

\begin{method}{\_findCandidates}{to\_cover}
 Return the list of lists \var{candidates}, which corresponds to a covering set $C$ in Algorithm~\ref{alg:searchCand}, such that each element in \var{to\_cover} is covered by all values of the appropriate list in \var{candidates}.  
\end{method}


\begin{method}{\_chooseQk\_FromCandidates}{candidates}
Take the previous intermediate weight coefficients set $\Q_{k}$ and constructs a new intermediate weight coefficients set $\Q_{k+1}$ from \var{candidates} by Algorithm~\ref{alg:extendWeightCoefSet}.
\end{method}


\begin{method}{\_getQk}{to\_cover}
Links together methods \fun{\_findCandidates}{to\_cover} and \fun{\_chooseQk\_FromCandidates}{} to return the intermediate weight coefficients set $\Q_{k+1}$.
\end{method}

\begin{method}{findWeightCoefficientsSet}{maxIterations}
According to  Algorithm~\ref{alg:weightCoefSet}, constructs a weight coefficients set $\Q$ by iterative using \fun{\_getQk}{}. A computation is aborted if the number of iterations exceeds \var{maxIterations}. 
\end{method}




\subsection*{Class WeightFunctionSearch}

This class implements Algorithm~\ref{alg:weightFunction_modified} of modified Phase 2 of the extending window method from Section \ref{sec:modifiedPhase2} with different methods of choice of possible weight coefficients sets and control of non-convergence. Methods 2a, 2b, 2c, 2d and 2e from Algorithm~\ref{alg:pickElement} correspond to 9, 15, 22, 23 and 14 respectively. A weight function $q$ is returned by method \fun{findWeightFunction}{}. The constructor of the class is

\begin{method}{\_\_init\_\_}{algForParallelAdd, weightCoefSet, method}
The ring generator $\omega$, base $\beta$, alphabet $\A$ and input alphabet $\B$ are initialized by the values obtained from \var{algForParallelAdd}. The weight coefficients set $\Q$ is set to \var{weightCoefSet}. The parameter \var{method} (the number of an experimental method or \verb+'2a'+, \verb+'2b'+, etc.) determines the way of the choice of a possible weight coefficients set $\Qwo{k}$ from $\Qwo{(k-1)}$, see Algorithm~\ref{alg:pickElement}. If \var{method} is \var{None}, then is the default method is 2b. Possible weight coefficients sets  are saved in a dictionary \var{\_Qw\_w}, which is set to be empty.
\end{method}

The following methods are used for search for a weight function $q$:

\begin{method}{\_find\_weightCoef\_for\_comb\_B}{combinations}
Take all combinations of input digits $\tupleo{(k-1)} \in \B^{k}$ in \var{combinations} such that $\#\Qwo{(k-1)}>1$, extend them by all letters $w_{-k}\in\B$ and find a possible weight coefficients set $\Qwo{k}$ by the method \fun{\_findQw}{$\tupleo{(k-1)}$}. If there is only one element in $\Qwo{k}$, it is saved as a solved input of the weight function $q$. Otherwise, the set $\Qwo{k}$ is saved in \var{\_Qw\_w} as an unsolved combination which requires extending of the window. The unsolved combinations are returned.  
\end{method}

\begin{method}{\_findQw}{w\_tuple}
Return a set $\Qwo{k}=\Q_{[\text{w\_tuple}]}$ by wrapping \fun{\_findQw\_once}{} into a while loop according to Algorithm~\ref{alg:possibleWeightCoefSetStable}. If $\Qwo{k}=\Qwo{(k-1)}$, then add a vertex $\tupleo{k}$ to a Rauzy graph $G_{k+1}$ and call \fun{\_checkCycles}{w\_tuple} in $G_{k}$.
\end{method}

\begin{method}{\_findQw\_once}{w\_tuple,Qw\_prev} 
Return a set $\Qwo{k}'=\Q'_{[\text{w\_tuple}]}$ obtained by Algorithm~\ref{alg:minimalSet} as a subset of \var{Qw\_prev}$=\Qwo{(k-1)}$. The set of possible carries from the right $\Qw{1}{k}$ is taken from the class attribute \var{\_Qw\_w}. The methods of Algorithm~\ref{alg:pickElement} are implemented here along with the experimental ones.
\end{method}

\begin{method}{\_checkCycles}{w\_tuple}
Using Algorithm~\ref{alg:checkCycles}, check if there is a cycle in the Rauzy graph $G_k$ starting in the vertex which label equals \var{w\_tuple} without the first digit. If yes, then an exception \emph{RuntimeErrorParAdd} is raised. 
\end{method}


\begin{method}{findWeightFunction}{max\_input\_length}
Attempt to construct a weight function $q$ by Algorithm~\ref{alg:weightFunction_modified}. It increments length of window and call the method \fun{\_find\_weightCoef\_for\_comb\_B}{} until a unique weight coefficient is found for all possible combinations of input digits. If the length of window exceeds \var{max\_input\_length}, then an exception \emph{RuntimeErrorParAdd} is raised. 
\end{method}


\begin{method}{check\_one\_letter\_inputs}{max\_input\_length}
Checks by Algorithm \ref{alg:oneletterSets} if there is a unique weight coefficient for inputs $({b,b,\dots,b})\in\B^r$ for some length of window $r$. An exception \emph{RuntimeErrorParAdd} is raised in the case of an infinite loop. Otherwise the list of input digits  $b$ which require the largest length of window is returned.
\end{method}







\subsection*{Class WeightFunction}
This class serves for saving a weight function $q$. The constructor is

\begin{method}{\_\_init\_\_}{B}
Set the input alphabet $\B$ to \var{B} and maximum length of window to 1. Initialize an attribute \var{\_mapping} for saving the weight function $q$ to an empty dictionary. 
\end{method}

The methods for saving and calling are following:

\begin{method}{addWeightCoefToInput}{\_input, coef}
Save the weight coefficient \var{coef} for the combination of digits \var{\_input} to \var{\_mapping}. The digits of \var{\_input} must be in the input alphabet $\B$.
\end{method}

\begin{method}{getWeightCoef}{w}
The digits of the list \var{w} are taken from the left until a weight coefficient in the dictionary \var{\_mapping} is found. 
\end{method}

The result of the method \fun{getWeightCoef}{} is used to make this class callable, i.e., if \var{\_q} is an instance of \emph{WeightFunction}, then \var{\_q}.\fun{getWeightCoef}{w} is the same as \var{\_q}(\var{w}).






\begin{upravit}
\subsection*{User interfaces}
We provide two interfaces for running of the implemented extending window method -- the~shell version and graphic user interface.

\subsubsection*{Shell}
SageMath must be installed. The implementation of the extending window method is launched in a shell by typing \verb+sage extending_window_method.sage <input_sample.sage>+. The parameters of the numeration system and setting of outputs and computation is given by the SageMath file \verb+input_sample.sage+. See Appendix \ref{app:inputSample} for an example of such a file.

The name of the numeration system, minimal polynomial of generator $\omega$, an approximate value of $\omega$, the base $\beta$, alphabet $\A$ and input alphabet $\B$ are set in the part INPUTS. The maximum number of iterations in Phase 1, maximal length of the window in Phase 2 and the launching of the sanity check are set in SETTING. 

The boolean values in the part SAVING determines which formats of the outputs are saved. All outputs are saved in the folder \verb+./outputs/<name>/+. General information about the computation can be saved in the TeX format, the computed weight function and local digit set conversion in the CSV file format. An inputs setting saved as a dictionary can be loaded by the interact interface. The log of the whole computation can be saved as a text file. There is also an option to save unsolved combinations in Phase 2 in the CSV file format in the case of the interruption of the program.

According to the boolean values in the part IMAGES, figures of the alphabet, input alphabet, weight coefficients set or part of the set $\Zomega$ with marked alphabet shifted into points which are divisible by the base $\beta$ are saved in the PNG format to the folder \verb+./outputs/<name>/+ \verb+img/+. Optionally, the weight coefficients set is plotted with the  bound given by the proof of Theorem~\ref{thm:suffCondPhase1}. Images of individual steps of both phases of the extending window method can be saved, too. For Phase 2, the search for the weight coefficient  is plotted for the digits given by \verb+phase2_input+.  

The program prints out all inputs and then it computes the weight function $q$ by calling \fun{findWeightFunction}{ max\_iterations, max\_input\_length}. The increments of the weight coefficients set in each iteration of Phase 1 are printed and then also the obtained weight coefficients set $\Q$. The longest tested combinations given by repetition of one letter are printed after the computation of \fun{check\_one\_letter\_inputs}{max\_input\_length}. During computing of Phase 2, the current length of window and the number of saved combinations are printed. At the end, the final length of window, elapsed time and info about saved outputs are printed.  

It is possible to batch process all input files in one folder by executing the bash script \verb+ewm_batch <folder_name>+.  

\subsubsection{Interact in SageMath Cloud}
The graphic user interface is implemented using an interact in SageMath Cloud. The parts of the interact are on Figure \ref{fig:interact1}, \ref{fig:interact2} and \ref{fig:interact3} in Appendix \ref{app:interact}. The functionality is basically the same as the shell version. An account on the website \url{https://cloud.sagemath.com} is needed to use the interact. Create a new project and load files \verb+extending_window_method_GUI.sagews+ and \verb+interact_ewm.sage+. After executing of the cell by Shift+Enter in the first one, the parameters of the numeration system are filled in the corresponding spaces or one of the previous settings is loaded by typing its name.  By default, the last inputs are shown in the form. The inputs are submitted by pressing the button Update. Using check-boxes, the formats for saving of the output are chosen and the search for the weight function is launched by pressing second button Update.

The printed output is similar to the shell output. In addition, it contains figures and it is formatted using \LaTeX. Moreover, the sanity check can be run for a given length, the weight coefficient for a tuple of  input digits is returned or images of individual steps of both phases are shown and saved.


\end{upravit}

\chapter{Testing}
	\komentar{tabulka vsech uspesnych plus nejake neuspesne a okomentovat}


\section{Comparing different choices in Phase 1 and 2}
\komentar{vsechny Mileniny rucne spoctene, taky by tady mohly byt i nejake neuspesne}
\afterpage{
\clearpage   % To flush out all floats, might not be what you want
\begin{landscape}
	\begin{table}[h]
		\begin{center}
		\begin{tabular}{l|c c c c|ccc|c c  c  c  c  c }
\multirow{2}{*}{Name} & \multirow{2}{*}{$\omega$} & \multirow{2}{*}{$m_\omega$} & \multirow{2}{*}{$\beta$} & \multirow{2}{*}{$m_\beta$} & \multirow{2}{*}{conj.} & \multirow{2}{*}{$\#\A$} & \multirow{2}{*}{min.} & \multicolumn{5}{c}{$\#\Q$} \\
  &  &  &  &  &  &  &  & 6 & 8 & 9 & 10 & 11 \\ \hline
Eisenstein\_1--block\_complex & $ \frac{1}{2} i \, \sqrt{3} - \frac{1}{2} $ & $ t^{2} + t + 1 $ & $ \omega - 1 $ & $ x^{2} + 3 \, x + 3 $ & yes & $ 7 $ & yes & 19 & 19 & 19 & 19 & 19 \\
Eisenstein\_1--block\_integer & $ \frac{1}{2} i \, \sqrt{3} - \frac{1}{2} $ & $ t^{2} + t + 1 $ & $ \omega - 1 $ & $ x^{2} + 3 \, x + 3 $ & yes & $ 7 $ & yes & 113 & 53 & 52 & 52 & 53 \\
Eisenstein\_2--block\_complex & $ \frac{1}{2} i \, \sqrt{3} - \frac{1}{2} $ & $ t^{2} + t + 1 $ & $ -3 \, \omega $ & $ x^{2} - 3 \, x + 9 $ & yes & $ 14 $ & no & 17 & 17 & 17 & 17 & 17 \\
Eisenstein\_2--block\_integer & $ \frac{1}{2} i \, \sqrt{3} - \frac{1}{2} $ & $ t^{2} + t + 1 $ & $ -3 \, \omega $ & $ x^{2} - 3 \, x + 9 $ & yes & $ 16 $ & no & 26 & 26 & 26 & 26 & 26 \\
Penney\_1--block\_complex & $ i - 1 $ & $ t^{2} + 2 \, t + 2 $ & $ \omega $ & $ x^{2} + 2 \, x + 2 $ & yes & $ 5 $ & yes & 45 & 45 & 45 & 45 & 45 \\
Penney\_1--block\_integer & $ i $ & $ t^{2} + 1 $ & $ \omega - 1 $ & $ x^{2} + 2 \, x + 2 $ & yes & $ 5 $ & yes & 97 & 27 & 27 & 27 & 27 \\
Penney\_2--block\_integer & $ i $ & $ t^{2} + 1 $ & $ -2 \, \omega $ & $ x^{2} + 4 $ & yes & $ 9 $ & no & 27 & 27 & 27 & 27 & 27 \\
Quadratic+1+0--17\_integer & $ -\frac{1}{2} \, \sqrt{17} + \frac{3}{2} $ & $ t^{2} - 3 \, t - 2 $ & $ 2 \, \omega - 3 $ & $ x^{2} - 17 $ & yes & $ 19 $ & no & 9 & 9 & 9 & 9 & 9 \\
Quadratic+1+0--17\_integer\_smaller & $ -\frac{1}{2} \, \sqrt{17} + \frac{3}{2} $ & $ t^{2} - 3 \, t - 2 $ & $ 2 \, \omega - 3 $ & $ x^{2} - 17 $ & yes & $ 18 $ & yes & 9 & 9 & 9 & 9 & 9 \\
Quadratic+1+0--21\_integer & $ -\frac{1}{2} \, \sqrt{21} + \frac{3}{2} $ & $ t^{2} - 3 \, t - 3 $ & $ 2 \, \omega - 3 $ & $ x^{2} - 21 $ & yes & $ 23 $ & no & 9 & 9 & 9 & 9 & 9 \\
Quadratic+1+0--21\_integer\_smaller & $ -\frac{1}{2} \, \sqrt{21} + \frac{3}{2} $ & $ t^{2} - 3 \, t - 3 $ & $ 2 \, \omega - 3 $ & $ x^{2} - 21 $ & yes & $ 22 $ & yes & 9 & 9 & 9 & 9 & 9 \\
Quadratic+1+2+3\_complex\_smaller & $ i \, \sqrt{2} - 1 $ & $ t^{2} + 2 \, t + 3 $ & $ -\omega - 2 $ & $ x^{2} + 2 \, x + 3 $ & yes & $ 6 $ & yes & 27 & 27 & 26 & 26 & 27 \\
Quadratic+1+3+4\_complex & $ \frac{1}{2} i \, \sqrt{7} - \frac{1}{2} $ & $ t^{2} + t + 2 $ & $ \omega - 1 $ & $ x^{2} + 3 \, x + 4 $ & yes & $ 8 $ & yes & 20 & 20 & 19 & 19 & 19 \\
Quadratic+1+3+5\_complex1 & $ \frac{1}{2} i \, \sqrt{11} - \frac{3}{2} $ & $ t^{2} + 3 \, t + 5 $ & $ \omega $ & $ x^{2} + 3 \, x + 5 $ & yes & $ 9 $ & yes & 19 & 11 & 17 & 17 & 11 \\
Quadratic+1+3+5\_complex2  & $ \frac{1}{2} i \, \sqrt{11} - \frac{3}{2} $ & $ t^{2} + 3 \, t + 5 $ & $ \omega $ & $ x^{2} + 3 \, x + 5 $ & yes & $ 9 $ & yes & 39 & 31 & 34 & 34 & 31 \\
Quadratic+1+4+5\_complex1 & $ i $ & $ t^{2} + 1 $ & $ \omega - 2 $ & $ x^{2} + 4 \, x + 5 $ & yes & $ 10 $ & yes & 19 & 17 & 17 & 17 & 17 \\
Quadratic+1+4+5\_complex2 & $ i $ & $ t^{2} + 1 $ & $ \omega - 2 $ & $ x^{2} + 4 \, x + 5 $ & yes & $ 10 $ & yes & 17 & 17 & 17 & 17 & 17 \\
\end{tabular}

		\end{center}
	\caption{Comparing methods in Phase 1}
	\label{tab:resultsQuadrNonint}
	\end{table}
\end{landscape}
}

\afterpage{
\clearpage   % To flush out all floats, might not be what you want
\begin{landscape}
	\begin{table}[h]
		\begin{center}
		\begin{tabular}{l|cc| ccc|  ccc|  ccc|  ccc}
\multirow{2}{*}{Name}  & Methods & \multirow{2}{*}{$\#\Q$}&\multicolumn{3}{c|}{$9$} & \multicolumn{3}{c|}{$15$} & \multicolumn{3}{c|}{$22$} & \multicolumn{3}{c}{$23$} \\
 & Phase 1&  &$bbb$ & Ph.2 & $r$ &$bbb$ & Ph.2 & $r$ &$bbb$ & Ph.2 & $r$ &$bbb$ & Ph.2 & $r$ \\ \hline
\multirow{1}{*}{Eisenstein\_1--block\_complex}& $12, 13, 14, 15, 16$ & $19$ &\checkmark & \checkmark & 3 & \checkmark & \checkmark & 3 & \checkmark & \checkmark & 3 & \checkmark & \checkmark & 3 \\
\hline
\multirow{2}{*}{Eisenstein\_1--block\_integer}& $12, 13, 15, 16$ & $57$ &\xmark & - & - & \xmark & - & - & \xmark & - & - & \xmark & - & - \\
& $14$ & $139$ &\xmark & - & - & \xmark & - & - & \xmark & - & - & \xmark & - & - \\
\hline
\multirow{1}{*}{Eisenstein\_2--block\_complex}& $12, 13, 14, 15, 16$ & $17$ &\xmark & - & - & \xmark & - & - & \xmark & - & - & \xmark & - & - \\
\hline
\multirow{1}{*}{Eisenstein\_2--block\_integer}& $12, 13, 14, 15, 16$ & $26$ &\xmark & - & - & \xmark & - & - & \xmark & - & - & \xmark & - & - \\
\hline
\multirow{1}{*}{Penney\_1--block\_complex}& $12, 13, 14, 15, 16$ & $45$ &\checkmark & \checkmark & 6 & \checkmark & \checkmark & 6 & \checkmark & \checkmark & 6 & \checkmark & \checkmark & 6 \\
\hline
\multirow{2}{*}{Penney\_1--block\_integer}& $12, 13, 15, 16$ & $27$ &\xmark & - & - & \xmark & - & - & \xmark & - & - & \xmark & - & - \\
& $14$ & $95$ &\xmark & - & - & \xmark & - & - & \xmark & - & - & \xmark & - & - \\
\hline
\multirow{1}{*}{Penney\_2--block\_integer}& $12, 13, 14, 15, 16$ & $27$ &\checkmark & \checkmark & 5 & \checkmark & \checkmark & 5 & \checkmark & \checkmark & 5 & \checkmark & \checkmark & 5 \\
\hline
\multirow{1}{*}{Quadratic+1+0--2\_integer}& $12, 13, 14, 15, 16$ & $9$ &\checkmark & \checkmark & 5 & \checkmark & \checkmark & 5 & \checkmark & \checkmark & 5 & \checkmark & \checkmark & 4 \\
\hline
\multirow{1}{*}{Quadratic+1+0--21\_integer}& $12, 13, 14, 15, 16$ & $9$ &\checkmark & \checkmark & 4 & \checkmark & \checkmark & 4 & \checkmark & \checkmark & 4 & \checkmark & \checkmark & 4 \\
\hline
\multirow{1}{*}{Quadratic+1+0--3\_integer}& $12, 13, 14, 15, 16$ & $9$ &\checkmark & \checkmark & 4 & \checkmark & \checkmark & 5 & \checkmark & \checkmark & 5 & \checkmark & \checkmark & 5 \\
\hline
\multirow{1}{*}{Quadratic+1+0--5\_integer}& $12, 13, 14, 15, 16$ & $9$ &\xmark & - & - & \checkmark & \checkmark & 3 & \checkmark & \checkmark & 2 & \checkmark & \checkmark & 2 \\
\hline
\multirow{1}{*}{Quadratic+1+2+3\_complex}& $12, 13, 14, 15, 16$ & $27$ &\checkmark & \xmark & - & \checkmark & \checkmark & 7 & \checkmark & \xmark & - & \checkmark & \xmark & - \\
\hline
\multirow{4}{*}{Quadratic+1+3+4\_complex}& $12$ & $20$ &\checkmark & \checkmark & 7 & \checkmark & \checkmark & 7 & \checkmark & \xmark & - & \xmark & - & - \\
& $16$ & $19$ &\checkmark & \xmark & - & \checkmark & \checkmark & 7 & \checkmark & \xmark & - & \xmark & - & - \\
& $13, 15$ & $20$ &\checkmark & \xmark & - & \checkmark & \checkmark & 7 & \checkmark & \xmark & - & \xmark & - & - \\
& $14$ & $21$ &\checkmark & \checkmark & 7 & \checkmark & \checkmark & 7 & \checkmark & \xmark & - & \xmark & - & - \\
\hline
\multirow{3}{*}{Quadratic+1+3+5\_complex1}& $14$ & $19$ &\xmark & - & - & \xmark & - & - & \xmark & - & - & \xmark & - & - \\
& $12, 16$ & $11$ &\xmark & - & - & \checkmark & \xmark & - & \xmark & - & - & \xmark & - & - \\
& $13, 15$ & $17$ &\xmark & - & - & \xmark & - & - & \xmark & - & - & \xmark & - & - \\
\hline
\multirow{3}{*}{Quadratic+1+3+5\_complex2 }& $12, 16$ & $33$ &\xmark & - & - & \checkmark & \xmark & - & \xmark & - & - & \xmark & - & - \\
& $13, 15$ & $39$ &\xmark & - & - & \checkmark & \xmark & - & \checkmark & \xmark & - & \xmark & - & - \\
& $14$ & $43$ &\xmark & - & - & \checkmark & \xmark & - & \checkmark & \xmark & - & \xmark & - & - \\
\hline
\multirow{2}{*}{Quadratic+1+4+5\_complex1}& $14$ & $19$ &\xmark & - & - & \xmark & - & - & \xmark & - & - & \xmark & - & - \\
& $12, 13, 15, 16$ & $17$ &\xmark & - & - & \xmark & - & - & \xmark & - & - & \xmark & - & - \\
\hline
\multirow{1}{*}{Quadratic+1+4+5\_complex2}& $12, 13, 14, 15, 16$ & $17$ &\checkmark & \checkmark & 3 & \checkmark & \checkmark & 3 & \checkmark & \checkmark & 3 & \checkmark & \checkmark & 3 \\
\hline
\end{tabular}

		\end{center}
	\caption{Comparing methods in Phase 2}
	\label{tab:resultsQuadrNonint}
	\end{table}
\end{landscape}
}

\section{Quadratic bases with non-integer alphabet}
\begin{table}[h]
	\begin{center}
	\begin{tabular}{l|c|cc c| c c| c| c c c }
Ex. &$\omega$ & $\beta$ & $m_\beta$ & conj. & $\#\A$ & min. & $\#\Q$ & $bb\dots b$ & Phase 2 & $r$   \\ \hline
\ref{ex:tAA} & $ \frac{1}{2} i \, \sqrt{3} - \frac{1}{2} $ & $ \omega - 1 $ & $ x^{2} + 3 \, x + 3 $ & ? & $ 7 $ & yes & $ 19 $ & \checkmark & \checkmark & 3 \\
\ref{ex:tAB} & $ \frac{1}{2} i \, \sqrt{3} - \frac{1}{2} $ & $ \omega - 1 $ & $ x^{2} + 3 \, x + 3 $ & ? & $ 7 $ & yes & $ 19 $ & \checkmark & \checkmark & 3 \\
\ref{ex:tAC} & $ \frac{1}{2} i \, \sqrt{3} - \frac{1}{2} $ & $ \omega - 1 $ & $ x^{2} + 3 \, x + 3 $ & ? & $ 7 $ & yes & $ 19 $ & \checkmark & \checkmark & 3 \\
\ref{ex:tAD} & $ i - 1 $ & $ \omega $ & $ x^{2} + 2 \, x + 2 $ & ? & $ 5 $ & yes & $ 45 $ & \checkmark & \checkmark & 6 \\
\ref{ex:tAE} & $ i - 1 $ & $ \omega $ & $ x^{2} + 2 \, x + 2 $ & ? & $ 5 $ & yes & $ 45 $ & \checkmark & \checkmark & 6 \\
\ref{ex:tAF} & $ i \, \sqrt{2} - 1 $ & $ -\omega - 2 $ & $ x^{2} + 2 \, x + 3 $ & ? & $ 6 $ & yes & $ 27 $ & \checkmark & \checkmark & 7 \\
\ref{ex:tAG} & $ \frac{1}{2} i \, \sqrt{11} - \frac{3}{2} $ & $ \omega $ & $ x^{2} + 3 \, x + 5 $ & ? & $ 9 $ & yes & $ 11 $ & \checkmark & \xmark & - \\
\ref{ex:tAH} & $ \frac{1}{2} i \, \sqrt{11} - \frac{3}{2} $ & $ \omega $ & $ x^{2} + 3 \, x + 5 $ & ? & $ 9 $ & yes & $ 11 $ & \checkmark & \xmark & - \\
\end{tabular}
	\end{center}
\caption{Quadratic bases with a non-integer alphabet}
\label{tab:resultsQuadrNonint}
\end{table}


\section{Quadratic bases with integer alphabet}
\begin{table}[h]
	\begin{center}
	\begin{tabular}{c|cc c| c c| c| c c c|l }
$\omega$ & $\beta$ & $m_\beta$ & conj. & $\#\A$ & min. & $\#\Q$ & $bb\dots b$ & Phase 2 & $r$& Ex.   \\ \hline
$ \frac{1}{2} i \, \sqrt{11} + \frac{1}{2} $ & $ -i \, \sqrt{11} $ & $ x^{2} + 11 $ & no & $ 13 $ & no & $ 9 $ & \checkmark & \checkmark & 2 \\
$ \frac{1}{2} i \, \sqrt{11} + \frac{1}{2} $ & $ -i \, \sqrt{11} $ & $ x^{2} + 11 $ & no & $ 12 $ & yes & $ 9 $ & \checkmark & \checkmark & 4  & \ref{ex:integerAB}\\
$ \frac{1}{2} i \, \sqrt{7} - \frac{1}{2} $ & $ -i \, \sqrt{7} $ & $ x^{2} + 7 $ & no & $ 9 $ & no & $ 9 $ & \checkmark & \checkmark & 2 \\
$ \frac{1}{2} i \, \sqrt{7} - \frac{1}{2} $ & $ -i \, \sqrt{7} $ & $ x^{2} + 7 $ & no & $ 8 $ & yes & $ 9 $ & \checkmark & \checkmark & 4  & \ref{ex:integerAD}\\
$ \frac{1}{2} i \, \sqrt{3} + \frac{1}{2} $ & $ -\frac{3}{2} i \, \sqrt{3} + \frac{1}{2} $ & $ x^{2} - x + 7 $ & no & $ 11 $ & no & $ 9 $ & \checkmark & \checkmark & 2  & \ref{ex:integerAE} \\
$ i \, \sqrt{3} $ & $ i \, \sqrt{3} $ & $ x^{2} + 3 $ & no & $ 4 $ & yes & $ 9 $ & \checkmark & \checkmark & 4 \\
$ i \, \sqrt{2} $ & $ i \, \sqrt{2} $ & $ x^{2} + 2 $ & no & $ 3 $ & yes & $ 9 $ & \checkmark & \checkmark & 4  & \ref{ex:integerAG}\\
$ \sqrt{2} $ & $ -\sqrt{2} $ & $ x^{2} - 2 $ & yes & $ 3 $ & yes & $ 9 $ & \checkmark & \checkmark & 5  & \ref{ex:integerAH}\\
$ \sqrt{3} - 1 $ & $ -\sqrt{3} $ & $ x^{2} - 3 $ & yes & $ 4 $ & yes & $ 9 $ & \checkmark & \checkmark & 5 \\
$ \frac{1}{2} \, \sqrt{5} + \frac{1}{2} $ & $ -\sqrt{5} $ & $ x^{2} - 5 $ & yes & $ 6 $ & yes & $ 9 $ & \checkmark & \checkmark & 4  & \ref{ex:integerAJ}\\
$ -\sqrt{5} + 1 $ & $ -\sqrt{5} $ & $ x^{2} - 5 $ & yes & $ 6 $ & yes & $ 9 $ & \checkmark & \checkmark & 4  & \ref{ex:integerAK}\\
$ -\sqrt{6} + 1 $ & $ -\sqrt{6} $ & $ x^{2} - 6 $ & yes & $ 7 $ & yes & $ 9 $ & \checkmark & \checkmark & 4 \\
$ \sqrt{6} - 1 $ & $ \sqrt{6} $ & $ x^{2} - 6 $ & yes & $ 7 $ & yes & $ 9 $ & \checkmark & \checkmark & 4 \\
$ -\sqrt{7} + 2 $ & $ \sqrt{7} $ & $ x^{2} - 7 $ & yes & $ 8 $ & yes & $ 9 $ & \checkmark & \checkmark & 4 \\
$ \frac{1}{2} \, \sqrt{13} + \frac{1}{2} $ & $ -\sqrt{13} $ & $ x^{2} - 13 $ & yes & $ 15 $ & no & $ 9 $ & \checkmark & \checkmark & 2  & \ref{ex:integerAO}\\
$ \frac{1}{2} \, \sqrt{13} + \frac{1}{2} $ & $ -\sqrt{13} $ & $ x^{2} - 13 $ & yes & $ 14 $ & yes & $ 9 $ & \checkmark & \checkmark & 4 \\
$ -\frac{1}{2} \, \sqrt{17} + \frac{3}{2} $ & $ -\sqrt{17} $ & $ x^{2} - 17 $ & yes & $ 18 $ & yes & $ 9 $ & \checkmark & \checkmark & 4  & \ref{ex:integerAQ}\\
$ -\frac{1}{2} \, \sqrt{21} + \frac{3}{2} $ & $ -\sqrt{21} $ & $ x^{2} - 21 $ & yes & $ 22 $ & yes & $ 9 $ & \checkmark & \checkmark & 4  & \ref{ex:integerAR}\\
\end{tabular}
	\end{center}
\caption{Quadratic bases with an integer alphabet}
\label{tab:resultsQuadrInt}
\end{table}



\section{Cubic bases}


\afterpage{
\clearpage   % To flush out all floats, might not be what you want
\begin{landscape}
	\begin{table}[h]
		\begin{center}
		\begin{tabular}{l|c|cc c| c c| c| c c c }
Ex. &$\omega$ & $\beta$ & $m_\beta$ & conj. & $\#\A$ & min. & $\#\Q$ & $bb\dots b$ & Phase 2 & $r$   \\ \hline
\ref{ex:cubicAA} & $ 2^{\frac{1}{3}} $ & $ -2^{\frac{1}{3}} $ & $ x^{3} + 2 $ & no & $ 3 $ & yes & $ 27 $ & \checkmark & \checkmark & 6 \\
\ref{ex:cubicAB} & $ 2^{\frac{1}{3}} $ & $ 2^{\frac{1}{3}} $ & $ x^{3} - 2 $ & yes & $ 3 $ & yes & $ 27 $ & \checkmark & \checkmark & 6 \\
\ref{ex:cubicAC} & $ {\left(\frac{1}{9} \, \sqrt{19} \sqrt{3} + 1\right)}^{\frac{1}{3}} + \frac{2}{3 \, {\left(\frac{1}{9} \, \sqrt{19} \sqrt{3} + 1\right)}^{\frac{1}{3}}} $ & $ {\left(\sqrt{57} - \frac{197}{27}\right)}^{\frac{1}{3}} - \frac{14}{9 \, {\left(\sqrt{57} - \frac{197}{27}\right)}^{\frac{1}{3}}} - \frac{2}{3} $ & $ x^{3} + 2 \, x^{2} + 6 \, x + 18 $ & no & $ 31 $ & no & $ 83 $ & \checkmark & \xmark & None \\
\ref{ex:cubicAD} & $ {\left(\frac{1}{9} \, \sqrt{29} \sqrt{3} + \frac{28}{27}\right)}^{\frac{1}{3}} + \frac{1}{9 \, {\left(\frac{1}{9} \, \sqrt{29} \sqrt{3} + \frac{28}{27}\right)}^{\frac{1}{3}}} + \frac{1}{3} $ & $ {\left(\frac{2}{9} \, \sqrt{29} \sqrt{3} - 2\right)}^{\frac{1}{3}} - \frac{2}{3 \, {\left(\frac{2}{9} \, \sqrt{29} \sqrt{3} - 2\right)}^{\frac{1}{3}}} - 1 $ & $ x^{3} + 3 \, x^{2} + 5 \, x + 7 $ & no & $ 16 $ & yes & $ 99 $ & \checkmark & \xmark & - \\
\end{tabular}
		\end{center}
	\caption{Cubic bases}
	\label{tab:resultsQuadrInt}
\end{table}
\end{landscape}
}

%\chapter*{Summary}
\newpage
\bibliography{literatura}
\addcontentsline{toc}{chapter}{References}
\bibliographystyle{amsplain}

\appendix
\chapter*{Appendices}
\pagenumbering{Roman}
\addcontentsline{toc}{chapter}{Appendices}
\renewcommand{\thesection}{\Alph{section}}
\counterwithin{exmp}{section}
\section{Illustration of Phase 1}
\label{app:phase1}   
Figures \ref{img:phase1img3} -- \ref{img:phase1img12} illustrates first and last iterations of the construction of the weight coefficients set $\Q$ for the Eisenstein base $\beta = -\frac{3}{2} + \frac{\imath \sqrt{3}}{2}$ with the complex alphabet $\mathcal{A} =\{0, 1, -1, \omega, -\omega, -\omega - 1, \omega + 1\}$ and input alphabet $\B=\A+\A$. The second last iteration is skipped.

\figurehascaptionOne{1 = The starting set $\Q_0{=}\{0\}$.}
\figurehascaptionOne{2 = The set $\B+\Q_0$ need to be covered.}
\figurehascaptionOne{3 = The set $\Q_0$ does not cover the set $\B+\Q_0${,} i.e.{,} the set $\A+\beta \cdot \Q_0$ is not superset of $\B+\Q_0$.}
\figurehascaptionOne{4 = The set $\Q_0$ is extended to $\Q_1$ to cover all elements of $\B+\Q_0$.}
\figurehascaptionOne{5 = The set $\B+\Q_1$ need to be covered.}
\figurehascaptionOne{6 = The set $\Q_1$ does not cover the set $\B+\Q_1${,} i.e.{,} the set $\A+\beta \cdot \Q_1$ is not superset of $\B+\Q_1$.}
\figurehascaptionOne{7 = The set $\Q_1$ is extended to $\Q_2$ to cover all elements of $\B+\Q_1$.}
\figurehascaptionOne{8 = The set $\B+\Q_2$ need to be covered.}
\figurehascaptionOne{9 = The set $\Q_2$ does not cover the set $\B+\Q_2${,} i.e.{,} the set $\A+\beta \cdot \Q_2$ is not superset of $\B+\Q_2$.}
\figurehascaptionOne{10 = The set $\Q_2$ is extended to $\Q_3$ to cover all elements of $\B+\Q_2$.}
\figurehascaptionOne{11 = The set $\B+\Q_3$ need to be covered.}
\figurehascaptionOne{12 = In the last iteration{,} the set $\Q_3$ covers the set $\B+\Q_3${,} i.e.{,} the set $\A+\beta \cdot \Q_3$ is superset of $\B+\Q_2$. The weight coefficients set $\Q$ equals $\Q_3$.}
\figurehascaptionOne{13 = The final weight coefficients $\Q{=}\Q_3$.}


\foreach \n in {3,4,6,7,12} {%
\begin{SCfigure}[][htbp]
    \centering
    \caption{\getcaptionOne{\n}}
    \label{img:phase1img\n}
    \includegraphics[height=0.27\textheight]{img/eisenstein/phase1_image_\n.png}
\end{SCfigure}
    }

\newpage


\section{Illustration of Phase 2}
The construction of set $\Q_{[\omega,1,2]}$ for the Eisenstein base $\beta = -\frac{3}{2} + \frac{\imath \sqrt{3}}{2}$ with the complex alphabet $\mathcal{A} =\{0, 1, -1, \omega, -\omega, -\omega - 1, \omega + 1\}$  and input alphabet $\B=\A+\A$ is illustrated on Figures \ref{img:phase2img3} -- \ref{img:phase2img7}.
\label{app:phase2}    

\figurehascaptionTwo{1 = Phase 2 starts with the weight coefficients set $\Q$ from Phase 1.}
\figurehascaptionTwo{2 = The set $\omega+\Q$ need to be covered.}
\figurehascaptionTwo{3 = The elements of $\omega+\Q$ are covered by the set $\Q_{[\omega]}\subset\Q$.}
\figurehascaptionTwo{4 = The set $\omega+\Q_{[1]}$ need to be covered.}
\figurehascaptionTwo{5 = The elements of $\omega+\Q_{[1]}$ are covered by the set $\Q_{[\omega,1]}\subset\Q_{[\omega]}$.}
\figurehascaptionTwo{6 = The set $\omega+\Q_{[1,2]}$ need to be covered.}
\figurehascaptionTwo{7 = The elements of $\omega+\Q_{[1]}$ are covered by the set $\Q_{[\omega,1,2]}\subset\Q_{[\omega,1]}$ which has only one element{.} This element is the output of the weight function $q{(\omega,1,2)}$.}



\foreach \n in {3,5,7} {%
\begin{SCfigure}[][htbp]
    \centering
    \caption{\getcaptionTwo{\n}}
    \label{img:phase2img\n}
    \includegraphics[height=0.23\textheight]{img/eisenstein/phase2_image_\n.png}
\end{SCfigure}
    }

\newpage

\section{Interfaces}
\label{app:interfaces}
File \verb+ewm_inputs.sage+:
\lstinputlisting[language=Python]{ewm_inputs.sage}

File \verb+ewm_gspreadsheet.sage+:
\lstinputlisting[language=Python]{ewm_gspreadsheet.sage}




\section{Tested examples}
\label{app:examples}

\subsection*{Unsuccessful examples comparing different methods}
The reasons of failure of Phase~2 for numeration systems in Section~\ref{sec:compareMethods} can be found here. See Tables \ref{tab:resultsPhaseOne}, \ref{tab:resultsPhaseTwo} and \ref{tab:alphabets} for parameters of the numeration systems.
\begin{exmp}
\label{ex:compareAA}


\rule{0cm}{0cm}

\begin{tabular}{ll}
$\omega=  \frac{1}{2} i \, \sqrt{3} - \frac{1}{2} $  & $\beta= \omega - 1 = \frac{1}{2} i \, \sqrt{3} - \frac{3}{2} $\\
$m_\omega(t)=  t^{2} + t + 1 $  & $m_\beta(x)=  x^{2} + 3 \, x + 3 $\\
Real conjugate of $\beta$ greater than 1:   &  no \\ \hline
\multicolumn{2}{l}{\begin{minipage}{\textwidth}\begin{dmath*}\A = \left\{0, 1, -1, \omega, -\omega, -\omega - 1, \omega + 1\right\}  \end{dmath*}\end{minipage} }\\
$\#\A= $ 7 $ $ & $\A$ is minimal. \\
\multicolumn{2}{l}{\begin{minipage}{\textwidth}\begin{dmath*}\B =\A+\A \end{dmath*}\end{minipage} }\\[10pt]
\multicolumn{2}{l}{\begin{minipage}{\textwidth}$\A$ divided into congruence classes modulo $\beta$: \begin{dmath*} \left\{\left\{0\right\}, \left\{1, \omega, -\omega - 1\right\}, \left\{-1, -\omega, \omega + 1\right\}\right\}  \end{dmath*}\end{minipage} }\\[10pt]
\multicolumn{2}{l}{\begin{minipage}{\textwidth}$\A$ divided into congruence classes modulo $\beta-1$: \begin{dmath*} \left\{\left\{0\right\}, \left\{1\right\}, \left\{-1\right\}, \left\{\omega\right\}, \left\{-\omega\right\}, \left\{-\omega - 1\right\}, \left\{\omega + 1\right\}\right\}  \end{dmath*}\end{minipage} }\\
 & \\ \hline
 & \\
\end{tabular}

\begin{tabular}{ll}
Phase 1 (methods $12, 13, 14, 15, 16$): &
\checkmark, $\#\mathcal{Q} =19$ \\ 
Method  9: &\\
$b,b,\dots,b$ inputs: & \checkmark \\
Phase 2: & \checkmark , $r= 3$ \\
Method  15: &\\
$b,b,\dots,b$ inputs: & \checkmark \\
Phase 2: & \checkmark , $r= 3$ \\
Method  22: &\\
$b,b,\dots,b$ inputs: & \checkmark \\
Phase 2: & \checkmark , $r= 3$ \\
Method  23: &\\
$b,b,\dots,b$ inputs: & \checkmark \\
Phase 2: & \checkmark , $r= 3$ \\
\hline
\end{tabular}

\end{exmp}




\begin{exmp}
\label{ex:compareAB}


\rule{0cm}{0cm}

\begin{tabular}{ll}
$\omega=  \frac{1}{2} i \, \sqrt{3} - \frac{1}{2} $  & $\beta= \omega - 1 = \frac{1}{2} i \, \sqrt{3} - \frac{3}{2} $\\
$m_\omega(t)=  t^{2} + t + 1 $  & $m_\beta(x)=  x^{2} + 3 \, x + 3 $\\
Real conjugate of $\beta$ greater than 1:   &  no \\ \hline
\multicolumn{2}{l}{\begin{minipage}{\textwidth}\begin{dmath*}\A = \left\{-3, -2, -1, 0, 1, 2, 3\right\}  \end{dmath*}\end{minipage} }\\
$\#\A= $ 7 $ $ & $\A$ is minimal. \\
\multicolumn{2}{l}{\begin{minipage}{\textwidth}\begin{dmath*}\B =\A+\A \end{dmath*}\end{minipage} }\\[10pt]
\multicolumn{2}{l}{\begin{minipage}{\textwidth}$\A$ divided into congruence classes modulo $\beta$: \begin{dmath*} \left\{\left\{-3, 0, 3\right\}, \left\{-2, 1\right\}, \left\{-1, 2\right\}\right\}  \end{dmath*}\end{minipage} }\\[10pt]
\multicolumn{2}{l}{\begin{minipage}{\textwidth}$\A$ divided into congruence classes modulo $\beta-1$: \begin{dmath*} \left\{\left\{-3\right\}, \left\{-2\right\}, \left\{-1\right\}, \left\{0\right\}, \left\{1\right\}, \left\{2\right\}, \left\{3\right\}\right\}  \end{dmath*}\end{minipage} }\\
 & \\ \hline
 & \\
\end{tabular}

\begin{tabular}{ll}
Phase 1 (methods $12, 13, 15, 16$): &
\checkmark, $\#\mathcal{Q} =57$ \\ 
Method  9: &\\
Failing $b,b,\dots,b$ inputs: & $\{2, 3, 5, 6, -5, -4, -3\}$ \\
Method  15: &\\
Failing $b,b,\dots,b$ inputs: & $\{2, 3, 6, -6, -4, -3\}$ \\
Method  22: &\\
Failing $b,b,\dots,b$ inputs: & $\{0, 1, 3, 4, 6, -6, -4, -3, -1\}$ \\
Method  23: &\\
Failing $b,b,\dots,b$ inputs: & $\{3, 4, 6, -6, -4, -3\}$ \\
\hline
Phase 1 (methods $14$): &
\checkmark, $\#\mathcal{Q} =139$ \\ 
Method  9: &\\
Failing $b,b,\dots,b$ inputs: & $\{0, 2, 4, 5, -2, -5, -4\}$ \\
Method  15: &\\
Failing $b,b,\dots,b$ inputs: & $\{0, 2, 4, 5, -2, -5, -4\}$ \\
Method  22: &\\
Failing $b,b,\dots,b$ inputs: & $\{0, 2, 3, 5, 6, -2, -6, -5, -3\}$ \\
Method  23: &\\
Failing $b,b,\dots,b$ inputs: & $\{0, 3, 4, 5, 6, -6, -5, -4, -3\}$ \\
\hline
\end{tabular}

\end{exmp}




\begin{exmp}
\label{ex:compareAC}


\rule{0cm}{0cm}

\begin{tabular}{ll}
$\omega=  \frac{1}{2} i \, \sqrt{3} - \frac{1}{2} $  & $\beta= -3 \, \omega = -\frac{3}{2} i \, \sqrt{3} + \frac{3}{2} $\\
$m_\omega(t)=  t^{2} + t + 1 $  & $m_\beta(x)=  x^{2} - 3 \, x + 9 $\\
Real conjugate of $\beta$ greater than 1:   &  no \\ \hline
\multicolumn{2}{l}{\begin{minipage}{\textwidth}\begin{dmath*}\A = \left\{0, 1, \omega, \omega + 1, 2 \, \omega, 2 \, \omega - 1, \omega - 1, -1, -2, -\omega, -\omega - 1, -\omega - 2, -2 \, \omega, -2 \, \omega - 1\right\}  \end{dmath*}\end{minipage} }\\
$\#\A= $ 14 $ $ & $\A$ is not minimal. \\
\multicolumn{2}{l}{\begin{minipage}{\textwidth}\begin{dmath*}\B =\A+\A \end{dmath*}\end{minipage} }\\[10pt]
\multicolumn{2}{l}{\begin{minipage}{\textwidth}$\A$ divided into congruence classes modulo $\beta$: \begin{dmath*} \left\{\left\{0\right\}, \left\{1, -2\right\}, \left\{\omega, -2 \, \omega\right\}, \left\{\omega + 1\right\}, \left\{2 \, \omega, -\omega\right\}, \left\{2 \, \omega - 1, -\omega - 1\right\}, \left\{\omega - 1, -2 \, \omega - 1\right\}, \left\{-1\right\}, \left\{-\omega - 2\right\}\right\}  \end{dmath*}\end{minipage} }\\[10pt]
\multicolumn{2}{l}{\begin{minipage}{\textwidth}$\A$ divided into congruence classes modulo $\beta-1$: \begin{dmath*} \left\{\left\{0\right\}, \left\{1, \omega - 1\right\}, \left\{\omega, -2 \, \omega - 1\right\}, \left\{\omega + 1, 2 \, \omega - 1, -\omega - 2, -2 \, \omega\right\}, \left\{2 \, \omega, -\omega - 1\right\}, \left\{-1\right\}, \left\{-2, -\omega\right\}\right\}  \end{dmath*}\end{minipage} }\\
 & \\ \hline
 & \\
\end{tabular}

\begin{tabular}{ll}
Phase 1 (methods $12, 13, 14, 15, 16$): &
\checkmark, $\#\mathcal{Q} =17$ \\ 
Method  9: &\\
Failing $b,b,\dots,b$ inputs: & $\{2\omega - 1, \omega + 1, -2\omega, -4, -\omega - 2\}$ \\
Method  15: &\\
Failing $b,b,\dots,b$ inputs: & $\{2\omega - 1, \omega + 1, -2\omega, -4, -\omega - 2\}$ \\
Method  22: &\\
Failing $b,b,\dots,b$ inputs: & $\{2\omega - 1, \omega + 1, -2\omega, -4, -\omega - 2\}$ \\
Method  23: &\\
Failing $b,b,\dots,b$ inputs: & $\{2\omega - 1, \omega + 1, -2\omega, -4, -\omega - 2\}$ \\
\hline
\end{tabular}

\end{exmp}




\begin{exmp}
\label{ex:compareAD}


\rule{0cm}{0cm}

\begin{tabular}{ll}
$\omega=  \frac{1}{2} i \, \sqrt{3} - \frac{1}{2} $  & $\beta= -3 \, \omega = -\frac{3}{2} i \, \sqrt{3} + \frac{3}{2} $\\
$m_\omega(t)=  t^{2} + t + 1 $  & $m_\beta(x)=  x^{2} - 3 \, x + 9 $\\
Real conjugate of $\beta$ greater than 1:   &  no \\ \hline
\multicolumn{2}{l}{\begin{minipage}{\textwidth}\begin{dmath*}\A = \left\{-\omega + 3, -\omega + 2, -\omega + 1, -\omega, 2, 1, 0, -1, \omega + 1, \omega, \omega - 1, \omega - 2, 2 \, \omega, 2 \, \omega - 1, 2 \, \omega - 2, 2 \, \omega - 3\right\}  \end{dmath*}\end{minipage} }\\
$\#\A= $ 16 $ $ & $\A$ is not minimal. \\
\multicolumn{2}{l}{\begin{minipage}{\textwidth}\begin{dmath*}\B =\A+\A \end{dmath*}\end{minipage} }\\[10pt]
\multicolumn{2}{l}{\begin{minipage}{\textwidth}$\A$ divided into congruence classes modulo $\beta$: \begin{dmath*} \left\{\left\{-\omega + 3, -\omega, 2 \, \omega, 2 \, \omega - 3\right\}, \left\{-\omega + 2, 2 \, \omega - 1\right\}, \left\{-\omega + 1, 2 \, \omega - 2\right\}, \left\{2, -1\right\}, \left\{1\right\}, \left\{0\right\}, \left\{\omega + 1, \omega - 2\right\}, \left\{\omega\right\}, \left\{\omega - 1\right\}\right\}  \end{dmath*}\end{minipage} }\\[10pt]
\multicolumn{2}{l}{\begin{minipage}{\textwidth}$\A$ divided into congruence classes modulo $\beta-1$: \begin{dmath*} \left\{\left\{-\omega + 3, 1, \omega - 1, 2 \, \omega - 3\right\}, \left\{-\omega + 2, 0, \omega - 2\right\}, \left\{-\omega + 1, -1\right\}, \left\{-\omega\right\}, \left\{2, \omega, 2 \, \omega - 2\right\}, \left\{\omega + 1, 2 \, \omega - 1\right\}, \left\{2 \, \omega\right\}\right\}  \end{dmath*}\end{minipage} }\\
 & \\ \hline
 & \\
\end{tabular}

\begin{tabular}{ll}
Phase 1 (methods $12, 13, 14, 15, 16$): &
\checkmark, $\#\mathcal{Q} =26$ \\ 
Method  9: &\\
Failing $b,b,\dots,b$ inputs: & $\{0, 1, 2, 2\omega - 4, \omega - 2, 4\omega, 3\omega - 5, \omega - 1, -\omega + 3, -2\omega + 5, 2\omega - 3\}$ \\
Method  15: &\\
Failing $b,b,\dots,b$ inputs: & $\{0, 1, 2, 2\omega - 4, 2\omega - 2, \omega - 2, \omega, 4\omega, 3\omega - 5, \omega - 1, -\omega + 3, -2\omega + 5, 2\omega - 3\}$ \\
Method  22: &\\
Failing $b,b,\dots,b$ inputs: & $\{0, 1, 2, 2\omega - 4, \omega - 2, \omega, 4\omega, 3\omega - 5, \omega - 1, -\omega + 3, -2\omega + 5, 2\omega - 3\}$ \\
Method  23: &\\
Failing $b,b,\dots,b$ inputs: & $\{1, 2, 2\omega - 4, \omega - 2, 4\omega, 3\omega - 5, \omega - 1, -\omega + 3, -2\omega + 5, 2\omega - 3\}$ \\
\hline
\end{tabular}

\end{exmp}




\begin{exmp}
\label{ex:compareAE}


\rule{0cm}{0cm}

\begin{tabular}{ll}
$\omega=  i - 1 $  & $\beta= \omega = i - 1 $\\
$m_\omega(t)=  t^{2} + 2 \, t + 2 $  & $m_\beta(x)=  x^{2} + 2 \, x + 2 $\\
Real conjugate of $\beta$ greater than 1:   &  no \\ \hline
\multicolumn{2}{l}{\begin{minipage}{\textwidth}\begin{dmath*}\A = \left\{0, \omega + 1, -\omega - 1, 1, -1\right\}  \end{dmath*}\end{minipage} }\\
$\#\A= $ 5 $ $ & $\A$ is minimal. \\
\multicolumn{2}{l}{\begin{minipage}{\textwidth}\begin{dmath*}\B =\A+\A \end{dmath*}\end{minipage} }\\[10pt]
\multicolumn{2}{l}{\begin{minipage}{\textwidth}$\A$ divided into congruence classes modulo $\beta$: \begin{dmath*} \left\{\left\{0\right\}, \left\{\omega + 1, -\omega - 1, 1, -1\right\}\right\}  \end{dmath*}\end{minipage} }\\[10pt]
\multicolumn{2}{l}{\begin{minipage}{\textwidth}$\A$ divided into congruence classes modulo $\beta-1$: \begin{dmath*} \left\{\left\{0\right\}, \left\{\omega + 1\right\}, \left\{-\omega - 1\right\}, \left\{1\right\}, \left\{-1\right\}\right\}  \end{dmath*}\end{minipage} }\\
 & \\ \hline
 & \\
\end{tabular}

\begin{tabular}{ll}
Phase 1 (methods $12, 13, 14, 15, 16$): &
\checkmark, $\#\mathcal{Q} =45$ \\ 
Method  9: &\\
$b,b,\dots,b$ inputs: & \checkmark \\
Phase 2: & \checkmark , $r= 6$ \\
Method  15: &\\
$b,b,\dots,b$ inputs: & \checkmark \\
Phase 2: & \checkmark , $r= 6$ \\
Method  22: &\\
$b,b,\dots,b$ inputs: & \checkmark \\
Phase 2: & \checkmark , $r= 6$ \\
Method  23: &\\
$b,b,\dots,b$ inputs: & \checkmark \\
Phase 2: & \checkmark , $r= 6$ \\
\hline
\end{tabular}

\end{exmp}




\begin{exmp}
\label{ex:compareAF}


\rule{0cm}{0cm}

\begin{tabular}{ll}
$\omega=  i $  & $\beta= \omega - 1 = i - 1 $\\
$m_\omega(t)=  t^{2} + 1 $  & $m_\beta(x)=  x^{2} + 2 \, x + 2 $\\
Real conjugate of $\beta$ greater than 1:   &  no \\ \hline
\multicolumn{2}{l}{\begin{minipage}{\textwidth}\begin{dmath*}\A = \left\{-2, -1, 0, 1, 2\right\}  \end{dmath*}\end{minipage} }\\
$\#\A= $ 5 $ $ & $\A$ is minimal. \\
\multicolumn{2}{l}{\begin{minipage}{\textwidth}\begin{dmath*}\B = \left\{-3, -2, -1, 0, 1, 2, 3\right\}  \end{dmath*}\end{minipage} }\\[10pt]
\multicolumn{2}{l}{\begin{minipage}{\textwidth}$\A$ divided into congruence classes modulo $\beta$: \begin{dmath*} \left\{\left\{-2, 0, 2\right\}, \left\{-1, 1\right\}\right\}  \end{dmath*}\end{minipage} }\\[10pt]
\multicolumn{2}{l}{\begin{minipage}{\textwidth}$\A$ divided into congruence classes modulo $\beta-1$: \begin{dmath*} \left\{\left\{-2\right\}, \left\{-1\right\}, \left\{0\right\}, \left\{1\right\}, \left\{2\right\}\right\}  \end{dmath*}\end{minipage} }\\
 & \\ \hline
 & \\
\end{tabular}

\begin{tabular}{ll}
Phase 1 (methods $12, 13, 15, 16$): &
\checkmark, $\#\mathcal{Q} =27$ \\ 
Method  9: &\\
Failing $b,b,\dots,b$ inputs: & $\{-3, -2, 1, 2\}$ \\
Method  15: &\\
Failing $b,b,\dots,b$ inputs: & $\{-2, 2\}$ \\
Method  22: &\\
Failing $b,b,\dots,b$ inputs: & $\{-3, -2, 2, 3\}$ \\
Method  23: &\\
Failing $b,b,\dots,b$ inputs: & $\{-3, -2, 2, 3\}$ \\
\hline
Phase 1 (methods $14$): &
\checkmark, $\#\mathcal{Q} =95$ \\ 
Method  9: &\\
Failing $b,b,\dots,b$ inputs: & $\{-3, 0, 1, 3\}$ \\
Method  15: &\\
Failing $b,b,\dots,b$ inputs: & $\{-3, 0, 1, 3\}$ \\
Method  22: &\\
Failing $b,b,\dots,b$ inputs: & $\{0, 3\}$ \\
Method  23: &\\
Failing $b,b,\dots,b$ inputs: & $\{0, 3\}$ \\
\hline
\end{tabular}

\end{exmp}




\begin{exmp}
\label{ex:compareAG}


\rule{0cm}{0cm}

\begin{tabular}{ll}
$\omega=  i $  & $\beta= -2 \, \omega = -2 i $\\
$m_\omega(t)=  t^{2} + 1 $  & $m_\beta(x)=  x^{2} + 4 $\\
Real conjugate of $\beta$ greater than 1:   &  no \\ \hline
\multicolumn{2}{l}{\begin{minipage}{\textwidth}\begin{dmath*}\A = \left\{0, 1, -1, \omega, -\omega, \omega - 1, -\omega + 1, \omega - 2, -\omega + 2\right\}  \end{dmath*}\end{minipage} }\\
$\#\A= $ 9 $ $ & $\A$ is not minimal. \\
\multicolumn{2}{l}{\begin{minipage}{\textwidth}\begin{dmath*}\B =\A+\A \end{dmath*}\end{minipage} }\\[10pt]
\multicolumn{2}{l}{\begin{minipage}{\textwidth}$\A$ divided into congruence classes modulo $\beta$: \begin{dmath*} \left\{\left\{0\right\}, \left\{1, -1\right\}, \left\{\omega, -\omega, \omega - 2, -\omega + 2\right\}, \left\{\omega - 1, -\omega + 1\right\}\right\}  \end{dmath*}\end{minipage} }\\[10pt]
\multicolumn{2}{l}{\begin{minipage}{\textwidth}$\A$ divided into congruence classes modulo $\beta-1$: \begin{dmath*} \left\{\left\{0, \omega - 2, -\omega + 2\right\}, \left\{1, \omega - 1\right\}, \left\{-1, -\omega + 1\right\}, \left\{\omega\right\}, \left\{-\omega\right\}\right\}  \end{dmath*}\end{minipage} }\\
 & \\ \hline
 & \\
\end{tabular}

\begin{tabular}{ll}
Phase 1 (methods $12, 13, 14, 15, 16$): &
\checkmark, $\#\mathcal{Q} =27$ \\ 
Method  9: &\\
$b,b,\dots,b$ inputs: & \checkmark \\
Phase 2: & \checkmark , $r= 5$ \\
Method  15: &\\
$b,b,\dots,b$ inputs: & \checkmark \\
Phase 2: & \checkmark , $r= 5$ \\
Method  22: &\\
$b,b,\dots,b$ inputs: & \checkmark \\
Phase 2: & \checkmark , $r= 5$ \\
Method  23: &\\
$b,b,\dots,b$ inputs: & \checkmark \\
Phase 2: & \checkmark , $r= 5$ \\
\hline
\end{tabular}

\end{exmp}




\begin{exmp}
\label{ex:compareAH}


\rule{0cm}{0cm}

\begin{tabular}{ll}
$\omega=  \sqrt{2} $  & $\beta= \omega = \sqrt{2} $\\
$m_\omega(t)=  t^{2} - 2 $  & $m_\beta(x)=  x^{2} - 2 $\\
Real conjugate of $\beta$ greater than 1:   &  yes \\ \hline
\multicolumn{2}{l}{\begin{minipage}{\textwidth}\begin{dmath*}\A = \left\{0, 1, -1\right\}  \end{dmath*}\end{minipage} }\\
$\#\A= $ 3 $ $ & $\A$ is minimal. \\
\multicolumn{2}{l}{\begin{minipage}{\textwidth}\begin{dmath*}\B =\A+\A \end{dmath*}\end{minipage} }\\[10pt]
\multicolumn{2}{l}{\begin{minipage}{\textwidth}$\A$ divided into congruence classes modulo $\beta$: \begin{dmath*} \left\{\left\{0\right\}, \left\{1, -1\right\}\right\}  \end{dmath*}\end{minipage} }\\[10pt]
\multicolumn{2}{l}{\begin{minipage}{\textwidth}$\A$ divided into congruence classes modulo $\beta-1$: \begin{dmath*} \left\{\left\{0, 1, -1\right\}\right\}  \end{dmath*}\end{minipage} }\\
 & \\ \hline
 & \\
\end{tabular}

\begin{tabular}{ll}
Phase 1 (methods $12, 13, 14, 15, 16$): &
\checkmark, $\#\mathcal{Q} =9$ \\ 
Method  9: &\\
$b,b,\dots,b$ inputs: & \checkmark \\
Phase 2: & \checkmark , $r= 5$ \\
Method  15: &\\
$b,b,\dots,b$ inputs: & \checkmark \\
Phase 2: & \checkmark , $r= 5$ \\
Method  22: &\\
$b,b,\dots,b$ inputs: & \checkmark \\
Phase 2: & \checkmark , $r= 5$ \\
Method  23: &\\
$b,b,\dots,b$ inputs: & \checkmark \\
Phase 2: & \checkmark , $r= 4$ \\
\hline
\end{tabular}

\end{exmp}




\begin{exmp}
\label{ex:compareAI}


\rule{0cm}{0cm}

\begin{tabular}{ll}
$\omega=  -\frac{1}{2} \, \sqrt{21} + \frac{3}{2} $  & $\beta= 2 \, \omega - 3 = -\sqrt{21} $\\
$m_\omega(t)=  t^{2} - 3 \, t - 3 $  & $m_\beta(x)=  x^{2} - 21 $\\
Real conjugate of $\beta$ greater than 1:   &  yes \\ \hline
\multicolumn{2}{l}{\begin{minipage}{\textwidth}\begin{dmath*}\A = \left\{-10, -9, -8, -7, -6, -5, -4, -3, -2, -1, 0, 1, 2, 3, 4, 5, 6, 7, 8, 9, 10, 11\right\}  \end{dmath*}\end{minipage} }\\
$\#\A= $ 22 $ $ & $\A$ is minimal. \\
\multicolumn{2}{l}{\begin{minipage}{\textwidth}\begin{dmath*}\B =\A+\A \end{dmath*}\end{minipage} }\\[10pt]
\multicolumn{2}{l}{\begin{minipage}{\textwidth}$\A$ divided into congruence classes modulo $\beta$: \begin{dmath*} \left\{\left\{-10, 11\right\}, \left\{-9\right\}, \left\{-8\right\}, \left\{-7\right\}, \left\{-6\right\}, \left\{-5\right\}, \left\{-4\right\}, \left\{-3\right\}, \left\{-2\right\}, \left\{-1\right\}, \left\{0\right\}, \left\{1\right\}, \left\{2\right\}, \left\{3\right\}, \left\{4\right\}, \left\{5\right\}, \left\{6\right\}, \left\{7\right\}, \left\{8\right\}, \left\{9\right\}, \left\{10\right\}\right\}  \end{dmath*}\end{minipage} }\\[10pt]
\multicolumn{2}{l}{\begin{minipage}{\textwidth}$\A$ divided into congruence classes modulo $\beta-1$: \begin{dmath*} \left\{\left\{-10, 0, 10\right\}, \left\{-9, 1, 11\right\}, \left\{-8, 2\right\}, \left\{-7, 3\right\}, \left\{-6, 4\right\}, \left\{-5, 5\right\}, \left\{-4, 6\right\}, \left\{-3, 7\right\}, \left\{-2, 8\right\}, \left\{-1, 9\right\}\right\}  \end{dmath*}\end{minipage} }\\
 & \\ \hline
 & \\
\end{tabular}

\begin{tabular}{ll}
Phase 1 (methods $12, 13, 14, 15, 16$): &
\checkmark, $\#\mathcal{Q} =9$ \\ 
Method  9: &\\
$b,b,\dots,b$ inputs: & \checkmark \\
Phase 2: & \checkmark , $r= 4$ \\
Method  15: &\\
$b,b,\dots,b$ inputs: & \checkmark \\
Phase 2: & \checkmark , $r= 4$ \\
Method  22: &\\
$b,b,\dots,b$ inputs: & \checkmark \\
Phase 2: & \checkmark , $r= 4$ \\
Method  23: &\\
$b,b,\dots,b$ inputs: & \checkmark \\
Phase 2: & \checkmark , $r= 4$ \\
\hline
\end{tabular}

\end{exmp}




\begin{exmp}
\label{ex:compareAJ}


\rule{0cm}{0cm}

\begin{tabular}{ll}
$\omega=  \sqrt{3} - 1 $  & $\beta= -\omega - 1 = -\sqrt{3} $\\
$m_\omega(t)=  t^{2} + 2 \, t - 2 $  & $m_\beta(x)=  x^{2} - 3 $\\
Real conjugate of $\beta$ greater than 1:   &  yes \\ \hline
\multicolumn{2}{l}{\begin{minipage}{\textwidth}\begin{dmath*}\A = \left\{0, 1, -1, 2\right\}  \end{dmath*}\end{minipage} }\\
$\#\A= $ 4 $ $ & $\A$ is minimal. \\
\multicolumn{2}{l}{\begin{minipage}{\textwidth}\begin{dmath*}\B =\A+\A \end{dmath*}\end{minipage} }\\[10pt]
\multicolumn{2}{l}{\begin{minipage}{\textwidth}$\A$ divided into congruence classes modulo $\beta$: \begin{dmath*} \left\{\left\{0\right\}, \left\{1\right\}, \left\{-1, 2\right\}\right\}  \end{dmath*}\end{minipage} }\\[10pt]
\multicolumn{2}{l}{\begin{minipage}{\textwidth}$\A$ divided into congruence classes modulo $\beta-1$: \begin{dmath*} \left\{\left\{0, 2\right\}, \left\{1, -1\right\}\right\}  \end{dmath*}\end{minipage} }\\
 & \\ \hline
 & \\
\end{tabular}

\begin{tabular}{ll}
Phase 1 (methods $12, 13, 14, 15, 16$): &
\checkmark, $\#\mathcal{Q} =9$ \\ 
Method  9: &\\
$b,b,\dots,b$ inputs: & \checkmark \\
Phase 2: & \checkmark , $r= 4$ \\
Method  15: &\\
$b,b,\dots,b$ inputs: & \checkmark \\
Phase 2: & \checkmark , $r= 5$ \\
Method  22: &\\
$b,b,\dots,b$ inputs: & \checkmark \\
Phase 2: & \checkmark , $r= 5$ \\
Method  23: &\\
$b,b,\dots,b$ inputs: & \checkmark \\
Phase 2: & \checkmark , $r= 5$ \\
\hline
\end{tabular}

\end{exmp}




\begin{exmp}
\label{ex:compareAK}


\rule{0cm}{0cm}

\begin{tabular}{ll}
$\omega=  \frac{1}{2} \, \sqrt{5} - \frac{1}{2} $  & $\beta= 2 \, \omega + 1 = \sqrt{5} $\\
$m_\omega(t)=  t^{2} + t - 1 $  & $m_\beta(x)=  x^{2} - 5 $\\
Real conjugate of $\beta$ greater than 1:   &  yes \\ \hline
\multicolumn{2}{l}{\begin{minipage}{\textwidth}\begin{dmath*}\A = \left\{-3, -2, -1, 0, 1, 2, 3, 4\right\}  \end{dmath*}\end{minipage} }\\
$\#\A= $ 8 $ $ & $\A$ is not minimal. \\
\multicolumn{2}{l}{\begin{minipage}{\textwidth}\begin{dmath*}\B =\A+\A \end{dmath*}\end{minipage} }\\[10pt]
\multicolumn{2}{l}{\begin{minipage}{\textwidth}$\A$ divided into congruence classes modulo $\beta$: \begin{dmath*} \left\{\left\{-3, 2\right\}, \left\{-2, 3\right\}, \left\{-1, 4\right\}, \left\{0\right\}, \left\{1\right\}\right\}  \end{dmath*}\end{minipage} }\\[10pt]
\multicolumn{2}{l}{\begin{minipage}{\textwidth}$\A$ divided into congruence classes modulo $\beta-1$: \begin{dmath*} \left\{\left\{-3, -1, 1, 3\right\}, \left\{-2, 0, 2, 4\right\}\right\}  \end{dmath*}\end{minipage} }\\
 & \\ \hline
 & \\
\end{tabular}

\begin{tabular}{ll}
Phase 1 (methods $12, 13, 14, 15, 16$): &
\checkmark, $\#\mathcal{Q} =9$ \\ 
Method  9: &\\
Failing $b,b,\dots,b$ inputs: & $\{-2\}$ \\
Method  15: &\\
$b,b,\dots,b$ inputs: & \checkmark \\
Phase 2: & \checkmark , $r= 3$ \\
Method  22: &\\
$b,b,\dots,b$ inputs: & \checkmark \\
Phase 2: & \checkmark , $r= 2$ \\
Method  23: &\\
$b,b,\dots,b$ inputs: & \checkmark \\
Phase 2: & \checkmark , $r= 2$ \\
\hline
\end{tabular}

\end{exmp}




\begin{exmp}
\label{ex:compareAL}


\rule{0cm}{0cm}

\begin{tabular}{ll}
$\omega=  i \, \sqrt{2} - 1 $  & $\beta= -\omega - 2 = -i \, \sqrt{2} - 1 $\\
$m_\omega(t)=  t^{2} + 2 \, t + 3 $  & $m_\beta(x)=  x^{2} + 2 \, x + 3 $\\
Real conjugate of $\beta$ greater than 1:   &  no \\ \hline
\multicolumn{2}{l}{\begin{minipage}{\textwidth}\begin{dmath*}\A = \left\{0, \omega + 1, -\omega - 1, 1, -1, \omega\right\}  \end{dmath*}\end{minipage} }\\
$\#\A= $ 6 $ $ & $\A$ is minimal. \\
\multicolumn{2}{l}{\begin{minipage}{\textwidth}\begin{dmath*}\B =\A+\A \end{dmath*}\end{minipage} }\\[10pt]
\multicolumn{2}{l}{\begin{minipage}{\textwidth}$\A$ divided into congruence classes modulo $\beta$: \begin{dmath*} \left\{\left\{0\right\}, \left\{\omega + 1, -1\right\}, \left\{-\omega - 1, 1, \omega\right\}\right\}  \end{dmath*}\end{minipage} }\\[10pt]
\multicolumn{2}{l}{\begin{minipage}{\textwidth}$\A$ divided into congruence classes modulo $\beta-1$: \begin{dmath*} \left\{\left\{0\right\}, \left\{\omega + 1\right\}, \left\{-\omega - 1\right\}, \left\{1\right\}, \left\{-1\right\}, \left\{\omega\right\}\right\}  \end{dmath*}\end{minipage} }\\
 & \\ \hline
 & \\
\end{tabular}

\begin{tabular}{ll}
Phase 1 (methods $12, 13, 14, 15, 16$): &
\checkmark, $\#\mathcal{Q} =27$ \\ 
Method  9: &\\
$b,b,\dots,b$ inputs: & \checkmark \\
\multicolumn{2}{l}{\begin{minipage}{\textwidth} Phase 2 fails because  the sequence $(1, -2\omega - 2, -\omega - 2, 2\omega + 2, -2\omega - 2, -\omega - 2, 2\omega + 2, -2\omega - 2, \dots ,-2\omega - 2, -\omega - 2, 2\omega + 2, -2\omega - 2, \dots)$ leads to an infinite loop.\end{minipage} }\\
Method  15: &\\
$b,b,\dots,b$ inputs: & \checkmark \\
Phase 2: & \checkmark , $r= 7$ \\
Method  22: &\\
$b,b,\dots,b$ inputs: & \checkmark \\
\multicolumn{2}{l}{\begin{minipage}{\textwidth} Phase 2 fails because  the sequence $(1, -2\omega - 2, -\omega - 2, 2\omega + 2, -2\omega - 2, -\omega - 2, 2\omega + 2, -2\omega - 2, \dots ,-2\omega - 2, -\omega - 2, 2\omega + 2, -2\omega - 2, \dots)$ leads to an infinite loop.\end{minipage} }\\
Method  23: &\\
$b,b,\dots,b$ inputs: & \checkmark \\
\multicolumn{2}{l}{\begin{minipage}{\textwidth} Phase 2 fails because  the sequence $(0, \omega + 2, -1, -1, 2\omega + 1, -1, -1, 2\omega + 1, -1, \dots ,-1, -1, 2\omega + 1, -1, \dots)$ leads to an infinite loop.\end{minipage} }\\
\hline
\end{tabular}

\end{exmp}




\begin{exmp}
\label{ex:compareAM}


\rule{0cm}{0cm}

\begin{tabular}{ll}
$\omega=  \frac{1}{2} i \, \sqrt{7} - \frac{1}{2} $  & $\beta= \omega - 1 = \frac{1}{2} i \, \sqrt{7} - \frac{3}{2} $\\
$m_\omega(t)=  t^{2} + t + 2 $  & $m_\beta(x)=  x^{2} + 3 \, x + 4 $\\
Real conjugate of $\beta$ greater than 1:   &  no \\ \hline
\multicolumn{2}{l}{\begin{minipage}{\textwidth}\begin{dmath*}\A = \left\{0, \omega + 1, -\omega - 1, 1, -1, \omega, -\omega, \omega + 2\right\}  \end{dmath*}\end{minipage} }\\
$\#\A= $ 8 $ $ & $\A$ is minimal. \\
\multicolumn{2}{l}{\begin{minipage}{\textwidth}\begin{dmath*}\B =\A+\A \end{dmath*}\end{minipage} }\\[10pt]
\multicolumn{2}{l}{\begin{minipage}{\textwidth}$\A$ divided into congruence classes modulo $\beta$: \begin{dmath*} \left\{\left\{0\right\}, \left\{\omega + 1, -\omega - 1\right\}, \left\{1, \omega\right\}, \left\{-1, -\omega, \omega + 2\right\}\right\}  \end{dmath*}\end{minipage} }\\[10pt]
\multicolumn{2}{l}{\begin{minipage}{\textwidth}$\A$ divided into congruence classes modulo $\beta-1$: \begin{dmath*} \left\{\left\{0\right\}, \left\{\omega + 1\right\}, \left\{-\omega - 1\right\}, \left\{1\right\}, \left\{-1\right\}, \left\{\omega\right\}, \left\{-\omega\right\}, \left\{\omega + 2\right\}\right\}  \end{dmath*}\end{minipage} }\\
 & \\ \hline
 & \\
\end{tabular}

\begin{tabular}{ll}
Phase 1 (methods $12$): &
\checkmark, $\#\mathcal{Q} =20$ \\ 
Method  9: &\\
$b,b,\dots,b$ inputs: & \checkmark \\
Phase 2: & \checkmark , $r= 7$ \\
Method  15: &\\
$b,b,\dots,b$ inputs: & \checkmark \\
Phase 2: & \checkmark , $r= 7$ \\
Method  22: &\\
$b,b,\dots,b$ inputs: & \checkmark \\
\multicolumn{2}{l}{\begin{minipage}{\textwidth} Phase 2 fails because  the sequence $(2, 2\omega + 1, -\omega - 2, -2, 2\omega, 2\omega + 1, -\omega - 2, -2, 2\omega, \dots ,2\omega + 1, -\omega - 2, -2, 2\omega, \dots)$ leads to an infinite loop.\end{minipage} }\\
Method  23: &\\
Failing $b,b,\dots,b$ inputs: & $\{\omega + 3\}$ \\
\hline
Phase 1 (methods $16$): &
\checkmark, $\#\mathcal{Q} =19$ \\ 
Method  9: &\\
$b,b,\dots,b$ inputs: & \checkmark \\
\multicolumn{2}{l}{\begin{minipage}{\textwidth} Phase 2 fails because  the sequence $(2, 2\omega + 3, -2\omega - 2, 2, 2\omega + 3, -2\omega - 2, 2, \dots ,2, 2\omega + 3, -2\omega - 2, 2, \dots)$ leads to an infinite loop.\end{minipage} }\\
Method  15: &\\
$b,b,\dots,b$ inputs: & \checkmark \\
Phase 2: & \checkmark , $r= 7$ \\
Method  22: &\\
$b,b,\dots,b$ inputs: & \checkmark \\
\multicolumn{2}{l}{\begin{minipage}{\textwidth} Phase 2 fails because  the sequence $(2, 2\omega + 2, -\omega + 1, 2, 2\omega + 2, -\omega + 1, 2, \dots ,2, 2\omega + 2, -\omega + 1, 2, \dots)$ leads to an infinite loop.\end{minipage} }\\
Method  23: &\\
Failing $b,b,\dots,b$ inputs: & $\{\omega + 3\}$ \\
\hline
Phase 1 (methods $13, 15$): &
\checkmark, $\#\mathcal{Q} =20$ \\ 
Method  9: &\\
$b,b,\dots,b$ inputs: & \checkmark \\
\multicolumn{2}{l}{\begin{minipage}{\textwidth} Phase 2 fails because  the sequence $(2, 2\omega + 3, -2\omega - 2, 2, 2\omega + 3, -2\omega - 2, 2, \dots ,2, 2\omega + 3, -2\omega - 2, 2, \dots)$ leads to an infinite loop.\end{minipage} }\\
Method  15: &\\
$b,b,\dots,b$ inputs: & \checkmark \\
Phase 2: & \checkmark , $r= 7$ \\
Method  22: &\\
$b,b,\dots,b$ inputs: & \checkmark \\
\multicolumn{2}{l}{\begin{minipage}{\textwidth} Phase 2 fails because  the sequence $(2, 2\omega + 2, -\omega + 1, 2, 2\omega + 2, -\omega + 1, 2, \dots ,2, 2\omega + 2, -\omega + 1, 2, \dots)$ leads to an infinite loop.\end{minipage} }\\
Method  23: &\\
Failing $b,b,\dots,b$ inputs: & $\{\omega + 3\}$ \\
\hline
Phase 1 (methods $14$): &
\checkmark, $\#\mathcal{Q} =21$ \\ 
Method  9: &\\
$b,b,\dots,b$ inputs: & \checkmark \\
Phase 2: & \checkmark , $r= 7$ \\
Method  15: &\\
$b,b,\dots,b$ inputs: & \checkmark \\
Phase 2: & \checkmark , $r= 7$ \\
Method  22: &\\
$b,b,\dots,b$ inputs: & \checkmark \\
\multicolumn{2}{l}{\begin{minipage}{\textwidth} Phase 2 fails because  the sequence $(2, 2\omega + 2, -\omega + 1, 2, 2\omega + 2, -\omega + 1, 2, \dots ,2, 2\omega + 2, -\omega + 1, 2, \dots)$ leads to an infinite loop.\end{minipage} }\\
Method  23: &\\
Failing $b,b,\dots,b$ inputs: & $\{\omega + 3\}$ \\
\hline
\end{tabular}

\end{exmp}




\begin{exmp}
\label{ex:compareAN}


\rule{0cm}{0cm}

\begin{tabular}{ll}
$\omega=  \frac{1}{2} i \, \sqrt{11} - \frac{3}{2} $  & $\beta= \omega = \frac{1}{2} i \, \sqrt{11} - \frac{3}{2} $\\
$m_\omega(t)=  t^{2} + 3 \, t + 5 $  & $m_\beta(x)=  x^{2} + 3 \, x + 5 $\\
Real conjugate of $\beta$ greater than 1:   &  no \\ \hline
\multicolumn{2}{l}{\begin{minipage}{\textwidth}\begin{dmath*}\A = \left\{0, 1, -1, \omega + 1, -\omega - 1, \omega + 2, -\omega - 2, \omega + 3, -\omega - 3\right\}  \end{dmath*}\end{minipage} }\\
$\#\A= $ 9 $ $ & $\A$ is minimal. \\
\multicolumn{2}{l}{\begin{minipage}{\textwidth}\begin{dmath*}\B =\A+\A \end{dmath*}\end{minipage} }\\[10pt]
\multicolumn{2}{l}{\begin{minipage}{\textwidth}$\A$ divided into congruence classes modulo $\beta$: \begin{dmath*} \left\{\left\{0\right\}, \left\{1, \omega + 1\right\}, \left\{-1, -\omega - 1\right\}, \left\{\omega + 2, -\omega - 3\right\}, \left\{-\omega - 2, \omega + 3\right\}\right\}  \end{dmath*}\end{minipage} }\\[10pt]
\multicolumn{2}{l}{\begin{minipage}{\textwidth}$\A$ divided into congruence classes modulo $\beta-1$: \begin{dmath*} \left\{\left\{0\right\}, \left\{1\right\}, \left\{-1\right\}, \left\{\omega + 1\right\}, \left\{-\omega - 1\right\}, \left\{\omega + 2\right\}, \left\{-\omega - 2\right\}, \left\{\omega + 3\right\}, \left\{-\omega - 3\right\}\right\}  \end{dmath*}\end{minipage} }\\
 & \\ \hline
 & \\
\end{tabular}

\begin{tabular}{ll}
Phase 1 (methods $14$): &
\checkmark, $\#\mathcal{Q} =19$ \\ 
Method  9: &\\
Failing $b,b,\dots,b$ inputs: & $\{2\omega + 2, \omega + 4, -\omega - 4, -2\omega - 2\}$ \\
Method  15: &\\
Failing $b,b,\dots,b$ inputs: & $\{2\omega + 2, -2\omega - 2\}$ \\
Method  22: &\\
Failing $b,b,\dots,b$ inputs: & $\{2\omega + 2, \omega + 4, -\omega - 4, -2\omega - 2\}$ \\
Method  23: &\\
Failing $b,b,\dots,b$ inputs: & $\{2\omega + 2, 2\omega + 4, \omega + 4, -\omega - 4, -2\omega - 4, -2\omega - 2\}$ \\
\hline
Phase 1 (methods $12, 16$): &
\checkmark, $\#\mathcal{Q} =11$ \\ 
Method  9: &\\
Failing $b,b,\dots,b$ inputs: & $\{-2\omega - 2\}$ \\
Method  15: &\\
$b,b,\dots,b$ inputs: & \checkmark \\
\multicolumn{2}{l}{\begin{minipage}{\textwidth} Phase 2 fails because  the sequence $(2, 2\omega + 2, 2\omega + 2, 2, 2\omega + 2, 2\omega + 2, \dots ,2, 2\omega + 2, 2\omega + 2, \dots)$ leads to an infinite loop.\end{minipage} }\\
Method  22: &\\
Failing $b,b,\dots,b$ inputs: & $\{2\omega + 2, -2\omega - 2\}$ \\
Method  23: &\\
Failing $b,b,\dots,b$ inputs: & $\{2\omega + 2, -2\omega - 2\}$ \\
\hline
Phase 1 (methods $13, 15$): &
\checkmark, $\#\mathcal{Q} =17$ \\ 
Method  9: &\\
Failing $b,b,\dots,b$ inputs: & $\{2\omega + 2, -2\omega - 2\}$ \\
Method  15: &\\
Failing $b,b,\dots,b$ inputs: & $\{2\omega + 2, -2\omega - 2\}$ \\
Method  22: &\\
Failing $b,b,\dots,b$ inputs: & $\{2\omega + 2, -2\omega - 2\}$ \\
Method  23: &\\
Failing $b,b,\dots,b$ inputs: & $\{2\omega + 2, 2\omega + 4, -2\omega - 4, -2\omega - 2\}$ \\
\hline
\end{tabular}

\end{exmp}




\begin{exmp}
\label{ex:compareAO}


\rule{0cm}{0cm}

\begin{tabular}{ll}
$\omega=  \frac{1}{2} i \, \sqrt{11} - \frac{3}{2} $  & $\beta= \omega = \frac{1}{2} i \, \sqrt{11} - \frac{3}{2} $\\
$m_\omega(t)=  t^{2} + 3 \, t + 5 $  & $m_\beta(x)=  x^{2} + 3 \, x + 5 $\\
Real conjugate of $\beta$ greater than 1:   &  no \\ \hline
\multicolumn{2}{l}{\begin{minipage}{\textwidth}\begin{dmath*}\A = \left\{0, \omega + 1, -\omega - 1, 1, -1, 2 \, \omega + 2, -2 \, \omega - 2, \omega + 2, -\omega - 2\right\}  \end{dmath*}\end{minipage} }\\
$\#\A= $ 9 $ $ & $\A$ is minimal. \\
\multicolumn{2}{l}{\begin{minipage}{\textwidth}\begin{dmath*}\B =\A+\A \end{dmath*}\end{minipage} }\\[10pt]
\multicolumn{2}{l}{\begin{minipage}{\textwidth}$\A$ divided into congruence classes modulo $\beta$: \begin{dmath*} \left\{\left\{0\right\}, \left\{\omega + 1, 1\right\}, \left\{-\omega - 1, -1\right\}, \left\{2 \, \omega + 2, \omega + 2\right\}, \left\{-2 \, \omega - 2, -\omega - 2\right\}\right\}  \end{dmath*}\end{minipage} }\\[10pt]
\multicolumn{2}{l}{\begin{minipage}{\textwidth}$\A$ divided into congruence classes modulo $\beta-1$: \begin{dmath*} \left\{\left\{0\right\}, \left\{\omega + 1\right\}, \left\{-\omega - 1\right\}, \left\{1\right\}, \left\{-1\right\}, \left\{2 \, \omega + 2\right\}, \left\{-2 \, \omega - 2\right\}, \left\{\omega + 2\right\}, \left\{-\omega - 2\right\}\right\}  \end{dmath*}\end{minipage} }\\
 & \\ \hline
 & \\
\end{tabular}

\begin{tabular}{ll}
Phase 1 (methods $12, 16$): &
\checkmark, $\#\mathcal{Q} =33$ \\ 
Method  9: &\\
Failing $b,b,\dots,b$ inputs: & $\{-3\omega - 4, 2\omega + 2, 2\omega + 3, \omega + 3, -2\omega - 4, -2\omega - 3, -2\omega - 2, -\omega - 3\}$ \\
Method  15: &\\
$b,b,\dots,b$ inputs: & \checkmark \\
\multicolumn{2}{l}{\begin{minipage}{\textwidth} Phase 2 fails because  the sequence $(0, 1, 2\omega + 1, -4\omega - 4, 4\omega + 4, 0, 1, 4\omega + 4, 0, 1, 4\omega + 4, \dots ,4\omega + 4, 0, 1, 4\omega + 4, \dots)$ leads to an infinite loop.\end{minipage} }\\
Method  22: &\\
Failing $b,b,\dots,b$ inputs: & $\{-3\omega - 3, 3\omega + 3\}$ \\
Method  23: &\\
Failing $b,b,\dots,b$ inputs: & $\{-3\omega - 4, 2\omega + 3, 2\omega + 4, \omega + 3, -2\omega - 4, -2\omega - 3, -\omega - 3, 3\omega + 4\}$ \\
\hline
Phase 1 (methods $13, 15$): &
\checkmark, $\#\mathcal{Q} =39$ \\ 
Method  9: &\\
Failing $b,b,\dots,b$ inputs: & $\{2\omega + 2, -2\omega - 4, -2\omega - 3, -2\omega - 2, -\omega - 3\}$ \\
Method  15: &\\
$b,b,\dots,b$ inputs: & \checkmark \\
\multicolumn{2}{l}{\begin{minipage}{\textwidth} Phase 2 fails because  the sequence $(0, 0, -4\omega - 4, 3\omega + 4, 0, 4\omega + 4, 0, 4\omega + 4, 0, 4\omega + 4, \dots ,0, 4\omega + 4, 0, 4\omega + 4, \dots)$ leads to an infinite loop.\end{minipage} }\\
Method  22: &\\
$b,b,\dots,b$ inputs: & \checkmark \\
\multicolumn{2}{l}{\begin{minipage}{\textwidth} Phase 2 fails because  the sequence $(0, 0, -2\omega - 1, 3\omega + 3, -4\omega - 4, \omega, -2\omega - 3, 4\omega + 4, -\omega, 2\omega + 3, -3\omega - 3, 4\omega + 4, 2\omega + 2, 4\omega + 4, 2\omega + 2, 4\omega + 4, 2\omega + 2, \dots ,4\omega + 4, 2\omega + 2, 4\omega + 4, 2\omega + 2, \dots)$ leads to an infinite loop.\end{minipage} }\\
Method  23: &\\
Failing $b,b,\dots,b$ inputs: & $\{-3\omega - 4, 2\omega + 3, 2\omega + 4, \omega + 3, -2\omega - 4, -2\omega - 3, -\omega - 3, 3\omega + 4\}$ \\
\hline
Phase 1 (methods $14$): &
\checkmark, $\#\mathcal{Q} =43$ \\ 
Method  9: &\\
Failing $b,b,\dots,b$ inputs: & $\{-3\omega - 3, 2\omega + 2, -2\omega - 3, -\omega - 3, 3\omega + 3\}$ \\
Method  15: &\\
$b,b,\dots,b$ inputs: & \checkmark \\
\multicolumn{2}{l}{\begin{minipage}{\textwidth} Phase 2 fails because  the sequence $(0, -1, -3\omega - 3, \omega, -3\omega - 4, -3\omega - 4, -3\omega - 3, 4\omega + 4, -3\omega - 4, 2\omega + 1, 0, -1, -3\omega - 3, \omega, \dots ,0, -1, -3\omega - 3, \omega, \dots)$ leads to an infinite loop.\end{minipage} }\\
Method  22: &\\
$b,b,\dots,b$ inputs: & \checkmark \\
\multicolumn{2}{l}{\begin{minipage}{\textwidth} Phase 2 fails because  the sequence $(0, 0, -4\omega - 4, 2\omega + 1, -\omega - 2, -3\omega - 4, 2\omega + 2, -\omega - 1, -\omega, 3\omega + 3, -2\omega - 4, 2\omega + 1, -\omega - 2, -3\omega - 4, 2\omega + 2, \dots ,2\omega + 1, -\omega - 2, -3\omega - 4, 2\omega + 2, \dots)$ leads to an infinite loop.\end{minipage} }\\
Method  23: &\\
Failing $b,b,\dots,b$ inputs: & $\{2\omega + 3, \omega + 3, -2\omega - 3, -\omega - 3\}$ \\
\hline
\end{tabular}

\end{exmp}




\begin{exmp}
\label{ex:compareAP}


\rule{0cm}{0cm}

\begin{tabular}{ll}
$\omega=  i $  & $\beta= \omega - 2 = i - 2 $\\
$m_\omega(t)=  t^{2} + 1 $  & $m_\beta(x)=  x^{2} + 4 \, x + 5 $\\
Real conjugate of $\beta$ greater than 1:   &  no \\ \hline
\multicolumn{2}{l}{\begin{minipage}{\textwidth}\begin{dmath*}\A = \left\{\omega + 2, \omega + 1, \omega, \omega - 1, 1, 0, -1, -\omega + 1, -\omega, -\omega - 1\right\}  \end{dmath*}\end{minipage} }\\
$\#\A= $ 10 $ $ & $\A$ is minimal. \\
\multicolumn{2}{l}{\begin{minipage}{\textwidth}\begin{dmath*}\B =\A+\A \end{dmath*}\end{minipage} }\\[10pt]
\multicolumn{2}{l}{\begin{minipage}{\textwidth}$\A$ divided into congruence classes modulo $\beta$: \begin{dmath*} \left\{\left\{\omega + 2, -1, -\omega + 1\right\}, \left\{\omega + 1, -\omega\right\}, \left\{\omega, -\omega - 1\right\}, \left\{\omega - 1, 1\right\}, \left\{0\right\}\right\}  \end{dmath*}\end{minipage} }\\[10pt]
\multicolumn{2}{l}{\begin{minipage}{\textwidth}$\A$ divided into congruence classes modulo $\beta-1$: \begin{dmath*} \left\{\left\{\omega + 2\right\}, \left\{\omega + 1\right\}, \left\{\omega\right\}, \left\{\omega - 1\right\}, \left\{1\right\}, \left\{0\right\}, \left\{-1\right\}, \left\{-\omega + 1\right\}, \left\{-\omega\right\}, \left\{-\omega - 1\right\}\right\}  \end{dmath*}\end{minipage} }\\
 & \\ \hline
 & \\
\end{tabular}

\begin{tabular}{ll}
Phase 1 (methods $14$): &
\checkmark, $\#\mathcal{Q} =19$ \\ 
Method  9: &\\
Failing $b,b,\dots,b$ inputs: & $\{2, -\omega - 2, -2\omega\}$ \\
Method  15: &\\
Failing $b,b,\dots,b$ inputs: & $\{-\omega - 2\}$ \\
Method  22: &\\
Failing $b,b,\dots,b$ inputs: & $\{2\omega - 2, 2\omega + 2, \omega + 3, -\omega - 2, -2\omega + 2\}$ \\
Method  23: &\\
Failing $b,b,\dots,b$ inputs: & $\{2\omega - 2, 2\omega + 2, \omega + 3, -\omega - 2, -2\omega + 2\}$ \\
\hline
Phase 1 (methods $12, 13, 15, 16$): &
\checkmark, $\#\mathcal{Q} =17$ \\ 
Method  9: &\\
Failing $b,b,\dots,b$ inputs: & $\{2, -\omega - 2, -2\omega\}$ \\
Method  15: &\\
Failing $b,b,\dots,b$ inputs: & $\{-\omega - 2\}$ \\
Method  22: &\\
Failing $b,b,\dots,b$ inputs: & $\{2\omega + 2, -\omega - 2\}$ \\
Method  23: &\\
Failing $b,b,\dots,b$ inputs: & $\{2\omega + 2, -\omega - 2\}$ \\
\hline
\end{tabular}

\end{exmp}




\begin{exmp}
\label{ex:compareAQ}


\rule{0cm}{0cm}

\begin{tabular}{ll}
$\omega=  i $  & $\beta= \omega - 2 = i - 2 $\\
$m_\omega(t)=  t^{2} + 1 $  & $m_\beta(x)=  x^{2} + 4 \, x + 5 $\\
Real conjugate of $\beta$ greater than 1:   &  no \\ \hline
\multicolumn{2}{l}{\begin{minipage}{\textwidth}\begin{dmath*}\A = \left\{\omega + 2, \omega + 1, \omega, 2, 1, 0, -1, -\omega + 1, -\omega, -\omega - 1\right\}  \end{dmath*}\end{minipage} }\\
$\#\A= $ 10 $ $ & $\A$ is minimal. \\
\multicolumn{2}{l}{\begin{minipage}{\textwidth}\begin{dmath*}\B =\A+\A \end{dmath*}\end{minipage} }\\[10pt]
\multicolumn{2}{l}{\begin{minipage}{\textwidth}$\A$ divided into congruence classes modulo $\beta$: \begin{dmath*} \left\{\left\{\omega + 2, -1, -\omega + 1\right\}, \left\{\omega + 1, -\omega\right\}, \left\{\omega, 2, -\omega - 1\right\}, \left\{1\right\}, \left\{0\right\}\right\}  \end{dmath*}\end{minipage} }\\[10pt]
\multicolumn{2}{l}{\begin{minipage}{\textwidth}$\A$ divided into congruence classes modulo $\beta-1$: \begin{dmath*} \left\{\left\{\omega + 2\right\}, \left\{\omega + 1\right\}, \left\{\omega\right\}, \left\{2\right\}, \left\{1\right\}, \left\{0\right\}, \left\{-1\right\}, \left\{-\omega + 1\right\}, \left\{-\omega\right\}, \left\{-\omega - 1\right\}\right\}  \end{dmath*}\end{minipage} }\\
 & \\ \hline
 & \\
\end{tabular}

\begin{tabular}{ll}
Phase 1 (methods $12, 13, 14, 15, 16$): &
\checkmark, $\#\mathcal{Q} =17$ \\ 
Method  9: &\\
$b,b,\dots,b$ inputs: & \checkmark \\
Phase 2: & \checkmark , $r= 3$ \\
Method  15: &\\
$b,b,\dots,b$ inputs: & \checkmark \\
Phase 2: & \checkmark , $r= 3$ \\
Method  22: &\\
$b,b,\dots,b$ inputs: & \checkmark \\
Phase 2: & \checkmark , $r= 3$ \\
Method  23: &\\
$b,b,\dots,b$ inputs: & \checkmark \\
Phase 2: & \checkmark , $r= 3$ \\
\hline
\end{tabular}

\end{exmp}






\newpage
\subsection*{Quadratic bases with integer alphabet}
The following examples show alphabets  divided into congruence classes modulo $\beta$ and $\beta-1$ for some numeration systems in Table~\ref{tab:resultsQuadrInt}.
\begin{exmp}
\label{ex:integerAA}


\rule{0cm}{0cm}

\begin{tabular}{ll}
$\omega=  \frac{1}{2} i \, \sqrt{11} + \frac{1}{2} $  & $\beta= -2 \, \omega + 1 = -i \, \sqrt{11} $\\
$m_\omega(t)=  t^{2} - t + 3 $  & $m_\beta(x)=  x^{2} + 11 $\\
Real conjugate of $\beta$ greater than 1:   &  no \\
$\#\A= $ 13 $ $ & $\A$ is not minimal. \\
\multicolumn{2}{l}{\begin{minipage}{\textwidth}\begin{dmath*}\A = \left\{-6, -5, -4, -3, -2, -1, 0, 1, 2, 3, 4, 5, 6\right\}  \end{dmath*}\end{minipage} }\\
\multicolumn{2}{l}{\begin{minipage}{\textwidth}$\A$ divided into congruence classes modulo $\beta$: \begin{dmath*} \left\{\left\{-6, 5\right\}, \left\{-5, 6\right\}, \left\{-4\right\}, \left\{-3\right\}, \left\{-2\right\}, \left\{-1\right\}, \left\{0\right\}, \left\{1\right\}, \left\{2\right\}, \left\{3\right\}, \left\{4\right\}\right\}  \end{dmath*}\end{minipage} }\\[10pt]
\multicolumn{2}{l}{\begin{minipage}{\textwidth}$\A$ divided into congruence classes modulo $\beta-1$: \begin{dmath*} \left\{\left\{-6, 0, 6\right\}, \left\{-5, 1\right\}, \left\{-4, 2\right\}, \left\{-3, 3\right\}, \left\{-2, 4\right\}, \left\{-1, 5\right\}\right\}  \end{dmath*}\end{minipage} }\\
 & \\ \hline
 & \\
Phase 1 (method  9): &
\checkmark, $\#\mathcal{Q} = $ 9 $ $ \\ 
$b,b,\dots,b$ inputs (method  15): & \checkmark \\
Phase 2 (method  15): & \checkmark , $r= 2$ \\
\end{tabular}

\end{exmp}




\begin{exmp}
\label{ex:integerAB}


\rule{0cm}{0cm}

\begin{tabular}{ll}
$\omega=  \frac{1}{2} i \, \sqrt{11} + \frac{1}{2} $  & $\beta= -2 \, \omega + 1 = -i \, \sqrt{11} $\\
$m_\omega(t)=  t^{2} - t + 3 $  & $m_\beta(x)=  x^{2} + 11 $\\
Real conjugate of $\beta$ greater than 1:   &  no \\
$\#\A= $ 12 $ $ & $\A$ is minimal. \\
\multicolumn{2}{l}{\begin{minipage}{\textwidth}\begin{dmath*}\A = \left\{-5, -4, -3, -2, -1, 0, 1, 2, 3, 4, 5, 6\right\}  \end{dmath*}\end{minipage} }\\
\multicolumn{2}{l}{\begin{minipage}{\textwidth}$\A$ divided into congruence classes modulo $\beta$: \begin{dmath*} \left\{\left\{-5, 6\right\}, \left\{-4\right\}, \left\{-3\right\}, \left\{-2\right\}, \left\{-1\right\}, \left\{0\right\}, \left\{1\right\}, \left\{2\right\}, \left\{3\right\}, \left\{4\right\}, \left\{5\right\}\right\}  \end{dmath*}\end{minipage} }\\[10pt]
\multicolumn{2}{l}{\begin{minipage}{\textwidth}$\A$ divided into congruence classes modulo $\beta-1$: \begin{dmath*} \left\{\left\{-5, 1\right\}, \left\{-4, 2\right\}, \left\{-3, 3\right\}, \left\{-2, 4\right\}, \left\{-1, 5\right\}, \left\{0, 6\right\}\right\}  \end{dmath*}\end{minipage} }\\
 & \\ \hline
 & \\
Phase 1 (method  9): &
\checkmark, $\#\mathcal{Q} = $ 9 $ $ \\ 
$b,b,\dots,b$ inputs (method  15): & \checkmark \\
Phase 2 (method  15): & \checkmark , $r= 4$ \\
\end{tabular}

\end{exmp}




\begin{exmp}
\label{ex:integerAC}


\rule{0cm}{0cm}

\begin{tabular}{ll}
$\omega=  \frac{1}{2} i \, \sqrt{7} - \frac{1}{2} $  & $\beta= -2 \, \omega - 1 = -i \, \sqrt{7} $\\
$m_\omega(t)=  t^{2} + t + 2 $  & $m_\beta(x)=  x^{2} + 7 $\\
Real conjugate of $\beta$ greater than 1:   &  no \\
$\#\A= $ 9 $ $ & $\A$ is not minimal. \\
\multicolumn{2}{l}{\begin{minipage}{\textwidth}\begin{dmath*}\A = \left\{-4, -3, -2, -1, 0, 1, 2, 3, 4\right\}  \end{dmath*}\end{minipage} }\\
\multicolumn{2}{l}{\begin{minipage}{\textwidth}$\A$ divided into congruence classes modulo $\beta$: \begin{dmath*} \left\{\left\{-4, 3\right\}, \left\{-3, 4\right\}, \left\{-2\right\}, \left\{-1\right\}, \left\{0\right\}, \left\{1\right\}, \left\{2\right\}\right\}  \end{dmath*}\end{minipage} }\\[10pt]
\multicolumn{2}{l}{\begin{minipage}{\textwidth}$\A$ divided into congruence classes modulo $\beta-1$: \begin{dmath*} \left\{\left\{-4, 0, 4\right\}, \left\{-3, 1\right\}, \left\{-2, 2\right\}, \left\{-1, 3\right\}\right\}  \end{dmath*}\end{minipage} }\\
 & \\ \hline
 & \\
Phase 1 (method  9): &
\checkmark, $\#\mathcal{Q} = $ 9 $ $ \\ 
$b,b,\dots,b$ inputs (method  15): & \checkmark \\
Phase 2 (method  15): & \checkmark , $r= 2$ \\
\end{tabular}

\end{exmp}




\begin{exmp}
\label{ex:integerAD}


\rule{0cm}{0cm}

\begin{tabular}{ll}
$\omega=  \frac{1}{2} i \, \sqrt{7} - \frac{1}{2} $  & $\beta= -2 \, \omega - 1 = -i \, \sqrt{7} $\\
$m_\omega(t)=  t^{2} + t + 2 $  & $m_\beta(x)=  x^{2} + 7 $\\
Real conjugate of $\beta$ greater than 1:   &  no \\
$\#\A= $ 8 $ $ & $\A$ is minimal. \\
\multicolumn{2}{l}{\begin{minipage}{\textwidth}\begin{dmath*}\A = \left\{-3, -2, -1, 0, 1, 2, 3, 4\right\}  \end{dmath*}\end{minipage} }\\
\multicolumn{2}{l}{\begin{minipage}{\textwidth}$\A$ divided into congruence classes modulo $\beta$: \begin{dmath*} \left\{\left\{-3, 4\right\}, \left\{-2\right\}, \left\{-1\right\}, \left\{0\right\}, \left\{1\right\}, \left\{2\right\}, \left\{3\right\}\right\}  \end{dmath*}\end{minipage} }\\[10pt]
\multicolumn{2}{l}{\begin{minipage}{\textwidth}$\A$ divided into congruence classes modulo $\beta-1$: \begin{dmath*} \left\{\left\{-3, 1\right\}, \left\{-2, 2\right\}, \left\{-1, 3\right\}, \left\{0, 4\right\}\right\}  \end{dmath*}\end{minipage} }\\
 & \\ \hline
 & \\
Phase 1 (method  9): &
\checkmark, $\#\mathcal{Q} = $ 9 $ $ \\ 
$b,b,\dots,b$ inputs (method  15): & \checkmark \\
Phase 2 (method  15): & \checkmark , $r= 4$ \\
\end{tabular}

\end{exmp}




\begin{exmp}
\label{ex:integerAE}


\rule{0cm}{0cm}

\begin{tabular}{ll}
$\omega=  \frac{1}{2} i \, \sqrt{3} + \frac{1}{2} $  & $\beta= -3 \, \omega + 2 = -\frac{3}{2} i \, \sqrt{3} + \frac{1}{2} $\\
$m_\omega(t)=  t^{2} - t + 1 $  & $m_\beta(x)=  x^{2} - x + 7 $\\
Real conjugate of $\beta$ greater than 1:   &  no \\
$\#\A= $ 11 $ $ & $\A$ is not minimal. \\
\multicolumn{2}{l}{\begin{minipage}{\textwidth}\begin{dmath*}\A = \left\{-5, -4, -3, -2, -1, 0, 1, 2, 3, 4, 5\right\}  \end{dmath*}\end{minipage} }\\
\multicolumn{2}{l}{\begin{minipage}{\textwidth}$\A$ divided into congruence classes modulo $\beta$: \begin{dmath*} \left\{\left\{-5, 2\right\}, \left\{-4, 3\right\}, \left\{-3, 4\right\}, \left\{-2, 5\right\}, \left\{-1\right\}, \left\{0\right\}, \left\{1\right\}\right\}  \end{dmath*}\end{minipage} }\\[10pt]
\multicolumn{2}{l}{\begin{minipage}{\textwidth}$\A$ divided into congruence classes modulo $\beta-1$: \begin{dmath*} \left\{\left\{-5, 2\right\}, \left\{-4, 3\right\}, \left\{-3, 4\right\}, \left\{-2, 5\right\}, \left\{-1\right\}, \left\{0\right\}, \left\{1\right\}\right\}  \end{dmath*}\end{minipage} }\\
 & \\ \hline
 & \\
Phase 1 (method  12): &
\checkmark, $\#\mathcal{Q} = $ 9 $ $ \\ 
$b,b,\dots,b$ inputs (method  15): & \checkmark \\
Phase 2 (method  15): & \checkmark , $r= 2$ \\
\end{tabular}

\end{exmp}




\begin{exmp}
\label{ex:integerAF}


\rule{0cm}{0cm}

\begin{tabular}{ll}
$\omega=  i \, \sqrt{3} $  & $\beta= \omega = i \, \sqrt{3} $\\
$m_\omega(t)=  t^{2} + 3 $  & $m_\beta(x)=  x^{2} + 3 $\\
Real conjugate of $\beta$ greater than 1:   &  no \\
$\#\A= $ 4 $ $ & $\A$ is minimal. \\
\multicolumn{2}{l}{\begin{minipage}{\textwidth}\begin{dmath*}\A = \left\{-1, 0, 1, 2\right\}  \end{dmath*}\end{minipage} }\\
\multicolumn{2}{l}{\begin{minipage}{\textwidth}$\A$ divided into congruence classes modulo $\beta$: \begin{dmath*} \left\{\left\{-1, 2\right\}, \left\{0\right\}, \left\{1\right\}\right\}  \end{dmath*}\end{minipage} }\\[10pt]
\multicolumn{2}{l}{\begin{minipage}{\textwidth}$\A$ divided into congruence classes modulo $\beta-1$: \begin{dmath*} \left\{\left\{-1\right\}, \left\{0\right\}, \left\{1\right\}, \left\{2\right\}\right\}  \end{dmath*}\end{minipage} }\\
 & \\ \hline
 & \\
Phase 1 (method  12): &
\checkmark, $\#\mathcal{Q} = $ 9 $ $ \\ 
$b,b,\dots,b$ inputs (method  15): & \checkmark \\
Phase 2 (method  15): & \checkmark , $r= 4$ \\
\end{tabular}

\end{exmp}




\begin{exmp}
\label{ex:integerAG}


\rule{0cm}{0cm}

\begin{tabular}{ll}
$\omega=  i \, \sqrt{2} $  & $\beta= -\omega = -i \, \sqrt{2} $\\
$m_\omega(t)=  t^{2} + 2 $  & $m_\beta(x)=  x^{2} + 2 $\\
Real conjugate of $\beta$ greater than 1:   &  ? \\
$\#\A= $ 3 $ $ & $\A$ is minimal. \\
\multicolumn{2}{l}{\begin{minipage}{\textwidth}\begin{dmath*}\A = \left\{-1, 0, 1\right\}  \end{dmath*}\end{minipage} }\\
\multicolumn{2}{l}{\begin{minipage}{\textwidth}$\A$ divided into congruence classes modulo $\beta$: \begin{dmath*} \left\{\left\{-1, 1\right\}, \left\{0\right\}\right\}  \end{dmath*}\end{minipage} }\\[10pt]
\multicolumn{2}{l}{\begin{minipage}{\textwidth}$\A$ divided into congruence classes modulo $\beta-1$: \begin{dmath*} \left\{\left\{-1\right\}, \left\{0\right\}, \left\{1\right\}\right\}  \end{dmath*}\end{minipage} }\\
 & \\ \hline
 & \\
Phase 1 (method  8): &
\checkmark, $\#\mathcal{Q} = $ 9 $ $ \\ 
$b,b,\dots,b$ inputs (method  15): & \checkmark \\
Phase 2 (method  15): & \checkmark , $r= 4$ \\
\end{tabular}

\end{exmp}




\begin{exmp}
\label{ex:integerAH}


\rule{0cm}{0cm}

\begin{tabular}{ll}
$\omega=  \sqrt{2} $  & $\beta= -\omega = -\sqrt{2} $\\
$m_\omega(t)=  t^{2} - 2 $  & $m_\beta(x)=  x^{2} - 2 $\\
Real conjugate of $\beta$ greater than 1:   &  yes \\
$\#\A= $ 3 $ $ & $\A$ is minimal. \\
\multicolumn{2}{l}{\begin{minipage}{\textwidth}\begin{dmath*}\A = \left\{0, 1, -1\right\}  \end{dmath*}\end{minipage} }\\
\multicolumn{2}{l}{\begin{minipage}{\textwidth}$\A$ divided into congruence classes modulo $\beta$: \begin{dmath*} \left\{\left\{0\right\}, \left\{1, -1\right\}\right\}  \end{dmath*}\end{minipage} }\\[10pt]
\multicolumn{2}{l}{\begin{minipage}{\textwidth}$\A$ divided into congruence classes modulo $\beta-1$: \begin{dmath*} \left\{\left\{0, 1, -1\right\}\right\}  \end{dmath*}\end{minipage} }\\
 & \\ \hline
 & \\
Phase 1 (method  9): &
\checkmark, $\#\mathcal{Q} = $ 9 $ $ \\ 
$b,b,\dots,b$ inputs (method  21): & \checkmark \\
Phase 2 (method  21): & \checkmark , $r= 4$ \\
\end{tabular}

\end{exmp}




\begin{exmp}
\label{ex:integerAI}


\rule{0cm}{0cm}

\begin{tabular}{ll}
$\omega=  \sqrt{3} - 1 $  & $\beta= -\omega - 1 = -\sqrt{3} $\\
$m_\omega(t)=  t^{2} + 2 \, t - 2 $  & $m_\beta(x)=  x^{2} - 3 $\\
Real conjugate of $\beta$ greater than 1:   &  yes \\
$\#\A= $ 4 $ $ & $\A$ is minimal. \\
\multicolumn{2}{l}{\begin{minipage}{\textwidth}\begin{dmath*}\A = \left\{0, 1, -1, 2\right\}  \end{dmath*}\end{minipage} }\\
\multicolumn{2}{l}{\begin{minipage}{\textwidth}$\A$ divided into congruence classes modulo $\beta$: \begin{dmath*} \left\{\left\{0\right\}, \left\{1\right\}, \left\{-1, 2\right\}\right\}  \end{dmath*}\end{minipage} }\\[10pt]
\multicolumn{2}{l}{\begin{minipage}{\textwidth}$\A$ divided into congruence classes modulo $\beta-1$: \begin{dmath*} \left\{\left\{0, 2\right\}, \left\{1, -1\right\}\right\}  \end{dmath*}\end{minipage} }\\
 & \\ \hline
 & \\
Phase 1 (method  6): &
\checkmark, $\#\mathcal{Q} = $ 9 $ $ \\ 
$b,b,\dots,b$ inputs (method  23): & \checkmark \\
Phase 2 (method  23): & \checkmark , $r= 5$ \\
\end{tabular}

\end{exmp}




\begin{exmp}
\label{ex:integerAJ}


\rule{0cm}{0cm}

\begin{tabular}{ll}
$\omega=  \frac{1}{2} \, \sqrt{5} + \frac{1}{2} $  & $\beta= -2 \, \omega + 1 = -\sqrt{5} $\\
$m_\omega(t)=  t^{2} - t - 1 $  & $m_\beta(x)=  x^{2} - 5 $\\
Real conjugate of $\beta$ greater than 1:   &  yes \\
$\#\A= $ 7 $ $ & $\A$ is not minimal. \\
\multicolumn{2}{l}{\begin{minipage}{\textwidth}\begin{dmath*}\A = \left\{-3, -2, -1, 0, 1, 2, 3\right\}  \end{dmath*}\end{minipage} }\\
\multicolumn{2}{l}{\begin{minipage}{\textwidth}$\A$ divided into congruence classes modulo $\beta$: \begin{dmath*} \left\{\left\{-3, 2\right\}, \left\{-2, 3\right\}, \left\{-1\right\}, \left\{0\right\}, \left\{1\right\}\right\}  \end{dmath*}\end{minipage} }\\[10pt]
\multicolumn{2}{l}{\begin{minipage}{\textwidth}$\A$ divided into congruence classes modulo $\beta-1$: \begin{dmath*} \left\{\left\{-3, -1, 1, 3\right\}, \left\{-2, 0, 2\right\}\right\}  \end{dmath*}\end{minipage} }\\
 & \\ \hline
 & \\
Phase 1 (method  9): &
\checkmark, $\#\mathcal{Q} = $ 9 $ $ \\ 
$b,b,\dots,b$ inputs (method  15): & \checkmark \\
Phase 2 (method  15): & \checkmark , $r= 2$ \\
\end{tabular}

\end{exmp}




\begin{exmp}
\label{ex:integerAK}


\rule{0cm}{0cm}

\begin{tabular}{ll}
$\omega=  \frac{1}{2} \, \sqrt{5} + \frac{1}{2} $  & $\beta= -2 \, \omega + 1 = -\sqrt{5} $\\
$m_\omega(t)=  t^{2} - t - 1 $  & $m_\beta(x)=  x^{2} - 5 $\\
Real conjugate of $\beta$ greater than 1:   &  yes \\
$\#\A= $ 6 $ $ & $\A$ is not minimal. \\
\multicolumn{2}{l}{\begin{minipage}{\textwidth}\begin{dmath*}\A = \left\{-2, -1, 0, 1, 2, 3\right\}  \end{dmath*}\end{minipage} }\\
\multicolumn{2}{l}{\begin{minipage}{\textwidth}$\A$ divided into congruence classes modulo $\beta$: \begin{dmath*} \left\{\left\{-2, 3\right\}, \left\{-1\right\}, \left\{0\right\}, \left\{1\right\}, \left\{2\right\}\right\}  \end{dmath*}\end{minipage} }\\[10pt]
\multicolumn{2}{l}{\begin{minipage}{\textwidth}$\A$ divided into congruence classes modulo $\beta-1$: \begin{dmath*} \left\{\left\{-2, 0, 2\right\}, \left\{-1, 1, 3\right\}\right\}  \end{dmath*}\end{minipage} }\\
 & \\ \hline
 & \\
Phase 1 (method  9): &
\checkmark, $\#\mathcal{Q} = $ 9 $ $ \\ 
$b,b,\dots,b$ inputs (method  15): & \checkmark \\
Phase 2 (method  15): & \checkmark , $r= 4$ \\
\end{tabular}

\end{exmp}




\begin{exmp}
\label{ex:integerAL}


\rule{0cm}{0cm}

\begin{tabular}{ll}
$\omega=  -\sqrt{5} + 1 $  & $\beta= \omega - 1 = -\sqrt{5} $\\
$m_\omega(t)=  t^{2} - 2 \, t - 4 $  & $m_\beta(x)=  x^{2} - 5 $\\
Real conjugate of $\beta$ greater than 1:   &  yes \\
$\#\A= $ 6 $ $ & $\A$ is minimal. \\
\multicolumn{2}{l}{\begin{minipage}{\textwidth}\begin{dmath*}\A = \left\{-2, -1, 0, 1, 2, 3\right\}  \end{dmath*}\end{minipage} }\\
\multicolumn{2}{l}{\begin{minipage}{\textwidth}$\A$ divided into congruence classes modulo $\beta$: \begin{dmath*} \left\{\left\{-2, 3\right\}, \left\{-1\right\}, \left\{0\right\}, \left\{1\right\}, \left\{2\right\}\right\}  \end{dmath*}\end{minipage} }\\[10pt]
\multicolumn{2}{l}{\begin{minipage}{\textwidth}$\A$ divided into congruence classes modulo $\beta-1$: \begin{dmath*} \left\{\left\{-2, 2\right\}, \left\{-1, 3\right\}, \left\{0\right\}, \left\{1\right\}\right\}  \end{dmath*}\end{minipage} }\\
 & \\ \hline
 & \\
Phase 1 (method  13): &
\checkmark, $\#\mathcal{Q} = $ 9 $ $ \\ 
$b,b,\dots,b$ inputs (method  23): & \checkmark \\
Phase 2 (method  23): & \checkmark , $r= 5$ \\
\end{tabular}

\end{exmp}




\begin{exmp}
\label{ex:integerAM}


\rule{0cm}{0cm}

\begin{tabular}{ll}
$\omega=  -\sqrt{6} + 1 $  & $\beta= \omega - 1 = -\sqrt{6} $\\
$m_\omega(t)=  t^{2} - 2 \, t - 5 $  & $m_\beta(x)=  x^{2} - 6 $\\
Real conjugate of $\beta$ greater than 1:   &  yes \\
$\#\A= $ 7 $ $ & $\A$ is minimal. \\
\multicolumn{2}{l}{\begin{minipage}{\textwidth}\begin{dmath*}\A = \left\{-2, -1, 0, 1, 2, 3, 4\right\}  \end{dmath*}\end{minipage} }\\
\multicolumn{2}{l}{\begin{minipage}{\textwidth}$\A$ divided into congruence classes modulo $\beta$: \begin{dmath*} \left\{\left\{-2, 4\right\}, \left\{-1\right\}, \left\{0\right\}, \left\{1\right\}, \left\{2\right\}, \left\{3\right\}\right\}  \end{dmath*}\end{minipage} }\\[10pt]
\multicolumn{2}{l}{\begin{minipage}{\textwidth}$\A$ divided into congruence classes modulo $\beta-1$: \begin{dmath*} \left\{\left\{-2, 3\right\}, \left\{-1, 4\right\}, \left\{0\right\}, \left\{1\right\}, \left\{2\right\}\right\}  \end{dmath*}\end{minipage} }\\
 & \\ \hline
 & \\
Phase 1 (method  13): &
\checkmark, $\#\mathcal{Q} = $ 9 $ $ \\ 
$b,b,\dots,b$ inputs (method  23): & \checkmark \\
Phase 2 (method  23): & \checkmark , $r= 5$ \\
\end{tabular}

\end{exmp}




\begin{exmp}
\label{ex:integerAN}


\rule{0cm}{0cm}

\begin{tabular}{ll}
$\omega=  \sqrt{6} - 1 $  & $\beta= \omega + 1 = \sqrt{6} $\\
$m_\omega(t)=  t^{2} + 2 \, t - 5 $  & $m_\beta(x)=  x^{2} - 6 $\\
Real conjugate of $\beta$ greater than 1:   &  yes \\
$\#\A= $ 7 $ $ & $\A$ is minimal. \\
\multicolumn{2}{l}{\begin{minipage}{\textwidth}\begin{dmath*}\A = \left\{-2, -1, 0, 1, 2, 3, 4\right\}  \end{dmath*}\end{minipage} }\\
\multicolumn{2}{l}{\begin{minipage}{\textwidth}$\A$ divided into congruence classes modulo $\beta$: \begin{dmath*} \left\{\left\{-2, 4\right\}, \left\{-1\right\}, \left\{0\right\}, \left\{1\right\}, \left\{2\right\}, \left\{3\right\}\right\}  \end{dmath*}\end{minipage} }\\[10pt]
\multicolumn{2}{l}{\begin{minipage}{\textwidth}$\A$ divided into congruence classes modulo $\beta-1$: \begin{dmath*} \left\{\left\{-2, 3\right\}, \left\{-1, 4\right\}, \left\{0\right\}, \left\{1\right\}, \left\{2\right\}\right\}  \end{dmath*}\end{minipage} }\\
 & \\ \hline
 & \\
Phase 1 (method  13): &
\checkmark, $\#\mathcal{Q} = $ 9 $ $ \\ 
$b,b,\dots,b$ inputs (method  23): & \checkmark \\
Phase 2 (method  23): & \checkmark , $r= 4$ \\
\end{tabular}

\end{exmp}




\begin{exmp}
\label{ex:integerAO}


\rule{0cm}{0cm}

\begin{tabular}{ll}
$\omega=  -\sqrt{7} + 2 $  & $\beta= -\omega + 2 = \sqrt{7} $\\
$m_\omega(t)=  t^{2} - 4 \, t - 3 $  & $m_\beta(x)=  x^{2} - 7 $\\
Real conjugate of $\beta$ greater than 1:   &  yes \\
$\#\A= $ 8 $ $ & $\A$ is minimal. \\
\multicolumn{2}{l}{\begin{minipage}{\textwidth}\begin{dmath*}\A = \left\{-3, -2, -1, 0, 1, 2, 3, 4\right\}  \end{dmath*}\end{minipage} }\\
\multicolumn{2}{l}{\begin{minipage}{\textwidth}$\A$ divided into congruence classes modulo $\beta$: \begin{dmath*} \left\{\left\{-3, 4\right\}, \left\{-2\right\}, \left\{-1\right\}, \left\{0\right\}, \left\{1\right\}, \left\{2\right\}, \left\{3\right\}\right\}  \end{dmath*}\end{minipage} }\\[10pt]
\multicolumn{2}{l}{\begin{minipage}{\textwidth}$\A$ divided into congruence classes modulo $\beta-1$: \begin{dmath*} \left\{\left\{-3, 3\right\}, \left\{-2, 4\right\}, \left\{-1\right\}, \left\{0\right\}, \left\{1\right\}, \left\{2\right\}\right\}  \end{dmath*}\end{minipage} }\\
 & \\ \hline
 & \\
Phase 1 (method  13): &
\checkmark, $\#\mathcal{Q} = $ 9 $ $ \\ 
$b,b,\dots,b$ inputs (method  23): & \checkmark \\
Phase 2 (method  23): & \checkmark , $r= 4$ \\
\end{tabular}

\end{exmp}




\begin{exmp}
\label{ex:integerAP}


\rule{0cm}{0cm}

\begin{tabular}{ll}
$\omega=  \frac{1}{2} \, \sqrt{13} + \frac{1}{2} $  & $\beta= -2 \, \omega + 1 = -\sqrt{13} $\\
$m_\omega(t)=  t^{2} - t - 3 $  & $m_\beta(x)=  x^{2} - 13 $\\
Real conjugate of $\beta$ greater than 1:   &  yes \\
$\#\A= $ 15 $ $ & $\A$ is not minimal. \\
\multicolumn{2}{l}{\begin{minipage}{\textwidth}\begin{dmath*}\A = \left\{-7, -6, -5, -4, -3, -2, -1, 0, 1, 2, 3, 4, 5, 6, 7\right\}  \end{dmath*}\end{minipage} }\\
\multicolumn{2}{l}{\begin{minipage}{\textwidth}$\A$ divided into congruence classes modulo $\beta$: \begin{dmath*} \left\{\left\{-7, 6\right\}, \left\{-6, 7\right\}, \left\{-5\right\}, \left\{-4\right\}, \left\{-3\right\}, \left\{-2\right\}, \left\{-1\right\}, \left\{0\right\}, \left\{1\right\}, \left\{2\right\}, \left\{3\right\}, \left\{4\right\}, \left\{5\right\}\right\}  \end{dmath*}\end{minipage} }\\[10pt]
\multicolumn{2}{l}{\begin{minipage}{\textwidth}$\A$ divided into congruence classes modulo $\beta-1$: \begin{dmath*} \left\{\left\{-7, -1, 5\right\}, \left\{-6, 0, 6\right\}, \left\{-5, 1, 7\right\}, \left\{-4, 2\right\}, \left\{-3, 3\right\}, \left\{-2, 4\right\}\right\}  \end{dmath*}\end{minipage} }\\
 & \\ \hline
 & \\
Phase 1 (method  9): &
\checkmark, $\#\mathcal{Q} = $ 9 $ $ \\ 
$b,b,\dots,b$ inputs (method  15): & \checkmark \\
Phase 2 (method  15): & \checkmark , $r= 2$ \\
\end{tabular}

\end{exmp}




\begin{exmp}
\label{ex:integerAQ}


\rule{0cm}{0cm}

\begin{tabular}{ll}
$\omega=  \frac{1}{2} \, \sqrt{13} + \frac{1}{2} $  & $\beta= -2 \, \omega + 1 = -\sqrt{13} $\\
$m_\omega(t)=  t^{2} - t - 3 $  & $m_\beta(x)=  x^{2} - 13 $\\
Real conjugate of $\beta$ greater than 1:   &  yes \\
$\#\A= $ 14 $ $ & $\A$ is not minimal. \\
\multicolumn{2}{l}{\begin{minipage}{\textwidth}\begin{dmath*}\A = \left\{-6, -5, -4, -3, -2, -1, 0, 1, 2, 3, 4, 5, 6, 7\right\}  \end{dmath*}\end{minipage} }\\
\multicolumn{2}{l}{\begin{minipage}{\textwidth}$\A$ divided into congruence classes modulo $\beta$: \begin{dmath*} \left\{\left\{-6, 7\right\}, \left\{-5\right\}, \left\{-4\right\}, \left\{-3\right\}, \left\{-2\right\}, \left\{-1\right\}, \left\{0\right\}, \left\{1\right\}, \left\{2\right\}, \left\{3\right\}, \left\{4\right\}, \left\{5\right\}, \left\{6\right\}\right\}  \end{dmath*}\end{minipage} }\\[10pt]
\multicolumn{2}{l}{\begin{minipage}{\textwidth}$\A$ divided into congruence classes modulo $\beta-1$: \begin{dmath*} \left\{\left\{-6, 0, 6\right\}, \left\{-5, 1, 7\right\}, \left\{-4, 2\right\}, \left\{-3, 3\right\}, \left\{-2, 4\right\}, \left\{-1, 5\right\}\right\}  \end{dmath*}\end{minipage} }\\
 & \\ \hline
 & \\
Phase 1 (method  9): &
\checkmark, $\#\mathcal{Q} = $ 9 $ $ \\ 
$b,b,\dots,b$ inputs (method  15): & \checkmark \\
Phase 2 (method  15): & \checkmark , $r= 4$ \\
\end{tabular}

\end{exmp}




\begin{exmp}
\label{ex:integerAR}


\rule{0cm}{0cm}

\begin{tabular}{ll}
$\omega=  -\frac{1}{2} \, \sqrt{17} + \frac{3}{2} $  & $\beta= 2 \, \omega - 3 = -\sqrt{17} $\\
$m_\omega(t)=  t^{2} - 3 \, t - 2 $  & $m_\beta(x)=  x^{2} - 17 $\\
Real conjugate of $\beta$ greater than 1:   &  yes \\
$\#\A= $ 18 $ $ & $\A$ is minimal. \\
\multicolumn{2}{l}{\begin{minipage}{\textwidth}\begin{dmath*}\A = \left\{-8, -7, -6, -5, -4, -3, -2, -1, 0, 1, 2, 3, 4, 5, 6, 7, 8, 9\right\}  \end{dmath*}\end{minipage} }\\
\multicolumn{2}{l}{\begin{minipage}{\textwidth}$\A$ divided into congruence classes modulo $\beta$: \begin{dmath*} \left\{\left\{-8, 9\right\}, \left\{-7\right\}, \left\{-6\right\}, \left\{-5\right\}, \left\{-4\right\}, \left\{-3\right\}, \left\{-2\right\}, \left\{-1\right\}, \left\{0\right\}, \left\{1\right\}, \left\{2\right\}, \left\{3\right\}, \left\{4\right\}, \left\{5\right\}, \left\{6\right\}, \left\{7\right\}, \left\{8\right\}\right\}  \end{dmath*}\end{minipage} }\\[10pt]
\multicolumn{2}{l}{\begin{minipage}{\textwidth}$\A$ divided into congruence classes modulo $\beta-1$: \begin{dmath*} \left\{\left\{-8, 0, 8\right\}, \left\{-7, 1, 9\right\}, \left\{-6, 2\right\}, \left\{-5, 3\right\}, \left\{-4, 4\right\}, \left\{-3, 5\right\}, \left\{-2, 6\right\}, \left\{-1, 7\right\}\right\}  \end{dmath*}\end{minipage} }\\
 & \\ \hline
 & \\
Phase 1 (method  6): &
\checkmark, $\#\mathcal{Q} = $ 9 $ $ \\ 
$b,b,\dots,b$ inputs (method  22): & \checkmark \\
Phase 2 (method  22): & \checkmark , $r= 4$ \\
\end{tabular}

\end{exmp}




\begin{exmp}
\label{ex:integerAS}


\rule{0cm}{0cm}

\begin{tabular}{ll}
$\omega=  -\frac{1}{2} \, \sqrt{21} + \frac{3}{2} $  & $\beta= 2 \, \omega - 3 = -\sqrt{21} $\\
$m_\omega(t)=  t^{2} - 3 \, t - 3 $  & $m_\beta(x)=  x^{2} - 21 $\\
Real conjugate of $\beta$ greater than 1:   &  yes \\
$\#\A= $ 22 $ $ & $\A$ is minimal. \\
\multicolumn{2}{l}{\begin{minipage}{\textwidth}\begin{dmath*}\A = \left\{-10, -9, -8, -7, -6, -5, -4, -3, -2, -1, 0, 1, 2, 3, 4, 5, 6, 7, 8, 9, 10, 11\right\}  \end{dmath*}\end{minipage} }\\
\multicolumn{2}{l}{\begin{minipage}{\textwidth}$\A$ divided into congruence classes modulo $\beta$: \begin{dmath*} \left\{\left\{-10, 11\right\}, \left\{-9\right\}, \left\{-8\right\}, \left\{-7\right\}, \left\{-6\right\}, \left\{-5\right\}, \left\{-4\right\}, \left\{-3\right\}, \left\{-2\right\}, \left\{-1\right\}, \left\{0\right\}, \left\{1\right\}, \left\{2\right\}, \left\{3\right\}, \left\{4\right\}, \left\{5\right\}, \left\{6\right\}, \left\{7\right\}, \left\{8\right\}, \left\{9\right\}, \left\{10\right\}\right\}  \end{dmath*}\end{minipage} }\\[10pt]
\multicolumn{2}{l}{\begin{minipage}{\textwidth}$\A$ divided into congruence classes modulo $\beta-1$: \begin{dmath*} \left\{\left\{-10, 0, 10\right\}, \left\{-9, 1, 11\right\}, \left\{-8, 2\right\}, \left\{-7, 3\right\}, \left\{-6, 4\right\}, \left\{-5, 5\right\}, \left\{-4, 6\right\}, \left\{-3, 7\right\}, \left\{-2, 8\right\}, \left\{-1, 9\right\}\right\}  \end{dmath*}\end{minipage} }\\
 & \\ \hline
 & \\
Phase 1 (method  9): &
\checkmark, $\#\mathcal{Q} = $ 9 $ $ \\ 
$b,b,\dots,b$ inputs (method  9): & \checkmark \\
Phase 2 (method  9): & \checkmark , $r= 4$ \\
\end{tabular}

\end{exmp}







\subsection*{Killed examples}
The computation of a weight function for the following numeration systems was killed because of memory limits.
\begin{exmp}
\label{ex:killAB}

\rule{0cm}{0cm}

\begin{tabular}{ll}
$\omega=  -\frac{1}{2} \, \sqrt{37} + \frac{5}{2} $  & $\beta= -\omega - 3 = \frac{1}{2} \, \sqrt{37} - \frac{11}{2} $\\
$m_\omega(t)=  t^{2} - 5 \, t - 3 $  & $m_\beta(x)=  x^{2} + 11 \, x + 21 $\\
Real conjugate of $\beta$ greater than 1:   &  no \\
$\#\A= $ 33 $ $ & $\A$ is minimal. \\
\multicolumn{2}{l}{\begin{minipage}{\textwidth}\begin{dmath*}\A = \left\{0, 1, -1, \omega + 1, -\omega - 1, -\omega + 1, \omega - 1, \omega, -\omega, 2 \, \omega + 2, -2 \, \omega - 2, \omega + 2, -\omega - 2, -2 \, \omega + 2, 2 \, \omega - 2, 2 \, \omega + 1, -2 \, \omega - 1, -2 \, \omega + 1, 2 \, \omega - 1, 2 \, \omega, -2 \, \omega, -2 \, \omega + 3, 2 \, \omega - 3, -3 \, \omega + 3, 3 \, \omega - 3, -3 \, \omega + 2, 3 \, \omega - 2, -3 \, \omega + 1, 3 \, \omega - 1, 3 \, \omega, -3 \, \omega, -3 \, \omega + 4, 3 \, \omega - 4\right\}  \end{dmath*}\end{minipage} }\\
 & \\
Phase 1 (method  9): &
\checkmark, $\#\mathcal{Q} = $ 17 $ $ \\ 
$b,b,\dots,b$ inputs (method  2c): & \checkmark, maximal length of window: $ 3 $ \\
\multicolumn{2}{l}{\begin{minipage}{\textwidth} Computation of Phase 2 (method  2c) was killed when the length of window 5 was being proccessed. Numbers of saved combinations for each finished length are: (0, 12399, 682670, 2721482)\end{minipage} }\\
\end{tabular}

\end{exmp}



\begin{exmp}
\label{ex:killAA}

\rule{0cm}{0cm}

\begin{tabular}{ll}
$\omega=  -\frac{1}{2} \, \sqrt{29} + \frac{3}{2} $  & $\beta= 3 \, \omega + 1 = -\frac{3}{2} \, \sqrt{29} + \frac{11}{2} $\\
$m_\omega(t)=  t^{2} - 3 \, t - 5 $  & $m_\beta(x)=  x^{2} - 11 \, x - 35 $\\
Real conjugate of $\beta$ greater than 1:   &  yes \\
$\#\A= $ 49 $ $ & $\A$ is not minimal. \\
\multicolumn{2}{l}{\begin{minipage}{\textwidth}\begin{dmath*}\A = \left\{0, 1, -1, \omega + 1, -\omega - 1, -\omega + 1, \omega - 1, \omega, -\omega, 2 \, \omega + 2, -2 \, \omega - 2, \omega + 2, -\omega - 2, 2, -2, 3 \, \omega + 3, -3 \, \omega - 3, 2 \, \omega + 3, -2 \, \omega - 3, \omega + 3, -\omega - 3, 4 \, \omega + 4, -4 \, \omega - 4, 3 \, \omega + 4, -3 \, \omega - 4, 2 \, \omega + 4, -2 \, \omega - 4, 5 \, \omega + 5, -5 \, \omega - 5, 4 \, \omega + 5, -4 \, \omega - 5, 3 \, \omega + 5, -3 \, \omega - 5, 6 \, \omega + 6, -6 \, \omega - 6, 5 \, \omega + 6, -5 \, \omega - 6, 4 \, \omega + 6, -4 \, \omega - 6, 7 \, \omega + 7, -7 \, \omega - 7, 6 \, \omega + 7, -6 \, \omega - 7, 5 \, \omega + 7, -5 \, \omega - 7, -2 \, \omega + 5, 2 \, \omega - 5, 4 \, \omega + 7, -4 \, \omega - 7\right\}  \end{dmath*}\end{minipage} }\\
 & \\
Phase 1 (method  9): &
\checkmark, $\#\mathcal{Q} = $ 46 $ $ \\ 
$b,b,\dots,b$ inputs (method  21): & \checkmark, maximal length of window: $ 5 $ \\
\multicolumn{2}{l}{\begin{minipage}{\textwidth} Computation of Phase 2 (method  21) was killed. % when the length of window 1 was being proccessed. Numbers of saved combinations for each finished length are: ()
\end{minipage} }\\
\end{tabular}

\end{exmp}




\newpage
\begin{exmp}
\label{ex:killAC}


\rule{0cm}{0cm}

\begin{tabular}{ll}
$\omega=  {\left(\frac{1}{9} \, \sqrt{19} \sqrt{3} + 1\right)}^{\frac{1}{3}} + \frac{2}{3 \, {\left(\frac{1}{9} \, \sqrt{19} \sqrt{3} + 1\right)}^{\frac{1}{3}}} $  & $\beta= -2 \, \omega^{2} + \omega + 2 = {\left(\sqrt{57} - \frac{197}{27}\right)}^{\frac{1}{3}} - \frac{14}{9 \, {\left(\sqrt{57} - \frac{197}{27}\right)}^{\frac{1}{3}}} - \frac{2}{3} $\\
$m_\omega(t)=  t^{3} - 2 \, t - 2 $  & $m_\beta(x)=  x^{3} + 2 \, x^{2} + 6 \, x + 18 $\\
Real conjugate of $\beta$ greater than 1:   &  no \\
$\#\A= $ 31 $ $ & $\A$ is not minimal. \\
\multicolumn{2}{l}{\begin{minipage}{\textwidth}\begin{dmath*}\A = \left\{0, 1, -1, \omega^{2} + \omega + 1, -\omega^{2} - \omega - 1, \omega + 1, -\omega - 1, -\omega^{2} + \omega + 1, \omega^{2} - \omega - 1, \omega^{2} + 1, -\omega^{2} - 1, -\omega^{2} + 1, \omega^{2} - 1, \omega^{2} - \omega + 1, -\omega^{2} + \omega - 1, -\omega + 1, \omega - 1, -\omega^{2} - \omega + 1, \omega^{2} + \omega - 1, \omega^{2} + \omega, -\omega^{2} - \omega, \omega, -\omega, \omega^{2} + 2, -\omega^{2} - 2, 2, -2, -\omega^{2} + \omega, \omega^{2} - \omega, \omega^{2}, -\omega^{2}\right\}  \end{dmath*}\end{minipage} }\\
 & \\
Phase 1 (method  9): &
\checkmark, $\#\mathcal{Q} = $ 83 $ $ \\ 
$b,b,\dots,b$ inputs (method  21): & \checkmark, maximal length of window: $ 5 $ \\
\multicolumn{2}{l}{\begin{minipage}{\textwidth} Computation of Phase 2 (method  21) was killed when the length of window 4 was being proccessed. Numbers of saved combinations for each finished length are: (0, 71, 1887261)\end{minipage} }\\
\end{tabular}

\end{exmp}


\begin{exmp}
\label{ex:killAD}


\rule{0cm}{0cm}

\begin{tabular}{ll}
$\omega=  {\left(\frac{1}{9} \, \sqrt{29} \sqrt{3} + \frac{28}{27}\right)}^{\frac{1}{3}} + \frac{1}{9 \, {\left(\frac{1}{9} \, \sqrt{29} \sqrt{3} + \frac{28}{27}\right)}^{\frac{1}{3}}} + \frac{1}{3} $ \\
\multicolumn{2}{l}{ $\beta= -\omega^{2} + \omega - 1 = {\left(\frac{2}{9} \, \sqrt{29} \sqrt{3} - 2\right)}^{\frac{1}{3}} - \frac{2}{3 \, {\left(\frac{2}{9} \, \sqrt{29} \sqrt{3} - 2\right)}^{\frac{1}{3}}} - 1 $}\\
$m_\omega(t)=  t^{3} - t^{2} - 2 $  & $m_\beta(x)=  x^{3} + 3 \, x^{2} + 5 \, x + 7 $\\
Real conjugate of $\beta$ greater than 1:   &  no \\
$\#\A= $ 16 $ $ & $\A$ is minimal. \\
\multicolumn{2}{l}{\begin{minipage}{\textwidth}\begin{dmath*}\A = \left\{0, 1, -1, \omega^{2} + \omega + 1, -\omega^{2} - \omega - 1, \omega + 1, \omega^{2} + 1, -\omega^{2} - 1, -\omega^{2} + 1, \omega^{2} + \omega, \omega, -\omega, -\omega^{2} + \omega, \omega^{2} - \omega, \omega^{2}, -\omega^{2}\right\}  \end{dmath*}\end{minipage} }\\
 & \\
Phase 1 (method  9): &
\checkmark, $\#\mathcal{Q} = $ 99 $ $ \\ 
$b,b,\dots,b$ inputs (method  21): & \checkmark, maximal length of window: $ 6 $ \\
\multicolumn{2}{l}{\begin{minipage}{\textwidth} Computation of Phase 2 (method  21) was killed when the length of window 4 was being proccessed. Numbers of saved combinations for each finished length are: (73, 5329, 315494)\end{minipage} }\\
\end{tabular}

\end{exmp}




\end{document}