\section{Illustration of Phase 1}
\label{app:phase1}   
Figures \ref{img:phase1img3} -- \ref{img:phase1img12} illustrates first and last iterations of the construction of the weight coefficients set $\Q$ for the Eisenstein base $\beta = -\frac{3}{2} + \frac{\imath \sqrt{3}}{2}$ with the complex alphabet $\mathcal{A} =\{0, 1, -1, \omega, -\omega, -\omega - 1, \omega + 1\}$ and input alphabet $\B=\A+\A$. The second last iteration is skipped.

\figurehascaptionOne{1 = The starting set $\Q_0{=}\{0\}$.}
\figurehascaptionOne{2 = The set $\B+\Q_0$ need to be covered.}
\figurehascaptionOne{3 = The set $\Q_0$ does not cover the set $\B+\Q_0${,} i.e.{,} the set $\A+\beta \cdot \Q_0$ is not superset of $\B+\Q_0$.}
\figurehascaptionOne{4 = The set $\Q_0$ is extended to $\Q_1$ to cover all elements of $\B+\Q_0$.}
\figurehascaptionOne{5 = The set $\B+\Q_1$ need to be covered.}
\figurehascaptionOne{6 = The set $\Q_1$ does not cover the set $\B+\Q_1${,} i.e.{,} the set $\A+\beta \cdot \Q_1$ is not superset of $\B+\Q_1$.}
\figurehascaptionOne{7 = The set $\Q_1$ is extended to $\Q_2$ to cover all elements of $\B+\Q_1$.}
\figurehascaptionOne{8 = The set $\B+\Q_2$ need to be covered.}
\figurehascaptionOne{9 = The set $\Q_2$ does not cover the set $\B+\Q_2${,} i.e.{,} the set $\A+\beta \cdot \Q_2$ is not superset of $\B+\Q_2$.}
\figurehascaptionOne{10 = The set $\Q_2$ is extended to $\Q_3$ to cover all elements of $\B+\Q_2$.}
\figurehascaptionOne{11 = The set $\B+\Q_3$ need to be covered.}
\figurehascaptionOne{12 = In the last iteration{,} the set $\Q_3$ covers the set $\B+\Q_3${,} i.e.{,} the set $\A+\beta \cdot \Q_3$ is superset of $\B+\Q_2$. The weight coefficients set $\Q$ equals $\Q_3$.}
\figurehascaptionOne{13 = The final weight coefficients $\Q{=}\Q_3$.}


\foreach \n in {3,4,6,7,12} {%
\begin{SCfigure}[][htbp]
    \centering
    \caption{\getcaptionOne{\n}}
    \label{img:phase1img\n}
    \includegraphics[height=0.27\textheight]{img/eisenstein/phase1_image_\n.png}
\end{SCfigure}
    }

\newpage


\section{Illustration of Phase 2}
The construction of set $\Q_{[\omega,1,2]}$ for the Eisenstein base $\beta = -\frac{3}{2} + \frac{\imath \sqrt{3}}{2}$ with the complex alphabet $\mathcal{A} =\{0, 1, -1, \omega, -\omega, -\omega - 1, \omega + 1\}$  and input alphabet $\B=\A+\A$ is illustrated on Figures \ref{img:phase2img3} -- \ref{img:phase2img7}.
\label{app:phase2}    

\figurehascaptionTwo{1 = Phase 2 starts with the weight coefficients set $\Q$ from Phase 1.}
\figurehascaptionTwo{2 = The set $\omega+\Q$ need to be covered.}
\figurehascaptionTwo{3 = The elements of $\omega+\Q$ are covered by the set $\Q_{[\omega]}\subset\Q$.}
\figurehascaptionTwo{4 = The set $\omega+\Q_{[1]}$ need to be covered.}
\figurehascaptionTwo{5 = The elements of $\omega+\Q_{[1]}$ are covered by the set $\Q_{[\omega,1]}\subset\Q_{[\omega]}$.}
\figurehascaptionTwo{6 = The set $\omega+\Q_{[1,2]}$ need to be covered.}
\figurehascaptionTwo{7 = The elements of $\omega+\Q_{[1]}$ are covered by the set $\Q_{[\omega,1,2]}\subset\Q_{[\omega,1]}$ which has only one element{.} This element is the output of the weight function $q{(\omega,1,2)}$.}



\foreach \n in {3,5,7} {%
\begin{SCfigure}[][htbp]
    \centering
    \caption{\getcaptionTwo{\n}}
    \label{img:phase2img\n}
    \includegraphics[height=0.23\textheight]{img/eisenstein/phase2_image_\n.png}
\end{SCfigure}
    }

\newpage

\section{Interfaces}
\label{app:interfaces}
File \verb+ewm_inputs.sage+:
\lstinputlisting[language=Python]{ewm_inputs.sage}

File \verb+ewm_gspreadsheet.sage+:
\lstinputlisting[language=Python]{ewm_gspreadsheet.sage}




\section{Tested examples}
\label{app:examples}

\subsection*{Unsuccessful examples comparing different methods}
The reasons of failure of Phase~2 for numeration systems in Section~\ref{sec:compareMethods} can be found here. See Tables \ref{tab:resultsPhaseOne}, \ref{tab:resultsPhaseTwo} and \ref{tab:alphabets} for parameters of the numeration systems.
\begin{exmp}
\label{ex:compareAB}

\textbf{Eisenstein\_1--block\_integer}

Phase 1 (methods 1b, 1c, 1d, 1e):
\begin{enumerate}[ ]
\item  2a -- Check of $b,b,\dots,b$ inputs fails for $b\in \{2, 3, 5, 6, -5, -4, -3\}$.
\item  2b -- Check of $b,b,\dots,b$ inputs fails for $b\in \{2, 3, 6, -6, -4, -3\}$.
\item  2c -- Check of $b,b,\dots,b$ inputs fails for $b\in \{0, 1, 3, 4, 6, -6, -4, -3, -1\}$.
\item  2d -- Check of $b,b,\dots,b$ inputs fails for $b\in \{3, 4, 6, -6, -4, -3\}$.
\item  2e -- Check of $b,b,\dots,b$ inputs fails for $b\in \{2, 3, 6, -2, -6, -4, -3\}$.
\end{enumerate}


Phase 1 (methods 1a):
\begin{enumerate}[ ]
\item  2a -- Check of $b,b,\dots,b$ inputs fails for $b\in \{0, 2, 4, 5, -2, -5, -4\}$.
\item  2b -- Check of $b,b,\dots,b$ inputs fails for $b\in \{0, 2, 4, 5, -2, -5, -4\}$.
\item  2c -- Check of $b,b,\dots,b$ inputs fails for $b\in \{0, 2, 3, 5, 6, -2, -6, -5, -3\}$.
\item  2d -- Check of $b,b,\dots,b$ inputs fails for $b\in \{0, 3, 4, 5, 6, -6, -5, -4, -3\}$.
\item  2e -- Check of $b,b,\dots,b$ inputs fails for $b\in \{0, 2, 4, 5, -2, -5, -4\}$.
\end{enumerate}


\end{exmp}




\begin{exmp}
\label{ex:compareAC}

\textbf{Eisenstein\_2--block\_complex}

Phase 1 (methods 1a, 1b, 1c, 1d, 1e):
\begin{enumerate}[ ]
\item  2a -- Check of $b,b,\dots,b$ inputs fails for $b\in \{2\omega - 1, \omega + 1, -2\omega, -4, -\omega - 2\}$.
\item  2b -- Check of $b,b,\dots,b$ inputs fails for $b\in \{2\omega - 1, \omega + 1, -2\omega, -4, -\omega - 2\}$.
\item  2c -- Check of $b,b,\dots,b$ inputs fails for $b\in \{2\omega - 1, \omega + 1, -2\omega, -4, -\omega - 2\}$.
\item  2d -- Check of $b,b,\dots,b$ inputs fails for $b\in \{2\omega - 1, \omega + 1, -2\omega, -4, -\omega - 2\}$.
\item  2e -- Check of $b,b,\dots,b$ inputs fails for $b\in \{2\omega - 1, \omega + 1, -2\omega, -4, -\omega - 2\}$.
\end{enumerate}


\end{exmp}




\begin{exmp}
\label{ex:compareAD}

\textbf{Eisenstein\_2--block\_integer}

Phase 1 (methods 1a, 1b, 1c, 1d, 1e):
\begin{enumerate}[ ]
\item  2a -- Check of $b,b,\dots,b$ inputs fails for $b\in \{0, 1, 2, 2\omega - 4, \omega - 2, 4\omega, 3\omega - 5, \omega - 1, -\omega + 3, -2\omega + 5, 2\omega - 3\}$.
\item  2b -- Check of $b,b,\dots,b$ inputs fails for $b\in \{0, 1, 2, 2\omega - 4, 2\omega - 2, \omega - 2, \omega, 4\omega, 3\omega - 5, \omega - 1, -\omega + 3, -2\omega + 5, 2\omega - 3\}$.
\item  2c -- Check of $b,b,\dots,b$ inputs fails for $b\in \{0, 1, 2, 2\omega - 4, \omega - 2, \omega, 4\omega, 3\omega - 5, \omega - 1, -\omega + 3, -2\omega + 5, 2\omega - 3\}$.
\item  2d -- Check of $b,b,\dots,b$ inputs fails for $b\in \{1, 2, 2\omega - 4, \omega - 2, 4\omega, 3\omega - 5, \omega - 1, -\omega + 3, -2\omega + 5, 2\omega - 3\}$.
\item  2e -- Check of $b,b,\dots,b$ inputs fails for $b\in \{0, 1, 2, 2\omega - 4, 2\omega - 2, \omega - 2, \omega, 4\omega, 3\omega - 5, \omega - 1, -\omega + 3, -2\omega + 5, 2\omega - 3\}$.
\end{enumerate}


\end{exmp}




\begin{exmp}
\label{ex:compareAF}

\textbf{Penney\_1--block\_integer}

Phase 1 (methods 1b, 1c, 1d, 1e):
\begin{enumerate}[ ]
\item  2a -- Check of $b,b,\dots,b$ inputs fails for $b\in \{0, 3, 4, -4, -3\}$.
\item  2b -- Check of $b,b,\dots,b$ inputs fails for $b\in \{0, 3, -3\}$.
\item  2c -- Check of $b,b,\dots,b$ inputs fails for $b\in \{0, 3, 4, -4, -3, -2\}$.
\item  2d -- Check of $b,b,\dots,b$ inputs fails for $b\in \{0, 3, 4, -4, -3, -2\}$.
\item  2e -- Check of $b,b,\dots,b$ inputs fails for $b\in \{0, 3, -3\}$.
\end{enumerate}


Phase 1 (methods 1a):
\begin{enumerate}[ ]
\item  2a -- Check of $b,b,\dots,b$ inputs fails for $b\in \{0, 1, 2, -1, -2\}$.
\item  2b -- Check of $b,b,\dots,b$ inputs fails for $b\in \{0, 1, 2, -1, -2\}$.
\item  2c -- Check of $b,b,\dots,b$ inputs fails for $b\in \{0, 1, 2, 3, 4, -4, -3, -2\}$.
\item  2d -- Check of $b,b,\dots,b$ inputs fails for $b\in \{0, 1, 2, 3, 4, -4, -3, -2\}$.
\item  2e -- Check of $b,b,\dots,b$ inputs fails for $b\in \{2, -2\}$.
\end{enumerate}


\end{exmp}




\begin{exmp}
\label{ex:compareAK}

\textbf{Quadratic+1+0--5\_integer}

Phase 1 (methods 1a, 1b, 1c, 1d, 1e):
\begin{enumerate}[ ]
\item  2a -- Check of $b,b,\dots,b$ inputs fails for $b\in \{-2\}$.
\end{enumerate}


\end{exmp}




\begin{exmp}
\label{ex:compareAL}

\textbf{Quadratic+1+2+3\_complex}

Phase 1 (methods 1a, 1b, 1c, 1d, 1e):
\begin{enumerate}[ ]
\item  2a -- Phase 2   fails because  the sequence $(1, -2\omega - 2, -\omega - 2, 2\omega + 2, -2\omega - 2, -\omega - 2, 2\omega + 2, -2\omega - 2, \dots ,-2\omega - 2, -\omega - 2, 2\omega + 2, -2\omega - 2, \dots)$ leads to an infinite loop.
\item  2c -- Phase 2   fails because  the sequence $(1, -2\omega - 2, -\omega - 2, 2\omega + 2, -2\omega - 2, -\omega - 2, 2\omega + 2, -2\omega - 2, \dots ,-2\omega - 2, -\omega - 2, 2\omega + 2, -2\omega - 2, \dots)$ leads to an infinite loop.
\item  2d -- Phase 2   fails because  the sequence $(0, \omega + 2, -1, -1, 2\omega + 1, -1, -1, 2\omega + 1, -1, \dots ,-1, -1, 2\omega + 1, -1, \dots)$ leads to an infinite loop.
\end{enumerate}


\end{exmp}




\begin{exmp}
\label{ex:compareAM}

\textbf{Quadratic+1+3+4\_complex}

Phase 1 (methods 1b):
\begin{enumerate}[ ]
\item  2c -- Phase 2   fails because  the sequence $(2, 2\omega + 1, -\omega - 2, -2, 2\omega, 2\omega + 1, -\omega - 2, -2, 2\omega, \dots ,2\omega + 1, -\omega - 2, -2, 2\omega, \dots)$ leads to an infinite loop.
\item  2d -- Check of $b,b,\dots,b$ inputs fails for $b\in \{\omega + 3\}$.
\end{enumerate}


Phase 1 (methods 1c):
\begin{enumerate}[ ]
\item  2a -- Phase 2   fails because  the sequence $(2, 2\omega + 3, -2\omega - 2, 2, 2\omega + 3, -2\omega - 2, 2, \dots ,2, 2\omega + 3, -2\omega - 2, 2, \dots)$ leads to an infinite loop.
\item  2c -- Phase 2   fails because  the sequence $(2, 2\omega + 2, -\omega + 1, 2, 2\omega + 2, -\omega + 1, 2, \dots ,2, 2\omega + 2, -\omega + 1, 2, \dots)$ leads to an infinite loop.
\item  2d -- Check of $b,b,\dots,b$ inputs fails for $b\in \{\omega + 3\}$.
\end{enumerate}


Phase 1 (methods 1d, 1e):
\begin{enumerate}[ ]
\item  2a -- Phase 2   fails because  the sequence $(2, 2\omega + 3, -2\omega - 2, 2, 2\omega + 3, -2\omega - 2, 2, \dots ,2, 2\omega + 3, -2\omega - 2, 2, \dots)$ leads to an infinite loop.
\item  2c -- Phase 2   fails because  the sequence $(2, 2\omega + 2, -\omega + 1, 2, 2\omega + 2, -\omega + 1, 2, \dots ,2, 2\omega + 2, -\omega + 1, 2, \dots)$ leads to an infinite loop.
\item  2d -- Check of $b,b,\dots,b$ inputs fails for $b\in \{\omega + 3\}$.
\end{enumerate}


Phase 1 (methods 1a):
\begin{enumerate}[ ]
\item  2c -- Phase 2   fails because  the sequence $(2, 2\omega + 2, -\omega + 1, 2, 2\omega + 2, -\omega + 1, 2, \dots ,2, 2\omega + 2, -\omega + 1, 2, \dots)$ leads to an infinite loop.
\item  2d -- Check of $b,b,\dots,b$ inputs fails for $b\in \{\omega + 3\}$.
\end{enumerate}


\end{exmp}




\begin{exmp}
\label{ex:compareAN}

\textbf{Quadratic+1+3+5\_complex1}

Phase 1 (methods 1a):
\begin{enumerate}[ ]
\item  2a -- Check of $b,b,\dots,b$ inputs fails for $b\in \{2\omega + 2, \omega + 4, -\omega - 4, -2\omega - 2\}$.
\item  2b -- Check of $b,b,\dots,b$ inputs fails for $b\in \{2\omega + 2, -2\omega - 2\}$.
\item  2c -- Check of $b,b,\dots,b$ inputs fails for $b\in \{2\omega + 2, \omega + 4, -\omega - 4, -2\omega - 2\}$.
\item  2d -- Check of $b,b,\dots,b$ inputs fails for $b\in \{2\omega + 2, 2\omega + 4, \omega + 4, -\omega - 4, -2\omega - 4, -2\omega - 2\}$.
\item  2e -- Check of $b,b,\dots,b$ inputs fails for $b\in \{2\omega + 2, -2\omega - 2\}$.
\end{enumerate}


Phase 1 (methods 1b, 1c):
\begin{enumerate}[ ]
\item  2a -- Check of $b,b,\dots,b$ inputs fails for $b\in \{-2\omega - 2\}$.
\item  2b -- Phase 2   fails because  the sequence $(2, 2\omega + 2, 2\omega + 2, 2, 2\omega + 2, 2\omega + 2, \dots ,2, 2\omega + 2, 2\omega + 2, \dots)$ leads to an infinite loop.
\item  2c -- Check of $b,b,\dots,b$ inputs fails for $b\in \{2\omega + 2, -2\omega - 2\}$.
\item  2d -- Check of $b,b,\dots,b$ inputs fails for $b\in \{2\omega + 2, -2\omega - 2\}$.
\item  2e -- Phase 2   fails because  the sequence $(2, 2\omega + 2, 2\omega + 2, 2, 2\omega + 2, 2\omega + 2, \dots ,2, 2\omega + 2, 2\omega + 2, \dots)$ leads to an infinite loop.
\end{enumerate}


Phase 1 (methods 1d, 1e):
\begin{enumerate}[ ]
\item  2a -- Check of $b,b,\dots,b$ inputs fails for $b\in \{2\omega + 2, -2\omega - 2\}$.
\item  2b -- Check of $b,b,\dots,b$ inputs fails for $b\in \{2\omega + 2, -2\omega - 2\}$.
\item  2c -- Check of $b,b,\dots,b$ inputs fails for $b\in \{2\omega + 2, -2\omega - 2\}$.
\item  2d -- Check of $b,b,\dots,b$ inputs fails for $b\in \{2\omega + 2, 2\omega + 4, -2\omega - 4, -2\omega - 2\}$.
\item  2e -- Check of $b,b,\dots,b$ inputs fails for $b\in \{2\omega + 2, -2\omega - 2\}$.
\end{enumerate}


\end{exmp}




\begin{exmp}
\label{ex:compareAO}

\textbf{Quadratic+1+3+5\_complex2 }

Phase 1 (methods 1b, 1c):
\begin{enumerate}[ ]
\item  2a -- Check of $b,b,\dots,b$ inputs fails for $b\in \{-3\omega - 4, 2\omega + 2, 2\omega + 3, \omega + 3, -2\omega - 4, -2\omega - 3, -2\omega - 2, -\omega - 3\}$.
\item  2b -- Phase 2   fails because  the sequence $(0, 1, 2\omega + 1, -4\omega - 4, 4\omega + 4, 0, 1, 4\omega + 4, 0, 1, 4\omega + 4, \dots ,4\omega + 4, 0, 1, 4\omega + 4, \dots)$ leads to an infinite loop.
\item  2c -- Check of $b,b,\dots,b$ inputs fails for $b\in \{-3\omega - 3, 3\omega + 3\}$.
\item  2d -- Check of $b,b,\dots,b$ inputs fails for $b\in \{-3\omega - 4, 2\omega + 3, 2\omega + 4, \omega + 3, -2\omega - 4, -2\omega - 3, -\omega - 3, 3\omega + 4\}$.
\item  2e -- Phase 2   fails because  the sequence $(0, 1, 2\omega + 1, -4\omega - 4, 4\omega + 4, 0, 1, 4\omega + 4, 0, 1, 4\omega + 4, \dots ,4\omega + 4, 0, 1, 4\omega + 4, \dots)$ leads to an infinite loop.
\end{enumerate}


Phase 1 (methods 1d, 1e):
\begin{enumerate}[ ]
\item  2a -- Check of $b,b,\dots,b$ inputs fails for $b\in \{2\omega + 2, -2\omega - 4, -2\omega - 3, -2\omega - 2, -\omega - 3\}$.
\item  2b -- Phase 2   fails because  the sequence $(0, 0, -4\omega - 4, 3\omega + 4, 0, 4\omega + 4, 0, 4\omega + 4, 0, 4\omega + 4, \dots ,0, 4\omega + 4, 0, 4\omega + 4, \dots)$ leads to an infinite loop.
\item  2c -- Phase 2   fails because  the sequence $(0, 0, -2\omega - 1, 3\omega + 3, -4\omega - 4, \omega, -2\omega - 3, 4\omega + 4, -\omega, 2\omega + 3, -3\omega - 3, 4\omega + 4, 2\omega + 2, 4\omega + 4, 2\omega + 2, 4\omega + 4, 2\omega + 2, \dots ,4\omega + 4, 2\omega + 2, 4\omega + 4, 2\omega + 2, \dots)$ leads to an infinite loop.
\item  2d -- Check of $b,b,\dots,b$ inputs fails for $b\in \{-3\omega - 4, 2\omega + 3, 2\omega + 4, \omega + 3, -2\omega - 4, -2\omega - 3, -\omega - 3, 3\omega + 4\}$.
\item  2e -- Phase 2   fails because  the sequence $(0, -3\omega - 4, 3\omega + 3, -3\omega - 4, -4\omega - 4, 4\omega + 4, 0, 0, 4\omega + 4, -3\omega - 4, -1, -3\omega - 4, -1, -3\omega - 4, -1, \dots ,-3\omega - 4, -1, -3\omega - 4, -1, \dots)$ leads to an infinite loop.
\end{enumerate}


Phase 1 (methods 1a):
\begin{enumerate}[ ]
\item  2a -- Check of $b,b,\dots,b$ inputs fails for $b\in \{-3\omega - 3, 2\omega + 2, -2\omega - 3, -\omega - 3, 3\omega + 3\}$.
\item  2b -- Phase 2   fails because  the sequence $(0, -1, -3\omega - 3, \omega, -3\omega - 4, -3\omega - 4, -3\omega - 3, 4\omega + 4, -3\omega - 4, 2\omega + 1, 0, -1, -3\omega - 3, \omega, \dots ,0, -1, -3\omega - 3, \omega, \dots)$ leads to an infinite loop.
\item  2c -- Phase 2   fails because  the sequence $(0, 0, -4\omega - 4, 2\omega + 1, -\omega - 2, -3\omega - 4, 2\omega + 2, -\omega - 1, -\omega, 3\omega + 3, -2\omega - 4, 2\omega + 1, -\omega - 2, -3\omega - 4, 2\omega + 2, \dots ,2\omega + 1, -\omega - 2, -3\omega - 4, 2\omega + 2, \dots)$ leads to an infinite loop.
\item  2d -- Check of $b,b,\dots,b$ inputs fails for $b\in \{2\omega + 3, \omega + 3, -2\omega - 3, -\omega - 3\}$.
\item  2e -- Phase 2   fails because  the sequence $(0, -3\omega - 4, 3\omega + 3, -\omega, 2\omega + 1, -2\omega - 1, 1, 2\omega + 1, 0, -1, -3\omega - 3, \omega, -3\omega - 4, -3\omega - 4, -3\omega - 3, 4\omega + 4, -3\omega - 4, 2\omega + 1, 0, -1, -3\omega - 3, \dots ,2\omega + 1, 0, -1, -3\omega - 3, \dots)$ leads to an infinite loop.
\end{enumerate}


\end{exmp}




\begin{exmp}
\label{ex:compareAP}

\textbf{Quadratic+1+4+5\_complex1}

Phase 1 (Lemma \ref{lem:suffCondPhase1}):
\begin{enumerate}[ ]
\item  2a -- Phase 2   fails because  the sequence $(2, 2\omega - 1, -2\omega + 1, 2\omega - 2, -\omega + 2, 2, 2\omega - 2, -\omega + 2, 2, 2\omega - 2, \dots ,2\omega - 2, -\omega + 2, 2, 2\omega - 2, \dots)$ leads to an infinite loop.
\item  2b -- Check of $b,b,\dots,b$ inputs fails for $b\in \{-\omega - 2, -2\omega + 1\}$.
\item  2c -- Check of $b,b,\dots,b$ inputs fails for $b\in \{-\omega - 2\}$.
\item  2d -- Check of $b,b,\dots,b$ inputs fails for $b\in \{-\omega - 2\}$.
\end{enumerate}


Phase 1 (methods 1a):
\begin{enumerate}[ ]
\item  2a -- Check of $b,b,\dots,b$ inputs fails for $b\in \{2, -\omega - 2, -2\omega\}$.
\item  2b -- Check of $b,b,\dots,b$ inputs fails for $b\in \{-\omega - 2\}$.
\item  2c -- Check of $b,b,\dots,b$ inputs fails for $b\in \{2\omega - 2, 2\omega + 2, \omega + 3, -\omega - 2, -2\omega + 2\}$.
\item  2d -- Check of $b,b,\dots,b$ inputs fails for $b\in \{2\omega - 2, 2\omega + 2, \omega + 3, -\omega - 2, -2\omega + 2\}$.
\item  2e -- Check of $b,b,\dots,b$ inputs fails for $b\in \{-\omega - 2\}$.
\end{enumerate}


Phase 1 (methods 1b, 1c, 1d, 1e):
\begin{enumerate}[ ]
\item  2a -- Check of $b,b,\dots,b$ inputs fails for $b\in \{2, -\omega - 2, -2\omega\}$.
\item  2b -- Check of $b,b,\dots,b$ inputs fails for $b\in \{-\omega - 2\}$.
\item  2c -- Check of $b,b,\dots,b$ inputs fails for $b\in \{2\omega + 2, -\omega - 2\}$.
\item  2d -- Check of $b,b,\dots,b$ inputs fails for $b\in \{2\omega + 2, -\omega - 2\}$.
\item  2e -- Check of $b,b,\dots,b$ inputs fails for $b\in \{-\omega - 2\}$.
\end{enumerate}


\end{exmp}




\begin{exmp}
\label{ex:compareAR}

\textbf{Cubic+1+0+0+2\_integer}

Phase 1 (methods 1a, 1b, 1c, 1d, 1e):
\begin{enumerate}[ ]
\item  2a -- Phase 2   fails because  the sequence $(0, 1, 0, 1, 0, 1, 0, \dots ,0, 1, 0, 1, 0, \dots)$ leads to an infinite loop.
\item  2b -- Phase 2   fails because  the sequence $(0, 1, 0, -1, 0, 1, 0, -1, 0, \dots ,0, 1, 0, -1, 0, \dots)$ leads to an infinite loop.
\item  2c -- Phase 2   fails because  the sequence $(0, 1, 0, -1, 0, 1, 0, -1, 0, \dots ,0, 1, 0, -1, 0, \dots)$ leads to an infinite loop.
\item  2e -- Phase 2   fails because  the sequence $(0, 1, 0, -1, 0, 1, 0, -1, 0, \dots ,0, 1, 0, -1, 0, \dots)$ leads to an infinite loop.
\end{enumerate}


\end{exmp}




\begin{exmp}
\label{ex:compareAS}

\textbf{Cubic+1+0+0--2\_integer}

Phase 1 (methods 1a, 1b, 1c, 1d, 1e):
\begin{enumerate}[ ]
\item  2a -- Phase 2   fails because  the sequence $(0, 1, 0, 1, 0, 1, 0, \dots ,0, 1, 0, 1, 0, \dots)$ leads to an infinite loop.
\item  2b -- Phase 2   fails because  the sequence $(0, 1, 0, -1, 0, 1, 0, -1, 0, \dots ,0, 1, 0, -1, 0, \dots)$ leads to an infinite loop.
\item  2c -- Phase 2   fails because  the sequence $(0, 1, 0, -1, 0, 1, 0, -1, 0, \dots ,0, 1, 0, -1, 0, \dots)$ leads to an infinite loop.
\item  2e -- Phase 2   fails because  the sequence $(0, 1, 0, -1, 0, 1, 0, -1, 0, \dots ,0, 1, 0, -1, 0, \dots)$ leads to an infinite loop.
\end{enumerate}


\end{exmp}






\subsection*{Quadratic bases with integer alphabet}
The following examples show alphabets  divided into congruence classes modulo $\beta$ and $\beta-1$ for some numeration systems in Table~\ref{tab:resultsQuadrInt}.



\begin{exmp}
\label{ex:integerAB}

An alphabet $\A$ divided into congruence classes modulo $\beta$: 
$$ \left\{\left\{-5, 6\right\}, \left\{-4\right\}, \left\{-3\right\}, \left\{-2\right\}, \left\{-1\right\}, \left\{0\right\}, \left\{1\right\}, \left\{2\right\}, \left\{3\right\}, \left\{4\right\}, \left\{5\right\}\right\} \,,$$

and modulo $\beta - 1$: $$ \left\{\left\{-5, 1\right\}, \left\{-4, 2\right\}, \left\{-3, 3\right\}, \left\{-2, 4\right\}, \left\{-1, 5\right\}, \left\{0, 6\right\}\right\} \,.$$

\end{exmp}




%
%\begin{exmp}
%\label{ex:integerAD}
%
%An alphabet $\A$ divided into congruence classes modulo $\beta$: 
%$$ \left\{\left\{-3, 4\right\}, \left\{-2\right\}, \left\{-1\right\}, \left\{0\right\}, \left\{1\right\}, \left\{2\right\}, \left\{3\right\}\right\} \,,$$
%
%and modulo $\beta - 1$: $$ \left\{\left\{-3, 1\right\}, \left\{-2, 2\right\}, \left\{-1, 3\right\}, \left\{0, 4\right\}\right\} \,.$$
%
%\end{exmp}


\begin{exmp}
\label{ex:integerAE}

An alphabet $\A$ divided into congruence classes modulo $\beta$: 
$$ \left\{\left\{-5, 2\right\}, \left\{-4, 3\right\}, \left\{-3, 4\right\}, \left\{-2, 5\right\}, \left\{-1\right\}, \left\{0\right\}, \left\{1\right\}\right\} \,,$$

and modulo $\beta - 1$: $$ \left\{\left\{-5, 2\right\}, \left\{-4, 3\right\}, \left\{-3, 4\right\}, \left\{-2, 5\right\}, \left\{-1\right\}, \left\{0\right\}, \left\{1\right\}\right\} \,.$$

\end{exmp}




%\begin{exmp}
%\label{ex:integerAG}
%
%An alphabet $\A$ divided into congruence classes modulo $\beta$: 
%$$ \left\{\left\{-1, 1\right\}, \left\{0\right\}\right\} \,,$$
%
%and modulo $\beta - 1$: $$ \left\{\left\{-1\right\}, \left\{0\right\}, \left\{1\right\}\right\} \,.$$
%
%\end{exmp}
%
%
%\begin{exmp}
%\label{ex:integerAH}
%
%An alphabet $\A$ divided into congruence classes modulo $\beta$: 
%$$ \left\{\left\{0\right\}, \left\{1, -1\right\}\right\} \,,$$
%
%and modulo $\beta - 1$: $$ \left\{\left\{0, 1, -1\right\}\right\} \,.$$
%
%\end{exmp}




\begin{exmp}
\label{ex:integerAJ}

An alphabet $\A$ divided into congruence classes modulo $\beta$: 
$$ \left\{\left\{-2, 3\right\}, \left\{-1\right\}, \left\{0\right\}, \left\{1\right\}, \left\{2\right\}\right\} \,,$$

and modulo $\beta - 1$: $$ \left\{\left\{-2, 0, 2\right\}, \left\{-1, 1, 3\right\}\right\} \,.$$

\end{exmp}


\begin{exmp}
\label{ex:integerAK}

An alphabet $\A$ divided into congruence classes modulo $\beta$: 
$$ \left\{\left\{-2, 3\right\}, \left\{-1\right\}, \left\{0\right\}, \left\{1\right\}, \left\{2\right\}\right\} \,,$$

and modulo $\beta - 1$: $$ \left\{\left\{-2, 2\right\}, \left\{-1, 3\right\}, \left\{0\right\}, \left\{1\right\}\right\} \,.$$

\end{exmp}



\begin{exmp}
\label{ex:integerAO}

An alphabet $\A$ divided into congruence classes modulo $\beta$: 
$$ \left\{\left\{-7, 6\right\}, \left\{-6, 7\right\}, \left\{-5\right\}, \left\{-4\right\}, \left\{-3\right\}, \left\{-2\right\}, \left\{-1\right\}, \left\{0\right\}, \left\{1\right\}, \left\{2\right\}, \left\{3\right\}, \left\{4\right\}, \left\{5\right\}\right\} \,,$$

and modulo $\beta - 1$: $$ \left\{\left\{-7, -1, 5\right\}, \left\{-6, 0, 6\right\}, \left\{-5, 1, 7\right\}, \left\{-4, 2\right\}, \left\{-3, 3\right\}, \left\{-2, 4\right\}\right\} \,.$$

\end{exmp}



%\begin{exmp}
%\label{ex:integerAQ}
%
%An alphabet $\A$ divided into congruence classes modulo $\beta$: 
%$$ \left\{\left\{-8, 9\right\}, \left\{-7\right\}, \left\{-6\right\}, \left\{-5\right\}, \left\{-4\right\}, \left\{-3\right\}, \left\{-2\right\}, \left\{-1\right\}, \left\{0\right\}, \left\{1\right\}, \left\{2\right\}, \left\{3\right\}, \left\{4\right\}, \left\{5\right\}, \left\{6\right\}, \left\{7\right\}, \left\{8\right\}\right\} \,,$$
%
%and modulo $\beta - 1$: $$ \left\{\left\{-8, 0, 8\right\}, \left\{-7, 1, 9\right\}, \left\{-6, 2\right\}, \left\{-5, 3\right\}, \left\{-4, 4\right\}, \left\{-3, 5\right\}, \left\{-2, 6\right\}, \left\{-1, 7\right\}\right\} \,.$$
%
%\end{exmp}


\begin{exmp}
\label{ex:integerAR}

An alphabet $\A$ divided into congruence classes modulo $\beta$: 
\begin{align*}
\{\left\{-10, 11\right\}, \left\{-9\right\}, \left\{-8\right\}, \left\{-7\right\},& \left\{-6\right\}, \left\{-5\right\}, \left\{-4\right\}, \left\{-3\right\}, \left\{-2\right\},  \\
\left\{-1\right\}, \left\{0\right\}, \left\{1\right\}, \left\{2\right\}, \left\{3\right\}, \left\{4\right\},& \left\{5\right\}, \left\{6\right\}, \left\{7\right\}, \left\{8\right\}, \left\{9\right\}, \left\{10\right\}\} \,,
\end{align*}

and modulo $\beta - 1$: \begin{align*} 
\{\left\{-10, 0, 10\right\}, \left\{-9, 1, 11\right\},& \left\{-8, 2\right\}, \left\{-7, 3\right\}, \left\{-6, 4\right\}, \\
 \left\{-5, 5\right\}, \left\{-4, 6\right\}, \{-3,\, & 7\}, \left\{-2, 8\right\}, \left\{-1, 9\right\}\} \,.
 \end{align*}

\end{exmp}




\subsection*{Killed examples}
The computation of a weight function for the following numeration systems was killed because of memory limits.
\begin{exmp}
\label{ex:killAB}

\rule{0cm}{0cm}

\begin{tabular}{ll}
$\omega=  -\frac{1}{2} \, \sqrt{37} + \frac{5}{2} $  & $\beta= -\omega - 3 = \frac{1}{2} \, \sqrt{37} - \frac{11}{2} $\\
$m_\omega(t)=  t^{2} - 5 \, t - 3 $  & $m_\beta(x)=  x^{2} + 11 \, x + 21 $\\
Real conjugate of $\beta$ greater than 1:   &  no \\
$\#\A= $ 33 $ $ & $\A$ is minimal. \\
\multicolumn{2}{l}{\begin{minipage}{\textwidth}\begin{dmath*}\A = \left\{0, 1, -1, \omega + 1, -\omega - 1, -\omega + 1, \omega - 1, \omega, -\omega, 2 \, \omega + 2, -2 \, \omega - 2, \omega + 2, -\omega - 2, -2 \, \omega + 2, 2 \, \omega - 2, 2 \, \omega + 1, -2 \, \omega - 1, -2 \, \omega + 1, 2 \, \omega - 1, 2 \, \omega, -2 \, \omega, -2 \, \omega + 3, 2 \, \omega - 3, -3 \, \omega + 3, 3 \, \omega - 3, -3 \, \omega + 2, 3 \, \omega - 2, -3 \, \omega + 1, 3 \, \omega - 1, 3 \, \omega, -3 \, \omega, -3 \, \omega + 4, 3 \, \omega - 4\right\}  \end{dmath*}\end{minipage} }\\
 & \\
Phase 1 (method  9): &
\checkmark, $\#\mathcal{Q} = $ 17 $ $ \\ 
$b,b,\dots,b$ inputs (method  2c): & \checkmark, maximal length of window: $ 3 $ \\
\multicolumn{2}{l}{\begin{minipage}{\textwidth} Computation of Phase 2 (method  2c) was killed when the length of window 5 was being proccessed. Numbers of saved combinations for each finished length are: (0, 12399, 682670, 2721482)\end{minipage} }\\
\end{tabular}

\end{exmp}



\begin{exmp}
\label{ex:killAA}

\rule{0cm}{0cm}

\begin{tabular}{ll}
$\omega=  -\frac{1}{2} \, \sqrt{29} + \frac{3}{2} $  & $\beta= 3 \, \omega + 1 = -\frac{3}{2} \, \sqrt{29} + \frac{11}{2} $\\
$m_\omega(t)=  t^{2} - 3 \, t - 5 $  & $m_\beta(x)=  x^{2} - 11 \, x - 35 $\\
Real conjugate of $\beta$ greater than 1:   &  yes \\
$\#\A= $ 49 $ $ & $\A$ is not minimal. \\
\multicolumn{2}{l}{\begin{minipage}{\textwidth}\begin{dmath*}\A = \left\{0, 1, -1, \omega + 1, -\omega - 1, -\omega + 1, \omega - 1, \omega, -\omega, 2 \, \omega + 2, -2 \, \omega - 2, \omega + 2, -\omega - 2, 2, -2, 3 \, \omega + 3, -3 \, \omega - 3, 2 \, \omega + 3, -2 \, \omega - 3, \omega + 3, -\omega - 3, 4 \, \omega + 4, -4 \, \omega - 4, 3 \, \omega + 4, -3 \, \omega - 4, 2 \, \omega + 4, -2 \, \omega - 4, 5 \, \omega + 5, -5 \, \omega - 5, 4 \, \omega + 5, -4 \, \omega - 5, 3 \, \omega + 5, -3 \, \omega - 5, 6 \, \omega + 6, -6 \, \omega - 6, 5 \, \omega + 6, -5 \, \omega - 6, 4 \, \omega + 6, -4 \, \omega - 6, 7 \, \omega + 7, -7 \, \omega - 7, 6 \, \omega + 7, -6 \, \omega - 7, 5 \, \omega + 7, -5 \, \omega - 7, -2 \, \omega + 5, 2 \, \omega - 5, 4 \, \omega + 7, -4 \, \omega - 7\right\}  \end{dmath*}\end{minipage} }\\
 & \\
Phase 1 (method  9): &
\checkmark, $\#\mathcal{Q} = $ 46 $ $ \\ 
$b,b,\dots,b$ inputs (method  21): & \checkmark, maximal length of window: $ 5 $ \\
\multicolumn{2}{l}{\begin{minipage}{\textwidth} Computation of Phase 2 (method  21) was killed. % when the length of window 1 was being proccessed. Numbers of saved combinations for each finished length are: ()
\end{minipage} }\\
\end{tabular}

\end{exmp}





\begin{exmp}
\label{ex:killAC}


\rule{0cm}{0cm}

\begin{tabular}{ll}
$\omega=  {\left(\frac{1}{9} \, \sqrt{19} \sqrt{3} + 1\right)}^{\frac{1}{3}} + \frac{2}{3 \, {\left(\frac{1}{9} \, \sqrt{19} \sqrt{3} + 1\right)}^{\frac{1}{3}}} $  & $\beta= -2 \, \omega^{2} + \omega + 2 = {\left(\sqrt{57} - \frac{197}{27}\right)}^{\frac{1}{3}} - \frac{14}{9 \, {\left(\sqrt{57} - \frac{197}{27}\right)}^{\frac{1}{3}}} - \frac{2}{3} $\\
$m_\omega(t)=  t^{3} - 2 \, t - 2 $  & $m_\beta(x)=  x^{3} + 2 \, x^{2} + 6 \, x + 18 $\\
Real conjugate of $\beta$ greater than 1:   &  no \\
$\#\A= $ 31 $ $ & $\A$ is not minimal. \\
\multicolumn{2}{l}{\begin{minipage}{\textwidth}\begin{dmath*}\A = \left\{0, 1, -1, \omega^{2} + \omega + 1, -\omega^{2} - \omega - 1, \omega + 1, -\omega - 1, -\omega^{2} + \omega + 1, \omega^{2} - \omega - 1, \omega^{2} + 1, -\omega^{2} - 1, -\omega^{2} + 1, \omega^{2} - 1, \omega^{2} - \omega + 1, -\omega^{2} + \omega - 1, -\omega + 1, \omega - 1, -\omega^{2} - \omega + 1, \omega^{2} + \omega - 1, \omega^{2} + \omega, -\omega^{2} - \omega, \omega, -\omega, \omega^{2} + 2, -\omega^{2} - 2, 2, -2, -\omega^{2} + \omega, \omega^{2} - \omega, \omega^{2}, -\omega^{2}\right\}  \end{dmath*}\end{minipage} }\\
 & \\
Phase 1 (method  9): &
\checkmark, $\#\mathcal{Q} = $ 83 $ $ \\ 
$b,b,\dots,b$ inputs (method  21): & \checkmark, maximal length of window: $ 5 $ \\
\multicolumn{2}{l}{\begin{minipage}{\textwidth} Computation of Phase 2 (method  21) was killed when the length of window 4 was being proccessed. Numbers of saved combinations for each finished length are: (0, 71, 1887261)\end{minipage} }\\
\end{tabular}

\end{exmp}


\begin{exmp}
\label{ex:killAD}


\rule{0cm}{0cm}

\begin{tabular}{ll}
$\omega=  {\left(\frac{1}{9} \, \sqrt{29} \sqrt{3} + \frac{28}{27}\right)}^{\frac{1}{3}} + \frac{1}{9 \, {\left(\frac{1}{9} \, \sqrt{29} \sqrt{3} + \frac{28}{27}\right)}^{\frac{1}{3}}} + \frac{1}{3} $ \\
\multicolumn{2}{l}{ $\beta= -\omega^{2} + \omega - 1 = {\left(\frac{2}{9} \, \sqrt{29} \sqrt{3} - 2\right)}^{\frac{1}{3}} - \frac{2}{3 \, {\left(\frac{2}{9} \, \sqrt{29} \sqrt{3} - 2\right)}^{\frac{1}{3}}} - 1 $}\\
$m_\omega(t)=  t^{3} - t^{2} - 2 $  & $m_\beta(x)=  x^{3} + 3 \, x^{2} + 5 \, x + 7 $\\
Real conjugate of $\beta$ greater than 1:   &  no \\
$\#\A= $ 16 $ $ & $\A$ is minimal. \\
\multicolumn{2}{l}{\begin{minipage}{\textwidth}\begin{dmath*}\A = \left\{0, 1, -1, \omega^{2} + \omega + 1, -\omega^{2} - \omega - 1, \omega + 1, \omega^{2} + 1, -\omega^{2} - 1, -\omega^{2} + 1, \omega^{2} + \omega, \omega, -\omega, -\omega^{2} + \omega, \omega^{2} - \omega, \omega^{2}, -\omega^{2}\right\}  \end{dmath*}\end{minipage} }\\
 & \\
Phase 1 (method  9): &
\checkmark, $\#\mathcal{Q} = $ 99 $ $ \\ 
$b,b,\dots,b$ inputs (method  21): & \checkmark, maximal length of window: $ 6 $ \\
\multicolumn{2}{l}{\begin{minipage}{\textwidth} Computation of Phase 2 (method  21) was killed when the length of window 4 was being proccessed. Numbers of saved combinations for each finished length are: (73, 5329, 315494)\end{minipage} }\\
\end{tabular}

\end{exmp}


