\section{Illustration of Phase 1}
\label{app:phase1}   
Figures \ref{img:phase1img3} -- \ref{img:phase1img12} illustrates first and last iterations of the construction of the weight coefficients set $\Q$ for the Eisenstein base $\beta = -\frac{3}{2} + \frac{\imath \sqrt{3}}{2}$ with the complex alphabet $\mathcal{A} =\{0, 1, -1, \omega, -\omega, -\omega - 1, \omega + 1\}$ and input alphabet $\B=\A+\A$. The second last iteration is skipped.

\figurehascaptionOne{1 = The starting set $\Q_0{=}\{0\}$.}
\figurehascaptionOne{2 = The set $\B+\Q_0$ need to be covered.}
\figurehascaptionOne{3 = The set $\Q_0$ does not cover the set $\B+\Q_0${,} i.e.{,} the set $\A+\beta \cdot \Q_0$ is not superset of $\B+\Q_0$.}
\figurehascaptionOne{4 = The set $\Q_0$ is extended to $\Q_1$ to cover all elements of $\B+\Q_0$.}
\figurehascaptionOne{5 = The set $\B+\Q_1$ need to be covered.}
\figurehascaptionOne{6 = The set $\Q_1$ does not cover the set $\B+\Q_1${,} i.e.{,} the set $\A+\beta \cdot \Q_1$ is not superset of $\B+\Q_1$.}
\figurehascaptionOne{7 = The set $\Q_1$ is extended to $\Q_2$ to cover all elements of $\B+\Q_1$.}
\figurehascaptionOne{8 = The set $\B+\Q_2$ need to be covered.}
\figurehascaptionOne{9 = The set $\Q_2$ does not cover the set $\B+\Q_2${,} i.e.{,} the set $\A+\beta \cdot \Q_2$ is not superset of $\B+\Q_2$.}
\figurehascaptionOne{10 = The set $\Q_2$ is extended to $\Q_3$ to cover all elements of $\B+\Q_2$.}
\figurehascaptionOne{11 = The set $\B+\Q_3$ need to be covered.}
\figurehascaptionOne{12 = In the last iteration{,} the set $\Q_3$ covers the set $\B+\Q_3${,} i.e.{,} the set $\A+\beta \cdot \Q_3$ is superset of $\B+\Q_2$. The weight coefficients set $\Q$ equals $\Q_3$.}
\figurehascaptionOne{13 = The final weight coefficients $\Q{=}\Q_3$.}


\foreach \n in {3,4,6,7,12} {%
\begin{SCfigure}[][htbp]
    \centering
    \caption{\getcaptionOne{\n}}
    \label{img:phase1img\n}
    \includegraphics[height=0.27\textheight]{img/eisenstein/phase1_image_\n.png}
\end{SCfigure}
    }

\newpage


\section{Illustration of Phase 2}
The construction of set $\Q_{[\omega,1,2]}$ for the Eisenstein base $\beta = -\frac{3}{2} + \frac{\imath \sqrt{3}}{2}$ with the complex alphabet $\mathcal{A} =\{0, 1, -1, \omega, -\omega, -\omega - 1, \omega + 1\}$  and input alphabet $\B=\A+\A$ is illustrated on Figures \ref{img:phase2img3} -- \ref{img:phase2img7}.
\label{app:phase2}    

\figurehascaptionTwo{1 = Phase 2 starts with the weight coefficients set $\Q$ from Phase 1.}
\figurehascaptionTwo{2 = The set $\omega+\Q$ need to be covered.}
\figurehascaptionTwo{3 = The elements of $\omega+\Q$ are covered by the set $\Q_{[\omega]}\subset\Q$.}
\figurehascaptionTwo{4 = The set $\omega+\Q_{[1]}$ need to be covered.}
\figurehascaptionTwo{5 = The elements of $\omega+\Q_{[1]}$ are covered by the set $\Q_{[\omega,1]}\subset\Q_{[\omega]}$.}
\figurehascaptionTwo{6 = The set $\omega+\Q_{[1,2]}$ need to be covered.}
\figurehascaptionTwo{7 = The elements of $\omega+\Q_{[1]}$ are covered by the set $\Q_{[\omega,1,2]}\subset\Q_{[\omega,1]}$ which has only one element{.} This element is the output of the weight function $q{(\omega,1,2)}$.}



\foreach \n in {3,5,7} {%
\begin{SCfigure}[][htbp]
    \centering
    \caption{\getcaptionTwo{\n}}
    \label{img:phase2img\n}
    \includegraphics[height=0.23\textheight]{img/eisenstein/phase2_image_\n.png}
\end{SCfigure}
    }

\newpage

\section{Interfaces}
\label{app:interfaces}
File \verb+ewm_inputs.sage+:
\lstinputlisting[language=Python]{ewm_inputs.sage}

File \verb+ewm_gspreadsheet.sage+:
\lstinputlisting[language=Python]{ewm_gspreadsheet.sage}




\section{Tested examples}
\label{app:examples}

\subsection*{Unsuccessful examples comparing different methods}
The reasons of failure of Phase~2 for numeration systems in Section~\ref{sec:compareMethods} can be found here. See Tables \ref{tab:resultsPhaseOne}, \ref{tab:resultsPhaseTwo} and \ref{tab:alphabets} for parameters of the numeration systems.
\begin{exmp}
\label{ex:compareAA}


\rule{0cm}{0cm}

\begin{tabular}{ll}
$\omega=  \frac{1}{2} i \, \sqrt{3} - \frac{1}{2} $  & $\beta= \omega - 1 = \frac{1}{2} i \, \sqrt{3} - \frac{3}{2} $\\
$m_\omega(t)=  t^{2} + t + 1 $  & $m_\beta(x)=  x^{2} + 3 \, x + 3 $\\
Real conjugate of $\beta$ greater than 1:   &  no \\ \hline
\multicolumn{2}{l}{\begin{minipage}{\textwidth}\begin{dmath*}\A = \left\{0, 1, -1, \omega, -\omega, -\omega - 1, \omega + 1\right\}  \end{dmath*}\end{minipage} }\\
$\#\A= $ 7 $ $ & $\A$ is minimal. \\
\multicolumn{2}{l}{\begin{minipage}{\textwidth}\begin{dmath*}\B =\A+\A \end{dmath*}\end{minipage} }\\[10pt]
\multicolumn{2}{l}{\begin{minipage}{\textwidth}$\A$ divided into congruence classes modulo $\beta$: \begin{dmath*} \left\{\left\{0\right\}, \left\{1, \omega, -\omega - 1\right\}, \left\{-1, -\omega, \omega + 1\right\}\right\}  \end{dmath*}\end{minipage} }\\[10pt]
\multicolumn{2}{l}{\begin{minipage}{\textwidth}$\A$ divided into congruence classes modulo $\beta-1$: \begin{dmath*} \left\{\left\{0\right\}, \left\{1\right\}, \left\{-1\right\}, \left\{\omega\right\}, \left\{-\omega\right\}, \left\{-\omega - 1\right\}, \left\{\omega + 1\right\}\right\}  \end{dmath*}\end{minipage} }\\
 & \\ \hline
 & \\
\end{tabular}

\begin{tabular}{ll}
Phase 1 (methods $12, 13, 14, 15, 16$): &
\checkmark, $\#\mathcal{Q} =19$ \\ 
Method  9: &\\
$b,b,\dots,b$ inputs: & \checkmark \\
Phase 2: & \checkmark , $r= 3$ \\
Method  15: &\\
$b,b,\dots,b$ inputs: & \checkmark \\
Phase 2: & \checkmark , $r= 3$ \\
Method  22: &\\
$b,b,\dots,b$ inputs: & \checkmark \\
Phase 2: & \checkmark , $r= 3$ \\
Method  23: &\\
$b,b,\dots,b$ inputs: & \checkmark \\
Phase 2: & \checkmark , $r= 3$ \\
\hline
\end{tabular}

\end{exmp}




\begin{exmp}
\label{ex:compareAB}


\rule{0cm}{0cm}

\begin{tabular}{ll}
$\omega=  \frac{1}{2} i \, \sqrt{3} - \frac{1}{2} $  & $\beta= \omega - 1 = \frac{1}{2} i \, \sqrt{3} - \frac{3}{2} $\\
$m_\omega(t)=  t^{2} + t + 1 $  & $m_\beta(x)=  x^{2} + 3 \, x + 3 $\\
Real conjugate of $\beta$ greater than 1:   &  no \\ \hline
\multicolumn{2}{l}{\begin{minipage}{\textwidth}\begin{dmath*}\A = \left\{-3, -2, -1, 0, 1, 2, 3\right\}  \end{dmath*}\end{minipage} }\\
$\#\A= $ 7 $ $ & $\A$ is minimal. \\
\multicolumn{2}{l}{\begin{minipage}{\textwidth}\begin{dmath*}\B =\A+\A \end{dmath*}\end{minipage} }\\[10pt]
\multicolumn{2}{l}{\begin{minipage}{\textwidth}$\A$ divided into congruence classes modulo $\beta$: \begin{dmath*} \left\{\left\{-3, 0, 3\right\}, \left\{-2, 1\right\}, \left\{-1, 2\right\}\right\}  \end{dmath*}\end{minipage} }\\[10pt]
\multicolumn{2}{l}{\begin{minipage}{\textwidth}$\A$ divided into congruence classes modulo $\beta-1$: \begin{dmath*} \left\{\left\{-3\right\}, \left\{-2\right\}, \left\{-1\right\}, \left\{0\right\}, \left\{1\right\}, \left\{2\right\}, \left\{3\right\}\right\}  \end{dmath*}\end{minipage} }\\
 & \\ \hline
 & \\
\end{tabular}

\begin{tabular}{ll}
Phase 1 (methods $12, 13, 15, 16$): &
\checkmark, $\#\mathcal{Q} =57$ \\ 
Method  9: &\\
Failing $b,b,\dots,b$ inputs: & $\{2, 3, 5, 6, -5, -4, -3\}$ \\
Method  15: &\\
Failing $b,b,\dots,b$ inputs: & $\{2, 3, 6, -6, -4, -3\}$ \\
Method  22: &\\
Failing $b,b,\dots,b$ inputs: & $\{0, 1, 3, 4, 6, -6, -4, -3, -1\}$ \\
Method  23: &\\
Failing $b,b,\dots,b$ inputs: & $\{3, 4, 6, -6, -4, -3\}$ \\
\hline
Phase 1 (methods $14$): &
\checkmark, $\#\mathcal{Q} =139$ \\ 
Method  9: &\\
Failing $b,b,\dots,b$ inputs: & $\{0, 2, 4, 5, -2, -5, -4\}$ \\
Method  15: &\\
Failing $b,b,\dots,b$ inputs: & $\{0, 2, 4, 5, -2, -5, -4\}$ \\
Method  22: &\\
Failing $b,b,\dots,b$ inputs: & $\{0, 2, 3, 5, 6, -2, -6, -5, -3\}$ \\
Method  23: &\\
Failing $b,b,\dots,b$ inputs: & $\{0, 3, 4, 5, 6, -6, -5, -4, -3\}$ \\
\hline
\end{tabular}

\end{exmp}




\begin{exmp}
\label{ex:compareAC}


\rule{0cm}{0cm}

\begin{tabular}{ll}
$\omega=  \frac{1}{2} i \, \sqrt{3} - \frac{1}{2} $  & $\beta= -3 \, \omega = -\frac{3}{2} i \, \sqrt{3} + \frac{3}{2} $\\
$m_\omega(t)=  t^{2} + t + 1 $  & $m_\beta(x)=  x^{2} - 3 \, x + 9 $\\
Real conjugate of $\beta$ greater than 1:   &  no \\ \hline
\multicolumn{2}{l}{\begin{minipage}{\textwidth}\begin{dmath*}\A = \left\{0, 1, \omega, \omega + 1, 2 \, \omega, 2 \, \omega - 1, \omega - 1, -1, -2, -\omega, -\omega - 1, -\omega - 2, -2 \, \omega, -2 \, \omega - 1\right\}  \end{dmath*}\end{minipage} }\\
$\#\A= $ 14 $ $ & $\A$ is not minimal. \\
\multicolumn{2}{l}{\begin{minipage}{\textwidth}\begin{dmath*}\B =\A+\A \end{dmath*}\end{minipage} }\\[10pt]
\multicolumn{2}{l}{\begin{minipage}{\textwidth}$\A$ divided into congruence classes modulo $\beta$: \begin{dmath*} \left\{\left\{0\right\}, \left\{1, -2\right\}, \left\{\omega, -2 \, \omega\right\}, \left\{\omega + 1\right\}, \left\{2 \, \omega, -\omega\right\}, \left\{2 \, \omega - 1, -\omega - 1\right\}, \left\{\omega - 1, -2 \, \omega - 1\right\}, \left\{-1\right\}, \left\{-\omega - 2\right\}\right\}  \end{dmath*}\end{minipage} }\\[10pt]
\multicolumn{2}{l}{\begin{minipage}{\textwidth}$\A$ divided into congruence classes modulo $\beta-1$: \begin{dmath*} \left\{\left\{0\right\}, \left\{1, \omega - 1\right\}, \left\{\omega, -2 \, \omega - 1\right\}, \left\{\omega + 1, 2 \, \omega - 1, -\omega - 2, -2 \, \omega\right\}, \left\{2 \, \omega, -\omega - 1\right\}, \left\{-1\right\}, \left\{-2, -\omega\right\}\right\}  \end{dmath*}\end{minipage} }\\
 & \\ \hline
 & \\
\end{tabular}

\begin{tabular}{ll}
Phase 1 (methods $12, 13, 14, 15, 16$): &
\checkmark, $\#\mathcal{Q} =17$ \\ 
Method  9: &\\
Failing $b,b,\dots,b$ inputs: & $\{2\omega - 1, \omega + 1, -2\omega, -4, -\omega - 2\}$ \\
Method  15: &\\
Failing $b,b,\dots,b$ inputs: & $\{2\omega - 1, \omega + 1, -2\omega, -4, -\omega - 2\}$ \\
Method  22: &\\
Failing $b,b,\dots,b$ inputs: & $\{2\omega - 1, \omega + 1, -2\omega, -4, -\omega - 2\}$ \\
Method  23: &\\
Failing $b,b,\dots,b$ inputs: & $\{2\omega - 1, \omega + 1, -2\omega, -4, -\omega - 2\}$ \\
\hline
\end{tabular}

\end{exmp}




\begin{exmp}
\label{ex:compareAD}


\rule{0cm}{0cm}

\begin{tabular}{ll}
$\omega=  \frac{1}{2} i \, \sqrt{3} - \frac{1}{2} $  & $\beta= -3 \, \omega = -\frac{3}{2} i \, \sqrt{3} + \frac{3}{2} $\\
$m_\omega(t)=  t^{2} + t + 1 $  & $m_\beta(x)=  x^{2} - 3 \, x + 9 $\\
Real conjugate of $\beta$ greater than 1:   &  no \\ \hline
\multicolumn{2}{l}{\begin{minipage}{\textwidth}\begin{dmath*}\A = \left\{-\omega + 3, -\omega + 2, -\omega + 1, -\omega, 2, 1, 0, -1, \omega + 1, \omega, \omega - 1, \omega - 2, 2 \, \omega, 2 \, \omega - 1, 2 \, \omega - 2, 2 \, \omega - 3\right\}  \end{dmath*}\end{minipage} }\\
$\#\A= $ 16 $ $ & $\A$ is not minimal. \\
\multicolumn{2}{l}{\begin{minipage}{\textwidth}\begin{dmath*}\B =\A+\A \end{dmath*}\end{minipage} }\\[10pt]
\multicolumn{2}{l}{\begin{minipage}{\textwidth}$\A$ divided into congruence classes modulo $\beta$: \begin{dmath*} \left\{\left\{-\omega + 3, -\omega, 2 \, \omega, 2 \, \omega - 3\right\}, \left\{-\omega + 2, 2 \, \omega - 1\right\}, \left\{-\omega + 1, 2 \, \omega - 2\right\}, \left\{2, -1\right\}, \left\{1\right\}, \left\{0\right\}, \left\{\omega + 1, \omega - 2\right\}, \left\{\omega\right\}, \left\{\omega - 1\right\}\right\}  \end{dmath*}\end{minipage} }\\[10pt]
\multicolumn{2}{l}{\begin{minipage}{\textwidth}$\A$ divided into congruence classes modulo $\beta-1$: \begin{dmath*} \left\{\left\{-\omega + 3, 1, \omega - 1, 2 \, \omega - 3\right\}, \left\{-\omega + 2, 0, \omega - 2\right\}, \left\{-\omega + 1, -1\right\}, \left\{-\omega\right\}, \left\{2, \omega, 2 \, \omega - 2\right\}, \left\{\omega + 1, 2 \, \omega - 1\right\}, \left\{2 \, \omega\right\}\right\}  \end{dmath*}\end{minipage} }\\
 & \\ \hline
 & \\
\end{tabular}

\begin{tabular}{ll}
Phase 1 (methods $12, 13, 14, 15, 16$): &
\checkmark, $\#\mathcal{Q} =26$ \\ 
Method  9: &\\
Failing $b,b,\dots,b$ inputs: & $\{0, 1, 2, 2\omega - 4, \omega - 2, 4\omega, 3\omega - 5, \omega - 1, -\omega + 3, -2\omega + 5, 2\omega - 3\}$ \\
Method  15: &\\
Failing $b,b,\dots,b$ inputs: & $\{0, 1, 2, 2\omega - 4, 2\omega - 2, \omega - 2, \omega, 4\omega, 3\omega - 5, \omega - 1, -\omega + 3, -2\omega + 5, 2\omega - 3\}$ \\
Method  22: &\\
Failing $b,b,\dots,b$ inputs: & $\{0, 1, 2, 2\omega - 4, \omega - 2, \omega, 4\omega, 3\omega - 5, \omega - 1, -\omega + 3, -2\omega + 5, 2\omega - 3\}$ \\
Method  23: &\\
Failing $b,b,\dots,b$ inputs: & $\{1, 2, 2\omega - 4, \omega - 2, 4\omega, 3\omega - 5, \omega - 1, -\omega + 3, -2\omega + 5, 2\omega - 3\}$ \\
\hline
\end{tabular}

\end{exmp}




\begin{exmp}
\label{ex:compareAE}


\rule{0cm}{0cm}

\begin{tabular}{ll}
$\omega=  i - 1 $  & $\beta= \omega = i - 1 $\\
$m_\omega(t)=  t^{2} + 2 \, t + 2 $  & $m_\beta(x)=  x^{2} + 2 \, x + 2 $\\
Real conjugate of $\beta$ greater than 1:   &  no \\ \hline
\multicolumn{2}{l}{\begin{minipage}{\textwidth}\begin{dmath*}\A = \left\{0, \omega + 1, -\omega - 1, 1, -1\right\}  \end{dmath*}\end{minipage} }\\
$\#\A= $ 5 $ $ & $\A$ is minimal. \\
\multicolumn{2}{l}{\begin{minipage}{\textwidth}\begin{dmath*}\B =\A+\A \end{dmath*}\end{minipage} }\\[10pt]
\multicolumn{2}{l}{\begin{minipage}{\textwidth}$\A$ divided into congruence classes modulo $\beta$: \begin{dmath*} \left\{\left\{0\right\}, \left\{\omega + 1, -\omega - 1, 1, -1\right\}\right\}  \end{dmath*}\end{minipage} }\\[10pt]
\multicolumn{2}{l}{\begin{minipage}{\textwidth}$\A$ divided into congruence classes modulo $\beta-1$: \begin{dmath*} \left\{\left\{0\right\}, \left\{\omega + 1\right\}, \left\{-\omega - 1\right\}, \left\{1\right\}, \left\{-1\right\}\right\}  \end{dmath*}\end{minipage} }\\
 & \\ \hline
 & \\
\end{tabular}

\begin{tabular}{ll}
Phase 1 (methods $12, 13, 14, 15, 16$): &
\checkmark, $\#\mathcal{Q} =45$ \\ 
Method  9: &\\
$b,b,\dots,b$ inputs: & \checkmark \\
Phase 2: & \checkmark , $r= 6$ \\
Method  15: &\\
$b,b,\dots,b$ inputs: & \checkmark \\
Phase 2: & \checkmark , $r= 6$ \\
Method  22: &\\
$b,b,\dots,b$ inputs: & \checkmark \\
Phase 2: & \checkmark , $r= 6$ \\
Method  23: &\\
$b,b,\dots,b$ inputs: & \checkmark \\
Phase 2: & \checkmark , $r= 6$ \\
\hline
\end{tabular}

\end{exmp}




\begin{exmp}
\label{ex:compareAF}


\rule{0cm}{0cm}

\begin{tabular}{ll}
$\omega=  i $  & $\beta= \omega - 1 = i - 1 $\\
$m_\omega(t)=  t^{2} + 1 $  & $m_\beta(x)=  x^{2} + 2 \, x + 2 $\\
Real conjugate of $\beta$ greater than 1:   &  no \\ \hline
\multicolumn{2}{l}{\begin{minipage}{\textwidth}\begin{dmath*}\A = \left\{-2, -1, 0, 1, 2\right\}  \end{dmath*}\end{minipage} }\\
$\#\A= $ 5 $ $ & $\A$ is minimal. \\
\multicolumn{2}{l}{\begin{minipage}{\textwidth}\begin{dmath*}\B = \left\{-3, -2, -1, 0, 1, 2, 3\right\}  \end{dmath*}\end{minipage} }\\[10pt]
\multicolumn{2}{l}{\begin{minipage}{\textwidth}$\A$ divided into congruence classes modulo $\beta$: \begin{dmath*} \left\{\left\{-2, 0, 2\right\}, \left\{-1, 1\right\}\right\}  \end{dmath*}\end{minipage} }\\[10pt]
\multicolumn{2}{l}{\begin{minipage}{\textwidth}$\A$ divided into congruence classes modulo $\beta-1$: \begin{dmath*} \left\{\left\{-2\right\}, \left\{-1\right\}, \left\{0\right\}, \left\{1\right\}, \left\{2\right\}\right\}  \end{dmath*}\end{minipage} }\\
 & \\ \hline
 & \\
\end{tabular}

\begin{tabular}{ll}
Phase 1 (methods $12, 13, 15, 16$): &
\checkmark, $\#\mathcal{Q} =27$ \\ 
Method  9: &\\
Failing $b,b,\dots,b$ inputs: & $\{-3, -2, 1, 2\}$ \\
Method  15: &\\
Failing $b,b,\dots,b$ inputs: & $\{-2, 2\}$ \\
Method  22: &\\
Failing $b,b,\dots,b$ inputs: & $\{-3, -2, 2, 3\}$ \\
Method  23: &\\
Failing $b,b,\dots,b$ inputs: & $\{-3, -2, 2, 3\}$ \\
\hline
Phase 1 (methods $14$): &
\checkmark, $\#\mathcal{Q} =95$ \\ 
Method  9: &\\
Failing $b,b,\dots,b$ inputs: & $\{-3, 0, 1, 3\}$ \\
Method  15: &\\
Failing $b,b,\dots,b$ inputs: & $\{-3, 0, 1, 3\}$ \\
Method  22: &\\
Failing $b,b,\dots,b$ inputs: & $\{0, 3\}$ \\
Method  23: &\\
Failing $b,b,\dots,b$ inputs: & $\{0, 3\}$ \\
\hline
\end{tabular}

\end{exmp}




\begin{exmp}
\label{ex:compareAG}


\rule{0cm}{0cm}

\begin{tabular}{ll}
$\omega=  i $  & $\beta= -2 \, \omega = -2 i $\\
$m_\omega(t)=  t^{2} + 1 $  & $m_\beta(x)=  x^{2} + 4 $\\
Real conjugate of $\beta$ greater than 1:   &  no \\ \hline
\multicolumn{2}{l}{\begin{minipage}{\textwidth}\begin{dmath*}\A = \left\{0, 1, -1, \omega, -\omega, \omega - 1, -\omega + 1, \omega - 2, -\omega + 2\right\}  \end{dmath*}\end{minipage} }\\
$\#\A= $ 9 $ $ & $\A$ is not minimal. \\
\multicolumn{2}{l}{\begin{minipage}{\textwidth}\begin{dmath*}\B =\A+\A \end{dmath*}\end{minipage} }\\[10pt]
\multicolumn{2}{l}{\begin{minipage}{\textwidth}$\A$ divided into congruence classes modulo $\beta$: \begin{dmath*} \left\{\left\{0\right\}, \left\{1, -1\right\}, \left\{\omega, -\omega, \omega - 2, -\omega + 2\right\}, \left\{\omega - 1, -\omega + 1\right\}\right\}  \end{dmath*}\end{minipage} }\\[10pt]
\multicolumn{2}{l}{\begin{minipage}{\textwidth}$\A$ divided into congruence classes modulo $\beta-1$: \begin{dmath*} \left\{\left\{0, \omega - 2, -\omega + 2\right\}, \left\{1, \omega - 1\right\}, \left\{-1, -\omega + 1\right\}, \left\{\omega\right\}, \left\{-\omega\right\}\right\}  \end{dmath*}\end{minipage} }\\
 & \\ \hline
 & \\
\end{tabular}

\begin{tabular}{ll}
Phase 1 (methods $12, 13, 14, 15, 16$): &
\checkmark, $\#\mathcal{Q} =27$ \\ 
Method  9: &\\
$b,b,\dots,b$ inputs: & \checkmark \\
Phase 2: & \checkmark , $r= 5$ \\
Method  15: &\\
$b,b,\dots,b$ inputs: & \checkmark \\
Phase 2: & \checkmark , $r= 5$ \\
Method  22: &\\
$b,b,\dots,b$ inputs: & \checkmark \\
Phase 2: & \checkmark , $r= 5$ \\
Method  23: &\\
$b,b,\dots,b$ inputs: & \checkmark \\
Phase 2: & \checkmark , $r= 5$ \\
\hline
\end{tabular}

\end{exmp}




\begin{exmp}
\label{ex:compareAH}


\rule{0cm}{0cm}

\begin{tabular}{ll}
$\omega=  \sqrt{2} $  & $\beta= \omega = \sqrt{2} $\\
$m_\omega(t)=  t^{2} - 2 $  & $m_\beta(x)=  x^{2} - 2 $\\
Real conjugate of $\beta$ greater than 1:   &  yes \\ \hline
\multicolumn{2}{l}{\begin{minipage}{\textwidth}\begin{dmath*}\A = \left\{0, 1, -1\right\}  \end{dmath*}\end{minipage} }\\
$\#\A= $ 3 $ $ & $\A$ is minimal. \\
\multicolumn{2}{l}{\begin{minipage}{\textwidth}\begin{dmath*}\B =\A+\A \end{dmath*}\end{minipage} }\\[10pt]
\multicolumn{2}{l}{\begin{minipage}{\textwidth}$\A$ divided into congruence classes modulo $\beta$: \begin{dmath*} \left\{\left\{0\right\}, \left\{1, -1\right\}\right\}  \end{dmath*}\end{minipage} }\\[10pt]
\multicolumn{2}{l}{\begin{minipage}{\textwidth}$\A$ divided into congruence classes modulo $\beta-1$: \begin{dmath*} \left\{\left\{0, 1, -1\right\}\right\}  \end{dmath*}\end{minipage} }\\
 & \\ \hline
 & \\
\end{tabular}

\begin{tabular}{ll}
Phase 1 (methods $12, 13, 14, 15, 16$): &
\checkmark, $\#\mathcal{Q} =9$ \\ 
Method  9: &\\
$b,b,\dots,b$ inputs: & \checkmark \\
Phase 2: & \checkmark , $r= 5$ \\
Method  15: &\\
$b,b,\dots,b$ inputs: & \checkmark \\
Phase 2: & \checkmark , $r= 5$ \\
Method  22: &\\
$b,b,\dots,b$ inputs: & \checkmark \\
Phase 2: & \checkmark , $r= 5$ \\
Method  23: &\\
$b,b,\dots,b$ inputs: & \checkmark \\
Phase 2: & \checkmark , $r= 4$ \\
\hline
\end{tabular}

\end{exmp}




\begin{exmp}
\label{ex:compareAI}


\rule{0cm}{0cm}

\begin{tabular}{ll}
$\omega=  -\frac{1}{2} \, \sqrt{21} + \frac{3}{2} $  & $\beta= 2 \, \omega - 3 = -\sqrt{21} $\\
$m_\omega(t)=  t^{2} - 3 \, t - 3 $  & $m_\beta(x)=  x^{2} - 21 $\\
Real conjugate of $\beta$ greater than 1:   &  yes \\ \hline
\multicolumn{2}{l}{\begin{minipage}{\textwidth}\begin{dmath*}\A = \left\{-10, -9, -8, -7, -6, -5, -4, -3, -2, -1, 0, 1, 2, 3, 4, 5, 6, 7, 8, 9, 10, 11\right\}  \end{dmath*}\end{minipage} }\\
$\#\A= $ 22 $ $ & $\A$ is minimal. \\
\multicolumn{2}{l}{\begin{minipage}{\textwidth}\begin{dmath*}\B =\A+\A \end{dmath*}\end{minipage} }\\[10pt]
\multicolumn{2}{l}{\begin{minipage}{\textwidth}$\A$ divided into congruence classes modulo $\beta$: \begin{dmath*} \left\{\left\{-10, 11\right\}, \left\{-9\right\}, \left\{-8\right\}, \left\{-7\right\}, \left\{-6\right\}, \left\{-5\right\}, \left\{-4\right\}, \left\{-3\right\}, \left\{-2\right\}, \left\{-1\right\}, \left\{0\right\}, \left\{1\right\}, \left\{2\right\}, \left\{3\right\}, \left\{4\right\}, \left\{5\right\}, \left\{6\right\}, \left\{7\right\}, \left\{8\right\}, \left\{9\right\}, \left\{10\right\}\right\}  \end{dmath*}\end{minipage} }\\[10pt]
\multicolumn{2}{l}{\begin{minipage}{\textwidth}$\A$ divided into congruence classes modulo $\beta-1$: \begin{dmath*} \left\{\left\{-10, 0, 10\right\}, \left\{-9, 1, 11\right\}, \left\{-8, 2\right\}, \left\{-7, 3\right\}, \left\{-6, 4\right\}, \left\{-5, 5\right\}, \left\{-4, 6\right\}, \left\{-3, 7\right\}, \left\{-2, 8\right\}, \left\{-1, 9\right\}\right\}  \end{dmath*}\end{minipage} }\\
 & \\ \hline
 & \\
\end{tabular}

\begin{tabular}{ll}
Phase 1 (methods $12, 13, 14, 15, 16$): &
\checkmark, $\#\mathcal{Q} =9$ \\ 
Method  9: &\\
$b,b,\dots,b$ inputs: & \checkmark \\
Phase 2: & \checkmark , $r= 4$ \\
Method  15: &\\
$b,b,\dots,b$ inputs: & \checkmark \\
Phase 2: & \checkmark , $r= 4$ \\
Method  22: &\\
$b,b,\dots,b$ inputs: & \checkmark \\
Phase 2: & \checkmark , $r= 4$ \\
Method  23: &\\
$b,b,\dots,b$ inputs: & \checkmark \\
Phase 2: & \checkmark , $r= 4$ \\
\hline
\end{tabular}

\end{exmp}




\begin{exmp}
\label{ex:compareAJ}


\rule{0cm}{0cm}

\begin{tabular}{ll}
$\omega=  \sqrt{3} - 1 $  & $\beta= -\omega - 1 = -\sqrt{3} $\\
$m_\omega(t)=  t^{2} + 2 \, t - 2 $  & $m_\beta(x)=  x^{2} - 3 $\\
Real conjugate of $\beta$ greater than 1:   &  yes \\ \hline
\multicolumn{2}{l}{\begin{minipage}{\textwidth}\begin{dmath*}\A = \left\{0, 1, -1, 2\right\}  \end{dmath*}\end{minipage} }\\
$\#\A= $ 4 $ $ & $\A$ is minimal. \\
\multicolumn{2}{l}{\begin{minipage}{\textwidth}\begin{dmath*}\B =\A+\A \end{dmath*}\end{minipage} }\\[10pt]
\multicolumn{2}{l}{\begin{minipage}{\textwidth}$\A$ divided into congruence classes modulo $\beta$: \begin{dmath*} \left\{\left\{0\right\}, \left\{1\right\}, \left\{-1, 2\right\}\right\}  \end{dmath*}\end{minipage} }\\[10pt]
\multicolumn{2}{l}{\begin{minipage}{\textwidth}$\A$ divided into congruence classes modulo $\beta-1$: \begin{dmath*} \left\{\left\{0, 2\right\}, \left\{1, -1\right\}\right\}  \end{dmath*}\end{minipage} }\\
 & \\ \hline
 & \\
\end{tabular}

\begin{tabular}{ll}
Phase 1 (methods $12, 13, 14, 15, 16$): &
\checkmark, $\#\mathcal{Q} =9$ \\ 
Method  9: &\\
$b,b,\dots,b$ inputs: & \checkmark \\
Phase 2: & \checkmark , $r= 4$ \\
Method  15: &\\
$b,b,\dots,b$ inputs: & \checkmark \\
Phase 2: & \checkmark , $r= 5$ \\
Method  22: &\\
$b,b,\dots,b$ inputs: & \checkmark \\
Phase 2: & \checkmark , $r= 5$ \\
Method  23: &\\
$b,b,\dots,b$ inputs: & \checkmark \\
Phase 2: & \checkmark , $r= 5$ \\
\hline
\end{tabular}

\end{exmp}




\begin{exmp}
\label{ex:compareAK}


\rule{0cm}{0cm}

\begin{tabular}{ll}
$\omega=  \frac{1}{2} \, \sqrt{5} - \frac{1}{2} $  & $\beta= 2 \, \omega + 1 = \sqrt{5} $\\
$m_\omega(t)=  t^{2} + t - 1 $  & $m_\beta(x)=  x^{2} - 5 $\\
Real conjugate of $\beta$ greater than 1:   &  yes \\ \hline
\multicolumn{2}{l}{\begin{minipage}{\textwidth}\begin{dmath*}\A = \left\{-3, -2, -1, 0, 1, 2, 3, 4\right\}  \end{dmath*}\end{minipage} }\\
$\#\A= $ 8 $ $ & $\A$ is not minimal. \\
\multicolumn{2}{l}{\begin{minipage}{\textwidth}\begin{dmath*}\B =\A+\A \end{dmath*}\end{minipage} }\\[10pt]
\multicolumn{2}{l}{\begin{minipage}{\textwidth}$\A$ divided into congruence classes modulo $\beta$: \begin{dmath*} \left\{\left\{-3, 2\right\}, \left\{-2, 3\right\}, \left\{-1, 4\right\}, \left\{0\right\}, \left\{1\right\}\right\}  \end{dmath*}\end{minipage} }\\[10pt]
\multicolumn{2}{l}{\begin{minipage}{\textwidth}$\A$ divided into congruence classes modulo $\beta-1$: \begin{dmath*} \left\{\left\{-3, -1, 1, 3\right\}, \left\{-2, 0, 2, 4\right\}\right\}  \end{dmath*}\end{minipage} }\\
 & \\ \hline
 & \\
\end{tabular}

\begin{tabular}{ll}
Phase 1 (methods $12, 13, 14, 15, 16$): &
\checkmark, $\#\mathcal{Q} =9$ \\ 
Method  9: &\\
Failing $b,b,\dots,b$ inputs: & $\{-2\}$ \\
Method  15: &\\
$b,b,\dots,b$ inputs: & \checkmark \\
Phase 2: & \checkmark , $r= 3$ \\
Method  22: &\\
$b,b,\dots,b$ inputs: & \checkmark \\
Phase 2: & \checkmark , $r= 2$ \\
Method  23: &\\
$b,b,\dots,b$ inputs: & \checkmark \\
Phase 2: & \checkmark , $r= 2$ \\
\hline
\end{tabular}

\end{exmp}




\begin{exmp}
\label{ex:compareAL}


\rule{0cm}{0cm}

\begin{tabular}{ll}
$\omega=  i \, \sqrt{2} - 1 $  & $\beta= -\omega - 2 = -i \, \sqrt{2} - 1 $\\
$m_\omega(t)=  t^{2} + 2 \, t + 3 $  & $m_\beta(x)=  x^{2} + 2 \, x + 3 $\\
Real conjugate of $\beta$ greater than 1:   &  no \\ \hline
\multicolumn{2}{l}{\begin{minipage}{\textwidth}\begin{dmath*}\A = \left\{0, \omega + 1, -\omega - 1, 1, -1, \omega\right\}  \end{dmath*}\end{minipage} }\\
$\#\A= $ 6 $ $ & $\A$ is minimal. \\
\multicolumn{2}{l}{\begin{minipage}{\textwidth}\begin{dmath*}\B =\A+\A \end{dmath*}\end{minipage} }\\[10pt]
\multicolumn{2}{l}{\begin{minipage}{\textwidth}$\A$ divided into congruence classes modulo $\beta$: \begin{dmath*} \left\{\left\{0\right\}, \left\{\omega + 1, -1\right\}, \left\{-\omega - 1, 1, \omega\right\}\right\}  \end{dmath*}\end{minipage} }\\[10pt]
\multicolumn{2}{l}{\begin{minipage}{\textwidth}$\A$ divided into congruence classes modulo $\beta-1$: \begin{dmath*} \left\{\left\{0\right\}, \left\{\omega + 1\right\}, \left\{-\omega - 1\right\}, \left\{1\right\}, \left\{-1\right\}, \left\{\omega\right\}\right\}  \end{dmath*}\end{minipage} }\\
 & \\ \hline
 & \\
\end{tabular}

\begin{tabular}{ll}
Phase 1 (methods $12, 13, 14, 15, 16$): &
\checkmark, $\#\mathcal{Q} =27$ \\ 
Method  9: &\\
$b,b,\dots,b$ inputs: & \checkmark \\
\multicolumn{2}{l}{\begin{minipage}{\textwidth} Phase 2 fails because  the sequence $(1, -2\omega - 2, -\omega - 2, 2\omega + 2, -2\omega - 2, -\omega - 2, 2\omega + 2, -2\omega - 2, \dots ,-2\omega - 2, -\omega - 2, 2\omega + 2, -2\omega - 2, \dots)$ leads to an infinite loop.\end{minipage} }\\
Method  15: &\\
$b,b,\dots,b$ inputs: & \checkmark \\
Phase 2: & \checkmark , $r= 7$ \\
Method  22: &\\
$b,b,\dots,b$ inputs: & \checkmark \\
\multicolumn{2}{l}{\begin{minipage}{\textwidth} Phase 2 fails because  the sequence $(1, -2\omega - 2, -\omega - 2, 2\omega + 2, -2\omega - 2, -\omega - 2, 2\omega + 2, -2\omega - 2, \dots ,-2\omega - 2, -\omega - 2, 2\omega + 2, -2\omega - 2, \dots)$ leads to an infinite loop.\end{minipage} }\\
Method  23: &\\
$b,b,\dots,b$ inputs: & \checkmark \\
\multicolumn{2}{l}{\begin{minipage}{\textwidth} Phase 2 fails because  the sequence $(0, \omega + 2, -1, -1, 2\omega + 1, -1, -1, 2\omega + 1, -1, \dots ,-1, -1, 2\omega + 1, -1, \dots)$ leads to an infinite loop.\end{minipage} }\\
\hline
\end{tabular}

\end{exmp}




\begin{exmp}
\label{ex:compareAM}


\rule{0cm}{0cm}

\begin{tabular}{ll}
$\omega=  \frac{1}{2} i \, \sqrt{7} - \frac{1}{2} $  & $\beta= \omega - 1 = \frac{1}{2} i \, \sqrt{7} - \frac{3}{2} $\\
$m_\omega(t)=  t^{2} + t + 2 $  & $m_\beta(x)=  x^{2} + 3 \, x + 4 $\\
Real conjugate of $\beta$ greater than 1:   &  no \\ \hline
\multicolumn{2}{l}{\begin{minipage}{\textwidth}\begin{dmath*}\A = \left\{0, \omega + 1, -\omega - 1, 1, -1, \omega, -\omega, \omega + 2\right\}  \end{dmath*}\end{minipage} }\\
$\#\A= $ 8 $ $ & $\A$ is minimal. \\
\multicolumn{2}{l}{\begin{minipage}{\textwidth}\begin{dmath*}\B =\A+\A \end{dmath*}\end{minipage} }\\[10pt]
\multicolumn{2}{l}{\begin{minipage}{\textwidth}$\A$ divided into congruence classes modulo $\beta$: \begin{dmath*} \left\{\left\{0\right\}, \left\{\omega + 1, -\omega - 1\right\}, \left\{1, \omega\right\}, \left\{-1, -\omega, \omega + 2\right\}\right\}  \end{dmath*}\end{minipage} }\\[10pt]
\multicolumn{2}{l}{\begin{minipage}{\textwidth}$\A$ divided into congruence classes modulo $\beta-1$: \begin{dmath*} \left\{\left\{0\right\}, \left\{\omega + 1\right\}, \left\{-\omega - 1\right\}, \left\{1\right\}, \left\{-1\right\}, \left\{\omega\right\}, \left\{-\omega\right\}, \left\{\omega + 2\right\}\right\}  \end{dmath*}\end{minipage} }\\
 & \\ \hline
 & \\
\end{tabular}

\begin{tabular}{ll}
Phase 1 (methods $12$): &
\checkmark, $\#\mathcal{Q} =20$ \\ 
Method  9: &\\
$b,b,\dots,b$ inputs: & \checkmark \\
Phase 2: & \checkmark , $r= 7$ \\
Method  15: &\\
$b,b,\dots,b$ inputs: & \checkmark \\
Phase 2: & \checkmark , $r= 7$ \\
Method  22: &\\
$b,b,\dots,b$ inputs: & \checkmark \\
\multicolumn{2}{l}{\begin{minipage}{\textwidth} Phase 2 fails because  the sequence $(2, 2\omega + 1, -\omega - 2, -2, 2\omega, 2\omega + 1, -\omega - 2, -2, 2\omega, \dots ,2\omega + 1, -\omega - 2, -2, 2\omega, \dots)$ leads to an infinite loop.\end{minipage} }\\
Method  23: &\\
Failing $b,b,\dots,b$ inputs: & $\{\omega + 3\}$ \\
\hline
Phase 1 (methods $16$): &
\checkmark, $\#\mathcal{Q} =19$ \\ 
Method  9: &\\
$b,b,\dots,b$ inputs: & \checkmark \\
\multicolumn{2}{l}{\begin{minipage}{\textwidth} Phase 2 fails because  the sequence $(2, 2\omega + 3, -2\omega - 2, 2, 2\omega + 3, -2\omega - 2, 2, \dots ,2, 2\omega + 3, -2\omega - 2, 2, \dots)$ leads to an infinite loop.\end{minipage} }\\
Method  15: &\\
$b,b,\dots,b$ inputs: & \checkmark \\
Phase 2: & \checkmark , $r= 7$ \\
Method  22: &\\
$b,b,\dots,b$ inputs: & \checkmark \\
\multicolumn{2}{l}{\begin{minipage}{\textwidth} Phase 2 fails because  the sequence $(2, 2\omega + 2, -\omega + 1, 2, 2\omega + 2, -\omega + 1, 2, \dots ,2, 2\omega + 2, -\omega + 1, 2, \dots)$ leads to an infinite loop.\end{minipage} }\\
Method  23: &\\
Failing $b,b,\dots,b$ inputs: & $\{\omega + 3\}$ \\
\hline
Phase 1 (methods $13, 15$): &
\checkmark, $\#\mathcal{Q} =20$ \\ 
Method  9: &\\
$b,b,\dots,b$ inputs: & \checkmark \\
\multicolumn{2}{l}{\begin{minipage}{\textwidth} Phase 2 fails because  the sequence $(2, 2\omega + 3, -2\omega - 2, 2, 2\omega + 3, -2\omega - 2, 2, \dots ,2, 2\omega + 3, -2\omega - 2, 2, \dots)$ leads to an infinite loop.\end{minipage} }\\
Method  15: &\\
$b,b,\dots,b$ inputs: & \checkmark \\
Phase 2: & \checkmark , $r= 7$ \\
Method  22: &\\
$b,b,\dots,b$ inputs: & \checkmark \\
\multicolumn{2}{l}{\begin{minipage}{\textwidth} Phase 2 fails because  the sequence $(2, 2\omega + 2, -\omega + 1, 2, 2\omega + 2, -\omega + 1, 2, \dots ,2, 2\omega + 2, -\omega + 1, 2, \dots)$ leads to an infinite loop.\end{minipage} }\\
Method  23: &\\
Failing $b,b,\dots,b$ inputs: & $\{\omega + 3\}$ \\
\hline
Phase 1 (methods $14$): &
\checkmark, $\#\mathcal{Q} =21$ \\ 
Method  9: &\\
$b,b,\dots,b$ inputs: & \checkmark \\
Phase 2: & \checkmark , $r= 7$ \\
Method  15: &\\
$b,b,\dots,b$ inputs: & \checkmark \\
Phase 2: & \checkmark , $r= 7$ \\
Method  22: &\\
$b,b,\dots,b$ inputs: & \checkmark \\
\multicolumn{2}{l}{\begin{minipage}{\textwidth} Phase 2 fails because  the sequence $(2, 2\omega + 2, -\omega + 1, 2, 2\omega + 2, -\omega + 1, 2, \dots ,2, 2\omega + 2, -\omega + 1, 2, \dots)$ leads to an infinite loop.\end{minipage} }\\
Method  23: &\\
Failing $b,b,\dots,b$ inputs: & $\{\omega + 3\}$ \\
\hline
\end{tabular}

\end{exmp}




\begin{exmp}
\label{ex:compareAN}


\rule{0cm}{0cm}

\begin{tabular}{ll}
$\omega=  \frac{1}{2} i \, \sqrt{11} - \frac{3}{2} $  & $\beta= \omega = \frac{1}{2} i \, \sqrt{11} - \frac{3}{2} $\\
$m_\omega(t)=  t^{2} + 3 \, t + 5 $  & $m_\beta(x)=  x^{2} + 3 \, x + 5 $\\
Real conjugate of $\beta$ greater than 1:   &  no \\ \hline
\multicolumn{2}{l}{\begin{minipage}{\textwidth}\begin{dmath*}\A = \left\{0, 1, -1, \omega + 1, -\omega - 1, \omega + 2, -\omega - 2, \omega + 3, -\omega - 3\right\}  \end{dmath*}\end{minipage} }\\
$\#\A= $ 9 $ $ & $\A$ is minimal. \\
\multicolumn{2}{l}{\begin{minipage}{\textwidth}\begin{dmath*}\B =\A+\A \end{dmath*}\end{minipage} }\\[10pt]
\multicolumn{2}{l}{\begin{minipage}{\textwidth}$\A$ divided into congruence classes modulo $\beta$: \begin{dmath*} \left\{\left\{0\right\}, \left\{1, \omega + 1\right\}, \left\{-1, -\omega - 1\right\}, \left\{\omega + 2, -\omega - 3\right\}, \left\{-\omega - 2, \omega + 3\right\}\right\}  \end{dmath*}\end{minipage} }\\[10pt]
\multicolumn{2}{l}{\begin{minipage}{\textwidth}$\A$ divided into congruence classes modulo $\beta-1$: \begin{dmath*} \left\{\left\{0\right\}, \left\{1\right\}, \left\{-1\right\}, \left\{\omega + 1\right\}, \left\{-\omega - 1\right\}, \left\{\omega + 2\right\}, \left\{-\omega - 2\right\}, \left\{\omega + 3\right\}, \left\{-\omega - 3\right\}\right\}  \end{dmath*}\end{minipage} }\\
 & \\ \hline
 & \\
\end{tabular}

\begin{tabular}{ll}
Phase 1 (methods $14$): &
\checkmark, $\#\mathcal{Q} =19$ \\ 
Method  9: &\\
Failing $b,b,\dots,b$ inputs: & $\{2\omega + 2, \omega + 4, -\omega - 4, -2\omega - 2\}$ \\
Method  15: &\\
Failing $b,b,\dots,b$ inputs: & $\{2\omega + 2, -2\omega - 2\}$ \\
Method  22: &\\
Failing $b,b,\dots,b$ inputs: & $\{2\omega + 2, \omega + 4, -\omega - 4, -2\omega - 2\}$ \\
Method  23: &\\
Failing $b,b,\dots,b$ inputs: & $\{2\omega + 2, 2\omega + 4, \omega + 4, -\omega - 4, -2\omega - 4, -2\omega - 2\}$ \\
\hline
Phase 1 (methods $12, 16$): &
\checkmark, $\#\mathcal{Q} =11$ \\ 
Method  9: &\\
Failing $b,b,\dots,b$ inputs: & $\{-2\omega - 2\}$ \\
Method  15: &\\
$b,b,\dots,b$ inputs: & \checkmark \\
\multicolumn{2}{l}{\begin{minipage}{\textwidth} Phase 2 fails because  the sequence $(2, 2\omega + 2, 2\omega + 2, 2, 2\omega + 2, 2\omega + 2, \dots ,2, 2\omega + 2, 2\omega + 2, \dots)$ leads to an infinite loop.\end{minipage} }\\
Method  22: &\\
Failing $b,b,\dots,b$ inputs: & $\{2\omega + 2, -2\omega - 2\}$ \\
Method  23: &\\
Failing $b,b,\dots,b$ inputs: & $\{2\omega + 2, -2\omega - 2\}$ \\
\hline
Phase 1 (methods $13, 15$): &
\checkmark, $\#\mathcal{Q} =17$ \\ 
Method  9: &\\
Failing $b,b,\dots,b$ inputs: & $\{2\omega + 2, -2\omega - 2\}$ \\
Method  15: &\\
Failing $b,b,\dots,b$ inputs: & $\{2\omega + 2, -2\omega - 2\}$ \\
Method  22: &\\
Failing $b,b,\dots,b$ inputs: & $\{2\omega + 2, -2\omega - 2\}$ \\
Method  23: &\\
Failing $b,b,\dots,b$ inputs: & $\{2\omega + 2, 2\omega + 4, -2\omega - 4, -2\omega - 2\}$ \\
\hline
\end{tabular}

\end{exmp}




\begin{exmp}
\label{ex:compareAO}


\rule{0cm}{0cm}

\begin{tabular}{ll}
$\omega=  \frac{1}{2} i \, \sqrt{11} - \frac{3}{2} $  & $\beta= \omega = \frac{1}{2} i \, \sqrt{11} - \frac{3}{2} $\\
$m_\omega(t)=  t^{2} + 3 \, t + 5 $  & $m_\beta(x)=  x^{2} + 3 \, x + 5 $\\
Real conjugate of $\beta$ greater than 1:   &  no \\ \hline
\multicolumn{2}{l}{\begin{minipage}{\textwidth}\begin{dmath*}\A = \left\{0, \omega + 1, -\omega - 1, 1, -1, 2 \, \omega + 2, -2 \, \omega - 2, \omega + 2, -\omega - 2\right\}  \end{dmath*}\end{minipage} }\\
$\#\A= $ 9 $ $ & $\A$ is minimal. \\
\multicolumn{2}{l}{\begin{minipage}{\textwidth}\begin{dmath*}\B =\A+\A \end{dmath*}\end{minipage} }\\[10pt]
\multicolumn{2}{l}{\begin{minipage}{\textwidth}$\A$ divided into congruence classes modulo $\beta$: \begin{dmath*} \left\{\left\{0\right\}, \left\{\omega + 1, 1\right\}, \left\{-\omega - 1, -1\right\}, \left\{2 \, \omega + 2, \omega + 2\right\}, \left\{-2 \, \omega - 2, -\omega - 2\right\}\right\}  \end{dmath*}\end{minipage} }\\[10pt]
\multicolumn{2}{l}{\begin{minipage}{\textwidth}$\A$ divided into congruence classes modulo $\beta-1$: \begin{dmath*} \left\{\left\{0\right\}, \left\{\omega + 1\right\}, \left\{-\omega - 1\right\}, \left\{1\right\}, \left\{-1\right\}, \left\{2 \, \omega + 2\right\}, \left\{-2 \, \omega - 2\right\}, \left\{\omega + 2\right\}, \left\{-\omega - 2\right\}\right\}  \end{dmath*}\end{minipage} }\\
 & \\ \hline
 & \\
\end{tabular}

\begin{tabular}{ll}
Phase 1 (methods $12, 16$): &
\checkmark, $\#\mathcal{Q} =33$ \\ 
Method  9: &\\
Failing $b,b,\dots,b$ inputs: & $\{-3\omega - 4, 2\omega + 2, 2\omega + 3, \omega + 3, -2\omega - 4, -2\omega - 3, -2\omega - 2, -\omega - 3\}$ \\
Method  15: &\\
$b,b,\dots,b$ inputs: & \checkmark \\
\multicolumn{2}{l}{\begin{minipage}{\textwidth} Phase 2 fails because  the sequence $(0, 1, 2\omega + 1, -4\omega - 4, 4\omega + 4, 0, 1, 4\omega + 4, 0, 1, 4\omega + 4, \dots ,4\omega + 4, 0, 1, 4\omega + 4, \dots)$ leads to an infinite loop.\end{minipage} }\\
Method  22: &\\
Failing $b,b,\dots,b$ inputs: & $\{-3\omega - 3, 3\omega + 3\}$ \\
Method  23: &\\
Failing $b,b,\dots,b$ inputs: & $\{-3\omega - 4, 2\omega + 3, 2\omega + 4, \omega + 3, -2\omega - 4, -2\omega - 3, -\omega - 3, 3\omega + 4\}$ \\
\hline
Phase 1 (methods $13, 15$): &
\checkmark, $\#\mathcal{Q} =39$ \\ 
Method  9: &\\
Failing $b,b,\dots,b$ inputs: & $\{2\omega + 2, -2\omega - 4, -2\omega - 3, -2\omega - 2, -\omega - 3\}$ \\
Method  15: &\\
$b,b,\dots,b$ inputs: & \checkmark \\
\multicolumn{2}{l}{\begin{minipage}{\textwidth} Phase 2 fails because  the sequence $(0, 0, -4\omega - 4, 3\omega + 4, 0, 4\omega + 4, 0, 4\omega + 4, 0, 4\omega + 4, \dots ,0, 4\omega + 4, 0, 4\omega + 4, \dots)$ leads to an infinite loop.\end{minipage} }\\
Method  22: &\\
$b,b,\dots,b$ inputs: & \checkmark \\
\multicolumn{2}{l}{\begin{minipage}{\textwidth} Phase 2 fails because  the sequence $(0, 0, -2\omega - 1, 3\omega + 3, -4\omega - 4, \omega, -2\omega - 3, 4\omega + 4, -\omega, 2\omega + 3, -3\omega - 3, 4\omega + 4, 2\omega + 2, 4\omega + 4, 2\omega + 2, 4\omega + 4, 2\omega + 2, \dots ,4\omega + 4, 2\omega + 2, 4\omega + 4, 2\omega + 2, \dots)$ leads to an infinite loop.\end{minipage} }\\
Method  23: &\\
Failing $b,b,\dots,b$ inputs: & $\{-3\omega - 4, 2\omega + 3, 2\omega + 4, \omega + 3, -2\omega - 4, -2\omega - 3, -\omega - 3, 3\omega + 4\}$ \\
\hline
Phase 1 (methods $14$): &
\checkmark, $\#\mathcal{Q} =43$ \\ 
Method  9: &\\
Failing $b,b,\dots,b$ inputs: & $\{-3\omega - 3, 2\omega + 2, -2\omega - 3, -\omega - 3, 3\omega + 3\}$ \\
Method  15: &\\
$b,b,\dots,b$ inputs: & \checkmark \\
\multicolumn{2}{l}{\begin{minipage}{\textwidth} Phase 2 fails because  the sequence $(0, -1, -3\omega - 3, \omega, -3\omega - 4, -3\omega - 4, -3\omega - 3, 4\omega + 4, -3\omega - 4, 2\omega + 1, 0, -1, -3\omega - 3, \omega, \dots ,0, -1, -3\omega - 3, \omega, \dots)$ leads to an infinite loop.\end{minipage} }\\
Method  22: &\\
$b,b,\dots,b$ inputs: & \checkmark \\
\multicolumn{2}{l}{\begin{minipage}{\textwidth} Phase 2 fails because  the sequence $(0, 0, -4\omega - 4, 2\omega + 1, -\omega - 2, -3\omega - 4, 2\omega + 2, -\omega - 1, -\omega, 3\omega + 3, -2\omega - 4, 2\omega + 1, -\omega - 2, -3\omega - 4, 2\omega + 2, \dots ,2\omega + 1, -\omega - 2, -3\omega - 4, 2\omega + 2, \dots)$ leads to an infinite loop.\end{minipage} }\\
Method  23: &\\
Failing $b,b,\dots,b$ inputs: & $\{2\omega + 3, \omega + 3, -2\omega - 3, -\omega - 3\}$ \\
\hline
\end{tabular}

\end{exmp}




\begin{exmp}
\label{ex:compareAP}


\rule{0cm}{0cm}

\begin{tabular}{ll}
$\omega=  i $  & $\beta= \omega - 2 = i - 2 $\\
$m_\omega(t)=  t^{2} + 1 $  & $m_\beta(x)=  x^{2} + 4 \, x + 5 $\\
Real conjugate of $\beta$ greater than 1:   &  no \\ \hline
\multicolumn{2}{l}{\begin{minipage}{\textwidth}\begin{dmath*}\A = \left\{\omega + 2, \omega + 1, \omega, \omega - 1, 1, 0, -1, -\omega + 1, -\omega, -\omega - 1\right\}  \end{dmath*}\end{minipage} }\\
$\#\A= $ 10 $ $ & $\A$ is minimal. \\
\multicolumn{2}{l}{\begin{minipage}{\textwidth}\begin{dmath*}\B =\A+\A \end{dmath*}\end{minipage} }\\[10pt]
\multicolumn{2}{l}{\begin{minipage}{\textwidth}$\A$ divided into congruence classes modulo $\beta$: \begin{dmath*} \left\{\left\{\omega + 2, -1, -\omega + 1\right\}, \left\{\omega + 1, -\omega\right\}, \left\{\omega, -\omega - 1\right\}, \left\{\omega - 1, 1\right\}, \left\{0\right\}\right\}  \end{dmath*}\end{minipage} }\\[10pt]
\multicolumn{2}{l}{\begin{minipage}{\textwidth}$\A$ divided into congruence classes modulo $\beta-1$: \begin{dmath*} \left\{\left\{\omega + 2\right\}, \left\{\omega + 1\right\}, \left\{\omega\right\}, \left\{\omega - 1\right\}, \left\{1\right\}, \left\{0\right\}, \left\{-1\right\}, \left\{-\omega + 1\right\}, \left\{-\omega\right\}, \left\{-\omega - 1\right\}\right\}  \end{dmath*}\end{minipage} }\\
 & \\ \hline
 & \\
\end{tabular}

\begin{tabular}{ll}
Phase 1 (methods $14$): &
\checkmark, $\#\mathcal{Q} =19$ \\ 
Method  9: &\\
Failing $b,b,\dots,b$ inputs: & $\{2, -\omega - 2, -2\omega\}$ \\
Method  15: &\\
Failing $b,b,\dots,b$ inputs: & $\{-\omega - 2\}$ \\
Method  22: &\\
Failing $b,b,\dots,b$ inputs: & $\{2\omega - 2, 2\omega + 2, \omega + 3, -\omega - 2, -2\omega + 2\}$ \\
Method  23: &\\
Failing $b,b,\dots,b$ inputs: & $\{2\omega - 2, 2\omega + 2, \omega + 3, -\omega - 2, -2\omega + 2\}$ \\
\hline
Phase 1 (methods $12, 13, 15, 16$): &
\checkmark, $\#\mathcal{Q} =17$ \\ 
Method  9: &\\
Failing $b,b,\dots,b$ inputs: & $\{2, -\omega - 2, -2\omega\}$ \\
Method  15: &\\
Failing $b,b,\dots,b$ inputs: & $\{-\omega - 2\}$ \\
Method  22: &\\
Failing $b,b,\dots,b$ inputs: & $\{2\omega + 2, -\omega - 2\}$ \\
Method  23: &\\
Failing $b,b,\dots,b$ inputs: & $\{2\omega + 2, -\omega - 2\}$ \\
\hline
\end{tabular}

\end{exmp}




\begin{exmp}
\label{ex:compareAQ}


\rule{0cm}{0cm}

\begin{tabular}{ll}
$\omega=  i $  & $\beta= \omega - 2 = i - 2 $\\
$m_\omega(t)=  t^{2} + 1 $  & $m_\beta(x)=  x^{2} + 4 \, x + 5 $\\
Real conjugate of $\beta$ greater than 1:   &  no \\ \hline
\multicolumn{2}{l}{\begin{minipage}{\textwidth}\begin{dmath*}\A = \left\{\omega + 2, \omega + 1, \omega, 2, 1, 0, -1, -\omega + 1, -\omega, -\omega - 1\right\}  \end{dmath*}\end{minipage} }\\
$\#\A= $ 10 $ $ & $\A$ is minimal. \\
\multicolumn{2}{l}{\begin{minipage}{\textwidth}\begin{dmath*}\B =\A+\A \end{dmath*}\end{minipage} }\\[10pt]
\multicolumn{2}{l}{\begin{minipage}{\textwidth}$\A$ divided into congruence classes modulo $\beta$: \begin{dmath*} \left\{\left\{\omega + 2, -1, -\omega + 1\right\}, \left\{\omega + 1, -\omega\right\}, \left\{\omega, 2, -\omega - 1\right\}, \left\{1\right\}, \left\{0\right\}\right\}  \end{dmath*}\end{minipage} }\\[10pt]
\multicolumn{2}{l}{\begin{minipage}{\textwidth}$\A$ divided into congruence classes modulo $\beta-1$: \begin{dmath*} \left\{\left\{\omega + 2\right\}, \left\{\omega + 1\right\}, \left\{\omega\right\}, \left\{2\right\}, \left\{1\right\}, \left\{0\right\}, \left\{-1\right\}, \left\{-\omega + 1\right\}, \left\{-\omega\right\}, \left\{-\omega - 1\right\}\right\}  \end{dmath*}\end{minipage} }\\
 & \\ \hline
 & \\
\end{tabular}

\begin{tabular}{ll}
Phase 1 (methods $12, 13, 14, 15, 16$): &
\checkmark, $\#\mathcal{Q} =17$ \\ 
Method  9: &\\
$b,b,\dots,b$ inputs: & \checkmark \\
Phase 2: & \checkmark , $r= 3$ \\
Method  15: &\\
$b,b,\dots,b$ inputs: & \checkmark \\
Phase 2: & \checkmark , $r= 3$ \\
Method  22: &\\
$b,b,\dots,b$ inputs: & \checkmark \\
Phase 2: & \checkmark , $r= 3$ \\
Method  23: &\\
$b,b,\dots,b$ inputs: & \checkmark \\
Phase 2: & \checkmark , $r= 3$ \\
\hline
\end{tabular}

\end{exmp}






\newpage
\subsection*{Quadratic bases with integer alphabet}
The following examples show alphabets  divided into congruence classes modulo $\beta$ and $\beta-1$ for some numeration systems in Table~\ref{tab:resultsQuadrInt}.
\begin{exmp}
\label{ex:integerAA}


\rule{0cm}{0cm}

\begin{tabular}{ll}
$\omega=  \frac{1}{2} i \, \sqrt{11} + \frac{1}{2} $  & $\beta= -2 \, \omega + 1 = -i \, \sqrt{11} $\\
$m_\omega(t)=  t^{2} - t + 3 $  & $m_\beta(x)=  x^{2} + 11 $\\
Real conjugate of $\beta$ greater than 1:   &  no \\
$\#\A= $ 13 $ $ & $\A$ is not minimal. \\
\multicolumn{2}{l}{\begin{minipage}{\textwidth}\begin{dmath*}\A = \left\{-6, -5, -4, -3, -2, -1, 0, 1, 2, 3, 4, 5, 6\right\}  \end{dmath*}\end{minipage} }\\
\multicolumn{2}{l}{\begin{minipage}{\textwidth}$\A$ divided into congruence classes modulo $\beta$: \begin{dmath*} \left\{\left\{-6, 5\right\}, \left\{-5, 6\right\}, \left\{-4\right\}, \left\{-3\right\}, \left\{-2\right\}, \left\{-1\right\}, \left\{0\right\}, \left\{1\right\}, \left\{2\right\}, \left\{3\right\}, \left\{4\right\}\right\}  \end{dmath*}\end{minipage} }\\[10pt]
\multicolumn{2}{l}{\begin{minipage}{\textwidth}$\A$ divided into congruence classes modulo $\beta-1$: \begin{dmath*} \left\{\left\{-6, 0, 6\right\}, \left\{-5, 1\right\}, \left\{-4, 2\right\}, \left\{-3, 3\right\}, \left\{-2, 4\right\}, \left\{-1, 5\right\}\right\}  \end{dmath*}\end{minipage} }\\
 & \\ \hline
 & \\
Phase 1 (method  9): &
\checkmark, $\#\mathcal{Q} = $ 9 $ $ \\ 
$b,b,\dots,b$ inputs (method  15): & \checkmark \\
Phase 2 (method  15): & \checkmark , $r= 2$ \\
\end{tabular}

\end{exmp}




\begin{exmp}
\label{ex:integerAB}


\rule{0cm}{0cm}

\begin{tabular}{ll}
$\omega=  \frac{1}{2} i \, \sqrt{11} + \frac{1}{2} $  & $\beta= -2 \, \omega + 1 = -i \, \sqrt{11} $\\
$m_\omega(t)=  t^{2} - t + 3 $  & $m_\beta(x)=  x^{2} + 11 $\\
Real conjugate of $\beta$ greater than 1:   &  no \\
$\#\A= $ 12 $ $ & $\A$ is minimal. \\
\multicolumn{2}{l}{\begin{minipage}{\textwidth}\begin{dmath*}\A = \left\{-5, -4, -3, -2, -1, 0, 1, 2, 3, 4, 5, 6\right\}  \end{dmath*}\end{minipage} }\\
\multicolumn{2}{l}{\begin{minipage}{\textwidth}$\A$ divided into congruence classes modulo $\beta$: \begin{dmath*} \left\{\left\{-5, 6\right\}, \left\{-4\right\}, \left\{-3\right\}, \left\{-2\right\}, \left\{-1\right\}, \left\{0\right\}, \left\{1\right\}, \left\{2\right\}, \left\{3\right\}, \left\{4\right\}, \left\{5\right\}\right\}  \end{dmath*}\end{minipage} }\\[10pt]
\multicolumn{2}{l}{\begin{minipage}{\textwidth}$\A$ divided into congruence classes modulo $\beta-1$: \begin{dmath*} \left\{\left\{-5, 1\right\}, \left\{-4, 2\right\}, \left\{-3, 3\right\}, \left\{-2, 4\right\}, \left\{-1, 5\right\}, \left\{0, 6\right\}\right\}  \end{dmath*}\end{minipage} }\\
 & \\ \hline
 & \\
Phase 1 (method  9): &
\checkmark, $\#\mathcal{Q} = $ 9 $ $ \\ 
$b,b,\dots,b$ inputs (method  15): & \checkmark \\
Phase 2 (method  15): & \checkmark , $r= 4$ \\
\end{tabular}

\end{exmp}




\begin{exmp}
\label{ex:integerAC}


\rule{0cm}{0cm}

\begin{tabular}{ll}
$\omega=  \frac{1}{2} i \, \sqrt{7} - \frac{1}{2} $  & $\beta= -2 \, \omega - 1 = -i \, \sqrt{7} $\\
$m_\omega(t)=  t^{2} + t + 2 $  & $m_\beta(x)=  x^{2} + 7 $\\
Real conjugate of $\beta$ greater than 1:   &  no \\
$\#\A= $ 9 $ $ & $\A$ is not minimal. \\
\multicolumn{2}{l}{\begin{minipage}{\textwidth}\begin{dmath*}\A = \left\{-4, -3, -2, -1, 0, 1, 2, 3, 4\right\}  \end{dmath*}\end{minipage} }\\
\multicolumn{2}{l}{\begin{minipage}{\textwidth}$\A$ divided into congruence classes modulo $\beta$: \begin{dmath*} \left\{\left\{-4, 3\right\}, \left\{-3, 4\right\}, \left\{-2\right\}, \left\{-1\right\}, \left\{0\right\}, \left\{1\right\}, \left\{2\right\}\right\}  \end{dmath*}\end{minipage} }\\[10pt]
\multicolumn{2}{l}{\begin{minipage}{\textwidth}$\A$ divided into congruence classes modulo $\beta-1$: \begin{dmath*} \left\{\left\{-4, 0, 4\right\}, \left\{-3, 1\right\}, \left\{-2, 2\right\}, \left\{-1, 3\right\}\right\}  \end{dmath*}\end{minipage} }\\
 & \\ \hline
 & \\
Phase 1 (method  9): &
\checkmark, $\#\mathcal{Q} = $ 9 $ $ \\ 
$b,b,\dots,b$ inputs (method  15): & \checkmark \\
Phase 2 (method  15): & \checkmark , $r= 2$ \\
\end{tabular}

\end{exmp}




\begin{exmp}
\label{ex:integerAD}


\rule{0cm}{0cm}

\begin{tabular}{ll}
$\omega=  \frac{1}{2} i \, \sqrt{7} - \frac{1}{2} $  & $\beta= -2 \, \omega - 1 = -i \, \sqrt{7} $\\
$m_\omega(t)=  t^{2} + t + 2 $  & $m_\beta(x)=  x^{2} + 7 $\\
Real conjugate of $\beta$ greater than 1:   &  no \\
$\#\A= $ 8 $ $ & $\A$ is minimal. \\
\multicolumn{2}{l}{\begin{minipage}{\textwidth}\begin{dmath*}\A = \left\{-3, -2, -1, 0, 1, 2, 3, 4\right\}  \end{dmath*}\end{minipage} }\\
\multicolumn{2}{l}{\begin{minipage}{\textwidth}$\A$ divided into congruence classes modulo $\beta$: \begin{dmath*} \left\{\left\{-3, 4\right\}, \left\{-2\right\}, \left\{-1\right\}, \left\{0\right\}, \left\{1\right\}, \left\{2\right\}, \left\{3\right\}\right\}  \end{dmath*}\end{minipage} }\\[10pt]
\multicolumn{2}{l}{\begin{minipage}{\textwidth}$\A$ divided into congruence classes modulo $\beta-1$: \begin{dmath*} \left\{\left\{-3, 1\right\}, \left\{-2, 2\right\}, \left\{-1, 3\right\}, \left\{0, 4\right\}\right\}  \end{dmath*}\end{minipage} }\\
 & \\ \hline
 & \\
Phase 1 (method  9): &
\checkmark, $\#\mathcal{Q} = $ 9 $ $ \\ 
$b,b,\dots,b$ inputs (method  15): & \checkmark \\
Phase 2 (method  15): & \checkmark , $r= 4$ \\
\end{tabular}

\end{exmp}




\begin{exmp}
\label{ex:integerAE}


\rule{0cm}{0cm}

\begin{tabular}{ll}
$\omega=  \frac{1}{2} i \, \sqrt{3} + \frac{1}{2} $  & $\beta= -3 \, \omega + 2 = -\frac{3}{2} i \, \sqrt{3} + \frac{1}{2} $\\
$m_\omega(t)=  t^{2} - t + 1 $  & $m_\beta(x)=  x^{2} - x + 7 $\\
Real conjugate of $\beta$ greater than 1:   &  no \\
$\#\A= $ 11 $ $ & $\A$ is not minimal. \\
\multicolumn{2}{l}{\begin{minipage}{\textwidth}\begin{dmath*}\A = \left\{-5, -4, -3, -2, -1, 0, 1, 2, 3, 4, 5\right\}  \end{dmath*}\end{minipage} }\\
\multicolumn{2}{l}{\begin{minipage}{\textwidth}$\A$ divided into congruence classes modulo $\beta$: \begin{dmath*} \left\{\left\{-5, 2\right\}, \left\{-4, 3\right\}, \left\{-3, 4\right\}, \left\{-2, 5\right\}, \left\{-1\right\}, \left\{0\right\}, \left\{1\right\}\right\}  \end{dmath*}\end{minipage} }\\[10pt]
\multicolumn{2}{l}{\begin{minipage}{\textwidth}$\A$ divided into congruence classes modulo $\beta-1$: \begin{dmath*} \left\{\left\{-5, 2\right\}, \left\{-4, 3\right\}, \left\{-3, 4\right\}, \left\{-2, 5\right\}, \left\{-1\right\}, \left\{0\right\}, \left\{1\right\}\right\}  \end{dmath*}\end{minipage} }\\
 & \\ \hline
 & \\
Phase 1 (method  12): &
\checkmark, $\#\mathcal{Q} = $ 9 $ $ \\ 
$b,b,\dots,b$ inputs (method  15): & \checkmark \\
Phase 2 (method  15): & \checkmark , $r= 2$ \\
\end{tabular}

\end{exmp}




\begin{exmp}
\label{ex:integerAF}


\rule{0cm}{0cm}

\begin{tabular}{ll}
$\omega=  i \, \sqrt{3} $  & $\beta= \omega = i \, \sqrt{3} $\\
$m_\omega(t)=  t^{2} + 3 $  & $m_\beta(x)=  x^{2} + 3 $\\
Real conjugate of $\beta$ greater than 1:   &  no \\
$\#\A= $ 4 $ $ & $\A$ is minimal. \\
\multicolumn{2}{l}{\begin{minipage}{\textwidth}\begin{dmath*}\A = \left\{-1, 0, 1, 2\right\}  \end{dmath*}\end{minipage} }\\
\multicolumn{2}{l}{\begin{minipage}{\textwidth}$\A$ divided into congruence classes modulo $\beta$: \begin{dmath*} \left\{\left\{-1, 2\right\}, \left\{0\right\}, \left\{1\right\}\right\}  \end{dmath*}\end{minipage} }\\[10pt]
\multicolumn{2}{l}{\begin{minipage}{\textwidth}$\A$ divided into congruence classes modulo $\beta-1$: \begin{dmath*} \left\{\left\{-1\right\}, \left\{0\right\}, \left\{1\right\}, \left\{2\right\}\right\}  \end{dmath*}\end{minipage} }\\
 & \\ \hline
 & \\
Phase 1 (method  12): &
\checkmark, $\#\mathcal{Q} = $ 9 $ $ \\ 
$b,b,\dots,b$ inputs (method  15): & \checkmark \\
Phase 2 (method  15): & \checkmark , $r= 4$ \\
\end{tabular}

\end{exmp}




\begin{exmp}
\label{ex:integerAG}


\rule{0cm}{0cm}

\begin{tabular}{ll}
$\omega=  i \, \sqrt{2} $  & $\beta= -\omega = -i \, \sqrt{2} $\\
$m_\omega(t)=  t^{2} + 2 $  & $m_\beta(x)=  x^{2} + 2 $\\
Real conjugate of $\beta$ greater than 1:   &  ? \\
$\#\A= $ 3 $ $ & $\A$ is minimal. \\
\multicolumn{2}{l}{\begin{minipage}{\textwidth}\begin{dmath*}\A = \left\{-1, 0, 1\right\}  \end{dmath*}\end{minipage} }\\
\multicolumn{2}{l}{\begin{minipage}{\textwidth}$\A$ divided into congruence classes modulo $\beta$: \begin{dmath*} \left\{\left\{-1, 1\right\}, \left\{0\right\}\right\}  \end{dmath*}\end{minipage} }\\[10pt]
\multicolumn{2}{l}{\begin{minipage}{\textwidth}$\A$ divided into congruence classes modulo $\beta-1$: \begin{dmath*} \left\{\left\{-1\right\}, \left\{0\right\}, \left\{1\right\}\right\}  \end{dmath*}\end{minipage} }\\
 & \\ \hline
 & \\
Phase 1 (method  8): &
\checkmark, $\#\mathcal{Q} = $ 9 $ $ \\ 
$b,b,\dots,b$ inputs (method  15): & \checkmark \\
Phase 2 (method  15): & \checkmark , $r= 4$ \\
\end{tabular}

\end{exmp}




\begin{exmp}
\label{ex:integerAH}


\rule{0cm}{0cm}

\begin{tabular}{ll}
$\omega=  \sqrt{2} $  & $\beta= -\omega = -\sqrt{2} $\\
$m_\omega(t)=  t^{2} - 2 $  & $m_\beta(x)=  x^{2} - 2 $\\
Real conjugate of $\beta$ greater than 1:   &  yes \\
$\#\A= $ 3 $ $ & $\A$ is minimal. \\
\multicolumn{2}{l}{\begin{minipage}{\textwidth}\begin{dmath*}\A = \left\{0, 1, -1\right\}  \end{dmath*}\end{minipage} }\\
\multicolumn{2}{l}{\begin{minipage}{\textwidth}$\A$ divided into congruence classes modulo $\beta$: \begin{dmath*} \left\{\left\{0\right\}, \left\{1, -1\right\}\right\}  \end{dmath*}\end{minipage} }\\[10pt]
\multicolumn{2}{l}{\begin{minipage}{\textwidth}$\A$ divided into congruence classes modulo $\beta-1$: \begin{dmath*} \left\{\left\{0, 1, -1\right\}\right\}  \end{dmath*}\end{minipage} }\\
 & \\ \hline
 & \\
Phase 1 (method  9): &
\checkmark, $\#\mathcal{Q} = $ 9 $ $ \\ 
$b,b,\dots,b$ inputs (method  21): & \checkmark \\
Phase 2 (method  21): & \checkmark , $r= 4$ \\
\end{tabular}

\end{exmp}




\begin{exmp}
\label{ex:integerAI}


\rule{0cm}{0cm}

\begin{tabular}{ll}
$\omega=  \sqrt{3} - 1 $  & $\beta= -\omega - 1 = -\sqrt{3} $\\
$m_\omega(t)=  t^{2} + 2 \, t - 2 $  & $m_\beta(x)=  x^{2} - 3 $\\
Real conjugate of $\beta$ greater than 1:   &  yes \\
$\#\A= $ 4 $ $ & $\A$ is minimal. \\
\multicolumn{2}{l}{\begin{minipage}{\textwidth}\begin{dmath*}\A = \left\{0, 1, -1, 2\right\}  \end{dmath*}\end{minipage} }\\
\multicolumn{2}{l}{\begin{minipage}{\textwidth}$\A$ divided into congruence classes modulo $\beta$: \begin{dmath*} \left\{\left\{0\right\}, \left\{1\right\}, \left\{-1, 2\right\}\right\}  \end{dmath*}\end{minipage} }\\[10pt]
\multicolumn{2}{l}{\begin{minipage}{\textwidth}$\A$ divided into congruence classes modulo $\beta-1$: \begin{dmath*} \left\{\left\{0, 2\right\}, \left\{1, -1\right\}\right\}  \end{dmath*}\end{minipage} }\\
 & \\ \hline
 & \\
Phase 1 (method  6): &
\checkmark, $\#\mathcal{Q} = $ 9 $ $ \\ 
$b,b,\dots,b$ inputs (method  23): & \checkmark \\
Phase 2 (method  23): & \checkmark , $r= 5$ \\
\end{tabular}

\end{exmp}




\begin{exmp}
\label{ex:integerAJ}


\rule{0cm}{0cm}

\begin{tabular}{ll}
$\omega=  \frac{1}{2} \, \sqrt{5} + \frac{1}{2} $  & $\beta= -2 \, \omega + 1 = -\sqrt{5} $\\
$m_\omega(t)=  t^{2} - t - 1 $  & $m_\beta(x)=  x^{2} - 5 $\\
Real conjugate of $\beta$ greater than 1:   &  yes \\
$\#\A= $ 7 $ $ & $\A$ is not minimal. \\
\multicolumn{2}{l}{\begin{minipage}{\textwidth}\begin{dmath*}\A = \left\{-3, -2, -1, 0, 1, 2, 3\right\}  \end{dmath*}\end{minipage} }\\
\multicolumn{2}{l}{\begin{minipage}{\textwidth}$\A$ divided into congruence classes modulo $\beta$: \begin{dmath*} \left\{\left\{-3, 2\right\}, \left\{-2, 3\right\}, \left\{-1\right\}, \left\{0\right\}, \left\{1\right\}\right\}  \end{dmath*}\end{minipage} }\\[10pt]
\multicolumn{2}{l}{\begin{minipage}{\textwidth}$\A$ divided into congruence classes modulo $\beta-1$: \begin{dmath*} \left\{\left\{-3, -1, 1, 3\right\}, \left\{-2, 0, 2\right\}\right\}  \end{dmath*}\end{minipage} }\\
 & \\ \hline
 & \\
Phase 1 (method  9): &
\checkmark, $\#\mathcal{Q} = $ 9 $ $ \\ 
$b,b,\dots,b$ inputs (method  15): & \checkmark \\
Phase 2 (method  15): & \checkmark , $r= 2$ \\
\end{tabular}

\end{exmp}




\begin{exmp}
\label{ex:integerAK}


\rule{0cm}{0cm}

\begin{tabular}{ll}
$\omega=  \frac{1}{2} \, \sqrt{5} + \frac{1}{2} $  & $\beta= -2 \, \omega + 1 = -\sqrt{5} $\\
$m_\omega(t)=  t^{2} - t - 1 $  & $m_\beta(x)=  x^{2} - 5 $\\
Real conjugate of $\beta$ greater than 1:   &  yes \\
$\#\A= $ 6 $ $ & $\A$ is not minimal. \\
\multicolumn{2}{l}{\begin{minipage}{\textwidth}\begin{dmath*}\A = \left\{-2, -1, 0, 1, 2, 3\right\}  \end{dmath*}\end{minipage} }\\
\multicolumn{2}{l}{\begin{minipage}{\textwidth}$\A$ divided into congruence classes modulo $\beta$: \begin{dmath*} \left\{\left\{-2, 3\right\}, \left\{-1\right\}, \left\{0\right\}, \left\{1\right\}, \left\{2\right\}\right\}  \end{dmath*}\end{minipage} }\\[10pt]
\multicolumn{2}{l}{\begin{minipage}{\textwidth}$\A$ divided into congruence classes modulo $\beta-1$: \begin{dmath*} \left\{\left\{-2, 0, 2\right\}, \left\{-1, 1, 3\right\}\right\}  \end{dmath*}\end{minipage} }\\
 & \\ \hline
 & \\
Phase 1 (method  9): &
\checkmark, $\#\mathcal{Q} = $ 9 $ $ \\ 
$b,b,\dots,b$ inputs (method  15): & \checkmark \\
Phase 2 (method  15): & \checkmark , $r= 4$ \\
\end{tabular}

\end{exmp}




\begin{exmp}
\label{ex:integerAL}


\rule{0cm}{0cm}

\begin{tabular}{ll}
$\omega=  -\sqrt{5} + 1 $  & $\beta= \omega - 1 = -\sqrt{5} $\\
$m_\omega(t)=  t^{2} - 2 \, t - 4 $  & $m_\beta(x)=  x^{2} - 5 $\\
Real conjugate of $\beta$ greater than 1:   &  yes \\
$\#\A= $ 6 $ $ & $\A$ is minimal. \\
\multicolumn{2}{l}{\begin{minipage}{\textwidth}\begin{dmath*}\A = \left\{-2, -1, 0, 1, 2, 3\right\}  \end{dmath*}\end{minipage} }\\
\multicolumn{2}{l}{\begin{minipage}{\textwidth}$\A$ divided into congruence classes modulo $\beta$: \begin{dmath*} \left\{\left\{-2, 3\right\}, \left\{-1\right\}, \left\{0\right\}, \left\{1\right\}, \left\{2\right\}\right\}  \end{dmath*}\end{minipage} }\\[10pt]
\multicolumn{2}{l}{\begin{minipage}{\textwidth}$\A$ divided into congruence classes modulo $\beta-1$: \begin{dmath*} \left\{\left\{-2, 2\right\}, \left\{-1, 3\right\}, \left\{0\right\}, \left\{1\right\}\right\}  \end{dmath*}\end{minipage} }\\
 & \\ \hline
 & \\
Phase 1 (method  13): &
\checkmark, $\#\mathcal{Q} = $ 9 $ $ \\ 
$b,b,\dots,b$ inputs (method  23): & \checkmark \\
Phase 2 (method  23): & \checkmark , $r= 5$ \\
\end{tabular}

\end{exmp}




\begin{exmp}
\label{ex:integerAM}


\rule{0cm}{0cm}

\begin{tabular}{ll}
$\omega=  -\sqrt{6} + 1 $  & $\beta= \omega - 1 = -\sqrt{6} $\\
$m_\omega(t)=  t^{2} - 2 \, t - 5 $  & $m_\beta(x)=  x^{2} - 6 $\\
Real conjugate of $\beta$ greater than 1:   &  yes \\
$\#\A= $ 7 $ $ & $\A$ is minimal. \\
\multicolumn{2}{l}{\begin{minipage}{\textwidth}\begin{dmath*}\A = \left\{-2, -1, 0, 1, 2, 3, 4\right\}  \end{dmath*}\end{minipage} }\\
\multicolumn{2}{l}{\begin{minipage}{\textwidth}$\A$ divided into congruence classes modulo $\beta$: \begin{dmath*} \left\{\left\{-2, 4\right\}, \left\{-1\right\}, \left\{0\right\}, \left\{1\right\}, \left\{2\right\}, \left\{3\right\}\right\}  \end{dmath*}\end{minipage} }\\[10pt]
\multicolumn{2}{l}{\begin{minipage}{\textwidth}$\A$ divided into congruence classes modulo $\beta-1$: \begin{dmath*} \left\{\left\{-2, 3\right\}, \left\{-1, 4\right\}, \left\{0\right\}, \left\{1\right\}, \left\{2\right\}\right\}  \end{dmath*}\end{minipage} }\\
 & \\ \hline
 & \\
Phase 1 (method  13): &
\checkmark, $\#\mathcal{Q} = $ 9 $ $ \\ 
$b,b,\dots,b$ inputs (method  23): & \checkmark \\
Phase 2 (method  23): & \checkmark , $r= 5$ \\
\end{tabular}

\end{exmp}




\begin{exmp}
\label{ex:integerAN}


\rule{0cm}{0cm}

\begin{tabular}{ll}
$\omega=  \sqrt{6} - 1 $  & $\beta= \omega + 1 = \sqrt{6} $\\
$m_\omega(t)=  t^{2} + 2 \, t - 5 $  & $m_\beta(x)=  x^{2} - 6 $\\
Real conjugate of $\beta$ greater than 1:   &  yes \\
$\#\A= $ 7 $ $ & $\A$ is minimal. \\
\multicolumn{2}{l}{\begin{minipage}{\textwidth}\begin{dmath*}\A = \left\{-2, -1, 0, 1, 2, 3, 4\right\}  \end{dmath*}\end{minipage} }\\
\multicolumn{2}{l}{\begin{minipage}{\textwidth}$\A$ divided into congruence classes modulo $\beta$: \begin{dmath*} \left\{\left\{-2, 4\right\}, \left\{-1\right\}, \left\{0\right\}, \left\{1\right\}, \left\{2\right\}, \left\{3\right\}\right\}  \end{dmath*}\end{minipage} }\\[10pt]
\multicolumn{2}{l}{\begin{minipage}{\textwidth}$\A$ divided into congruence classes modulo $\beta-1$: \begin{dmath*} \left\{\left\{-2, 3\right\}, \left\{-1, 4\right\}, \left\{0\right\}, \left\{1\right\}, \left\{2\right\}\right\}  \end{dmath*}\end{minipage} }\\
 & \\ \hline
 & \\
Phase 1 (method  13): &
\checkmark, $\#\mathcal{Q} = $ 9 $ $ \\ 
$b,b,\dots,b$ inputs (method  23): & \checkmark \\
Phase 2 (method  23): & \checkmark , $r= 4$ \\
\end{tabular}

\end{exmp}




\begin{exmp}
\label{ex:integerAO}


\rule{0cm}{0cm}

\begin{tabular}{ll}
$\omega=  -\sqrt{7} + 2 $  & $\beta= -\omega + 2 = \sqrt{7} $\\
$m_\omega(t)=  t^{2} - 4 \, t - 3 $  & $m_\beta(x)=  x^{2} - 7 $\\
Real conjugate of $\beta$ greater than 1:   &  yes \\
$\#\A= $ 8 $ $ & $\A$ is minimal. \\
\multicolumn{2}{l}{\begin{minipage}{\textwidth}\begin{dmath*}\A = \left\{-3, -2, -1, 0, 1, 2, 3, 4\right\}  \end{dmath*}\end{minipage} }\\
\multicolumn{2}{l}{\begin{minipage}{\textwidth}$\A$ divided into congruence classes modulo $\beta$: \begin{dmath*} \left\{\left\{-3, 4\right\}, \left\{-2\right\}, \left\{-1\right\}, \left\{0\right\}, \left\{1\right\}, \left\{2\right\}, \left\{3\right\}\right\}  \end{dmath*}\end{minipage} }\\[10pt]
\multicolumn{2}{l}{\begin{minipage}{\textwidth}$\A$ divided into congruence classes modulo $\beta-1$: \begin{dmath*} \left\{\left\{-3, 3\right\}, \left\{-2, 4\right\}, \left\{-1\right\}, \left\{0\right\}, \left\{1\right\}, \left\{2\right\}\right\}  \end{dmath*}\end{minipage} }\\
 & \\ \hline
 & \\
Phase 1 (method  13): &
\checkmark, $\#\mathcal{Q} = $ 9 $ $ \\ 
$b,b,\dots,b$ inputs (method  23): & \checkmark \\
Phase 2 (method  23): & \checkmark , $r= 4$ \\
\end{tabular}

\end{exmp}




\begin{exmp}
\label{ex:integerAP}


\rule{0cm}{0cm}

\begin{tabular}{ll}
$\omega=  \frac{1}{2} \, \sqrt{13} + \frac{1}{2} $  & $\beta= -2 \, \omega + 1 = -\sqrt{13} $\\
$m_\omega(t)=  t^{2} - t - 3 $  & $m_\beta(x)=  x^{2} - 13 $\\
Real conjugate of $\beta$ greater than 1:   &  yes \\
$\#\A= $ 15 $ $ & $\A$ is not minimal. \\
\multicolumn{2}{l}{\begin{minipage}{\textwidth}\begin{dmath*}\A = \left\{-7, -6, -5, -4, -3, -2, -1, 0, 1, 2, 3, 4, 5, 6, 7\right\}  \end{dmath*}\end{minipage} }\\
\multicolumn{2}{l}{\begin{minipage}{\textwidth}$\A$ divided into congruence classes modulo $\beta$: \begin{dmath*} \left\{\left\{-7, 6\right\}, \left\{-6, 7\right\}, \left\{-5\right\}, \left\{-4\right\}, \left\{-3\right\}, \left\{-2\right\}, \left\{-1\right\}, \left\{0\right\}, \left\{1\right\}, \left\{2\right\}, \left\{3\right\}, \left\{4\right\}, \left\{5\right\}\right\}  \end{dmath*}\end{minipage} }\\[10pt]
\multicolumn{2}{l}{\begin{minipage}{\textwidth}$\A$ divided into congruence classes modulo $\beta-1$: \begin{dmath*} \left\{\left\{-7, -1, 5\right\}, \left\{-6, 0, 6\right\}, \left\{-5, 1, 7\right\}, \left\{-4, 2\right\}, \left\{-3, 3\right\}, \left\{-2, 4\right\}\right\}  \end{dmath*}\end{minipage} }\\
 & \\ \hline
 & \\
Phase 1 (method  9): &
\checkmark, $\#\mathcal{Q} = $ 9 $ $ \\ 
$b,b,\dots,b$ inputs (method  15): & \checkmark \\
Phase 2 (method  15): & \checkmark , $r= 2$ \\
\end{tabular}

\end{exmp}




\begin{exmp}
\label{ex:integerAQ}


\rule{0cm}{0cm}

\begin{tabular}{ll}
$\omega=  \frac{1}{2} \, \sqrt{13} + \frac{1}{2} $  & $\beta= -2 \, \omega + 1 = -\sqrt{13} $\\
$m_\omega(t)=  t^{2} - t - 3 $  & $m_\beta(x)=  x^{2} - 13 $\\
Real conjugate of $\beta$ greater than 1:   &  yes \\
$\#\A= $ 14 $ $ & $\A$ is not minimal. \\
\multicolumn{2}{l}{\begin{minipage}{\textwidth}\begin{dmath*}\A = \left\{-6, -5, -4, -3, -2, -1, 0, 1, 2, 3, 4, 5, 6, 7\right\}  \end{dmath*}\end{minipage} }\\
\multicolumn{2}{l}{\begin{minipage}{\textwidth}$\A$ divided into congruence classes modulo $\beta$: \begin{dmath*} \left\{\left\{-6, 7\right\}, \left\{-5\right\}, \left\{-4\right\}, \left\{-3\right\}, \left\{-2\right\}, \left\{-1\right\}, \left\{0\right\}, \left\{1\right\}, \left\{2\right\}, \left\{3\right\}, \left\{4\right\}, \left\{5\right\}, \left\{6\right\}\right\}  \end{dmath*}\end{minipage} }\\[10pt]
\multicolumn{2}{l}{\begin{minipage}{\textwidth}$\A$ divided into congruence classes modulo $\beta-1$: \begin{dmath*} \left\{\left\{-6, 0, 6\right\}, \left\{-5, 1, 7\right\}, \left\{-4, 2\right\}, \left\{-3, 3\right\}, \left\{-2, 4\right\}, \left\{-1, 5\right\}\right\}  \end{dmath*}\end{minipage} }\\
 & \\ \hline
 & \\
Phase 1 (method  9): &
\checkmark, $\#\mathcal{Q} = $ 9 $ $ \\ 
$b,b,\dots,b$ inputs (method  15): & \checkmark \\
Phase 2 (method  15): & \checkmark , $r= 4$ \\
\end{tabular}

\end{exmp}




\begin{exmp}
\label{ex:integerAR}


\rule{0cm}{0cm}

\begin{tabular}{ll}
$\omega=  -\frac{1}{2} \, \sqrt{17} + \frac{3}{2} $  & $\beta= 2 \, \omega - 3 = -\sqrt{17} $\\
$m_\omega(t)=  t^{2} - 3 \, t - 2 $  & $m_\beta(x)=  x^{2} - 17 $\\
Real conjugate of $\beta$ greater than 1:   &  yes \\
$\#\A= $ 18 $ $ & $\A$ is minimal. \\
\multicolumn{2}{l}{\begin{minipage}{\textwidth}\begin{dmath*}\A = \left\{-8, -7, -6, -5, -4, -3, -2, -1, 0, 1, 2, 3, 4, 5, 6, 7, 8, 9\right\}  \end{dmath*}\end{minipage} }\\
\multicolumn{2}{l}{\begin{minipage}{\textwidth}$\A$ divided into congruence classes modulo $\beta$: \begin{dmath*} \left\{\left\{-8, 9\right\}, \left\{-7\right\}, \left\{-6\right\}, \left\{-5\right\}, \left\{-4\right\}, \left\{-3\right\}, \left\{-2\right\}, \left\{-1\right\}, \left\{0\right\}, \left\{1\right\}, \left\{2\right\}, \left\{3\right\}, \left\{4\right\}, \left\{5\right\}, \left\{6\right\}, \left\{7\right\}, \left\{8\right\}\right\}  \end{dmath*}\end{minipage} }\\[10pt]
\multicolumn{2}{l}{\begin{minipage}{\textwidth}$\A$ divided into congruence classes modulo $\beta-1$: \begin{dmath*} \left\{\left\{-8, 0, 8\right\}, \left\{-7, 1, 9\right\}, \left\{-6, 2\right\}, \left\{-5, 3\right\}, \left\{-4, 4\right\}, \left\{-3, 5\right\}, \left\{-2, 6\right\}, \left\{-1, 7\right\}\right\}  \end{dmath*}\end{minipage} }\\
 & \\ \hline
 & \\
Phase 1 (method  6): &
\checkmark, $\#\mathcal{Q} = $ 9 $ $ \\ 
$b,b,\dots,b$ inputs (method  22): & \checkmark \\
Phase 2 (method  22): & \checkmark , $r= 4$ \\
\end{tabular}

\end{exmp}




\begin{exmp}
\label{ex:integerAS}


\rule{0cm}{0cm}

\begin{tabular}{ll}
$\omega=  -\frac{1}{2} \, \sqrt{21} + \frac{3}{2} $  & $\beta= 2 \, \omega - 3 = -\sqrt{21} $\\
$m_\omega(t)=  t^{2} - 3 \, t - 3 $  & $m_\beta(x)=  x^{2} - 21 $\\
Real conjugate of $\beta$ greater than 1:   &  yes \\
$\#\A= $ 22 $ $ & $\A$ is minimal. \\
\multicolumn{2}{l}{\begin{minipage}{\textwidth}\begin{dmath*}\A = \left\{-10, -9, -8, -7, -6, -5, -4, -3, -2, -1, 0, 1, 2, 3, 4, 5, 6, 7, 8, 9, 10, 11\right\}  \end{dmath*}\end{minipage} }\\
\multicolumn{2}{l}{\begin{minipage}{\textwidth}$\A$ divided into congruence classes modulo $\beta$: \begin{dmath*} \left\{\left\{-10, 11\right\}, \left\{-9\right\}, \left\{-8\right\}, \left\{-7\right\}, \left\{-6\right\}, \left\{-5\right\}, \left\{-4\right\}, \left\{-3\right\}, \left\{-2\right\}, \left\{-1\right\}, \left\{0\right\}, \left\{1\right\}, \left\{2\right\}, \left\{3\right\}, \left\{4\right\}, \left\{5\right\}, \left\{6\right\}, \left\{7\right\}, \left\{8\right\}, \left\{9\right\}, \left\{10\right\}\right\}  \end{dmath*}\end{minipage} }\\[10pt]
\multicolumn{2}{l}{\begin{minipage}{\textwidth}$\A$ divided into congruence classes modulo $\beta-1$: \begin{dmath*} \left\{\left\{-10, 0, 10\right\}, \left\{-9, 1, 11\right\}, \left\{-8, 2\right\}, \left\{-7, 3\right\}, \left\{-6, 4\right\}, \left\{-5, 5\right\}, \left\{-4, 6\right\}, \left\{-3, 7\right\}, \left\{-2, 8\right\}, \left\{-1, 9\right\}\right\}  \end{dmath*}\end{minipage} }\\
 & \\ \hline
 & \\
Phase 1 (method  9): &
\checkmark, $\#\mathcal{Q} = $ 9 $ $ \\ 
$b,b,\dots,b$ inputs (method  9): & \checkmark \\
Phase 2 (method  9): & \checkmark , $r= 4$ \\
\end{tabular}

\end{exmp}







\subsection*{Killed examples}
The computation of a weight function for the following numeration systems was killed because of memory limits.
\begin{exmp}
\label{ex:killAB}

\rule{0cm}{0cm}

\begin{tabular}{ll}
$\omega=  -\frac{1}{2} \, \sqrt{37} + \frac{5}{2} $  & $\beta= -\omega - 3 = \frac{1}{2} \, \sqrt{37} - \frac{11}{2} $\\
$m_\omega(t)=  t^{2} - 5 \, t - 3 $  & $m_\beta(x)=  x^{2} + 11 \, x + 21 $\\
Real conjugate of $\beta$ greater than 1:   &  no \\
$\#\A= $ 33 $ $ & $\A$ is minimal. \\
\multicolumn{2}{l}{\begin{minipage}{\textwidth}\begin{dmath*}\A = \left\{0, 1, -1, \omega + 1, -\omega - 1, -\omega + 1, \omega - 1, \omega, -\omega, 2 \, \omega + 2, -2 \, \omega - 2, \omega + 2, -\omega - 2, -2 \, \omega + 2, 2 \, \omega - 2, 2 \, \omega + 1, -2 \, \omega - 1, -2 \, \omega + 1, 2 \, \omega - 1, 2 \, \omega, -2 \, \omega, -2 \, \omega + 3, 2 \, \omega - 3, -3 \, \omega + 3, 3 \, \omega - 3, -3 \, \omega + 2, 3 \, \omega - 2, -3 \, \omega + 1, 3 \, \omega - 1, 3 \, \omega, -3 \, \omega, -3 \, \omega + 4, 3 \, \omega - 4\right\}  \end{dmath*}\end{minipage} }\\
 & \\
Phase 1 (method  9): &
\checkmark, $\#\mathcal{Q} = $ 17 $ $ \\ 
$b,b,\dots,b$ inputs (method  2c): & \checkmark, maximal length of window: $ 3 $ \\
\multicolumn{2}{l}{\begin{minipage}{\textwidth} Computation of Phase 2 (method  2c) was killed when the length of window 5 was being proccessed. Numbers of saved combinations for each finished length are: (0, 12399, 682670, 2721482)\end{minipage} }\\
\end{tabular}

\end{exmp}



\begin{exmp}
\label{ex:killAA}

\rule{0cm}{0cm}

\begin{tabular}{ll}
$\omega=  -\frac{1}{2} \, \sqrt{29} + \frac{3}{2} $  & $\beta= 3 \, \omega + 1 = -\frac{3}{2} \, \sqrt{29} + \frac{11}{2} $\\
$m_\omega(t)=  t^{2} - 3 \, t - 5 $  & $m_\beta(x)=  x^{2} - 11 \, x - 35 $\\
Real conjugate of $\beta$ greater than 1:   &  yes \\
$\#\A= $ 49 $ $ & $\A$ is not minimal. \\
\multicolumn{2}{l}{\begin{minipage}{\textwidth}\begin{dmath*}\A = \left\{0, 1, -1, \omega + 1, -\omega - 1, -\omega + 1, \omega - 1, \omega, -\omega, 2 \, \omega + 2, -2 \, \omega - 2, \omega + 2, -\omega - 2, 2, -2, 3 \, \omega + 3, -3 \, \omega - 3, 2 \, \omega + 3, -2 \, \omega - 3, \omega + 3, -\omega - 3, 4 \, \omega + 4, -4 \, \omega - 4, 3 \, \omega + 4, -3 \, \omega - 4, 2 \, \omega + 4, -2 \, \omega - 4, 5 \, \omega + 5, -5 \, \omega - 5, 4 \, \omega + 5, -4 \, \omega - 5, 3 \, \omega + 5, -3 \, \omega - 5, 6 \, \omega + 6, -6 \, \omega - 6, 5 \, \omega + 6, -5 \, \omega - 6, 4 \, \omega + 6, -4 \, \omega - 6, 7 \, \omega + 7, -7 \, \omega - 7, 6 \, \omega + 7, -6 \, \omega - 7, 5 \, \omega + 7, -5 \, \omega - 7, -2 \, \omega + 5, 2 \, \omega - 5, 4 \, \omega + 7, -4 \, \omega - 7\right\}  \end{dmath*}\end{minipage} }\\
 & \\
Phase 1 (method  9): &
\checkmark, $\#\mathcal{Q} = $ 46 $ $ \\ 
$b,b,\dots,b$ inputs (method  21): & \checkmark, maximal length of window: $ 5 $ \\
\multicolumn{2}{l}{\begin{minipage}{\textwidth} Computation of Phase 2 (method  21) was killed. % when the length of window 1 was being proccessed. Numbers of saved combinations for each finished length are: ()
\end{minipage} }\\
\end{tabular}

\end{exmp}




\newpage
\begin{exmp}
\label{ex:killAC}


\rule{0cm}{0cm}

\begin{tabular}{ll}
$\omega=  {\left(\frac{1}{9} \, \sqrt{19} \sqrt{3} + 1\right)}^{\frac{1}{3}} + \frac{2}{3 \, {\left(\frac{1}{9} \, \sqrt{19} \sqrt{3} + 1\right)}^{\frac{1}{3}}} $  & $\beta= -2 \, \omega^{2} + \omega + 2 = {\left(\sqrt{57} - \frac{197}{27}\right)}^{\frac{1}{3}} - \frac{14}{9 \, {\left(\sqrt{57} - \frac{197}{27}\right)}^{\frac{1}{3}}} - \frac{2}{3} $\\
$m_\omega(t)=  t^{3} - 2 \, t - 2 $  & $m_\beta(x)=  x^{3} + 2 \, x^{2} + 6 \, x + 18 $\\
Real conjugate of $\beta$ greater than 1:   &  no \\
$\#\A= $ 31 $ $ & $\A$ is not minimal. \\
\multicolumn{2}{l}{\begin{minipage}{\textwidth}\begin{dmath*}\A = \left\{0, 1, -1, \omega^{2} + \omega + 1, -\omega^{2} - \omega - 1, \omega + 1, -\omega - 1, -\omega^{2} + \omega + 1, \omega^{2} - \omega - 1, \omega^{2} + 1, -\omega^{2} - 1, -\omega^{2} + 1, \omega^{2} - 1, \omega^{2} - \omega + 1, -\omega^{2} + \omega - 1, -\omega + 1, \omega - 1, -\omega^{2} - \omega + 1, \omega^{2} + \omega - 1, \omega^{2} + \omega, -\omega^{2} - \omega, \omega, -\omega, \omega^{2} + 2, -\omega^{2} - 2, 2, -2, -\omega^{2} + \omega, \omega^{2} - \omega, \omega^{2}, -\omega^{2}\right\}  \end{dmath*}\end{minipage} }\\
 & \\
Phase 1 (method  9): &
\checkmark, $\#\mathcal{Q} = $ 83 $ $ \\ 
$b,b,\dots,b$ inputs (method  21): & \checkmark, maximal length of window: $ 5 $ \\
\multicolumn{2}{l}{\begin{minipage}{\textwidth} Computation of Phase 2 (method  21) was killed when the length of window 4 was being proccessed. Numbers of saved combinations for each finished length are: (0, 71, 1887261)\end{minipage} }\\
\end{tabular}

\end{exmp}


\begin{exmp}
\label{ex:killAD}


\rule{0cm}{0cm}

\begin{tabular}{ll}
$\omega=  {\left(\frac{1}{9} \, \sqrt{29} \sqrt{3} + \frac{28}{27}\right)}^{\frac{1}{3}} + \frac{1}{9 \, {\left(\frac{1}{9} \, \sqrt{29} \sqrt{3} + \frac{28}{27}\right)}^{\frac{1}{3}}} + \frac{1}{3} $ \\
\multicolumn{2}{l}{ $\beta= -\omega^{2} + \omega - 1 = {\left(\frac{2}{9} \, \sqrt{29} \sqrt{3} - 2\right)}^{\frac{1}{3}} - \frac{2}{3 \, {\left(\frac{2}{9} \, \sqrt{29} \sqrt{3} - 2\right)}^{\frac{1}{3}}} - 1 $}\\
$m_\omega(t)=  t^{3} - t^{2} - 2 $  & $m_\beta(x)=  x^{3} + 3 \, x^{2} + 5 \, x + 7 $\\
Real conjugate of $\beta$ greater than 1:   &  no \\
$\#\A= $ 16 $ $ & $\A$ is minimal. \\
\multicolumn{2}{l}{\begin{minipage}{\textwidth}\begin{dmath*}\A = \left\{0, 1, -1, \omega^{2} + \omega + 1, -\omega^{2} - \omega - 1, \omega + 1, \omega^{2} + 1, -\omega^{2} - 1, -\omega^{2} + 1, \omega^{2} + \omega, \omega, -\omega, -\omega^{2} + \omega, \omega^{2} - \omega, \omega^{2}, -\omega^{2}\right\}  \end{dmath*}\end{minipage} }\\
 & \\
Phase 1 (method  9): &
\checkmark, $\#\mathcal{Q} = $ 99 $ $ \\ 
$b,b,\dots,b$ inputs (method  21): & \checkmark, maximal length of window: $ 6 $ \\
\multicolumn{2}{l}{\begin{minipage}{\textwidth} Computation of Phase 2 (method  21) was killed when the length of window 4 was being proccessed. Numbers of saved combinations for each finished length are: (73, 5329, 315494)\end{minipage} }\\
\end{tabular}

\end{exmp}


