We have attempted to find a parallel addition algorithm for more than 5000 different numeration systems. Including various methods in Phase~1 and 2, the implementation of the extending window method has been launched over 7000 times. 

Most of the bases were given by polynomial combinations of $\omega$ with coefficients in a limited range, where $\omega$ was generated as a root of a monic polynomial with bounded integer coefficients. Mainly, $\omega$ are  quadratic, but cubic ones have been also tested. Alphabets for these bases were constructed automatically. 

Besides that, the extending window method was run for selected numeration systems in order to compare different methods in Phase~1 and 2, see  Section~\ref{sec:compareMethods}.

Processing such a number of inputs is enabled by the developed tools of non-convergence of Phase~2. The control of $bb\dots b$ inputs often reveals non-convergence very quickly, while an infinite loop in Phase~2 is avoided due to check of directed cycles in a Rauzy graph. This condition seems to be really strong -- so far we have only four examples which were killed because of lack of memory before non-convergence was revealed or a weight function found. We discuss them later.

Altogether, a parallel addition algorithm was found for about 140 cases. A possible explanation of such low success rate is the automatic construction of alphabets -- there may exist  alphabets which are better spread or more symmetric. Another reason is that we did not have enough computational power for testing with various methods in Phase~1 and 2:  different methods are more suitable for different numeration systems, as we will see in Section~\ref{sec:compareMethods}.
Nevertheless, we provide various successful examples of numeration systems with integer and non-integer alphabets. Some of them are listed in Section~\ref{sec:resultsExamples}.

In Section~\ref{sec:gspreadsheet}, we describe Google spreadsheet \href{https://docs.google.com/spreadsheets/d/1TnhrHdefHfHa0WSeVs4q6XVj3epjPlPlnoekE0E1xeM/edit?usp=sharing}{ParallelAddition\_results} which contains all tested inputs.

\section{Comparing different choices in Phase 1 and 2}
\label{sec:compareMethods}

As we mentioned, both phases of the extending window method are significantly dependent on the way of extending $\Q_k$ to $\Q{k+1}$ , respectively choice of $\Qwo{k}$ from $\Qwo{(k-1)}$. Various methods were designed and implemented. They were all tested on the numeration systems which are listed in Table~\ref{tab:resultsPhaseOne}. Parallel addition algorithms were found manually \cite{milena} for some of them, for instance Eisenstein\_1--block\_complex, Penney\_1--block\_complex, Penney\_2--block\_integer or Quadratic+1+4+5\_complex2. 
%\afterpage{
\clearpage   % To flush out all floats, might not be what you want
\begin{landscape}
	\begin{table}[h]
		\begin{center}
		\begin{tabular}{l|c c c c|ccc|c c  c  c  c  c }
\multirow{2}{*}{Name} & \multirow{2}{*}{$\omega$} & \multirow{2}{*}{$m_\omega$} & \multirow{2}{*}{$\beta$} & \multirow{2}{*}{$m_\beta$} & \multirow{2}{*}{conj.} & \multirow{2}{*}{$\#\A$} & \multirow{2}{*}{min.} & \multicolumn{5}{c}{$\#\Q$} \\
  &  &  &  &  &  &  &  & 6 & 8 & 9 & 10 & 11 \\ \hline
Eisenstein\_1--block\_complex & $ \frac{1}{2} i \, \sqrt{3} - \frac{1}{2} $ & $ t^{2} + t + 1 $ & $ \omega - 1 $ & $ x^{2} + 3 \, x + 3 $ & yes & $ 7 $ & yes & 19 & 19 & 19 & 19 & 19 \\
Eisenstein\_1--block\_integer & $ \frac{1}{2} i \, \sqrt{3} - \frac{1}{2} $ & $ t^{2} + t + 1 $ & $ \omega - 1 $ & $ x^{2} + 3 \, x + 3 $ & yes & $ 7 $ & yes & 113 & 53 & 52 & 52 & 53 \\
Eisenstein\_2--block\_complex & $ \frac{1}{2} i \, \sqrt{3} - \frac{1}{2} $ & $ t^{2} + t + 1 $ & $ -3 \, \omega $ & $ x^{2} - 3 \, x + 9 $ & yes & $ 14 $ & no & 17 & 17 & 17 & 17 & 17 \\
Eisenstein\_2--block\_integer & $ \frac{1}{2} i \, \sqrt{3} - \frac{1}{2} $ & $ t^{2} + t + 1 $ & $ -3 \, \omega $ & $ x^{2} - 3 \, x + 9 $ & yes & $ 16 $ & no & 26 & 26 & 26 & 26 & 26 \\
Penney\_1--block\_complex & $ i - 1 $ & $ t^{2} + 2 \, t + 2 $ & $ \omega $ & $ x^{2} + 2 \, x + 2 $ & yes & $ 5 $ & yes & 45 & 45 & 45 & 45 & 45 \\
Penney\_1--block\_integer & $ i $ & $ t^{2} + 1 $ & $ \omega - 1 $ & $ x^{2} + 2 \, x + 2 $ & yes & $ 5 $ & yes & 97 & 27 & 27 & 27 & 27 \\
Penney\_2--block\_integer & $ i $ & $ t^{2} + 1 $ & $ -2 \, \omega $ & $ x^{2} + 4 $ & yes & $ 9 $ & no & 27 & 27 & 27 & 27 & 27 \\
Quadratic+1+0--17\_integer & $ -\frac{1}{2} \, \sqrt{17} + \frac{3}{2} $ & $ t^{2} - 3 \, t - 2 $ & $ 2 \, \omega - 3 $ & $ x^{2} - 17 $ & yes & $ 19 $ & no & 9 & 9 & 9 & 9 & 9 \\
Quadratic+1+0--17\_integer\_smaller & $ -\frac{1}{2} \, \sqrt{17} + \frac{3}{2} $ & $ t^{2} - 3 \, t - 2 $ & $ 2 \, \omega - 3 $ & $ x^{2} - 17 $ & yes & $ 18 $ & yes & 9 & 9 & 9 & 9 & 9 \\
Quadratic+1+0--21\_integer & $ -\frac{1}{2} \, \sqrt{21} + \frac{3}{2} $ & $ t^{2} - 3 \, t - 3 $ & $ 2 \, \omega - 3 $ & $ x^{2} - 21 $ & yes & $ 23 $ & no & 9 & 9 & 9 & 9 & 9 \\
Quadratic+1+0--21\_integer\_smaller & $ -\frac{1}{2} \, \sqrt{21} + \frac{3}{2} $ & $ t^{2} - 3 \, t - 3 $ & $ 2 \, \omega - 3 $ & $ x^{2} - 21 $ & yes & $ 22 $ & yes & 9 & 9 & 9 & 9 & 9 \\
Quadratic+1+2+3\_complex\_smaller & $ i \, \sqrt{2} - 1 $ & $ t^{2} + 2 \, t + 3 $ & $ -\omega - 2 $ & $ x^{2} + 2 \, x + 3 $ & yes & $ 6 $ & yes & 27 & 27 & 26 & 26 & 27 \\
Quadratic+1+3+4\_complex & $ \frac{1}{2} i \, \sqrt{7} - \frac{1}{2} $ & $ t^{2} + t + 2 $ & $ \omega - 1 $ & $ x^{2} + 3 \, x + 4 $ & yes & $ 8 $ & yes & 20 & 20 & 19 & 19 & 19 \\
Quadratic+1+3+5\_complex1 & $ \frac{1}{2} i \, \sqrt{11} - \frac{3}{2} $ & $ t^{2} + 3 \, t + 5 $ & $ \omega $ & $ x^{2} + 3 \, x + 5 $ & yes & $ 9 $ & yes & 19 & 11 & 17 & 17 & 11 \\
Quadratic+1+3+5\_complex2  & $ \frac{1}{2} i \, \sqrt{11} - \frac{3}{2} $ & $ t^{2} + 3 \, t + 5 $ & $ \omega $ & $ x^{2} + 3 \, x + 5 $ & yes & $ 9 $ & yes & 39 & 31 & 34 & 34 & 31 \\
Quadratic+1+4+5\_complex1 & $ i $ & $ t^{2} + 1 $ & $ \omega - 2 $ & $ x^{2} + 4 \, x + 5 $ & yes & $ 10 $ & yes & 19 & 17 & 17 & 17 & 17 \\
Quadratic+1+4+5\_complex2 & $ i $ & $ t^{2} + 1 $ & $ \omega - 2 $ & $ x^{2} + 4 \, x + 5 $ & yes & $ 10 $ & yes & 17 & 17 & 17 & 17 & 17 \\
\end{tabular}

		\end{center}
	\caption{Comparing methods in Phase 1}
	\label{tab:resultsPhaseOne}
	\end{table}
\end{landscape}
%}

%\afterpage{
\clearpage   % To flush out all floats, might not be what you want
\begin{landscape}
	\begin{table}[h]
		\begin{center}
		\begin{tabular}{l|cc| ccc|  ccc|  ccc|  ccc}
\multirow{2}{*}{Name}  & Methods & \multirow{2}{*}{$\#\Q$}&\multicolumn{3}{c|}{$9$} & \multicolumn{3}{c|}{$15$} & \multicolumn{3}{c|}{$22$} & \multicolumn{3}{c}{$23$} \\
 & Phase 1&  &$bbb$ & Ph.2 & $r$ &$bbb$ & Ph.2 & $r$ &$bbb$ & Ph.2 & $r$ &$bbb$ & Ph.2 & $r$ \\ \hline
\multirow{1}{*}{Eisenstein\_1--block\_complex}& $12, 13, 14, 15, 16$ & $19$ &\checkmark & \checkmark & 3 & \checkmark & \checkmark & 3 & \checkmark & \checkmark & 3 & \checkmark & \checkmark & 3 \\
\hline
\multirow{2}{*}{Eisenstein\_1--block\_integer}& $12, 13, 15, 16$ & $57$ &\xmark & - & - & \xmark & - & - & \xmark & - & - & \xmark & - & - \\
& $14$ & $139$ &\xmark & - & - & \xmark & - & - & \xmark & - & - & \xmark & - & - \\
\hline
\multirow{1}{*}{Eisenstein\_2--block\_complex}& $12, 13, 14, 15, 16$ & $17$ &\xmark & - & - & \xmark & - & - & \xmark & - & - & \xmark & - & - \\
\hline
\multirow{1}{*}{Eisenstein\_2--block\_integer}& $12, 13, 14, 15, 16$ & $26$ &\xmark & - & - & \xmark & - & - & \xmark & - & - & \xmark & - & - \\
\hline
\multirow{1}{*}{Penney\_1--block\_complex}& $12, 13, 14, 15, 16$ & $45$ &\checkmark & \checkmark & 6 & \checkmark & \checkmark & 6 & \checkmark & \checkmark & 6 & \checkmark & \checkmark & 6 \\
\hline
\multirow{2}{*}{Penney\_1--block\_integer}& $12, 13, 15, 16$ & $27$ &\xmark & - & - & \xmark & - & - & \xmark & - & - & \xmark & - & - \\
& $14$ & $95$ &\xmark & - & - & \xmark & - & - & \xmark & - & - & \xmark & - & - \\
\hline
\multirow{1}{*}{Penney\_2--block\_integer}& $12, 13, 14, 15, 16$ & $27$ &\checkmark & \checkmark & 5 & \checkmark & \checkmark & 5 & \checkmark & \checkmark & 5 & \checkmark & \checkmark & 5 \\
\hline
\multirow{1}{*}{Quadratic+1+0--2\_integer}& $12, 13, 14, 15, 16$ & $9$ &\checkmark & \checkmark & 5 & \checkmark & \checkmark & 5 & \checkmark & \checkmark & 5 & \checkmark & \checkmark & 4 \\
\hline
\multirow{1}{*}{Quadratic+1+0--21\_integer}& $12, 13, 14, 15, 16$ & $9$ &\checkmark & \checkmark & 4 & \checkmark & \checkmark & 4 & \checkmark & \checkmark & 4 & \checkmark & \checkmark & 4 \\
\hline
\multirow{1}{*}{Quadratic+1+0--3\_integer}& $12, 13, 14, 15, 16$ & $9$ &\checkmark & \checkmark & 4 & \checkmark & \checkmark & 5 & \checkmark & \checkmark & 5 & \checkmark & \checkmark & 5 \\
\hline
\multirow{1}{*}{Quadratic+1+0--5\_integer}& $12, 13, 14, 15, 16$ & $9$ &\xmark & - & - & \checkmark & \checkmark & 3 & \checkmark & \checkmark & 2 & \checkmark & \checkmark & 2 \\
\hline
\multirow{1}{*}{Quadratic+1+2+3\_complex}& $12, 13, 14, 15, 16$ & $27$ &\checkmark & \xmark & - & \checkmark & \checkmark & 7 & \checkmark & \xmark & - & \checkmark & \xmark & - \\
\hline
\multirow{4}{*}{Quadratic+1+3+4\_complex}& $12$ & $20$ &\checkmark & \checkmark & 7 & \checkmark & \checkmark & 7 & \checkmark & \xmark & - & \xmark & - & - \\
& $16$ & $19$ &\checkmark & \xmark & - & \checkmark & \checkmark & 7 & \checkmark & \xmark & - & \xmark & - & - \\
& $13, 15$ & $20$ &\checkmark & \xmark & - & \checkmark & \checkmark & 7 & \checkmark & \xmark & - & \xmark & - & - \\
& $14$ & $21$ &\checkmark & \checkmark & 7 & \checkmark & \checkmark & 7 & \checkmark & \xmark & - & \xmark & - & - \\
\hline
\multirow{3}{*}{Quadratic+1+3+5\_complex1}& $14$ & $19$ &\xmark & - & - & \xmark & - & - & \xmark & - & - & \xmark & - & - \\
& $12, 16$ & $11$ &\xmark & - & - & \checkmark & \xmark & - & \xmark & - & - & \xmark & - & - \\
& $13, 15$ & $17$ &\xmark & - & - & \xmark & - & - & \xmark & - & - & \xmark & - & - \\
\hline
\multirow{3}{*}{Quadratic+1+3+5\_complex2 }& $12, 16$ & $33$ &\xmark & - & - & \checkmark & \xmark & - & \xmark & - & - & \xmark & - & - \\
& $13, 15$ & $39$ &\xmark & - & - & \checkmark & \xmark & - & \checkmark & \xmark & - & \xmark & - & - \\
& $14$ & $43$ &\xmark & - & - & \checkmark & \xmark & - & \checkmark & \xmark & - & \xmark & - & - \\
\hline
\multirow{2}{*}{Quadratic+1+4+5\_complex1}& $14$ & $19$ &\xmark & - & - & \xmark & - & - & \xmark & - & - & \xmark & - & - \\
& $12, 13, 15, 16$ & $17$ &\xmark & - & - & \xmark & - & - & \xmark & - & - & \xmark & - & - \\
\hline
\multirow{1}{*}{Quadratic+1+4+5\_complex2}& $12, 13, 14, 15, 16$ & $17$ &\checkmark & \checkmark & 3 & \checkmark & \checkmark & 3 & \checkmark & \checkmark & 3 & \checkmark & \checkmark & 3 \\
\hline
\end{tabular}

		\end{center}
	\caption{Comparing methods in Phase 2}
	\label{tab:resultsPhaseTwo}
	\end{table}
\end{landscape}
%}


\begin{table}[h]
	\begin{center}
	\begin{tabular}{lp{0.6\textwidth}}  Name & $\A$\\ \hline
Eisenstein\_1--block\_complex &
$ \left\{0, 1, -1, \omega, -\omega, -\omega - 1, \omega + 1\right\}  $ \\
Eisenstein\_1--block\_integer &
$ \left\{-3, -2, -1, 0, 1, 2, 3\right\}  $ \\
Eisenstein\_2--block\_complex &
$ \{0, 1, \omega, \omega + 1, 2 \, \omega, 2 \, \omega - 1, \omega - 1, -1, -2,$ \newline $-\omega, -\omega - 1, -\omega - 2, -2 \, \omega, -2 \, \omega - 1 \}  $ \\
Eisenstein\_2--block\_integer &
$ \{-\omega + 3, -\omega + 2, -\omega + 1, -\omega, 2, 1, 0, -1, \omega + 1,$ \newline $ \omega, \omega - 1, \omega - 2, 2 \, \omega, 2 \, \omega - 1, 2 \, \omega - 2, 2 \, \omega - 3\}  $ \\
Penney\_1--block\_complex &
$ \left\{0, \omega + 1, -\omega - 1, 1, -1\right\}  $ \\
Penney\_1--block\_integer &
$ \left\{-2, -1, 0, 1, 2\right\}  $ \\
Penney\_2--block\_integer &
$ \left\{0, 1, -1, \omega, -\omega, \omega - 1, -\omega + 1, \omega - 2, -\omega + 2\right\}  $ \\
Quadratic+1+0--2\_integer &
$ \left\{-1, 0, 1\right\}  $ \\
Quadratic+1+0--21\_integer &
$ \left\{-10, -9, -8, \dots, -1, 0, 1, \dots, 9, 10, 11\right\}  $ \\
Quadratic+1+0--3\_integer &
$ \left\{-1,0, 1,  2\right\}  $ \\
Quadratic+1+0--5\_integer &
$ \left\{-3, -2, -1, 0, 1, 2, 3, 4\right\}  $ \\
Quadratic+1+2+3\_complex &
$ \left\{0, \omega + 1, -\omega - 1, 1, -1, \omega\right\}  $ \\
Quadratic+1+3+4\_complex &
$ \left\{0, \omega + 1, -\omega - 1, 1, -1, \omega, -\omega, \omega + 2\right\}  $ \\
Quadratic+1+3+5\_complex1 &
$ \left\{0, 1, -1, \omega + 1, -\omega - 1, \omega + 2, -\omega - 2, \omega + 3, -\omega - 3\right\}  $ \\
Quadratic+1+3+5\_complex2  &
$ \left\{0, 1, -1, \omega + 1, -\omega - 1, \omega + 2, -\omega - 2, 2 \omega + 2, -2 \omega - 2\right\}  $ \\
Quadratic+1+4+5\_complex1 &
$ \left\{\omega + 2, \omega + 1, \omega, 1, 0, -1, -\omega + 1, -\omega, -\omega - 1, \omega - 1\right\}  $ \\
Quadratic+1+4+5\_complex2 &
$ \left\{\omega + 2, \omega + 1, \omega, 1, 0, -1, -\omega + 1, -\omega, -\omega - 1, 2\right\}  $ \\
Cubic+1+0+0+2\_integer &
$ \left\{-1, 0, 1\right\}  $ \\
Cubic+1+0+0--2\_integer &
$ \left\{-1, 0, 1\right\}  $ \\
\end{tabular}

	\end{center}
\caption{Alphabets for numeration systems in Table~\ref{tab:resultsPhaseOne} and \ref{tab:resultsPhaseTwo}}
\label{tab:alphabets}
\end{table}

The methods 1a, 1b, 1c, 1d and 1e, resp. 2a, 2b, 2c, 2d and 2e, which are desribed in Chapter~\ref{chap:diffChoices}, were selected as they represent groups of methods which behave similarly. Moreover, if a parallel addition algorithm was found for a numeration system, then at least one successful method is contained in the selection.



Let us explain Tables~\ref{tab:resultsPhaseOne}, \ref{tab:resultsPhaseTwo} and \ref{tab:alphabets}. Each row in Table~\ref{tab:resultsPhaseOne} represents one numeration system with a base $\beta\in\Zomega$ for a given algebraic integer $\omega$. The column \emph{conj.} signifies whether the base $\beta$ has a real conjugate greater than 1. The sizes of alphabets are listed and the column \emph{min.} says if the alphabets, which are listed in Table~\ref{tab:alphabets}, are minimal in the sense of the lower bound given by Theorem~\ref{thm:lowerBoundAlphabet}.  We remark that the size of an alphabet is compared with the bound regardless we work in $\Zbeta$ or $\Zomega$. 
There are the sizes of weight coefficients sets $\Q$ which were found  with various methods of Phase~1 in the last columns.

The results of Phase~2 for the selected numeration systems are shown in Table~\ref{tab:resultsPhaseTwo}. More rows for one numeration system correspond to distinct weight coefficients sets from Phase~1. The column $bb\dots b$ says whether control of $bb\dots b$ inputs was successful. If a weight function is found, it is denoted by \checkmark{} in the column \emph{Ph.2} and the length of window is in the column $r$. Symbol \xmark{} in the column \emph{Ph.2} means that a cycle in a Rauzy graph was found, i.e., Phase~2 does not converge. Reasons of non-convergence can be found in the appropriate example in Appendix~\ref{app:examples}.

We remark that Lemma~\ref{lem:suffCondPhase1} instead of some method of Phase~1 means that the set given by this lemma was used as $\Q$ instead of a computed one.  We see that the smaller weight coefficients set $\Q$ does not mean automatically better (Quadratic+1+4+5\_complex1 or Quadratic+1+3+4\_complex). An observation for many numeration systems is that if the extending window method is successful, then weight coefficients sets produced by different methods are similar. But it is not a rule. 

Unfortunately, there is also no best method for Phase~2. For example, method 2e is successful for all selected quadratic bases, but it fails for the cubic ones. Moreover, the length of window $r$ is not always minimal (Quadratic+1+0--5\_integer). On the other hand, the method 2d is the only one which finds a weight function for cubic bases, but it fails in many other cases. The method 2b seems to be often successful, but not with the optimal length of window.

An interesting example is Quadratic+1+4+5\_complex1. Only one element of the alphabet is different, comparing with Quadratic+1+4+5\_complex2. Whereas a weight function is easily found for Quadratic+1+4+5\_complex2 by all methods, the only successfull combination of methods for Quadratic+1+4+5\_complex1 is the weight coefficients set given by Lemma~\ref{lem:suffCondPhase1} and method 2e. We remark that many of elements of such $\Q$ are not used as outputs of the obtained weight function. Notice  in Example~\ref{ex:compareAP} that there are only few inputs digits $b$ such that  Quadratic+1+4+5\_complex1 fails in the check of $bb\dots b$ inputs.





\section{Examples of results}
\label{sec:resultsExamples}
We divide nume


\begin{table}[h]
	\begin{center}
	\begin{tabular}{l|c|cc c| c c| c| c c c }
Ex. &$\omega$ & $\beta$ & $m_\beta$ & conj. & $\#\A$ & min. & $\#\Q$ & $bb\dots b$ & Phase 2 & $r$   \\ \hline
\ref{ex:tAA} & $ \frac{1}{2} i \, \sqrt{3} - \frac{1}{2} $ & $ \omega - 1 $ & $ x^{2} + 3 \, x + 3 $ & ? & $ 7 $ & yes & $ 19 $ & \checkmark & \checkmark & 3 \\
\ref{ex:tAB} & $ \frac{1}{2} i \, \sqrt{3} - \frac{1}{2} $ & $ \omega - 1 $ & $ x^{2} + 3 \, x + 3 $ & ? & $ 7 $ & yes & $ 19 $ & \checkmark & \checkmark & 3 \\
\ref{ex:tAC} & $ \frac{1}{2} i \, \sqrt{3} - \frac{1}{2} $ & $ \omega - 1 $ & $ x^{2} + 3 \, x + 3 $ & ? & $ 7 $ & yes & $ 19 $ & \checkmark & \checkmark & 3 \\
\ref{ex:tAD} & $ i - 1 $ & $ \omega $ & $ x^{2} + 2 \, x + 2 $ & ? & $ 5 $ & yes & $ 45 $ & \checkmark & \checkmark & 6 \\
\ref{ex:tAE} & $ i - 1 $ & $ \omega $ & $ x^{2} + 2 \, x + 2 $ & ? & $ 5 $ & yes & $ 45 $ & \checkmark & \checkmark & 6 \\
\ref{ex:tAF} & $ i \, \sqrt{2} - 1 $ & $ -\omega - 2 $ & $ x^{2} + 2 \, x + 3 $ & ? & $ 6 $ & yes & $ 27 $ & \checkmark & \checkmark & 7 \\
\ref{ex:tAG} & $ \frac{1}{2} i \, \sqrt{11} - \frac{3}{2} $ & $ \omega $ & $ x^{2} + 3 \, x + 5 $ & ? & $ 9 $ & yes & $ 11 $ & \checkmark & \xmark & - \\
\ref{ex:tAH} & $ \frac{1}{2} i \, \sqrt{11} - \frac{3}{2} $ & $ \omega $ & $ x^{2} + 3 \, x + 5 $ & ? & $ 9 $ & yes & $ 11 $ & \checkmark & \xmark & - \\
\end{tabular}
	\end{center}
\caption{Quadratic bases with a non-integer alphabet (using methods 1d and 2b)}
\label{tab:resultsQuadrNonint}
\end{table}

 
\begin{table}[h]
	\begin{center}
	\begin{tabular}{c|cc c| c c| c| c c c|l }
$\omega$ & $\beta$ & $m_\beta$ & conj. & $\#\A$ & min. & $\#\Q$ & $bb\dots b$ & Phase 2 & $r$& Ex.   \\ \hline
$ \frac{1}{2} i \, \sqrt{11} + \frac{1}{2} $ & $ -i \, \sqrt{11} $ & $ x^{2} + 11 $ & no & $ 13 $ & no & $ 9 $ & \checkmark & \checkmark & 2 \\
$ \frac{1}{2} i \, \sqrt{11} + \frac{1}{2} $ & $ -i \, \sqrt{11} $ & $ x^{2} + 11 $ & no & $ 12 $ & yes & $ 9 $ & \checkmark & \checkmark & 4  & \ref{ex:integerAB}\\
$ \frac{1}{2} i \, \sqrt{7} - \frac{1}{2} $ & $ -i \, \sqrt{7} $ & $ x^{2} + 7 $ & no & $ 9 $ & no & $ 9 $ & \checkmark & \checkmark & 2 \\
$ \frac{1}{2} i \, \sqrt{7} - \frac{1}{2} $ & $ -i \, \sqrt{7} $ & $ x^{2} + 7 $ & no & $ 8 $ & yes & $ 9 $ & \checkmark & \checkmark & 4  & \ref{ex:integerAD}\\
$ \frac{1}{2} i \, \sqrt{3} + \frac{1}{2} $ & $ -\frac{3}{2} i \, \sqrt{3} + \frac{1}{2} $ & $ x^{2} - x + 7 $ & no & $ 11 $ & no & $ 9 $ & \checkmark & \checkmark & 2  & \ref{ex:integerAE} \\
$ i \, \sqrt{3} $ & $ i \, \sqrt{3} $ & $ x^{2} + 3 $ & no & $ 4 $ & yes & $ 9 $ & \checkmark & \checkmark & 4 \\
$ i \, \sqrt{2} $ & $ i \, \sqrt{2} $ & $ x^{2} + 2 $ & no & $ 3 $ & yes & $ 9 $ & \checkmark & \checkmark & 4  & \ref{ex:integerAG}\\
$ \sqrt{2} $ & $ -\sqrt{2} $ & $ x^{2} - 2 $ & yes & $ 3 $ & yes & $ 9 $ & \checkmark & \checkmark & 5  & \ref{ex:integerAH}\\
$ \sqrt{3} - 1 $ & $ -\sqrt{3} $ & $ x^{2} - 3 $ & yes & $ 4 $ & yes & $ 9 $ & \checkmark & \checkmark & 5 \\
$ \frac{1}{2} \, \sqrt{5} + \frac{1}{2} $ & $ -\sqrt{5} $ & $ x^{2} - 5 $ & yes & $ 6 $ & yes & $ 9 $ & \checkmark & \checkmark & 4  & \ref{ex:integerAJ}\\
$ -\sqrt{5} + 1 $ & $ -\sqrt{5} $ & $ x^{2} - 5 $ & yes & $ 6 $ & yes & $ 9 $ & \checkmark & \checkmark & 4  & \ref{ex:integerAK}\\
$ -\sqrt{6} + 1 $ & $ -\sqrt{6} $ & $ x^{2} - 6 $ & yes & $ 7 $ & yes & $ 9 $ & \checkmark & \checkmark & 4 \\
$ \sqrt{6} - 1 $ & $ \sqrt{6} $ & $ x^{2} - 6 $ & yes & $ 7 $ & yes & $ 9 $ & \checkmark & \checkmark & 4 \\
$ -\sqrt{7} + 2 $ & $ \sqrt{7} $ & $ x^{2} - 7 $ & yes & $ 8 $ & yes & $ 9 $ & \checkmark & \checkmark & 4 \\
$ \frac{1}{2} \, \sqrt{13} + \frac{1}{2} $ & $ -\sqrt{13} $ & $ x^{2} - 13 $ & yes & $ 15 $ & no & $ 9 $ & \checkmark & \checkmark & 2  & \ref{ex:integerAO}\\
$ \frac{1}{2} \, \sqrt{13} + \frac{1}{2} $ & $ -\sqrt{13} $ & $ x^{2} - 13 $ & yes & $ 14 $ & yes & $ 9 $ & \checkmark & \checkmark & 4 \\
$ -\frac{1}{2} \, \sqrt{17} + \frac{3}{2} $ & $ -\sqrt{17} $ & $ x^{2} - 17 $ & yes & $ 18 $ & yes & $ 9 $ & \checkmark & \checkmark & 4  & \ref{ex:integerAQ}\\
$ -\frac{1}{2} \, \sqrt{21} + \frac{3}{2} $ & $ -\sqrt{21} $ & $ x^{2} - 21 $ & yes & $ 22 $ & yes & $ 9 $ & \checkmark & \checkmark & 4  & \ref{ex:integerAR}\\
\end{tabular}
	\end{center}
\caption{Quadratic bases with an integer alphabet (using methods 1d and 2b)}
\label{tab:resultsQuadrInt}
\end{table}


\section{Google spreadsheet ParallelAddition\_results}
\label{sec:gspreadsheet}
All tested examples can be found in the spreadsheet \href{https://docs.google.com/spreadsheets/d/1TnhrHdefHfHa0WSeVs4q6XVj3epjPlPlnoekE0E1xeM/edit?usp=sharing}{ParallelAddition\_results}. Its structure is following -- the worksheet \verb+results+ is used for automatically saved results. Results may be copied to \verb+results_archive+ to reduce the number of rows in \verb+results+. The meaning of each column is obvious from the heading line. 

The worksheet \verb+inputs+ serves for loading inputs by the script \verb+ewm_gspreadsheet.sage+. The numeration systems for which a weight function was successfully found are in the worksheet \verb+successful+. We remark that they are listed with a single combination of methods for Phase~1 and 2. Nevertheless, all tested variants remain in \verb+results+ and \verb+results_archive+. 

Results for selected numeration systems which were tested with various  methods for Phase~1 and 2 are sorted in the worksheet \verb+comparePhase2+. Note that if the result of Phase~1 is same for more methods, then only one of them is used for testing of methods for Phase~2. This fact can be found in the column \emph{Groups of methods with the same result} in the worksheet \verb+comparePhase1+.

Very useful property of storing data in a worksheet is easy sorting and filtering.

The version of \href{https://docs.google.com/spreadsheets/d/1TnhrHdefHfHa0WSeVs4q6XVj3epjPlPlnoekE0E1xeM/edit?usp=sharing}{ParallelAddition\_results} which is on the attached DVD was downloaded on May 2, 2016.
